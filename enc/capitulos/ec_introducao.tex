%! Author = Jander Moreira
%! Date = 27/02/2023


Muito além dos computadores disponíveis em mesas de escritório, a computação hoje está presente em diversos dispositivos como veículos, celulares, televisores, câmeras, eletrodomésticos entre outros. O Engenheiro de Computação é um profissional capacitado tanto para conceber estes sistemas -- incluindo seu \textit{hardware} e \textit{software} -- quanto para integrá-los em sistemas ou soluções pré-existentes. Para tal, a formação desse profissional deve abordar conhecimentos técnicos aprofundados e também outros que lhe permitam atuar de maneira ética, contribuindo para a melhoria da vida em sociedade.

No entanto, sendo essa formação majoritariamente de base tecnológica, se faz necessário atualizá-la frente aos avanços do conhecimento nas áreas de engenharia e computação. Da mesma forma, deve-se considerar as novas demandas sociais decorrentes do contato cada vez mais frequente das pessoas com os dispositivos computacionais. Nesse contexto, apresenta-se aqui a reformulação curricular para o curso de Bacharelado em Engenharia de Computação (EC) da Universidade Federal de São Carlos (UFSCar).

Este Projeto Pedagógico de Curso (PPC), desenvolvido pelo Núcleo Docente Estruturante (NDE) do curso ao longo dos últimos anos, procurou seguir as recomendações mais recentes da \textit{Association for Computing Machinery} (ACM) e do \textit{Institute of Electrical and Electronics Engineers} (IEEE), principais associações de profissionais da área, atendendo às normativas do Conselho Nacional de Educação (CNE) e dos regimentos e normas da UFSCar.

Desde sua criação em 1992, passando por sua última reformulação em 2018, o curso sempre formou egressos com alta taxa de empregabilidade no mercado de trabalho. Assim, nesta reformulação, buscou-se continuar atendendo às demandas atuais de atuação profissional. Contudo, buscou-se refletir no PPC a vocação do Departamento de Computação da UFSCar para gerar novo conhecimento nas áreas específicas e com isso contribuir na formação de um profissional com maior escopo de atuação.

Em particular, este novo projeto pedagógico atende às novas Diretrizes Curriculares Nacionais (DCNs) para as Engenharias, instituídas pela Câmara de Educação Superior do Conselho Nacional de Educação (CES/CNE) através da Resolução nº 02/2019~\cite{CNE2019}, as quais tratam das competências que o egresso deve possuir para sua atuação profissional.

%% TODO: atualizar datas e reuniões
\textcolor{red}{Esse PPC foi aprovado pelo Conselho de Curso da EC em sua 45ª reunião ordinária de 12/07/2018 e pelo Conselho do Departamento de Computação em sua 3ª reunião extraordinária de 13/07/2018.}


% \section{Ficha Técnica do Curso}

% \menotti{Alterar os valores quando fecharmos a matriz curricular}

% \begin{table}[h!]
% %\caption{Ficha Técnica do Curso}
% \centering
% %%\footnotesize
% \begin{tabular}{rcc}
% \hline
% \multicolumn{3}{c}{\textbf{{\large Bacharelado em Engenharia de Computação - DC-UFSCar}}}  \\
% \hline
% \hline
% Formação: & \multicolumn{2}{c} {Bacharel em Engenharia de Computação} \\
% \hline
% Número de Vagas Anuais: & \multicolumn{2}{c} {30} \\
% \hline
% Regime Escolar:  & \multicolumn{2}{c} {Sistema de Créditos Semestral} \\
% \hline
% Turno de Funcionamento:  & \multicolumn{2}{c} {Diurno (Integral)} \\
% \hline
% Integralização Curricular Prevista:  & \multicolumn{2}{c} {10 Semestres } \\
% \hline
% Integralização Curricular Mínima: & \multicolumn{2}{c} {8 Semestres} \\
% \hline
% Integralização Curricular Máxima: & \multicolumn{2}{c} {18 Semestres} \\
% \hline
% \hline
% \multicolumn{1}{c}{\textbf{Atividades}} &  \hphantom{xxxxx}  \textbf{Créditos}  \hphantom{xxxxx}  &  \textbf{Carga Horária} \\
% \hline
% Disciplinas Obrigatórias:  &  xx   &	xx  \\
% \hline
% Disciplinas Optativas:     &  xx   &	xx  \\
% \hline
% Projeto Integrador:        &  xx  & xx \\
% \hline
% Estágio Curricular:        &  xx  &	 xx  \\
% \hline
% Atividades Complementares: &   xx     &  xx     \\
% \hline
% \textbf{Total:}           &   \textbf{xx}  &      \textbf{xx} \\
% \hline
% \end{tabular}
% \label{tab:x86}
% \end{table}


\section{Organização deste Documento}

% TODO: rever quando for fechar o documento
O restante deste documento está organizado da seguinte forma. No Capítulo~\ref{cha:MarcoReferencial} é apresentado o Marco Referencial do curso, contendo a área de conhecimento predominante e o campo de atuação profissional, justificativa de sua criação, objetivos, evolução institucional e histórico de suas avaliações e reformulações curriculares. O Marco Conceitual do curso, apresentado no Capítulo~\ref{cha:MarcoConceitual}, descreve o perfil do profissional a ser formado, bem como os saberes e as competências desejadas, em consonância com o estabelecido na Resolução CNE/CES n° 2/2019~\cite{CNE2019} e no ``Perfil do Profissional a ser formado na UFSCar''~\cite{ufscar2008perfil} . O Marco Estrutural do curso, descrevendo toda sua organização curricular, é apresentado no Capítulo~\ref{cha:MarcoEstrutural}. Por fim, no Capítulo~\ref{cha:implantacao} é descrito o plano de implantação do PPC, listando o pessoal e a infraestrutura disponíveis para o seu funcionamento.
