%! Author = Jander Moreira
%! Date = 27/02/2023

% ESTE TEXTO ESTAVA TODO COMENTADO NO ARQUIVO ORIGINAL (Jander)
% E CONTINUA COMENTADO AQUI...



%%%%%%%%%%%%%%%%%%%%%%%%%%%%%%%%%%%%%%%%%%%%%%%%%%%%%%%%%%%%%%%%%%%%
%\chapter{Avaliação de Aprendizagem}~\label{cha:avaliacao}

% \section{Adiantamento de conhecimento}
% Um dos objetivos deste curso, conforme descrito no Capítulo \autoref{intro}, é proporcionar experiências de aprendizagem onde o estudante desenvolve atividades complementares fora da sala de aula, seja em forma de Iniciação Científica, Projeto Integrador, ou mesmo projetos pessoais. Nessas experiências, é esperado que o aprendizado do estudante extrapole o domínio de conhecimento de um único projeto ou disciplina. Como resultado disso, é comum que o estudante desenvolva competências, por conta própria ou sob orientação docente, que são cobertas, parcialmente ou totalmente, em disciplinas que ainda não tenha cursado.

% Ao mesmo tempo, a Portaria GR/UFSCar no~522/06, em seu Art. 2o~, define que as avaliações devem, entre outras funções, ``diagnosticar o conhecimento prévio dos estudantes''. Neste sentido, o curso de Bacharelado em Ciência da Computação prevê a possibilidade do adiantamento do conhecimento, para que as competências adquiridas pelos estudantes nessas experiências de aprendizagem fora de sala de aula possam ser convertidas em frequência, nota ou créditos, totais ou parciais, estimulando assim uma maior busca por um aprendizado ativo e focado em atividades de cunho prático.

% Nesta seção são descritos os termos para o adiantamento de conhecimento para o curso de Bacharelado em Engenharia de Computação.

% \subsection{Condições para solicitação de exame de adiantamento de conhecimento}

% Se o estudante comprovar domínio de conhecimento de conteúdo relativo a uma parte ou à totalidade de qualquer das disciplinas disponíveis para exame de adiantamento de conhecimento (listadas na próxima seção), poderá requerê-lo a qualquer momento do curso. Caso a requisição seja aprovada, o estudante poderá ser dispensado parcialmente ou totalmente da disciplina.

% O exame de adiantamento de conhecimento é individual, e deve ser solicitado pelo estudante à Coordenação do Curso. A solicitação deve ser circunstanciada e indicar o domínio de conhecimento de conteúdo. É imprescindível que esse domínio de conhecimento tenha sido adquirido como resultado de alguma atividade de cunho prático realizada antes (sem limite de tempo???) ou durante o curso de graduação, e que se enquadre nas seguintes situações:

% \begin{itemize}
%     \item Projeto de iniciação científica, com ou sem bolsa;
%     \item Desempenho de atividade profissional, remunerada ou não remunerada;
%     \item Projeto acadêmico, realizado em disciplina da graduação;
%     \item Projeto integrador, conforme descrito no \autoref{integra}.
%     \item Outras situações????
% \end{itemize}

% É também imprescindível que os resultados da atividade realizada estejam disponíveis para análise e comprovação.

% Ao realizar a solicitação, o estudante deve indicar claramente para quais disciplinas e quais conteúdos e competências dentro das disciplinas o exame deve cobrir. O projeto pedagógico do curso ficará disponível para consulta do estudante e deve ser utilizado como referência para essa indicação.

% \subsection{Disciplinas disponíveis para adiantamento de conhecimento}

% Somente podem ser solicitados exames de adiantamento de conhecimento para as disciplinas listadas a seguir:

% \begin{itemize}
%     \item Construção de compiladores
%     \item Computação gráfica
%     \item Banco de dados
%     \item ...
% \end{itemize}

% \subsection{Sobre o exame de adiantamento de conhecimento}

% Uma vez aprovado o pedido de adiantamento de conhecimento, por parte da Coordenação do Curso, o processo será encaminhado ao coordenador da(s) disciplina(s) citadas (agora temos essa figura no curso, não?), que providenciará todos os atos e procedimentos para a convocação do estudante, elaboração e aplicação do exame, e emissão de parecer justificado.

% O formato e conteúdo do exame serão definidos pelo coordenador da(s) disciplina(s), ou por uma comissão de um ou mais docentes por ele indicados. O formato e conteúdo irão depender da natureza da solicitação, podendo versar sobre os tópicos indicados pelo estudante, mas também outros tópicos que possam ser julgados correlatos. O exame deve obrigatoriamente incluir uma demonstração de cunho prático, mas pode também contemplar atividades avaliativas teóricas, por escrito e/ou em formato oral.

% O parecer emitido após o exame deve indicar e justificar claramente e de forma objetiva se a solicitação foi aceita, total ou parcialmente.

% \subsection{Sobre os resultados do exame de adiantamento de conhecimento}

% São admitidas as seguintes possibilidades de resultado a serem indicados no parecer:

% \begin{itemize}

% \item O estudante é completamente dispensado da disciplina. Neste caso, a disciplina será lançada no histórico como APROVEITAMENTO POR EQUIVALÊNCIA (não sei como isso pode ser feito);

% \item O estudante é dispensado de cursar parte da disciplina. Neste caso, o parecer deve indicar claramente de quais conteúdos o estudante está dispensado. O estudante deve posteriormente se matricular na disciplina e entregar o parecer ao professor responsável no início do semestre letivo. O professor responsável pela oferta irá decidir e comunicar ao estudante de quais aulas ou provas o estudante estará dispensado, com base no parecer;

% \item O estudante é reprovado no exame. Neste caso, ele não poderá solicitar novamente adiantamento de conhecimento para as disciplinas envolvidas (para evitar abusos?).

% \end{itemize}

% Uma vez emitido, o parecer deve ser aprovado pelo conselho de curso, caso contrário não terá validade.

% A ausência não justificada ou a não realização pelo estudante do exame de adiantamento de conhecimento implicará em reprovação.

%%%%%%%%%%%%%%%%%%%%%%%%%%%%%%%%%%%%%%%%%%%%%%%%%%%%%%%%%%%%%%%%%%%%

% \chapter{Matriz Curricular}

% Figura gerada a partir da planilha que está em:
% https://docs.google.com/spreadsheets/d/18P6oRUgPfpKe1KhB07A2B20fqWytffCImKi9RDmnzCg/edit#gid=221983726
%
% \begin{sidewaysfigure}
%     \centering
%     \includegraphics[width=\textwidth]{imagens/GradeEnc2}
%     \caption{Matriz Curricular}
%     \label{fig:my_label}
% \end{sidewaysfigure}

% \newcommand{\sem}[1]{Semestre #1}
% \newcommand{\disc}[4]{#1\\{\scriptsize\textit{#2 #3 #4}}}

% \begin{tikzpicture}[scale=0.75, transform shape]
%     \tikzstyle{Padrao} = [draw=black, minimum width=2.75cm, minimum height=0.75cm, node distance=1cm, rounded corners, drop shadow, text centered]
%     \tikzstyle{Sem}    = [Padrao, fill=  blue!15]
%     \tikzstyle{Disc}   = [Padrao, fill=yellow!10]
%     \tikzstyle{DiscDM} = [Padrao, fill=yellow!10]


%     \node[name=s0]{};
%     \node[name=s1, below of=s0, Sem] {\sem{1}};
%     \node[name=s2, below of=s1, Sem] {\sem{2}};
%     \node[name=s3, below of=s2, Sem] {\sem{3}};
%     \node[name=s4, below of=s3, Sem] {\sem{4}};
%     \node[name=s5, below of=s4, Sem] {\sem{5}};
%     \node[name=s6, below of=s5, Sem] {\sem{6}};
%     \node[name=s7, below of=s6, Sem] {\sem{7}};
%     \node[name=s8, below of=s7, Sem] {\sem{8}};
%     \node[name=s9, below of=s8, Sem] {\sem{8}};
%     \node[name=s10, below of=s9, Sem] {\sem{10}};

%     \node[name=Calc1, right=0.25cm of s1, DiscDM]{\disc{Cálculo 1}{6}{5}{1}};
%     \node[name=Calc2, below of=Calc1, DiscDM]{\disc{Cálculo 2}{4}{4}{0}};
%     \node[name=Calc3, below of=Calc2, DiscDM]{\disc{Cálculo 3}{4}{4}{0}};

% \end{tikzpicture}

% Matriz do BCC:
% https://docs.google.com/spreadsheets/d/1ZJPOFfp3dmF_kGd_stOBlS2b-FSYeue4xiMfZCthqOk/edit#gid=1026637429


% \section{Matriz Curricular}


% \section{Metodologias de Ensino}

% falar das metodologias de ensino. quais disciplinas são ead, quais são pbl.


% \section{Integralização de Créditos}

% Para integralização do curso de Bacharelado em Ciência da Computação, o estudante deverá ser aprovado em 210 créditos (3150 horas-aula), além das 10 horas de atividades complementares. Os créditos para integralização do curso estão assim distribuídos:

% \begin{itemize}
% \item Formação Obrigatória XXXX créditos de disciplinas obrigatórias;
% \item Formação Complementar XXXX créditos, assim divididos:
% \begin{itemize}
% \item Disciplinas Optativas Profissionalizantes - XXXX créditos
% \item Disciplinas para o Desenvolvimento Humano e Complementar - XXXX créditos
% \item Estágio - 24 créditos.
% \end{itemize}
% \end{itemize}


% \section{Coordenador de Disciplinas}

% Professor responsável por gerar um plano de ensino comum para a disciplina ao longo do semestre. Esse professor é definido pelo NDE.  O coordenador da disciplina é subordinado ao coordenador da área e deve supervisionar as avaliações da disciplina em conjunto com o mesmo.

% \section{Carga Horária}

% \subsection{Carga Horária Semestral do Curso}
% O curso tem a seguinte carga horária semestral:
%  XXXXX horas de atividades divididas da seguinte maneira: XXXX horas de atividades em sala de aula (XXXX créditos)+ XXXXX horas de atividades extra-classe;

% \subsection{Carga Horária Semanal do Curso}
% \textbf{As atividades do curso deverão ser concebidas e planejadas prevendo a seguinte carga horária semanal aos estudantes: um total de 46 horas divididas da seguinte maneira:
% - 26 horas em sala de aula (26 créditos) + 4 horas de atividades extra-classe por disciplina (15 horas). Isto não deve ser interpretado como uma regra absoluta, uma vez que é natural que ocorram \lq\lq picos \rq\rq de trabalho ao longo do semestre. Porém, é uma diretriz que deverá ser perseguida o máximo possível.}


% \section{Disciplinas - Regulamento}

% As disciplinas devem ser definidas no projeto pedagógico de acordo com os itens detalhados a seguir.

% \begin{enumerate}
% \item Área da Disciplina: As disciplinas do curso são divididas em áreas de formação pré-estabelecidas no projeto pedagógico de acordo com os tópicos abordados nas mesmas.
% \item Coordenador da Disciplina: Professor responsável por gerar um plano de ensino comum para a disciplina ao longo do semestre.
% \item Metodologia de Ensino: Descrever a metodologia usada na disciplina, exemplo, aula expositiva, aula baseada em resolução de problemas, etc.
% \item Período (Semestre do Curso): O período que o estudante de perfil deve realizar a disciplina
% \item  Objetivo: Objetivo principal da disciplina.
% \item Pré-Requisito: Disciplinas que devem ter sido concluídas com êxito para cursar a mesma.
% \item Carga Horária Teórica e Prática: Definição do número de Horas de Crédito e Atividades Extra-Classe por Disciplina;
% \item Detalhamento de Esforço Semestral da Disciplina: A disciplina deverá ter especificado a carga horária correspondente ao esforço necessário para que o estudante consiga ter um aproveitamento satisfatório da mesma. Os itens a serem considerados na soma total do esforço:

% \begin{itemize}
% \item horas em sala de aula: aula teórica + aula prática;
% \item aulas de exercícios;
% \item trabalhos extra-classe valendo nota;
% \item tempo estimado de estudo, resolução de exercícios e preparação para aulas e preparação para avaliação;
% \end{itemize}

% \item Detalhamento de Esforço Semanal da Disciplina: Carga horária semanal para as atividades definidas anteriormente.
% \item Ementa: Descrição do conjunto tópicos a serem abordados na disciplina. Para cada tópico da ementa associar:

% \begin{itemize}
% \item Carga horária (teórico e prática; sala de aula e extra-classe);
% \item Competência a ser atingida do curso para cada tópico;
% \end{itemize}

% \item Plano de Ensino da Disciplina: O plano de Ensino corresponde ao detalhamento da atividade de uma disciplina para uma turma. Os planos de ensino da mesma disciplina do curso devem ser compatibilizados por semestre pelo Coordenador da Disciplina e supervisionado pelo Coordenador da Área em que a disciplina está inserida.

% \item Padrão de Avaliação: Especificar o padrão de avaliação para o estudante ser avaliado para atingir a competência associada a cada tópico.

% \item Bibliografia: A Bibliografia deve conter pelo menos 3livros textos e 3 livros de materiais complementares. O professor Coordenador da Disciplina deve checar se os livros da mesma existem no acervo da Biblioteca e caso não exista, providenciar um pedido de compra de livros junto a Biblioteca ou Departamento, protocolado e entregue à Secretaria de Coordenação de Curso.

% \item) Softwares Utilizados da Disciplina: Se a disciplina usar softwares listar os mesmos e as versões no plano de ensino. Usar preferencialmente softwares livres. Fica a cargo do Coordenador da Disciplina solicitar e disponibilizar a instalação do ambiente computacional 15 dias antes.

% \end{enumerate}

% \section{Núcleo de Conteúdos Básicos}
% A carga horária mínima do núcleo de conteúdos básicos previsto pela Resolução CNE/CES 11/2002 é de cerca de 30\% do total previsto. Como acreditamos que esta formação básica é fundamental para capacitar o egresso a aprender de forma autônoma e continuada, tornando-o mais versátil e capaz de acompanhar o desenvolvimento tecnológico (a especificidade se torna obsoleta rapidamente, o fundamental persiste), especial atenção foi direcionada para a composição deste núcleo. Essa formação básica será adquirida pelo estudante, através dos seguintes grupos de conhecimentos provenientes das ciências básicas e ciências humanas.

% Os conteúdos de Computação Clássica, relacionados à formação geral do Cientista de Computação, darão ao estudante, o aprendizado dos conceitos fundamentais que constituem a base da área computacional, necessários para a sua atuação profissional.

% \section{Núcleo de Conteúdos Profissionalizantes}

% De acordo com o parágrafo 3º, Artigo 6º, a Resolução CNE/CES 11/2002, o Núcleo de Conteúdos Profissionalizantes deve ser composto por cerca de 15\% de carga horária mínima. Neste núcleo se concentram os conteúdos de caráter profissionalizante dos cursos de computação. Os conteúdos oferecidos neste grupo capacitarão o estudante aplicar os conceitos adquiridos ao longo de todo o curso de Ciência da Computação.

% \section{Núcleo de Formação Específica}
% Segundo o parágrafo 4\textordmasculine, Artigo 6\textordmasculine, da Resolução CNE/CES n\textordmasculine 11/2002, os conteúdos abordados nos módulos se caracterizam pela especificidade em relação às extensões e aprofundamentos (...), bem como de outros conteúdos destinados a caracterizar modalidades. Desta forma, os conteúdos concentrados nesse núcleo definem o curso de Ciência da Computação. Sendo o curso de Ciência da Computação multidisciplinar,	incorporando	simultaneamente	formações	específicas	muito particulares com um perfil generalista, a construção de uma estrutura curricular que seja ao mesmo tempo flexível, abrangente e pontualmente profunda é um grande desafio.

% \section{Núcleo de Práticas Complementares}
% No projeto do curso existem três disciplinas, que darão ao estudante a oportunidade de realizar atividades acadêmicas, industriais e/ou sociais, que enriquecerão, profundamente, a sua formação profissional. Estas disciplinas são:

%\begin{itemize}
%\item Estágio Curricular
%\item Trabalho de Final de Curso
%\end{itemize}

% %%%%%%%%%%%%%%%%%%%%%%%%%%%%%%%%%%%%%%%%%%%%%%%%%%%%%%%%%%%%%%%%%%%%
% \chapter{Ementário de Disciplinas}
% %TODO: copiar do projeto ou do docs no Bcc estas disciplinas

% \section{Eixo: Fundamentos da Matemática e Estatística}
%     \disciplina{calculo1}{
    \titulo      {1}{Cálculo Diferencial e Integral 1}
    \objetivo    {Propiciar o aprendizado dos conceitos de limite, derivada e integral de funções de uma variável real. Propiciar a compreensão e o domínio dos conceitos e das técnicas de cálculo diferencial e integral. Desenvolver a habilidade de implementação desses conceitos e técnicas em problemas nos quais eles se constituem os modelos mais adequados. Desenvolver a linguagem matemática como forma universal de expressão da ciência.}
    % \requisitos{}
    \recomendadas{N/A}
    \ementa      {Números reais e funções de uma variável real. Limites e continuidade. Cálculo Diferencial e aplicações. Cálculo integral e aplicações.}
    \creditos{6 total (5 teóricos, 1 prático)}
    %     % \horas    {90 total (75 teóricas, 15 práticas)}
    % %    \extra       {6 horas}
    \codigo      {DM}{08.221-0}
    \bibliografia {
        GUIDORIZZI, H. L. Um curso de cálculo v. 1 – 5a. Edição, Rio de Janeiro: Livros Técnicos e Científicos, 2001.

        STEWART, J. Cálculo v. 1 – 5a. Edição, São Paulo: Pioneira Thomson Learning, 2006.

        SWOKOWSKI, E. W. Cálculo com geometria analítica v. 1 – 2a. Edição, São Paulo: McGraw-Hill do Brasil, 1994.
    }{
        ANTON, H. Cálculo v. 1, 10. ed, Porto Alegre, RS: Bookman, 2014.

        ÁVILA, G. Calculo: funções de uma variável v. 1 - 6a. Edição, Rio de Janeiro: Livros Técnicos e Científicos, 1994.

        FLEMMING, D. M.; GONCALVES, M. B. Cálculo A: funções, limite, derivação, integração – 6a. Edição, São Paulo: Prentice Hall, 2006.

        LEITHOLD, L. O cálculo com geometria analítica v. 1 – 3. ed, São Paulo: Harbra, 1991.

        THOMAS, G. B. et al. Cálculo, v.1 - 10a. Edição, Addison Wesley, 2002.
    }

    % Inserido por Murillo R. P. Homem, em 01/04/2023
     \competencias{
        cg-aprender/{ce-ap-1, ce-ap-2, ce-ap-3, ce-ap-4},
        cg-atuar/{ce-atuar-1, ce-atuar-2, ce-atuar-3, ce-atuar-4},
        }
        
}

%     \disciplina{algelin}{
    \titulo      {2}{Álgebra Linear}
    \objetivo    {Levar o aluno a entender e reconhecer as estruturas da Álgebra Linear,
        que aparecem em diversas áreas da matemática e, trabalhar com estas estruturas, tanto abstrata como concretamente (através de cálculo com representações matriciais).
    }
    \requisitos  {Geometria Analítica} % 81116 OU (215279 E 215384) OU 343510 OU 342017 OU 342190 OU 81515 OU 345083 OU 345970 OU 524182 % 
    \recomendadas{N/A}
    \ementa      {Espaços Vetoriais; Transformações Lineares; Diagonalização de
    Matrizes; Espaços com Produto Interno; Formas Bilineares e
    Quadráticas.}
    \creditos    {4 total (3 teóricos, 1 práticos)}
    %    \extra       {x horas}
    \codigo      {DM}{08.013-6} % Corrigido CoC DC -> DM % Sabrina 
    \bibliografia{
        BOLDRINI, J. L. et al. Álgebra Linear. 3. ed. São Paulo: Harbra, 1986.

        CALLIOLI, C. A.; DOMINGUES, H. H.; Costa, R. C. F. Álgebra linear e aplicações. 6. ed.São Paulo: Atual, 2013.

        COELHO, F. U.; LOURENÇO, M. L. Um curso de álgebra linear. 2. ed. São Paulo: EdUSP, 2010.
    }{
        ANTON, H.; BUSBY, R. C. Álgebra linear contemporânea. Porto Alegre, RS: Bookman, 2008.

        LIMA, E. L. Álgebra linear. 5. ed. Rio de Janeiro: IMPA. (Coleção Matemática Universitária), 2001.

        HOFFMAN, K.; KUNZE, R. Álgebra linear. 2. ed. Rio de Janeiro: Livros Técnicos e Científicos, 1979.

        LANG, S. Álgebra linear. São Paulo: Edgard Blucher, 1971.

        LIPSCHUTZ, S. Álgebra linear. São Paulo: McGraw-Hill do Brasil, 1973.

        MONTEIRO, L. H. J. Álgebra linear. São Paulo: Nobel, 1970.
    }

    % Inserido por Murillo R. P. Homem, em 01/04/2023
     \competencias{
        cg-aprender/{ce-ap-1, ce-ap-2, ce-ap-3, ce-ap-4},
        cg-atuar/{ce-atuar-1, ce-atuar-2, ce-atuar-3, ce-atuar-4},
        }
        
}

%     \disciplina{ga}{
    \titulo      {1}{Geometria Analítica}
    \objetivo    {Introduzir linguagem básica e ferramentas (matrizes e vetores), que permitam ao aluno analisar e resolver alguns problemas geométricos, no plano e espaço euclidianos, preparando-o para aplicações mais gerais do uso do mesmo tipo de ferramentas. Mais especificamente: Analisar e resolver problemas elementares que envolvem operações de matrizes e sistemas de equações lineares; Analisar soluções de problemas geométricos no plano e no espaço através do uso de vetores,matrizes e sistemas; Identificar configurações geométricas no plano e no espaço euclidiano a partir de suas equações, bem como deduzir equações para tais configurações. Resolver problemas que envolvem essas configurações.
    }
    % \requisitos  {N/A} % xxxxxxx
    \recomendadas{N/A}
    \ementa      {Matrizes; Sistemas Lineares; Eliminação Euclidiana. Vetores; produto escalar;vetorial e misto. Retas e Plano. Cônicas e Quadráticas.}
    \creditos    {4 total (3 teóricos, 1 práticos)}
    % %    \extra       {x horas}
    \codigo      {DM}{08.111-6}
    \bibliografia {
        BOULOS, P. e CAMARGO, I. Geometria analítica, um tratamento vetorial - 3. ed. - Pearson Editora, 2005.

        CAROLI, A., CALLIOLI, C. A., FEITOSA, M. O. Matrizes, vetores e geometria analitica - Editora Nobel, São Paulo, 1987.

        STEINBRUCH, A., WINTERLE, P. Geometria analítica - 2. ed. , Pearson Editora, São Paulo, 2006.
    }{
        BALDIN, Y. Y. e FURUYA, Y. K. S. Geometria analítica para todos e atividades com Octave e GeoGebra -  São Carlos: EDUFSCa, 2011r.

        CALLIOLI, C. A., DOMINGUES, H. H., COSTA, R. Álgebra linear e aplicações, 6 ed., São Paulo: Atual, 2007.

        LIMA, E. L. Geometria analítica e álgebra Linear - IMPA, 2001.

        LIPSCHUTZ, S. Álgebra linear: teoria e problemas - McGraw-Hill, 1994.

        WINTERLE, PVetores e geometria analítica - Makron Books, 2000.
    }

    % Inserido por Murillo R. P. Homem, em 01/04/2023
     \competencias{
        cg-aprender/{ce-ap-1, ce-ap-2, ce-ap-3, ce-ap-4},
        cg-atuar/{ce-atuar-1, ce-atuar-2, ce-atuar-3, ce-atuar-4},
        }
        
}
%     \disciplina{calculo2}{
    \titulo      {2}{Cálculo 2}
    \objetivo    {Aplicar os critérios de convergência para séries infinitas, bem como expandir funções em série de potências. Interpretar geometricamente os conceitos de funções de duas ou mais variáveis e ter habilidade nos cálculos de derivadas e dos máximos e mínimos de funções. Aplicar os teoremas das funções implícitas e inversas}
    \requisitos  {Cálculo Diferencial e Integral 1} % FAVOR NÃO ALTERAR PARA Cálculo 1, pois não é!!! 82210 OU 89109 OU 344001 OU 524034 OU 344001 OU 342009 OU 342211 OU 340456 OU 341258 OU (82619 E 82627) OU (215171 E 215287) OU (215287 E 215384)
    \recomendadas{N/A}
    \ementa      {Curvas e superfícies. Funções reais de várias variáveis. Diferenciabilidade de funções de várias variáveis. Fórmula de Taylor;  Máximos e mínimos; Multiplicadores de Lagrange. Derivação implícita e aplicações.}
    \creditos    {4 total (3 teóricos, 1 práticos)}
%    \extra       {x horas}
    \codigo      {DM}{08.920-6}
     \bibliografia {
        GUIDORIZZI, H. L. Um curso de cálculo. v. 2 – Livros Técnicos e Científicos, Rio de Janeiro, 2004 
        
        PINTO, D.;  MORGADO, M. C. F. Cálculo diferencial e integral de funções de várias variáveis -  UFRJ/SR-1, 1997. 
        
        THOMAS, G. B. et al. Cálculo, v.2 - 10a. Edição, Addison Wesley, 2003. 
    }{
        ÁVILA, G. S.S. Cálculo: funções de várias variáveis - 3a. Ediçao,  Rio de Janeiro: Livros Tecnicos e Cientificos, 1981. v.3. 258 p.  
        
        LIMA, E. L. Curso de análise v.2 - Projeto Euclides. Rio de Janeiro, IMPA, 1989. 
        
        RUDIN, W. Principles of mathematical analysis - 3. ed - McGraw-Hill, c1976. 
        
        STEWART, J.  Cálculo  v. 2, 4. ed,  Pioneira/Thomson Learning, São Paulo, 2001. 
        
        SWOKOWSKI, E. W. Cálculo com geometria analítica, v. 2 - 2. ed. – Markron Books, 1991.
    }

    % Inserido por Murillo R. P. Homem, em 01/04/2023
     \competencias{
        cg-aprender/{ce-ap-1, ce-ap-2, ce-ap-3, ce-ap-4},
        cg-atuar/{ce-atuar-1, ce-atuar-2, ce-atuar-3, ce-atuar-4},
        }
        
}

%     \disciplina{sereqdif}{
    \titulo      {3}{Séries e Equações Diferenciais}
    \objetivo    {Desenvolver as ideias gerais de modelos matemáticos de equações diferenciais ordinárias com aplicações à ciências físicas, químicas e engenharia; Desenvolver métodos elementares de resolução das equações clássicas de 1a. e 2a. ordem; Desenvolver métodos de resolução de equações diferenciais através de séries de potências; Representar funções em séries de potências e em séries de funções trigonométricas; Desenvolver métodos de resolução de equações diferenciais através de séries de potências; e Resolver equações diferenciais com uso de programas computacionais.}
    \requisitos  {Cálculo 1 } % 82210 OU 89109 OU 344001 OU 524034 OU 344001 OU 342009 OU 342211 OU 340456 OU 341258 OU (82619 E 82627) OU (215171 E 215287) OU (215287 E 215384)
    \recomendadas{N/A}
    \ementa      {Equações Diferenciais de 1a. Ordem; Equações Diferenciais de 2a. Ordem; Séries Numéricas. Séries de Potências. Noções sobre Séries de Fourier; e Soluções de Equações Diferenciais por Séries de Potências.}
    \creditos    {4 total (3 teóricos, 1 prático)}
    %    \extra       {x horas}
    \codigo      {DM}{08.940-0}
    \bibliografia {
        BOYCE, W.E.;DIPRIMA, R.C. (2001). Equações Diferenciais Elementares e Problemas de Valores de Contorno, 7. ed. Guanabara Koogan, Rio de Janeiro, 2001.

        FIGUEIREDO, D.G. ;NEVES, A.F. Equaçõe Diferenciais Aplicadas, Coleção Matemática Universitária, IMPA, Rio de Janeiro, 1977.

        GUIDORIZZI, L.H. Um Curso de Cálculo, v. 4. LTC, 2001.
    }{
        ZILL, D. G. Equações Diferenciais com Aplicações em Modelagem, Thomson, São Paulo, 2003.

        BASSANEZI R. C.; FERREIRA Jr. W. C. Equações Diferenciais com Aplicações, Editora Harbra Ltda, 1988.

        CODDINGTON E. A. .An Introduction to Ordinary Differential Equations, 1989.
        MATOS, P.M. Séries e Equações Diferenciais, 1a. edição, Printice Hall, São Paulo, 2001.

        STROGATZ, S. Nonlinear Dynamics and Chaos: With Applications to Physics, Biology, Chemistry, and Engineering, (Studies in Nonlinearity), Perseus Books Group, 2001.
    }

    % Inserido por Murillo R. P. Homem, em 01/04/2023
     \competencias{
        cg-aprender/{ce-ap-1, ce-ap-2, ce-ap-3, ce-ap-4},
        cg-atuar/{ce-atuar-1, ce-atuar-2, ce-atuar-3, ce-atuar-4},
        }
        
}
%     \disciplina{calculo3}{
    \titulo      {3}{Cálculo 3}
    \objetivo    {Generalizar os conceitos e técnicas do Cálculo Integral de funções de uma variável para funções de várias variáveis. Desenvolver a aplicação desses conceitos e técnicas em problemas correlatos.
    }
    \requisitos  {Cálculo 2} % 82260 OU 89206 OU 342076 OU 342220 OU 342475 OU 524042 OU 82635
    \recomendadas{N/A}
    \ementa      {Integração dupla; Integração tripla; Mudanças de coordenadas; Integral de linha; Diferenciais exatas e independência do caminho; Análise vetorial: Teorema de Gauss, Green e Stokes.}
    \creditos    {4 total (3 teóricos, 1 práticos)}
    %    \extra       {x horas}
    \codigo      {DM}{08.930-3}
    \bibliografia {
        GUIDORIZZI, H. L. Um curso de Cálculo. Volume 3, 5ª edição, Livros Técnicos e Científicos Editora, Rio de Janeiro, 2002.

        THOMAS, G.B. Cálculo. Volume 2, 10ª edição, Addison Wesley, São Paulo, 2003.

        SWOKOWSKI, E. W. Cálculo com Geometria Analítica. Volume 2, 2ª edição, Makron Books, São Paulo, 1995.
    }{
        ÁVILA, G. S. S., Cálculo das funções de múltiplas variáveis. Volume 3, 7ª edição, LTC Editora, Rio de Janeiro, 2006.

        LEITHOLD, L. Cálculo com Geometria Analítica. Volume 2, 2ª edição, Harbra, São Paulo, 1982.

        ANTON, H., Cálculo. Volume 2, 6ª edição, Bookman, Porto Alegre, 2000.

        PINTO, D.; FERREIRA MORGADO, M. Cálculo Diferencial e Integral de Funções de Várias Variáveis, 3a. edição, UFRJ, 2009.

        FLEMMING, D. M.; GONÇALVES, M. B. Cálculo B: funções de várias variáveis, integrais múltiplas, integrais curvilíneas e de superfície - 2. ed. - Pearson Prentice Hall, São Paulo, 2007.
    }

    % Inserido por Murillo R. P. Homem, em 01/04/2023
     \competencias{
        cg-aprender/{ce-ap-1, ce-ap-2, ce-ap-3, ce-ap-4},
        cg-atuar/{ce-atuar-1, ce-atuar-2, ce-atuar-3, ce-atuar-4},
        }
        
}

%     \disciplina{calnum}{
    \titulo      {4}{Cálculo Numérico}
    \objetivo    {Apresentar técnicas numéricas computacionais para resolução de problemas nos campos das ciências e da engenharia, levando em consideração suas especificidades, modelagem e aspectos computacionais vinculados a essas técnicas.
    }
    \requisitos  {Cálculo 1, Geometria Analítica e Construção de Algoritmos e Programação} % (82210 OU 89109 OU 524034 OU 82015 OU 344001 OU 342009 OU 342211 OU 340456 OU (82619 E 82627) OU (215171 E 215287)) E (81116 OU 524182 OU 345970 OU 81019 OU 81515 OU 342017 OU 342190 OU 343510 OU 345083) E (20320 OU 20109 OU 20168 OU 25070 OU ((481114 OU 343323) E 481220) OU 20184 OU 20125 OU 25011 OU 25054 OU 92410 OU 20583 OU 105180 OU 25070 OU 343137 OU 342068 OU 25470 OU 25488 OU 430048 OU 580015 OU 580287 OU 92410) OU 345970
    \recomendadas{N/A}
    \ementa      {Erros em processos numéricos;Solução numérica de sistemas de equações; lineares; Solução numérica de equações; Interpolação e aproximação de funções; Integração numérica;Solução numérica de equações diferenciais ordinárias}
    \creditos    {4 total (3 teóricos, 1 prático)}
    %    \extra       {x horas}
    \codigo      {DM}{08.302-0}
    \bibliografia {
        RUGGIERO, M.; LOPES, V. L. Cálculo Numérico: aspectos teóricos e computacionais, MacGraw-Hill, 1996.

        ARENALES, S.; DAREZZO, A. Cálculo Numérico - Aprendizagem com apoio de software, Editora Thomson, 2007.

        FRANCO, N. B. Cálculo Numérico, Pearson Prentice Hall, 2006.
    }{
        HUMES et al. Noções de Cálculo Numérico, MacGraw-Hill, 1984.

        BARROSO, C. L. et al. Cálculo Numérico com Aplicações, Harbra, 1987.

        BURDEN, R.L.,FAIRES, J.D. Numerical Analysis, PWS Publishing Company, 1996.

        CLÁUDIO, D. M. et al. Fundamentos de Matemática Computacional, Atlas, 1989).

        CONTE, S. D.,  Elementos de Análise Numérica, Ed. Globo, 1975.

        DEMIDOVICH, B. P. et al. Computacional Mathematics, Moscou, Mir Pub, 1987.

        SANTOS, V. R. Curso de Cálculo Numérico, LTC, 1977.

        SPERANDIO, D. et al. Cálculo Numérico: características matemáticas e computacionais dos métodos numéricos, Pearson/Prentice Hall, 2003.

        YONG, D. M. et al. (1972). Survery of Numerical Mathematics, Addison Wesley, 1972.}

    % Inserido por Murillo R. P. Homem, em 01/04/2023
     \competencias{
        cg-aprender/{ce-ap-1, ce-ap-2, ce-ap-3, ce-ap-4},
        cg-atuar/{ce-atuar-1, ce-atuar-2, ce-atuar-3, ce-atuar-4},
        }
        
}
%     \disciplina{probest}{
    \titulo      {4}{Probabilidade e Estatística}
    \objetivo    {Mostrar aos alunos conceitos de estatística, apresentando uma introdução aos princípios gerais, que serão úteis na área do aluno.}
    \requisitos  {N/A} % xxxxxxx
    \recomendadas{N/A}
    \ementa      {Experimento e Amostragem. Medidas Estatísticas dos Dados. Descrição Estatística dos Dados. Probabilidade. Variável Aleatória. Distribuições de Probabilidades Especiais. Distribuições Amostrais. Estimação de Parâmetros. Testes de Significância. Inferência Tratando-se de Duas Populações. Correlação e Previsão. Teste Qui-Quadrado.}
    \creditos    {4 total (4 teóricos)}
    %    \extra       {x horas}
    \codigo      {DEs}{15.001-0}
    \bibliografia {
        MORETTIN, P.A.; BUSSAB, W.O. Estatística Básica: 5. ed. São Paulo, Editora Saraiva, 2004.

        MONTGOMERY, D.C., RUNGER, G.C. Estatística Aplicada e Probabilidade para Engenheiros; LTC Editora, 2. ed. Rio Janeiro, 2003.

        MAGALHAES, M. N.; LIMA, A. C. P. Noções de Probabilidade e Estatística. 4. ed. São Paulo, EDUSP, 2002.
    }{
        WALPOLE, W.E.;MYERS, R.H.;MYERS, S.L.; YE, K. Prob. e Est. para Engenharia e Ciências, Pearson Prentice-Hall,São Paulo, 2009.

        MOORE, D. A Estatística Básica e Sua Prática, Editora LTC, Rio de Janeiro, 2005.

        COSTA NETO, P.L.O. Estatística, S.Paulo, Ed.Blucher, São Paulo, 1977.

        HOEL, P.G.; PORT, S.C.;STONE, C.J. Introdução à Teoria da Probabilidade. Ed. Interciência, 1978.

        MENDENHALL, W. Probabilidade e estatistica, Rio de Janeiro, RJ, Ed. Campus, 1985.
    }

    % Inserido por Murillo R. P. Homem, em 01/04/2023
     \competencias{
        cg-aprender/{ce-ap-1, ce-ap-2, ce-ap-3, ce-ap-4},
        cg-atuar/{ce-atuar-1, ce-atuar-2, ce-atuar-3, ce-atuar-4},
        }
            
}
% \section{Eixo: Física}
%     \disciplina{fisica1}{
    \titulo      {2}{Física 1}
    \objetivo    {Introduzir os princípios básicos da Física Clássica (Mecânica), tratados
    de forma elementar, desenvolvendo no estudante a intuição necessária para analisar fenômenos físicos sob os pontos de vista qualitativo e quantitativo. Despertar o interesse e ressaltar a necessidade do estudo desta matéria, mesmo para não especialistas.}
    \requisitos  {N/A} % xxxxxxx
    \recomendadas{N/A}
    \ementa      {Movimento de uma partícula em 1D, 2D e 3D; Leis de Newton; Aplicações das Leis de Newton – Equilíbrio de Líquidos (Arquimedes) - Forças Gravitacionais; Trabalho e Energia; Forças Conservativas – Energia Potencial; Conservação da Energia (Equação de Bernoulli); Sistemas de Várias Partículas – Centro de Massa; Colisões; Conservação do Momento Linear.
    }
    \creditos    {4 total (4 teóricos)}
    %    \extra       {x horas}
    \codigo      {DF}{09.901-5}
    \bibliografia {
        HALLIDAY, D.; RESNICK, R.; WALKER. J. Fundamentos de Física, volume 1, Mecânica, 9ª edição, GEN/LTC 2012.

        YOUNG, H. D.; FRIEDMAN. R. A.; SEARS ; ZEMANSKY. Física I: Mecânica, 12.ed.. Pearson, São Paulo.

        CHAVES,  A.; SAMPAIO,  J.F. Física Básica: Mecânica.
    }{
        NUSSENZVEIG, H.M. Curso de Física Básica, v. 1

        TIPLER,  P. A.; . MOSCA,  G. Física para cientista e engenheiros, v. 1, Mecânica, 6. ed., GEN/LTC, 2008.

        SERWAY R. A.; JEWETT, Jr. J. W. Princípios de Física, v. 1, Mecânica. 3 ed.. Editorial Thomson. 2005.

        FEYNMAN R.P.; Lectures on Physics, v. 1.

        KELLER, F.J.; GETTYS,  W.E.; SKOVE, M.J. Física, v. 1
    }

}

%     \disciplina{fisexpa}{
    \titulo      {2}{Física Experimental A}
    \objetivo    {Treinar o aluno para desenvolver atividades em laboratório. Familiarizá-lo
    com instrumentos de medidas de comprimento, tempo e temperatura. Ensinar o aluno a organizar dados experimentais, a determinar e processar erros, a construir e analisar gráficos, para que possa fazer uma avaliação crítica de seus resultados. Verificar experimentalmente as leis da Física.}
    \requisitos  {N/A} % xxxxxxx
    \recomendadas{N/A}
    \ementa      {Medidas e erros experimentais; Cinemática e dinâmica de partículas; Cinemática e dinâmica de corpos rígidos; Mecânica de meios contínuos;Termometria e calorimetria.}
    \creditos    {4 total (4 práticos)}
    %    \extra       {x horas}
    \codigo      {DF}{09.110-3}
    \bibliografia {
        INMETRO. Avaliação de dados de medição: guia para a expressão de incerteza de medição – GUM 2008. Traduzido de: Evaluation of measurement data: guide to the expression of uncertainty in measurement – GUM 2008. 1ª Ed. Duque de Caxias, RJ: INMETRO/CICMA/SEPIN, 2012. Disponível em:< http://www.inmetro.gov.br/infotec/publicacoes/gum\_final.pdf>. Acesso em: 13 Mar. 2013.

        HALLIDAY, D.; RESNICK, R.; WALKER, J. Fundamentos de Física: mecânica. [Fundamentals of physics]. Gerson Bazo Costamilan (Trad.). 4. ed. Rio de Janeiro: LTC, 1993

        VUOLO, J. H. Fundamentos da Teoria de Erros. 2. ed. São Paulo, SP: Editora Edgard Blücher LTDA
    }{
        INMETRO. Vocabulário internacional de termos fundamentais e gerais de Metrologia: portaria INMETRO nº 029 de 1995. 5. ed. Rio de Janeiro: Editiora SENAI, 2007.

        NUSSENZVEIG, H. M. Curso de Física Básica, 3. ed. São Paulo: Editora Edgard Blücher LTDA, 1996.

        CAMPOS, A. A; ALVES, E.S; SPEZIALI, N.L. Física Experimental Básica na Universidade, 2. ed. Belo Horizonte: Editora UFMG, 2008.

        DUPAS, M. A. Pesquisando e normalizando: noções básicas e recomendações úteis para a elaboração de trabalhos científicos. 6. ed. São Carlos: Editora EdUFSCar, 2009.

        WORSNOP, B. L.; FLINT, H. T. Curso Superior de Física Práctica - Tomo I. Buenos Aires: EUDEBA, 1964.
    }

}

%     \disciplina{fisica3}{
    \titulo      {3}{Física 3}
    \objetivo    {Nesta disciplina serão ministrados aos estudantes os fundamentos de eletricidade e magnetismo e suas aplicações. Os estudantes terão a oportunidade de aprender as equações de Maxwell. Serão criadas condições para que os mesmos possam adquirir uma base sólida nos assuntos a serem discutidos, resolver e discutir questões e problemas ao nível do que será ministrado e de acordo com as bibliografias recomendadas.}
    \requisitos  {Física 1} % 99015 OU 520136 OU 98108 OU 90018 OU 24023 OU 98019 OU 98051 OU 342092 OU 520098
    \recomendadas{N/A}
    \ementa      {Carga elétrica, força de Coulomb e conceito de campo elétrico; Cálculo do campo elétrico por integração direta e através da Lei de Gauss. Aplicações; Potencial elétrico. Materiais dielétricos e Capacitores; Corrente elétrica, circuitos simples e circuito RC; Campo magnético; Cálculo do campo magnético: Lei de Ampère e Biot-Savart; Indução eletromagnética e Lei de Faraday; Indutância e circuito RL; Propriedades magnéticas da matéria: diamagnetismo, paramagnetismo e ferromagnetismo.}
    \creditos    {4 total (4 teóricos)}
    %    \extra       {x horas}
    \codigo      {DF}{09.903-1}
    \bibliografia {
        Halliday, D.; Resnick , R.; Walker. J. Fundamentos da Física, 6ª ed., Rio de Janeiro: LTC, 2003.

        Young, H. D.; Friedman. R. A. Física III: Eletromagnetismo, 12ª ed., São Paulo: Addison Wesley, 2008.

        Tipler,  P. A.; Mosca. G. Física para cientistas e engenheiros, 5ª ed., Rio de Janeiro: Editora LTC, 2006.
    }{
        Serway R. A. ; Jewett Jr. J. W. Física: para cientistas e engenheiros, [Rio de Janeiro: LTC, 1996] ou [São Paulo: Cengage Learning, 2008].

        Nussenzveig,  H. M. Curso de Física Básica, 1. ed., São Paulo: Edgard Blucher, 1997.

        Keller , F. J.; Gettys, W. E.; Skove. M. J. Física, São Paulo: Makron Books, c1999.

        Feynman, R. P; Leighton, R. B.; Sands. M. The Feynman lectures on physics. Reading: Addison-Wesley, c1963.

        Chaves, A. S. Física: curso básico para estudantes de ciências físicas e engenharias. Rio de Janeiro: Reichmann \& Affonso, 2001.
    }
    % Inserido por Edilson Kato, em 11/03/2024
     \competencias{
        cg-aprender/{ce-ap-1, ce-ap-2, ce-ap-3, ce-ap-4},
        cg-atuar/{ce-atuar-1, ce-atuar-2, ce-atuar-3, ce-atuar-4},
        }
}

%     \disciplina{fisexpb}{
    \titulo      {3}{Física Experimental B}
    \objetivo    {Ao final da disciplina, o aluno deverá ter pleno conhecimento dos conceitos básicos, teórico-experimentais, de eletricidade, magnetismo e óptica geométrica. - Conhecerá os princípios de funcionamento e dominará a utilização de instrumentos de medidas elétricas, como: osciloscópio, voltímetro, amperímetro e ohmímetro. Saberá a função de vários componentes passivos, e poderá analisar e projetar circuitos elétricos simples, estando preparado para os cursos mais avançados, como os de Eletrônica. - Em óptica geométrica, verificará experimentalmente, as leis da reflexão e refração.
    }
    \requisitos  {N/A} % xxxxxxx
    \recomendadas{N/A}
    \ementa      {Medidas elétricas;Circuitos de corrente contínua; Indução eletromagnética; Resistência, capacitância e indutância; Circuitos de corrente alternada; Óptica geométrica: Dispositivos e instrumentos; Propriedades elétricas e magnéticas da matéria}
    \creditos    {4 total (4 práticos)}
    %    \extra       {x horas}
    \codigo      {DF}{09.111-1}
    \bibliografia {HALLIDAY, D.; RESNICK, R.; WALKER, J. Fundamentos de física. [Fundamentals of physics]. Gerson Bazo Costamilan (Trad.). 4 ed. Rio de Janeiro: LTC, c1993.

    TIPLER, P. A., Física para cientistas e engenheiros. [Physics for scientists and engineers]. Horacio Macedo (Trad.). 4 ed. Rio de Janeiro: LTC, c2000.

    NUSSENZVEIG, H. M. Curso de física básica. São Paulo: Edgard Blucher, 1997.
    }
    {
        BROPHY, J. J. Eletronica basica. Julio Cesar Goncalves Reis (Trad.). 3 ed. Rio de Janeiro: Guanabara Dois, 1978.

    CUTLER, P. Analise de circuitos CC, com problemas ilustrativos. Adalton Pereira de Toledo (Trad.). Sao Paulo: McGraw-Hill do Brasil, 1976.

    CUTLER, P. Analise de circuitos CA: com problemas ilustrativos. Adalton Pereira de Toledo (Trad.). Sao Paulo: McGraw-Hill do Brasil, 1976.

    NUSSENZVEIG, H. M.. Curso de Fisica Basica. 3 ed. Sao Paulo: Edgard Blucher, 1996.

    SERWAY, R. A. Física para cientistas e engenheiros com fisica moderna. [Physics for scientists and engineers with modern physics]. Horacio Macedo (Trad.). 3 ed. Rio de Janeiro: LTC, c1996.

    HALLIDAY, D; RESNICK, R; KRANE, K. S. Fisica III e IV. [Physics]. Denise Helena Sotero da Silva (Trad.). 4 ed. Rio de Janeiro: LTC, c1996.
    }

    % Inserido por Edilson Kato, em 11/03/2024
     \competencias{
        cg-aprender/{ce-ap-1, ce-ap-2, ce-ap-3, ce-ap-4},
        cg-atuar/{ce-atuar-1, ce-atuar-2, ce-atuar-3, ce-atuar-4},
        }
}

% \section{Eixo:Eletrônica}
%     \disciplina{circeletricos}{
    \titulo      {4}{Circuitos Elétricos}
    \objetivo    {Desenvolvimento das habilidades de modelagem e análise de circuitos elétricos estáticos e dinâmicos, no contexto dos sistemas lineares e de componentes de parâmetros discretos.}
    \requisitos  {Séries e Equações Diferenciais} % xxxxxxx CoC Removeu Física 2 
    % OU Física Experimental B % Sugestão Sabrina CoC
    \recomendadas{N/A}
    \ementa      {Natureza algébrica das variáveis representativas de tensão e corrente. Leis de Kirchhoff. Modelos ideais de componentes de parâmetros discretos. Princípio da Superposição e Teoremas de Thevenin e Norton. Método de análise baseado em tensões de Nó. Método de análise baseado em correntes de malha. Análise de circuitos estáticos. Análise de circuitos dinâmicos no domínio do tempo. Teorema da translação no tempo. Análise fasorial (no domínio complexo) em circuitos em regime senoidal.}
    \creditos    {4 total (4 teóricos)}
    %    \extra       {3 horas}
    \codigo      {DC}{1001352}
    \bibliografia {
        IRWIN, J. D. Análise básica de circuitos para engenharia. 7. ed. Rio de Janeiro: LTC, c2003.

        NILSSON, J. W.; RIEDEL, S. A. Electric circuits. 8. ed. Upper Saddle River: Pearson Prenttice Hall, c2008.

        JOHNSON, D. E.; JOHNSON, J. R.; HILBURN, J. L.; SCOTT, P. D. Electric circuit analysis. 3. ed. [s.l.]: John Wiley \& Sons, c1999
    }{
        ALEXANDER, C. K.; SADIKU, M. N. O. Fundamentals of electric circuits. 4. ed. Boston: McGraw-Hil, c2009.

        BOYLESTAD, R.. Introdução à análise de circuitos. 10. ed. São Paulo: Pearson Prenttice Hall, 2009.

        DORF, R. C.; SVOBODA, J. A. Introdução aos circuitos elétricos. 8. ed. Rio de Janeiro: LTC, 2012.

        HAYT JR, W. H.; KEMMERLY, J. E.; DURBIN, Steven M. Análise de circuitos em engenharia. 7. ed. São Paulo: McGraw-Hill do Brasil, 2008.

        MARKUS, O. Circuitos elétricos: corrente contínua e corrente alternada: teoria e exercícios. 9. ed. São Paulo: Érica, 2011.
    }

    % Jander 13/5/23
    \dataatualizacao{9/10/23} % Jander, Alexandre, Edilson, Fredy
    \competencias{
        % De:
        % cg-aprender/{ce-ap-1, ce-ap-3, ce-ap-4},
        % cg-produzir/{ce-pro-1, ce-pro-2, ce-pro-4}
        % Para:
        cg-aprender/{ce-ap-1, ce-ap-4},
        cg-produzir/{ce-pro-2, ce-pro-4},
        cg-atuar/{ce-atuar-4, ce-atuar-5},
    }
}
%     \disciplina{circeletronicos1}{
    \titulo      {5}{Circuitos Eletrônicos 1}
    \objetivo    {Promover o entendimento das características físicas dos materiais adotados para emergência dos comportamentos não lineares dos dispositivos eletrônicos básicos. Caracterizar os dispositivos eletrônicos básicos e suas propriedades não lineares. Desenvolver habilidades de modelagem, análise e síntese de circuitos eletrônicos. Gerar a capacitação em modelagem de circuitos eletrônicos por regiões de comportamento linear e resolução com verificação de hipóteses. Apresentar e desenvolver projetos dos principais circuitos funcionais e aplicações.}
    \requisitos  {Circuitos Elétricos} % xxxxxxx OU Fisica Experimental B % Sugestão Sabrina CoC 
    \recomendadas{N/A}
    \ementa      {Características e comportamentos de sistemas não lineares. Estratégias de análise de sistemas não lineares. Vantagens e desvantagens de sistemas não lineares. Materiais semicondutores básicos e suas propriedades. Concepção de dispositivos eletrônicos básicos. Caracterização do diodo: comportamentos e modelos: ideal, aproximado e teórico. Circuitos com diodos: modelagem e estratégia de análise de sistemas não lineares por regiões de comportamento. Síntese de circuitos com diodos. Caracterização do transistor de junção bipolar (TJB): comportamentos, modelos e configurações. Ponto de operação e circuito de polarização. Circuitos com transistores: modelagem e estratégia de análise. Síntese de circuitos com transistores TJB. Aplicações. Amplificadores operacionais: conceituação e propriedades. Análise e projeto de circuitos com base em amplificadores operacionais. Aplicações usuais e relevantes de amplificadores operacionais.}
    \creditos    {6 total (4 teóricos, 2 práticos)}
    %    \extra       {3 horas}
    \codigo      {DC}{1001533}
    \bibliografia {AMARAL, A. M. Raposo. Análise de circuitos e dispositivos eletrónicos. Porto: Publindústria, Edições Técnicas, 2013.

    TOOLEY, M.. Circuitos eletrônicos: fundamentos e aplicações. Rio de Janeiro: Elsevier, 2007.

    BOYLESTAD, R. L.; NASHELSKY, Louis. Dispositivos eletrônicos e teoria de circuitos. 11. ed. São Paulo: Pearson, 2013.
    }
    {
        MALVINO, A. P.; BATES, D. J. Eletrônica. 7. ed. Porto Alegre, RS: AMGH Editora, 2007.

    COMER, D. J.; COMER, Donald. Fundamentos de projeto de circuitos eletrônicos. Rio de Janeiro: LTC, 2005.

    SILVA, M. M. Introducao aos circuitos electricos e electronicos. 2. ed. Lisboa: Fundacao Calouste Gulbenkian, 2001

    HAYT, W.H.; Neudeck, G. W. ; Electronic Circuit Analysis and Design; Edition 2; Wiley;0 1984.

    BATARSEH, I.; Ahmad H.; Power Electronics: Circuit Analysis and Design; Springer; 2018.
    }

    \dataatualizacao{12/12/23} % Luciano
    \competencias{
        cg-aprender/{ce-ap-1, ce-ap-3},
        cg-produzir/{ce-pro-1, ce-pro-2, ce-pro-4},
        cg-atuar/{ce-atuar-1, ce-atuar-4}
    }
}

%     \disciplina{circeletronicos2}{
    \titulo      {6}{Circuitos Eletrônicos 2}
    \objetivo    {Caracterizar os dispositivos eletrônicos básicos e suas propriedades não lineares. Desenvolver habilidades de modelagem, análise e síntese de circuitos eletrônicos. Gerar a capacitação em modelagem de circuitos eletrônicos por regiões de comportamento linear e resolução com verificação de hipóteses. Apresentar estruturas de circuitos funcionais e aplicações.}
    \requisitos  {Circuitos Eletrônicos 1} % xxxxxxx
    \recomendadas{N/A}
    \ementa      {Caracterização do transistor de efeito de campo (FET): comportamentos, modelos e configurações. Ponto de operação e circuito de polarização de transistores FET. Circuitos com transistores FET: modelagem e estratégia de análise. Síntese de amplificadores com transistores FET. Transistores FET na síntese de circuitos eletrônicos em geral. Resposta em frequência do TJB e do FET. Circuitos integrados lineares e digitais. Circuitos osciladores. Fontes de alimentação.}
    \creditos    {6 total (4 teóricos, 2 práticos)}
    %    \extra       {3 horas}
    \codigo      {DC}{1001528}
    \bibliografia {
        AMARAL, A. M. Ra.. Análise de circuitos e dispositivos eletrónicos. Porto: Publindústria, Edições Técnicas, 2013.

        TOOLEY, M. Circuitos eletrônicos: fundamentos e aplicações. Rio de Janeiro: Elsevier, 2007.

        BOYLESTAD, R. L.; NASHELSKY, L. Dispositivos eletrônicos e teoria de circuitos. 11. ed. São Paulo: Pearson, 2013.
    }{
        MALVINO, A. P.; BATES, D. J. Eletrônica. 7. ed. Porto Alegre, RS: AMGH Editora, 2007.

        COMER, D.J.; COMER, Donald. Fundamentos de projeto de circuitos eletrônicos. Rio de Janeiro: LTC, 2005.

        SILVA, M. M. Introducao aos circuitos electricos e electronicos. 2. ed. Lisboa: Fundacao Calouste Gulbenkian, 2001

        HAYT, W. H.; NEUDECK, G. W. ; Electronic Circuit Analysis and Design; Edition 2; Wiley;0 1984.

        BATARSEH, I; AHMAD, H.; Power Electronics: Circuit Analysis and
        Design; Springer; 2018.
    }

    \dataatualizacao{9/10/23} % Jander, Alexandre, Edilson, Fredy
    \competencias{
        % cg-aprender/{ce-ap-1, ce-ap-3, ce-ap-4},
        % cg-produzir/{ce-pro-1, ce-pro-2, ce-pro-4},
        % cg-atuar/{ce-atuar-1, ce-atuar-5}
        % Para:
        cg-aprender/{ce-ap-1, ce-ap-3},
        cg-produzir/{ce-pro-1, ce-pro-2, ce-pro-4},
        cg-atuar/{ce-atuar-1, ce-atuar-4},
    }
}
% \section{Eixo:Algoritmos e Programação}
%     \disciplina{cap}{
    \titulo      {1}{Construção de Algoritmos e Programação}
    \objetivo    {Tornar os estudantes aptos a utilizar pensamento computacional e algorítmico para proposição de soluções de problemas. Capacitar os estudantes a mapear tais soluções em programas usando linguagem de programação.}
    \requisitos  {N/A} % xxxxxxx
    \recomendadas{N/A}
    \ementa      {Noções gerais da computação: organização de computadores, programas, linguagens e aplicações. Detalhamento de algoritmos estruturados e programação: tipos básicos de dados. Representação e manipulação de dados. Estruturas de controle de fluxo (condicionais e repetições). Modularização (sub-rotinas, passagem de parâmetros e escopo). Documentação. Estruturação básica de dados: variáveis compostas heterogêneas (registros) e homogêneas (vetores e matrizes). Operações em arquivos e sua manipulação. Alocação dinâmica de memória e ponteiros.}
    \creditos    {8 total (4 teóricos, 4 práticos)}
    %    \extra       {4 horas}
    \codigo      {DC}{1001350}
    \bibliografia {
        CIFERRI. R.R. Programação de Computadores, Edufscar, 2009.

        MEDINA, M. ; FERTIG. C. Algoritmos e Programação: Teoria e Prática, Novatec, 2005.

        SENNE,  E. Primeiro Curso de Programação em C, Visual Books, 2003.

        TREMBLAY, J.P.; BUNT. R.B. Ciência dos Computadores, McGraw-Hill, 1981.

        KERNIGHAN, B.W. ; RITCHIE, D.M. The C Programming Language (2nd Edition), 1988.
    }{
        HARBISON, S.P.; STEELE., G.L. C: a reference manual, 2002.

        KOCHAN; S.G. Programming in C: A complete introduction to the C programming language, 2004.

        KING,  K.N. C Programming: A Modern Approach, Norton \& Company, 1996.
    }
    % Jander, em 15/2/23
    \dataatualizacao{9/10/23} % Jander, Alexandre, Edilson, Fredy
    \competencias{%
        % Revisto sem modificações
        cg-aprender/{ce-ap-4},
        cg-produzir/{ce-pro-2, ce-pro-4},
        cg-atuar/{ce-atuar-1, ce-atuar-2},
        cg-empreender/{ce-emp-2},
    }
}

%     \disciplina{ipa}{
    \titulo      {1}{Introdução ao Pensamento Algorítmico}
    \objetivo    {Motivar e orientar os estudantes a desenvolver soluções sistemáticas para problemas diversos, contextualizados em situações cotidianas, de modo que estas possam ser implementadas a fim de usar o computador como ferramenta para obtenção de resultados. Desenvolver nos estudantes a habilidade de organizar e analisar dados de um problema, a fim de encontrar soluções utilizando técnicas de abstração, decomposição, reconhecimento de padrões e generalização, além da capacidade de analisar a eficiência de suas soluções.}
    \requisitos  {N/A} % xxxxxxx
    \recomendadas{N/A}
    \ementa      {Introdução ao pensamento algorítmico. Análise e especificação de problemas sob o aspecto de pensamento algorítmico. Técnicas de resolução de problemas: abstração, decomposição, reconhecimento de padrões e generalização. Representação e visualização de dados e soluções, com interpretação de resultados. Noções de paralelização. Noções de eficiência de um algoritmo. Introdução em alto nível de algoritmos de diversas áreas da ciência da computação: ordenação, busca, conectividade em grafos, caminhos mínimos, hashing, k-nn, criptografia.}
    \creditos       {2 total (2 teóricos)}
    %    \extra       {3 horas}
    \codigo      {DC}{1001349}
    \bibliografia {
        BHARGAVA, A. Y. Entendendo Algoritmos: Um guia ilustrado para programadores e outros curiosos. NOVATEC, 2017. 264 p. ISBN 978-85-752-2563-9.

        LOPES, A.; GARCIA, G.. Introdução à programação: 500 algoritmos resolvidos. Rio de Janeiro: Elsevier, 2002. 469 p. ISBN 978-85-352-1019-4.

        SPRAUL, V. A. Think like a programmer: an introduction to creative problem solving. No Starch Press, 2012. 256 p. ISBN 978-1593274245

        SOUZA, M. A. F. Algoritmos e lógica de programação: um texto introdutório para engenharia. 2.ed. São Paulo: Cengage Learning, 2014. 234 p. ISBN 9788522111299.
    }{
        FORBELLONE, A. L. V.; EBERSPACHER, H. F. Lógica de programação: a construção de algoritmos e estruturas de dados. 3. ed. São Paulo: Pearson Prentice Hall, 2008. 218 p. ISBN 978-85-7605-024-7.

        HOLLOWAY, J. P. Introdução a programação para engenharia: resolvendo problemas com algoritmos. Rio de Janeiro: LTC, 2006. 339 p. ISBN 8521614535.

        SKIENA, S. S. The algorithm design manual. New York: Springer-Verlag, c1998. 486 p. ISBN 0-387-94860-0.

        ERWIG, Martin. Once Upon an Algorithm: How Stories Explain Computing. MIT Press, 2017. 336 p. ISBN 978-0262036634

        FILHO, W. F.; PICTET, R. Computer Science Distilled: Learn the Art of Solving Computational Problems. Code Energy, 2017. 183 p. ISBN 0997316004

        TILLMAN, F. A.; CASSONE, D. T. A Professional's Guide to Problem Solving with Decision Science. Pioneering Partnerships LLC, 2018. 298 p. ISBN 978-0999767115
    }
    % Jander, em 15/2/23
    \competencias{
        cg-aprender/{ce-ap-3, ce-ap-4},
        cg-produzir/{ce-pro-2, ce-pro-4},
        cg-atuar/{ce-atuar-1, ce-atuar-2},
        cg-empreender/ce-emp-2,
    }
}
%     \disciplina{aed1}{
    \titulo      {2}{Algoritmos e Estruturas de Dados 1}
    \objetivo    {Tornar os estudantes aptos a utilizar técnicas básicas de programação em seus projetos; capacitar os estudantes a reconhecer, implementar e modificar algoritmos e estruturas de dados básicos; familiarizar os estudantes com noções de projeto e análise de algoritmos, através do estudo de uma linguagem algorítmica, exemplos e exercícios práticos; estimular os estudantes a avaliar quais técnicas de programação, algoritmos e estruturas de dados se adequam melhor a cada situação, problema ou aplicação.}
    \requisitos  {Construção de Algoritmos e Programação} % xxxxxxx
    \recomendadas{N/A}
    \ementa      {Introdução à recursão, com algoritmos e aplicações. Visão intuitiva sobre análise de correção (invariantes) e eficiência (complexidade) de algoritmos. Apresentação de busca linear e binária. Apresentação de algoritmos de ordenação elementares (insertion sort, selection sort e bubble sort). Apresentação de programação por retrocesso (backtracking) e enumeração. Noções de tipos abstratos de dados. Detalhamento de estruturas de dados como: listas (alocação estática e dinâmica, circulares, duplamente ligadas e com nó cabeça), matrizes e listas ortogonais, pilhas e filas (alocação sequencial e ligada) com aplicações. Detalhamento de árvores (definição, representação e propriedades), árvores binárias (manipulação e percursos) e árvores de busca (operações de busca, inserção e remoção). Apresentação de filas de prioridade com detalhamento das implementações triviais e com heap (alocação ligada e sequencial). Apresentação de exemplos e exercícios práticos, os quais podem envolver estruturas de dados compostas (como vetores de listas ligadas) e diferentes abordagens algorítmicas (gulosa, divisão e conquista, programação dinâmica, backtracking, busca em largura, etc).}
    \creditos    {4 total (4 teóricos)}
    % \extra       {4 horas}
    \codigo      {DC}{1001502}
    \bibliografia {
        FEOFILOFF. P. Algoritmos em Linguagem C, Elsevier, 2009.

        AARON M.; TENENBAUM, Y. L.; AUGENSTEIN,  M. J. Estruturas de dados usando C. São Paulo: Pearson Makron Books, 2009.

        FERRARI, R., RIBEIRO, M. X., DIAS, R. L., FALVO, M. Estruturas de Dados com Jogos. Rio de Janeiro – Elsevier, 2014.
    }{
        SEDGEWICK,  R. Algorithms in C++, Parts 1-4: fundamentals, data structures, sorting, searching. 3rd. ed., Boston: Addison - Wesley, 1998.

        SEDGEWICK, R. Algorithms in C++, Part 5: graph algorithms. 3rd. ed., Boston: Addison-Wesley, 2001.

        ZIVIANI, N. Projetos de algoritmos: com implementações em Pascal e C. 3. ed. rev. e ampl. São Paulo: Cengage Learning, 2012.

        ROBERTS,  E.S. Programming Abstractions in C: a Second Course in Computer Science, Addison-Wesley, 1997.

        CIFERRI,  R.R. Programação de Computadores, Edufscar, 2009.
    }
    % Jander, em 15/2/23
    \dataatualizacao{9/10/23} % Jander, Alexandre, Edilson, Fredy
    \competencias{
        % cg-produzir/{ce-pro-1, ce-pro-2, ce-pro-4, ce-pro-5},
        % cg-atuar/{ce-atuar-1, ce-atuar-2, ce-atuar-4},
        % cg-empreender/{ce-emp-1, ce-emp-2, ce-emp-4, ce-emp-5},
        % Para:
        cg-produzir/{ce-pro-1, ce-pro-2, ce-pro-4, ce-pro-5},
        cg-atuar/{ce-atuar-1, ce-atuar-4, ce-atuar-5},
        cg-empreender/{ce-emp-1, ce-emp-2},
    }
}
%     \disciplina{poo}{
    \titulo      {2}{Programação Orientada a Objetos}
    \objetivo    {Capacitar os estudantes nos conceitos básicos de programação orientada a objetos e suas características principais. Capacitar os estudantes na construção de programas utilizando uma linguagem baseada no paradigma de orientação a objetos.}
    \requisitos  {Construção de Algoritmos e Programação} % xxxxxxx
    \recomendadas{N/A}
    \ementa      {Histórico do paradigma orientado a objetos e comparação com o paradigma estruturado. Conceitos teóricos e práticos de orientação a objetos: abstração, classes, objetos, atributos e métodos, encapsulamento/visibilidade, herança, composição/agregação, sobrecarga, polimorfismo de inclusão e classes abstratas e polimorfismo paramétrico. Modularização. Alocação dinâmica de objetos. Tratamento de exceções.}
    \creditos    {4 total (2 teóricos, 2 práticos)}
    %    \extra       {4 horas}
    \codigo      {DC}{1001507}

    \bibliografia {
        DEITEL, H.M. \& DEITEL, P. J. - C++ Como Programar, 5ed, Pearson Prentice Hall, 2006 (Disponível BCo)

        PIZZOLATO, E. B. - Introdução à programação orientada a objetos com C++ e Java, EdUFSCar, 2010 (Disponível BCo)

        ECKEL, B. Thinking in C++. 2ed. Upper Saddle River: Prentice Hall, 2000. (Disponível BCo)
    }{
        SILVA FILHO, A. M. Introdução à programação orientada a objetos com C++, Elsevier, 2010 (Disponível BCo)

        DEITEL, Paul J.; DEITEL, Harvey M. C++ for programmers. Upper Saddle River, NJ: Prentice Hall, 2009. 1000 p. (Deitel Developer Series). ISBN 10-0-13-700130-9. (Disponível BCo)

        SCHILDT, Herbert. C++: the complete reference. 4. ed. New York: McGraw Hill, c2003. 1023 p. ISBN 0-07-222680-3.(Disponível BCo)
    }
    % Jander, 13/5/23
    \dataatualizacao{16/10/23} % Jander, Alexandre, Edilson, Fredy, Márcio, Alan
    \competencias{
        % cg-aprender/{ce-ap-1, ce-ap-2, ce-ap-3},
        % cg-produzir/{ce-pro-1, ce-pro-2},
        % cg-atuar/{ce-atuar-1, ce-atuar-5}
        % Para:
        cg-aprender/{ce-ap-4},
        cg-produzir/{ce-pro-5},
        cg-atuar/{ce-atuar-1, ce-atuar-3},
        cg-gerenciar/{ce-ger-1},
    }
}
    
    

%     \disciplina{aed2}{
    \titulo      {3}{Algoritmos e Estruturas de Dados 2}
    \objetivo    {Tornar os estudantes aptos a utilizar diversas técnicas de programação em seus projetos; capacitar os estudantes a reconhecer, implementar e modificar algoritmos e estruturas de dados amplamente utilizados; familiarizar os estudantes com o projeto e a análise de algoritmos, através do estudo de uma linguagem algorítmica, exemplos e exercícios práticos; estimular os estudantes a avaliar quais técnicas de programação, algoritmos e estruturas de dados se adequam melhor a cada situação, problema ou aplicação.}
    \requisitos  {Algoritmos e Estruturas de Dados 1} % xxxxxxx
    \recomendadas{N/A}
    \ementa      {Aprofundamento das noções de análise de correção (invariantes e indução matemática) e eficiência (complexidade de tempo e espaço) de algoritmos, incluindo a notação O. Detalhamento dos algoritmos de ordenação não-elementares (heap sort, merge sort e quick sort aleatorizado). Apresentação de algoritmo $O(n \log n)$ para cálculo de inversões entre sequências (adaptação do merge sort). Limitante inferior $\Omega (n \log n)$ para ordenação por comparação. Noções de algoritmos de ordenação não baseados em comparação e com tempo linear (bucket, counting e radix sort). Introdução de tabelas de símbolos com detalhamento de sua implementação usando estruturas de dados como: tabelas de espalhamento (hash tables), skip lists (estrutura probabilística), árvores de busca balanceadas (AVL ou rubro-negras e árvores de busca ótimas). Apresentação do algoritmo de Boyer-Moore e das árvores de prefixos para processamento de cadeias de caracteres. Introdução a grafos com diferentes tipos (simples, dirigido e ponderado) e representações (matrizes, listas de adjacência e listas ortogonais). Detalhamento de diversos algoritmos em grafos como: busca (com aplicação em conectividade), busca em largura (com aplicação em caminhos mínimos não ponderados), busca em profundidade (com aplicações em ordenação topológica e componentes fortemente conexos), caminhos mínimos em grafos sem custos negativos (algoritmo de Dijkstra com e sem heap). Apresentação de exemplos e exercícios práticos, os quais podem envolver estruturas de dados compostas (como heaps ou tabelas hash associados a vetores) e diferentes abordagens algorítmicas (gulosa, divisão e conquista, programação dinâmica, aleatorização etc).}
    \creditos    {4 total (4 teóricos)}
    %    \extra       {4 horas}
    \codigo      {DC}{1001490}
    \bibliografia {
        SEDGEWICK, R. Algorithms in C++, Part 5: graph algorithms. 3rd. ed., Boston: Addison-Wesley, 2001.

        ZIVIANI, N.Projeto de algoritmos: com implementações em Java e C++. 2. ed. São Paulo: Cengage Learning, 2011.

        FEOFILOFF. P. Algoritmos em Linguagem C, Elsevier, 2009.

        CORMEN, T.H. ; LEISERSON, C.E. ; RIVEST, R.L.; STEIN, C. Introduction to Algorithms, 3rd ed., McGraw-Hill, 2009.
    }{
        SEDGEWICK,  R. Algorithms in C++, Parts 1-4: fundamentals, data structures, sorting, searching. 3rd. ed., Boston: Addison - Wesley, 1998.

        BERMAN,  A. M. Data structures via C++: objects by evolution. New York: Oxford University Press, 1997.

        LANGSAM, Y. ;AUGENSTEIN, M. ; TENENBAUM, A. M. Data structures using C and C++. 2. ed. Upper Sadle River: Prentice Hall, 1996.

        ZIVIANI,  N. Projetos de algoritmos: com implementações em Pascal e C. 3. ed. rev. e ampl. São Paulo: Cengage Learning, 2012.

        DROZDEK,  A. Estruturas de dados e algoritmos em C++. São Paulo: Cengage Learning, 2010.
    }
    % Jander, 13/5/23
    \competencias{
        cg-aprender/{ce-ap-1, ce-ap-2},
        cg-produzir/{ce-pro-1, ce-pro-2, ce-pro-5},
        cg-empreender/{ce-emp-1, ce-emp-2}
    }
}
%     \disciplina{paa}{
    \titulo      {4}{Projeto e Análise de Algoritmos}
    \objetivo    {Tornar os estudantes aptos a aplicar estratégias algorítmicas avançadas a seus projetos; capacitar os estudantes a analisar a correção e o desempenho de algoritmos não-triviais; permitir aos estudantes consolidar os paradigmas de projeto de algoritmos (divisão e conquista, aleatorização, guloso, programação dinâmica), através de diversos exemplos e demonstrações; familiarizar os estudantes com noções da teoria da complexidade computacional; estimular os estudantes a avaliar quais técnicas de projeto, algoritmos e estruturas de dados se adequam melhor a cada situação, problema ou aplicação.}
    \requisitos  {Algoritmos e Estruturas de Dados 1} % xxxxxxx
    \recomendadas{N/A}
    \ementa      {Detalhamento das análises assintóticas (notação O, Omega e Theta). Aprofundamento de divisão-e-conquista: árvore de recorrência e teorema mestre (demonstração, interpretação e exemplos). Apresentação de aplicações em áreas distintas com definição do problema, algoritmo, recorrência, análises de correção e eficiência. Exemplos de aplicações: multiplicação de inteiros e matrizes, ordenação e seleção aleatorizados (Revisão de probabilidade). Revisão de grafos e apresentação da operação de contração de arestas com aplicação no algoritmo probabilístico de Karger para o problema do corte mínimo. Aprofundamento de algoritmos gulosos: aplicações em áreas distintas com definição do problema, algoritmo e invariantes, análises de correção e eficiência. Exemplos de aplicações: escalonamento de tarefas com peso em uma única máquina, coleção disjunta máxima de intervalos, códigos de Huffman, problema da árvore geradora mínima (algoritmo genérico) e abordagens de Prim (com e sem heap) e Kruskal (com detalhamento da estrutura union-find). Aprofundamento de programação dinâmica: princípios de PD (com exemplos); aplicações em áreas distintas com definição do problema, subestrutura ótima com demonstração, algoritmo, implementação eficiente, análises de correção e eficiência. Exemplos de aplicações: conjunto independente ponderado em grafos caminhos, alinhamento de sequências, problema da mochila, caminhos mínimos. Revisão do algoritmo para caminhos mínimos de Dijkstra com apresentação de contra-exemplo para o caso de grafos com custos negativos. Detalhamento dos algoritmos para caminhos mínimos de Bellman-Ford, Floyd-Warshall e Johnson. Introdução de NP-Completude pelo ponto de vista algorítmico: reduções; completude; definição e interpretação de NP-Completude (questão P vs NP). Noções de abordagens para tratar problemas NP-Completos e NP-Difíceis. Algoritmos exatos (Ex: busca exaustiva melhorada para Cobertura por Vértices e programação dinâmica para Caixeiro Viajante); algoritmos de aproximação (Ex: algoritmos guloso e de programação dinâmica para mochila); algoritmos de busca local (Ex: Corte Máximo e 2-SAT).}
    \creditos    {4 total (4 teóricos)}
    %    \extra       {4 horas}
    \codigo      {DC}{1001525}
    \bibliografia {
        S. Dasgupta, C.H. Papadimitriou, U.V. Vazirani. Algoritmos, McGraw-Hill, 2009.

        T.H. Cormen, C.E. Leiserson, R.L. Rivest, C. Stein. Introduction to Algorithms, 3rd ed., McGraw-Hill, 2009.

        R. Sedgewick, K. Wayne. Algorithms, 4th. ed., Addison-Wesley, 2011.
    }{
        J. Kleinberg, É. Tardos. Algorithm Design, Addison-Wesley, 2005.

        SEDGEWICK,  R. Algorithms in C++, Parts 1-4: fundamentals, data structures, sorting, searching. 3rd. ed., Boston: Addison - Wesley, 1998.

        SEDGEWICK, R. Algorithms in C++, Part 5: graph algorithms. 3rd. ed., Boston: Addison-Wesley, 2001.

        Nivio Ziviani. Projeto de algoritmos: com implementações em Java e C++. 2. ed. São Paulo: Cengage Learning, 2011.

        D.E. Knuth. The Art of Computer Programming, vols. 1 e 3, Addison-Wesley, 1973.

        K. H. Rosen. Discrete mathematics and its applications. 7th. ed. New York: McGraw Hill, 2013.
    }
    % Jander, 13/5/23
    \dataatualizacao{4/9/23} % Jander
    \competencias{
        % cg-aprender/{ce-ap-1, ce-ap-2},
        % cg-produzir/{ce-pro-2, ce-pro-4},
        % cg-atuar/{ce-atuar-4, ce-atuar-5}
        cg-aprender/{ce-ap-4},
        cg-produzir/{ce-pro-2, ce-pro-4},
        cg-atuar/{ce-atuar-4}
    }
}
%     \disciplina{ppd}{
    \titulo      {8}{Programação Paralela e Distribuída}
    \objetivo    {Familiarizar o estudante com os conceitos e termos básicos de sistemas paralelos, implementação e uso de concorrência, apresentar os tipos de arquitetura mais usados, descrever o suporte necessário para a programação de tais sistemas e apresentar algumas aplicações.}
    \requisitos  {Sistemas Operacionais} % xxxxx
    \recomendadas{Sistemas Distribídos} % TODO: Este não aparece no PPC da EnC, seria o caso de incluir como requisito?
    \ementa      {Revisão de arquiteturas paralelas: memória compartilhada e distribuída. Desenvolvimento de aplicações concorrentes: conceitos básicos da programação concorrente, definição, ativação e coordenação de processos, modelos de programação e técnicas de decomposição. Técnicas de otimização. Otimização sequencial: uso eficiente da memória, unit stride, blocking. Instruções vetoriais e super escalares, opções de otimização. Profiling e modelagem de desempenho. Controle de processos e paralelização fork-join. Programação com memória compartilhada e introdução ao OpenMP. Programação com memória distribuída e MPI. Programação de sistemas manycore como GPU e aceleradores: CUDA, OpenCL e outros. Programação paralela na nuvem. Avaliação de desempenho e teste de programas concorrentes.}
    \creditos    {4 total (2 teóricos, 2 práticos)}
    %    \extra       {x horas}
    \codigo      {DC}{1001483}
    \bibliografia {
        Grama,A.;Gupta,A.;Karypis,G.;Kumar,V. Introduction to Parallel Computing. Adisson- Wesley, 2003. (disponível na BCO).

        Dongarra, J.; Foster, I.; Fox, G.; Gropp, W.; White, A.; Torczon, L.; Kennedy, K. Sourcebook of Parallel Computing. Morgan Kaufmann Pub, 2003. (disonível na BCO).

        Foster, I. Designing and Building Parallel Programs. Addison-Wesley, 1995. www-unix.mcs.anl.gov/dbpp. (disponível na BCO).

        Casanova, H.; Legrand, A.; Robert, Y.. Parallel algorithms. Boca Raton, Fla.: CRC Press, 2009. 335 p. (disponível na BCO).

        Wilkinson, B. and Allen, M. Parallel Programming: Techniques and Applications Using Networked Workdstations and Parallel Computers. Pearson Prentice Hall, 2005. (disponível na BCO)

        Quinn, M. J. Parallel programming: in C with MPI and openMP. Boston: McGraw-Hill/Higher Education, 2004. (disponível na BCO).

        Lin, C.; Snyder, L.. Principles of parallel programming. Boston: Pearson Addison Wesley, 2009. (disponível na BCO).
    }{
        Flynn, M. J.; Rudd, K. W. Parallel Architectures. ACM Computing Surveys, v. 28, n.1, 1996.

        Chapman, B.; Jost, G. and van der Pas, R. Using OpenMP: Portable Shared Memory Parallel Programming. MIT Press, 2007.

        Robbins, K. A. and Robbins, S. Practical Unix Programming: A Guide to Concurrency, Communication, and Multithreading.. Prentice-Hall, Inc. 1996.

        Stevens, W. R. UNIX Network Programming: Interprocess Communications. 2nd ed. Prentice Hall, 1999.

        Stevens, W. R. Unix Network Programming: Networking APIs: Sockets and XTI, 2nd ed. Prentice Hall, 1999.

        Snir, M. et. al. MPI - The Complete Reference. The MPI Core, 2nd ed. MIT, 1998. (BCO)

        Gropp, W. et. al. MPI - The Complete Reference. The MPI Extensions, 2nd ed. MIT, 1998. (BCO)
    }
    % Jander, 13/5/23
    \dataatualizacao{23/10/23} % Edilson, Márcio, Luciano, Menotti, Helio
    \competencias{
        % cg-aprender/{ce-ap-1, ce-ap-4},
        % cg-produzir/{ce-pro-2, ce-pro-3, ce-pro-5},
        % cg-atuar/{ce-atuar-4, ce-atuar-5}
        cg-aprender/{ce-ap-1, ce-ap-2, ce-ap-4},
        cg-produzir/{ce-pro-2, ce-pro-3, ce-pro-4},
        cg-atuar/{ce-atuar-1, ce-atuar-2, ce-atuar-3, ce-atuar-4, ce-atuar-5},
        cg-pautar/{ce-paut-4}
    }
}
% \section{Eixo: Arquitetura de Computadores}
%     \disciplina{ld}{
    \titulo      {1}{Lógica Digital}
    \objetivo    {Ao final da disciplina o estudante deve ser capaz de projetar e analisar circuitos digitais combinatórios e sequenciais e executar sua implementação usando circuitos integrados e linguagem de descrição de hardware.}
    \requisitos  {N/A}
    \recomendadas{N/A}
    \ementa      {Conceitos fundamentais de Eletrônica Digital. Representação digital da informação. Álgebra Booleana. Tabelas verdade e portas lógicas. Expressões lógicas e formas canônicas. Estratégias de minimização de circuitos. Elementos de memória. Máquinas de estado (Mealy e Moore). Circuitos funcionais típicos (combinacionais e sequenciais).}
    \creditos    {6 total (4 teóricos, 2 práticos)}
    %    \extra       {3 horas}
    \codigo      {DC}{1001351}
    \bibliografia {
        TOCCI, Ronald J.; WIDMER, Neil S. MOSS, Gregory L. Sistemas digitais: princípios e aplicações. 11. ed. São Paulo: Pearson Prentice Hall, 2011. xx, 817 p. : il. ISBN 9788576059226.

        WAKERLY, John F. Digital design: principles and practices. 4. ed. Upper Saddle River: Pearson Prentice Hall, 2006. 895 p. ISBN 0-13-186389-4.

        FLOYD, Thomas L. Sistemas digitais: fundamentos e aplicações. 9. ed. Porto Alegre, RS: Artmed, 2007. xiii, 888 p. ISBN 9788560031931.
    }{
        Stephen Brown, Zvonko Vranesic, and Brown Stephen; Fundamentals of Digital Logic with Verilog Design; McGraw-Hill Companies,Inc.; Edition 2; 2007.

        PEDRONI, Volnei Antonio. Eletrônica digital moderna e VHDL: princípios digitais, eletrônica digital, projeto digital, microeletrônica e VHDL. Rio de Janeiro: Elsevier, 2010. 619 p. ISBN 9788535234657.

        ERCEGOVAC, Milos D.; LANG, Tomás. Digital arithmetic. San Frascisco: Morgan Kaufmann, c2004. 709 p. ISBN 1-55860-798-6.

        Victor P. Nelson, H. Troy Nagle, Bill D. Carroll, David Irwin; Digital Logic Circuit Analysis and Design; Edition 1
        Prentice Hall; 1995.

        Norman Balabanian e Bradley Carlson; Digital Logic Design Principles; Edition 1; Wiley; 2000.
    }
    \dataatualizacao{23/10/23} % Edilson, Márcio, Luciano, Menotti, Helio, Jander, 
    \competencias
    {
        cg-aprender/{ce-ap-1, ce-ap-2, ce-ap-3},
        cg-produzir/{ce-pro-1, ce-pro-2, ce-pro-4, ce-pro-5},
        cg-atuar/{ce-atuar-1, ce-atuar-2, ce-atuar-3, ce-atuar-4}
    }
}
%     \disciplina{sd}{
    \titulo      {2}{Sistemas Digitais} % sujeito a votação
    \objetivo    {Ao final da disciplina o estudante deve ser capaz de projetar, analisar e testar circuitos digitais síncronos e assíncronos complexos; e aplicar técnicas para solução de problemas inerentes a sua implementação.} % 
    \requisitos  {Lógica Digital} % Lógica Digital
    \recomendadas{N/A}
    \ementa      {Máquinas de estado algorítmicas. Arquiteturas eficientes dedicadas para solução de problemas (não von Neumann). Sincronização de relógio e temporização. Instabilidades e Falhas (Hazards e Glitches). Tecnologia de Implementação. Avaliação e testes de circuitos: Placas de Circuito Impresso, Ruídos, Alta frequência, Built-in Self-Test, Transmission-Line Effects.} % SUGESTÃO: Técnicas alternantivas para projetos de circuitos sequenciais; Sincronização de relógio e temporização; Instabilidades e Falhas (\emph{Hazards e Glitches}); Tecnologia de Implementação; Avaliação e testes de circuitos: Placas de Circuito Impresso, Ruídos, Alta frequência, \emph{Built-in Self-Test, Trnasmission line effects})}
    \creditos    {4 total (2 teóricos, 2 práticos)}
    %    \extra       {3 horas}
    \codigo      {DC}{1001539}
    \bibliografia {
        PEDRONI, Volnei Antonio. Eletrônica digital moderna e VHDL: princípios digitais, eletrônica digital, projeto digital, microeletrônica e VHDL. Rio de Janeiro: Elsevier, 2010. 619 p. ISBN 9788535234657.

        ARNOLD, Mark Gordon. Verilog digital computer design: Algorithms into hardware. Upper Saddle River: Prentice Hall PTR, c1999. 602 p. ISBN 0-13-639253-9.

        HAMBLEN, James O.; HALL, Tyson S.; FURMAN, Michael D. Rapid prototyping of digital systems. New York: Springer, c2008. xvii, 411 p. : il., grafs., ISBN 978038772670.
    }{
        Brown, Stephen D. Fundamentals of digital logic with Verilog design. Tata McGraw-Hill Education, 2007.\livrotexto{Cap. finais, conforme ementa}

        Lala. Parag K. Self-checking and Fault-tolerant Digital Design. Morgan Kaufmann, 2001.

        Harris, David. Skew-tolerant circuit design. Morgan Kaufmann, 2001.

        Oklobdzija, Vojin G., ed. Digital design and fabrication. CRC press, 2017.

        Agarwal, A. Lang, J. H. Foundations of Analog and Digital Electronic Circuits, Elsevier, 2005

        Wolf, Marilyn. Computers as components: principles of embedded computing system design. Elsevier, 2012.

        Szalapaj, Peter. Contemporary architecture and the digital design process. Routledge, 2014.
    }
    \dataatualizacao{23/10/23} % Edilson, Márcio, Luciano, Menotti, Helio, Jander    
    \competencias
    {
        cg-aprender/{ce-ap-1, ce-ap-2, ce-ap-3},
        cg-produzir/{ce-pro-1, ce-pro-2, ce-pro-4, ce-pro-5},
        cg-atuar/{ce-atuar-1, ce-atuar-2, ce-atuar-3, ce-atuar-4}
    }
}
%     \disciplina{arq1}{
    \titulo      {3}{Arquitetura e Organização de Computadores 1}
    \objetivo    {Ao final da disciplina o estudante deve ser capaz de entender os princípios da arquitetura e organização básica de computadores e a relação entre linguagens de alto nível e linguagens de máquina, bem como de criar um computador usando técnicas de implementação de unidades funcionais e analisar seu desempenho.}
    \requisitos  {Lógica Digital} % 02.437-6 -Lógica Digital
    \recomendadas{N/A}
    \ementa      {Conceitos fundamentais de Arquitetura de Computadores. Linguagem de máquina. Aritmética computacional. Organização do computador: monociclo, multiciclo e pipeline. Desempenho de computadores. Hierarquia de memória. Entrada/Saída: barramentos e dispositivos externos. Implementação de um processador completo usando linguagem de descrição de hardware.} %Incluir interrupções?
    \creditos    {6 total (4 teóricos, 2 práticos)}
    %    \extra       {3 horas}
    \codigo      {DC}{1001540} %02.735-9 antiga
    \bibliografia {
        PATTERSON, David A.; HENNESSY, John L. Organização e projeto de computadores: a interface harware/software. 3. ed. Rio de Janeiro: Elsevier, 2005. 484 p. ISBN 8535215212.

        HARRIS, David Money; HARRIS, Sarah L. Digital design and computer architecture. San Frascisco: Elsevier, 2007. 569 p. ISBN 978-0-12-370497-9.

        STALLINGS, William. Arquitetura e organização de computadores. 8. ed. São Paulo: Pearson, 2012. 624 p. ISBN 978-85-7605-564-8.

        SAITO, José Hiroki. Introdução à arquitetura e à organização de computadores: síntese do processador MIPS. São Carlos, SP: EdUFSCar, 2010. 189 p. (Coleção UAB-UFSCar. Sistemas de Informação). ISBN 978-85-7600-207-9.
    }{
        HENNESSY, John L.; PATTERSON, David A. Arquitetura de computadores: uma abordagem quantitativa. 3. ed. Rio de Janeiro: Campus, 2003. 827 p. ISBN 85-352-1110-1.

        STALLINGS, William. Arquitetura e organizacao de computadores: projeto para o desempenho. 5. ed. São Paulo: Prentice Hall, 2002. 786 p. ISBN 85-87918-53-2.
    }
    %\dataatualizacao{16/10/23} % Jander, Alexandre, Edilson, Fredy, Márcio, Alan
    \dataatualizacao{23/10/23} % Marcio, Luciano
    \competencias{
        % cg-aprender/{ce-ap-1, ce-ap-2, ce-ap-3},
        % cg-produzir/{ce-pro-1, ce-pro-2, ce-pro-4, ce-pro-5},
        % cg-atuar/{ce-atuar-1, ce-atuar-2, ce-atuar-3, ce-atuar-4}
       %cg-buscar/{ce-busc-1, ce-busc-4},        
        % Para:
        cg-aprender/{ce-ap-1, ce-ap-2, ce-ap-3},
        cg-atuar/{ce-atuar-1, ce-atuar-2}        
    }
}
%     \disciplina{arq2}{
    \titulo      {4}{Arquitetura e Organização de Computadores 2}
    \objetivo    {Ao final da disciplina o estudante deve ser capaz de entender a organização das principais arquiteturas modernas, bem como as técnicas de extração de paralelismo para o desenvolvimento visando alto desempenho.} % Destacar a capacitação para o projeto de novas arquiteturas 
    \requisitos  {Arquitetura e Organização de Computadores 1} % Arquitetura e Organização de Computadores I % Co-requisito Sistemas Operacionais
    \recomendadas{N/A}
    \ementa      {Linguagem de máquina de processadores modernos; Níveis de paralelismo: ILP, execução fora de ordem, SIMD, thread. Programação de baixo nível (System Programming) e Suporte ao Sistema Operacional. Interfaces de E/S, interrupções e timers.} % SUGESTÃO: Linguagem de máquina de processadores modernos (?); Níveis de paralelismo: ILP, execução fora de ordem, SIMD, \emph{thread} ; Programação de baixo nível (\emph{System Programming}) e Suporte ao Sistema Operacional.
    \creditos    {4 total (2 teóricos, 2 práticos)} % os dois créditos práticos não podem ser usados apenas para aprofundar os 2 teóricos! 
    %    \extra       {2 horas}
    \codigo      {DC}{1001541}
    \bibliografia {
        HENNESSY, John L.; PATTERSON, David A. Arquitetura de computadores: uma abordagem quantitativa. 3. ed. Rio de Janeiro: Campus, 2003. 827 p. ISBN 85-352-1110-1.

        STALLINGS, William. Arquitetura e organizacao de computadores: projeto para o desempenho. 5. ed. São Paulo: Prentice Hall, 2002. 786 p. ISBN 85-87918-53-2.

        HYDE, Randall. The art of assembly language. San Frascisco: No Starch Press, c2003. 903 p. ISBN 1-886411-97-2.

        IRVINE, Kip R. Assembly language for intel-based computers. 5. ed. Upper Saddle River: Prentice Hall, c2007. 722 p. ISBN 0-13-238310-1.
    }{
        PATTERSON, David A.; HENNESSY, John L. Organização e projeto de computadores: a interface harware/software. 3. ed. Rio de Janeiro: Elsevier, 2005. 484 p. ISBN 8535215212.

        HARRIS, David Money; HARRIS, Sarah L. Digital design and computer architecture. San Frascisco: Elsevier, 2007. 569 p. ISBN 978-0-12-370497-9.

        STALLINGS, William. Arquitetura e organização de computadores. 8. ed. São Paulo: Pearson, 2012. 624 p. ISBN 978-85-7605-564-8.
    }
    \dataatualizacao{23/10/23} % Marcio, Luciano
    \competencias{
        % cg-aprender/{ce-ap-1, ce-ap-2, ce-ap-3},
        % cg-produzir/{ce-pro-1, ce-pro-2, ce-pro-4, ce-pro-5},
        % cg-atuar/{ce-atuar-1, ce-atuar-2, ce-atuar-3, ce-atuar-4}
        cg-aprender/{ce-ap-1, ce-ap-2, ce-ap-3},
        cg-atuar/{ce-atuar-1},
    }
}
%     \disciplina{arqad}{
    \titulo      {7}{Arquiteturas de Alto Desempenho}
    \objetivo    {Ao final da disciplina o estudante deve ser capaz de entender e projetar os principais tipos de arquiteturas não convencionais para alto desempenho e baixo consumo energético.}
    \requisitos  {Arquitetura e Organização de Computadores 1}
    \recomendadas{N/A}
    \ementa      {Arquiteturas Heterogêneas: Aceleradores; ASIPs (Application Specific Instruction-set Processors); GPUs (Graphics Processing Units); DSPs (Digital Signal Processors); SoCs (System on Chip).}
    \creditos    {4 total (2 teóricos, 2 práticos)}
    %    \extra       {3 horas}
    \codigo      {DC}{1001537}
    \bibliografia {
        LASTOVETSKY, Alexey L.; DONGARRA, Jack J. High-performance heterogeneous computing. Hoboken, N.J.: Wiley, 2009. 267 p. (Wiley Series on Parallel and Distributed Computing). ISBN 9780470040393.

        KASTNER, Ryan. Arithmetic optimization techniques for hardware and software design. 1 online resource (viii, 187 ISBN 9780511712180 (ebook) .

        HENNESSY, John L.; PATTERSON, David A. Arquitetura de computadores: uma abordagem quantitativa. 3. ed. Rio de Janeiro: Campus, 2003. 827 p. ISBN 85-352-1110-1.

        STALLINGS, William. Arquitetura e organizacao de computadores: projeto para o desempenho. 5. ed. São Paulo: Prentice Hall, 2002. 786 p. ISBN 85-87918-53-2.
    }{
        PATTERSON, David A.; HENNESSY, John L. Organização e projeto de computadores: a interface harware/software. 3. ed. Rio de Janeiro: Elsevier, 2005. 484 p. ISBN 8535215212.

        HARRIS, David Money; HARRIS, Sarah L. Digital design and computer architecture. San Frascisco: Elsevier, 2007. 569 p. ISBN 978-0-12-370497-9.

        WOODS, Roger. WILEY INTERSCIENCE (ONLINE SERVICE). FPGA-based implementation of signal processing systems. Chichester, U.K.: John Wiley \& Sons, 2008. ISBN 9780470713785.
    }
    \dataatualizacao{12/12/23} %  Luciano
    \competencias{
        cg-aprender/{ce-ap-1, ce-ap-2, ce-ap-3},
        cg-produzir/{ce-pro-1, ce-pro-2, ce-pro-4, ce-pro-5},
        cg-atuar/{ce-atuar-1, ce-atuar-2, ce-atuar-3, ce-atuar-4}
    }
}
% \section{Eixo: Metodologia e Técnicas da Computação}
%     \disciplina{ihc}{
    \titulo      {8}{Interação Humano-Computador}
    \objetivo    {Tornar os estudantes aptos a considerar requisitos de usuário e aspectos de qualidade de uso na construção de sistemas computacionais interativos; capacitar os estudantes a fazer design de sistemas computacionais interativos, adotando modelos e técnicas bem estabelecidos; capacitar os estudantes a realizar avaliações de sistemas computacionais interativos, adotando modelos e técnicas bem estabelecidos.}
    \requisitos  {Construção de Algoritmos e Programação}
    \recomendadas{N/A}
    \ementa      {Visão geral da Interação Humano-Computador: histórico, áreas e disciplinas envolvidas. Apresentação do conceito de sistemas computacionais interativos. Apresentação de fundamentos teóricos: fatores humanos e ergonomia, modelos de engenharia, conceitos de qualidade de uso. Aprofundamento em design de sistemas computacionais interativos: abordagens ao design, modelagem da interação, apoio a decisões de design, técnicas e estilos de prototipação, documentação de decisões de design. Aprofundamento em avaliação de sistemas computacionais interativos: avaliação analítica e empírica, métodos e técnicas de avaliação de usabilidade e acessibilidade.}
    \creditos    {4 total (2 teóricos, 2 práticos)}
    %    \extra       {3 horas}
    \codigo      {DC}{1001508}
    \bibliografia {ROGERS, Yvonne; PREECE, Jennifer; SHARP, Helen. Design de interação: além da interação homem-computador. 3. ed. Porto Alegre, RS: Bookman, 2013. xiv, 585 p. : il. (color.) ISBN 9788582600061. %

    BARBOSA, Simone Diniz Junqueira; SILVA, Bruno Santana da. Interação humano-computador. Rio de Janeiro: Elsevier, 2010. 384 p. ISBN 9788535234183. %

    TULLIS, Tom; ALBERT, Bill. Measuring the user experience: collecting, analyzing, and presenting usability metrics. Burlington: Elsevier, 2008. 317 p. (The Morgan Kaufmann Series in Interactive Technologies). ISBN 978-0-12-373558-4.}
    {ROCHA, Heloisa; BARANAUSKAS, Maria Cecília Calani. (2003) Design e Avaliação de Interfaces Humano-Computador. São Paulo - Escola Computação: IME - USP, 2000. v.1. 242p. ISBN 85-88833-04-2%

    DIX, Alan; FINLAY, Janet; ABOWD, Gregory; BEALE, Russell. Human-Computer Interaction. 3rd Edition, Pearson, 2004. ISBN 978-0130461094%

    SHNEIDERMAN Ben; PLAISANT, Catherine. Designing the User Interface: Strategies for Effective Human-Computer Interaction 5th Edition,Pearson Addison-Wesley, 2009. ISBN 978-0321537355}

    \dataatualizacao{30/10/23} % Kelen, Luciano, Fedy, Alexandre, Matias 
    \competencias{
        cg-pautar/{ce-paut-2, ce-paut-3}, 
        cg-produzir/{ce-pro-1, ce-pro-2, ce-pro-4}, 
        cg-atuar/{ce-atuar-1, ce-atuar-3, ce-atuar-5}
    }
}
%     \disciplina{so}{
    \titulo      {5}{Sistemas Operacionais}
    \objetivo    {Familiarizar os estudantes com Sistemas Operacionais, apresentando seus objetivos, suas funcionalidades e aspectos de suas organizações internas. Familiarizar os estudantes com as políticas para o gerenciamento de processos e recursos. Familiarizar os estudantes com as funcionalidades providas pelos Sistemas Operacionais como gerenciadores de recursos. Tornar o estudante ciente dos algoritmos e das abstrações utilizadas em projetos de sistemas operacionais para o gerenciamento de atividades a executar (processos e threads) e para o armazenamento de dados (arquivos). Habilitar o estudante a identificar os requisitos existentes para diferentes tipos de sistemas computacionais e suas implicações no projeto do sistema operacional (sistemas de tempo-real, servidores, dispositivos com capacidades de software e hardware limitadas). Tornar os estudantes aptos a criar programas que usem eficientemente os recursos e serviços providos por sistemas operacionais. Tornar os estudantes aptos a entender e atuar no projeto e no desenvolvimento de sistemas operacionais.}
    \requisitos  {Arquitetura e Organização de Computadores 1}
    \recomendadas{N/A}
    \ementa      {Introdução. Interface do SO. Processos, threads e gerenciamento do processador. Gerenciamento de memória. Comunicação e sincronização de processos e threads. Gerenciamento de armazenamento. Estudo de caso com sistemas operacionais.}
    \creditos    {6 total (4 teóricos, 2 práticos)}
    %    \extra       {x horas}
    \codigo      {DC}{1001535}
    \bibliografia {TANENBAUM, A.S. "Sistemas Operacionais Modernos", 2. ed., Pearson Prentice Hall, 2008.
    % 
    SILBERSCHATZ, A.; GALVIN, P. B.; Gagne, G. Fundamentos de sistemas operacionais. Trad. 6. ed. LTC, 2009.
    % 
    TANENBAUM, A. S.; WOODHULL, A.S. Operating systems: design and implementation. 3 ed. Pearson Prentice Hall, 2009.}
    {STALLINGS, W. "Operating System: Internals and Design Principles",  6. ed., Prentice Hall, 2008. ISBN-10: 0136006329, ISBN-13: 978-0136006329.
    %
    MACHADO, F.B., MAIA, L.P. "Arquitetura de Sistemas Operacionais", 4. ed., LTC, 2007.ISBN: 8521615485, ISBN-13: 9788521615484.
    %
    DEITEL, H.M.; DEITEL, P.J. ; CHOFFNES. "Sistemas Operacionais", PRENTICE HALL BRASIL, 2007. ISBN: 8576050110, ISBN-13: 9788576050117.
    %
    GUIMARAES, C. C. Principios de sistemas operacionais. Rio de Janeiro: Campus, 1980. 222 p.
    %
    KIRNER, C.; MENDES, S. B. T. Sistemas operacionais distribuidos: aspectos gerais e analise de sua estrutura. Rio de Janeiro: Campus, 1988. 184 p. ISBN 85-7001-475- }

    % Comentado por Jander, para remover o erro de duplicação de competências
    % \competencias{
    % % Sistemas Operacionais
    % % Inserido por Murillo R. P. Homem, em 09/02/2023
    % % Compilado a partir do formulário preenchido por Hélio Guardia
    %     cg-aprender/{ce-ap-1, ce-ap-2, ce-ap-4},
    %     cg-produzir/{ce-pro-1, ce-pro-2, ce-pro-4},
    %     cg-atuar/{ce-atuar-1, ce-atuar-4}
    % }
    
    % Fredy Valente 06/03/2023
    \dataatualizacao{06/11/23} % Kelen, Luciano, Fedy, Alexandre, Kato, Helio, Jander, Menotti, Orides
    \competencias{
        %cg-aprender/{ce-ap-1, ce-ap-2, ce-ap-4},
        %cg-produzir/{ce-pro-1, ce-pro-2, ce-pro-4},
        %cg-atuar/{ce-atuar-1,  ce-atuar-4}
        cg-aprender/{ce-ap-1, ce-ap-2, ce-ap-4},
        cg-produzir/{ce-pro-2, ce-pro-4},
        cg-atuar/{ce-atuar-4, ce-atuar-5}
    }
}
%     \input{shared/6/6es.tex}
%     \disciplina{rc}{
    \titulo      {7}{Redes de Computadores}
    \objetivo    {Estudar as redes de computadores, abordando suas operações, funcionalidades e serviços. Apresentar tecnologias de conexão existentes, abordando aspectos de hardware e de protocolos e o projeto físico e lógico de redes.}
    \requisitos  {Sistemas Operacionais} % xxxxx
    \recomendadas{N/A}
    \ementa      {Transmissão de dados: camadas física e de enlace, sinalização, modulação e codificação, framing, endereçamento, camadas física e de enlace. Endereçamento lógico e físico, encaminhamento, roteamento e mobilidade na Internet. Endereçamento físico e lógico, roteamento fixo e dinâmico, mobilidade de nós, encaminhamento de pacotes. Controle de fluxo e de congestionamento: latência, bufferbloat, bandwidth, throughput, controle de fluxo fim a fim, controle de congestionamento na rede. Gerenciamento de rede: configuração, desempenho, contabilização, falha e segurança. Redes definidas por software (SDN), Redes de sensores, redes móveis, redes ad-hoc e redes veiculares. Qualidade de Serviço (QoS).}
    \creditos    {4 total (4 teóricos)}
    %    \extra       {x horas}
    \codigo      {DC}{1001504}
    \bibliografia {TANENBAUAN, A. "Computer Networks". Prentice-Hall, 3. ed., 1996. %G 004.6 T164c.2 (BCo) 
    % 
    KUROSE, J. F. ; ROSS, K. W. Redes de Computadores e a Internet: Uma abordagem top-down. Pearson Addison Wesley– 6ª Edição, 2014
    % 
    PETERSON, L. L.; DAVIE, B. S. Computer Networks: A Systems Approach, 5. ed,,  Editora Elsevier}
    {COMER, D. E. Redes de Computadores e Internet, 6. ed, Editora Bookman, 2016}

    % Fredy Valente 06/03/2023
    \dataatualizacao{06/11/23} % Kelen, Luciano, Fedy, Alexandre, Kato, Helio, Jander, Menotti, Orides    
    \competencias
    {
        cg-aprender/{ce-ap-1, ce-ap-2, ce-ap-4},
        cg-produzir/{ce-pro-2, ce-pro-4, ce-pro-5},
        cg-atuar/{ce-atuar-1, ce-atuar-2, ce-atuar-3, ce-atuar-4}
    }
}
%     \disciplina{bd}{
    \titulo      {6}{Banco de Dados}
    \objetivo    {Familiarizar os estudantes com os conceitos fundamentais sobre banco de dados; capacitar os estudantes para a realização de projetos de banco de dados; habilitar os estudantes para o desenvolvimento de sistemas de banco de dados; tornar os estudantes aptos a desenvolver um sistema de banco de dados utilizando um sistema gerenciador de banco de dados relacional.}
    \requisitos  {Algoritmos e Estrutura de Dados 1} % xxxxx
    \recomendadas{N/A}
    \ementa      {Conceitos básicos de banco de dados: arquitetura de um sistema de banco de dados; componentes de um sistema gerenciador de banco de dados, arquitetura cliente-servidor de banco de dados, modelos e esquemas de banco de dados. Projeto conceitual de banco de dados: modelo entidade-relacionamento e modelo entidade-relacionamento estendido. Projeto lógico de banco de dados: modelo relacional e mapeamento entre esquemas do nível conceitual para o nível lógico. Álgebra relacional. Linguagem SQL}
    \creditos   {4 total (2 teóricos, 2 práticos)}
    %    \extra       {x horas}
    \codigo      {DC}{1001493}
    \bibliografia {
        ELMASRI, Ramez; NAVATHE, Shamkant B. Sistemas de banco de dados. 6. ed. São Paulo: Pearson Addison Wesley, 2011. 788 p. ISBN 9788579360855. (disponível na BCO)

        RAMAKRISHNAN, Raghu; GEHRKE, Johannes. Sistemas de gerenciamento de banco de dados. 3. ed. São Paulo: McGraw-Hill, 2008. 884 p. ISBN 978-85-7726-027-0. (disponível na BCO)

        SILBERSCHATZ, Abraham; KORTH, Henry F.; SUDARSHAN, S. Sistema de bancos de dados. 6. ed. São Paulo: Elsevier, 2012. 861 p. ISBN 978-85-352-4535-6. (disponível na BCO)
    }{
        DATE, C. J. Introdução a sistemas de banco de dados. Rio de Janeiro: Elsevier, 2003. 865 p. ISBN 9788535212730. (disponível na BCO)

        HEUSER, Carlos Alberto. Projeto de banco de dados. 6. ed. Porto Alegre, RS: Bookman, 2009. 282 p. (Série Livros Didáticos Informática UFRGS ; v.4). ISBN 9788577803828.(disponível na BCO)

        GARCIA-MOLINA, Hector; ULLMAN, Jeffrey D.; WIDOM, Jennifer. Database system implementation. New Jersey: Prentice Hall, 2000. 653 p. ISBN 0-13-040264-8. (disponível na BCO)
    }
    \dataatualizacao{23/10/23} % Edilson, Márcio, Luciano, Menotti, Helio, Jander   
    \competencias{
        cg-aprender/{ce-ap-1, ce-ap-2},
        cg-produzir/{ce-pro-2, ce-pro-4},
        cg-atuar/{ce-atuar-4, ce-atuar-5}
    }
    % Murillo: retirei o item ce-paut-3
    % \competencias{
    %     cg-aprender/{ce-ap-3, ce-ap-4},
    %     cg-produzir/{ce-pro-1, ce-pro-2},
    %     cg-empreender/{ce-emp-1, ce-emp-2}, 
    %     cg-gerenciar/{ce-ger-1, ce-ger-3},
    %     cg-pautar/{ce-paut-1, ce-paut-4}, 
    %     cg-buscar/{ce-busc-1, ce-busc-3}
    % }

}

%     \disciplina{ia}{
    \titulo      {6}{Inteligência Artificial}
    \objetivo    {Capacitar o estudante para utilizar representação de conhecimento na construção de algoritmos a partir dos conceitos da IA. Propiciar ao estudante a aquisição dos conceitos relacionados à busca, representação de conhecimento, raciocínio automático e aprendizado de máquina. Desenvolver no estudante a competência para saber identificar problemas que podem ser resolvidos com técnicas da IA e quais técnicas podem ser adequadas a cada problema.}
    \requisitos  {Algoritmos e Estruturas de Dados 1} %  ou Programação e Algoritmos 2.

    \recomendadas{Probabilidade e Estátistica} %

    \ementa      {Caracterização da área de IA. Apresentação de métodos de busca desinformada e informada para a resolução de problemas: busca em largura, busca de custo uniforme, busca em profundidade, subida da encosta, têmpera simulada, algoritmos evolutivos. Introdução à representação de conhecimento baseada em lógica. Visão geral de métodos de raciocínio e inferência: algoritmos de encadeamento para frente e para trás, resolução e programação lógica. Introdução à representação de conhecimento incerto: quantificação de incerteza e raciocínio probabilístico. Noções de aprendizado de máquina supervisionado e não-supervisionado: classificação, regressão e agrupamento.}
    \creditos    {4 total (2 teóricos, 2 práticos)}
    %    \extra       {x horas}
    \codigo      {DC}{1001336}
    \bibliografia {
        RUSSELL, Stuart J; NORVIG, Peter. Artificial intelligence: a modern approach. 3. ed. Upper Saddle River: Prentice-Hall, c2010. 1131 p. ISBN 978-0-13-604259-4.
        % 
        LUGER, George F. Artificial intelligence: Structures and strategies for complex problem solving. 5. ed. Harlow: Addison Wesley Longman, c2005. 824 p. ISBN 0-321-26318-9.
        % 
        BRATKO, Ivan. Prolog: programming for artificial intelligence. 2. ed. Harlow: Addison-Wesley, 1990. 597 p. (International Computer Science Series). ISBN 0-201-41606-9.
    }
    {
        MITCHELL, Tom M. Machine learning. Boston: MCB/McGraw-Hill, 1997. 414 p. (McGraw-Hill Series in Computer Science). ISBN 0-07-042807-7;
    %
    BITTENCOURT, Guilherme. Inteligência artificial: ferramentas e teorias. 3. ed. Florianópolis, SC: Editora da UFSC, 2006. 371 p. : il., tabs. (Série Didática). ISBN 8532801382;
    %
    FACELI, Katti; LORENA, Ana Carolina; GAMA, João; CARVALHO, André Carlos Ponce de Leon Ferreira de. Inteligência artificial: uma abordagem de aprendizado de máquina. Rio de Janeiro: LTC, 2011. 378 p. ISBN 9788521618805;
    %
    COPPIN, Ben. Inteligência artificial. Grupo Gen-LTC, 2015.
    }
    \dataatualizacao{23/10/23} % Edilson, Márcio, Luciano, Menotti, Helio, Jander, 
    \competencias{
        cg-aprender/{ce-ap-1, ce-ap-3, ce-ap-4},
        cg-produzir/{ce-pro-1, ce-pro-2, ce-pro-4, ce-pro-5},
        cg-empreender/{ce-emp-3, ce-emp-4},
        cg-atuar/{ce-atuar-1, ce-atuar-3, ce-atuar-4}
    }
}
% \section{Eixo: Engenharias e Sistemas}
%     \disciplina{sistdin}{
    \titulo      {4}{Sistemas Dinâmicos}
    \objetivo    {Capacitar na elaboração de modelos físico-matemáticos visando possibilitar a análise ou projeto de sistemas. Prover teoria e ferramentas sistemáticas visando concluir sobre características gerais de sistemas em estudo ou visando comportamentos específicos requeridos ao projeto. Oferecer capacitação para seleção ou concepção de simuladores adequados para verificação de comportamentos em atividades de análise ou de síntese. A elaboração de um projeto que satisfaça as exigências de comportamento dinâmico previamente especificado somente se efetiva com a aplicação de conhecimentos técnicos de modelagem de dinâmica de sistemas.}
    \requisitos  {Séries e Equações Diferenciais e Física 1} % TODO: removi Fisica 2, pois não está na nossa grade (verificar)
    \recomendadas {N/A}
    \ementa      {Representação de modelos no domínio do tempo: entrada e saída e matricial (espaço de estados). Representação de modelos no domínio da frequência: Transformada de Laplace. Análise de sistemas e conceitos: modelos, aproximação, validação, protótipos e simuladores. Classificação geral de modelos de sistemas dinâmicos. Modelagem de Sistemas Lineares (Sistemas Elétricos, Sistemas Mecânico, Sistemas Fluídicos e Sistemas Térmicos), considerando as variáveis associadas à energia e fluxo, armazenamento, dissipação e balanço energético. Métricas de desempenho no tempo e na frequência e noções de identificação de parâmetros. Redução de ordem e técnicas de linearização para Sistemas ordem superior e sistemas não lineares. Técnicas computacionais para simulação de sistemas dinâmicos contínuos e discretos no tempo. Aplicações em sistemas diversos: fluídicos, eletro-hidráulicos, eletromecânicos, e termo-hidráulicos.}


    \creditos    {4 total (4 teóricos)}
    %    \extra       {3 horas}
    \codigo      {DC}{1001347}
    \bibliografia { CASTRUCCI, P. L; BITTAR, A. SALES, R. M. Controle automático. Rio de Janeiro: LTC, 2011. ISBN 978-85-216-1786-0.

    OGATA, Katsuhiko. System dynamics. 4. ed. Upper Saddle River, N.J.: Pearson Prentice Hall, 2004. ISBN 0-13-142462-9.

    CLOSE, C. M.; FREDERICK, D. K.; NEWELL, J. C. Modeling and analysis of dynamic systems. 3. ed. New York: John Wiley Sons, 2002. ISBN 0-41-39442-4.
    }
    {
        WELLSTEAD, P.E., Introduction to physical system modeling. Academic Press, New York, 1979;

    DOEBELIN, E.O., System modeling and response: theoretical and experimental approaches, John Wiley, New York, 1980;

    FELÍCIO, L. C.; Modelagem da dinâmica de sistemas e estudo da resposta. Rima 2ed.; 2010.

    AGUIRRE, L.A., Introdução à Identificação de Sistemas, Editora UFMG, 2003. (disponível na BCo)

        KARNOPP, D. C.; MARGOLIS, D. L.; ROSENBERG, R. C.; System Dynamics: Modeling, Simulation, and Control of Mechatronic Systems; Wiley; Edição 5; 2012. ISBN-10: 047088908X

    BROWN F, T,; Engineering System Dynamics: A Unified Graph-Centered Approach; CRC Press; Edição 2; 2001.;

    PALM III, W. J.; System Dynamics; McGraw Hill Education; Edição 3; 2013. ISBN-10: 0073398063;

    BROWN, F. T.; Engineering System Dynamics: A UYnified Graph-Centered Approach, CRC Press; Edição 2;
    2006. ISBN-10: 0849396484;

    SEELER, K. A.; System Dynamics: An Introduction for Mechanical Engineers; Springer; 2014. ISBN-10: 1461491517;

    KLUEVER, C.; Dynamic Systems: Modelin, Simulation, and Control; Wiley, Edição 1, 2015. ISBN-10: 1118289455.
    }
    
    %\dataatualizacao{30/10/23} % Kelen, Luciano, Fedy, Alexandre, Matias
    %\competencias{
    %     cg-aprender/{ce-ap-1, ce-ap-2, ce-ap-4},
    %     cg-atuar/{ce-atuar-1}
    % }    
    % Atualizado em 21/02/2024
    % Inserido por Murillo Rodrigo Petrucelli Homem 
    \competencias{
        cg-produzir/{ce-pro-1,ce-pro-2},
        cg-atuar/{ce-atuar-1,ce-atuar-4},
        cg-pautar/{ce-paut-4}   
    }
}

%     \disciplina{psd}{
    \titulo      {5}{Processamento de Sinais Digitais}
    \objetivo    {Prover embasamento teórico do ferramental matemático básico para a análise de sinais e sistemas no tempo contínuo e discreto com exemplos de aplicação em problemas de engenharia.}
    \requisitos  {Cálculo Diferencial e Integral 1, Geometria Analítica, Álgebra Linear e Construção de Algoritmos e Programação} % xxxxx
    \recomendadas{N/A}
    \ementa      {Introdução ao processamento de sinais. Fundamentos matemáticos de sinais e sistemas. Convolução de sinais. Análise em frequência de sinais. Série de Fourier, Transformada de Fourier e transformada Z. Amostragem e reconstrução de sinais de tempo contínuo: Teorema de Nyquist e efeito de Aliasing. Filtros digitais: análise, estruturas, técnicas de projeto e aspectos práticos de implementação.}
    \creditos    {6 total (4 teóricos, 2 práticos)}
    %    \extra       {x horas}
    \codigo      {DC}{1001486}
    \bibliografia {
        A.V. Oppenheim, A.S. Willsky e S.H. Nawab, "Signals and Systems", Segunda Edição, Prentice Hall, 1997;

        A.V. Oppenheim e R.W. Schafer, "Discrete Time Signal Processing", Prentice Hall, 1989;

        B.P. Lathi, "Sinais e sistemas lineares", 2. ed. Porto Alegre, RS: Bookman, 2007. 856 p. ISBN 978-85-60031-13-9.
    }{
        S.S. Soliman e M.D. Srninath, "Continuous and Discrete Signals And Systems", Segunda Edição, Prentice Hall, 1998.

        P. Denbigh, "System Analysis \& Signal Processsing", Addison Wesley, 1998.

        J.G.Proakis e D.G.Manolakis, Digital Signal Processing: Principles, Algorithms and Applications, 4a. edição, Pearson Prentice Hall, 2007.
    }
   \competencias{
        cg-aprender/{ce-ap-2, ce-ap-4},
        cg-produzir/{ce-pro-2, ce-pro-4, ce-pro-5},
        cg-atuar/{ce-atuar-1, ce-atuar-2, ce-atuar-3}
    }
}

%     \input{shared/7/7vc.tex}
%     % \input{shared/7/7sc.tex} % Optativa
%     \disciplina{controle1}{
    \titulo      {5}{Controle 1}
    \objetivo    {Desenvolver habilidades de modelagem, análise e projeto de sistemas de controle para ambientes de natureza dinâmica com característica linear em que as grandezas físicas devem evoluir de acordo com restrições ou requisitos desejados, baseada na teoria de controle clássico para sistemas em tempo contínuo SISO (Single Input Single Output).}
    \requisitos  {Sistemas Dinâmicos} % % xxx?)
    \recomendadas{N/A}
    \ementa      {O problema de controle. Comparação de malha aberta e fechada. diagrama de blocos. Análise temporal da resposta transitória de sistemas: sistemas de primeira, segunda ordem e superior, critérios de desempenho. Estabilidade: estabilidade de sistemas lineares, critério de estabilidade de Routh e erro em regime. Análise e projeto de sistemas de controle pelo método do lugar das raízes: Representação geométrica do lugar das raízes e método de construção, projeto de compensadores para melhoria da resposta transitória e do erro em regime. Análise e projeto de sistemas de controle no domínio da frequência: Diagramas de Bode, Diagrama de Nyquist, Estabilidade de sistemas com realimentação, projeto de compensadores para melhoria da resposta transitória e erro em regime. Implementação de controladores PID.}
    \creditos    {6 total (4 teóricos, 2 práticos)}
    %    \extra       {x horas}
    \codigo      {DC}{1001348}
    \bibliografia {
        CASTRUCCI, Plínio de Lauro; BITTAR, Anselmo; SALES, Roberto Moura. Controle automático. Rio de Janeiro: LTC, 2011. ISBN 978-85-216-1786-0.

        OGATA, Katsuhiko. Engenharia de controle moderno. 5. ed. São Paulo: Pearson, 2011. 809 p. ISBN 978-85-7605-810-6.

        NISE, Norman S. Control systems engineering. 2. ed. Redwood City: The Benjamin/Cummings, c1995. ISBN 0-8053-5424-7.
    }
    {FRANKLIN, Gene F.; POWELL, J. David; EMANI-NAEINI, Abbas. Feedback control of dynamic systems. 2. ed. Reading: Addison-Wesley, 1991.

    Kluever, C.; Dynamic Systems: Modelin, Simulation, and Control; Wiley, Edição 1, 2015. ISBN-10: 1118289455

    FRANKLIN, Gene F.; POWELL, J. David; WORKMAN, Michael L. Digital control of dynamic systems. 2. ed. Reading: Addison-Wesley, 1990. ISBN 0-201-11938-2.

    OGATA, Katsuhiko. Engenharia de controle moderno. 5. ed. São Paulo: Pearson, 2011. ISBN 978-85-7605-810-6.

    KUO, Benjamin C. Digital control systems. 2. ed. Ft. Worth: Saunders College Publishing, c1992. 751 p. (HRW Series in Electrical Engineering). ISBN 0-03-012884-6.

    ASTROM, Karl Johan; WITTENMARK, Bjorn. Adaptive control. Reading: Addison-Wesley, c1989. 526 p. (Addison-Wesley Series in Electrical and Computer Engineering Control Engineering).

    CRUZ, José Jaime da. Controle robusto multivariável. São Paulo: Edusp, 1996. 163 p. (Acadêmica ; v. 5). ISBN 9788531403413.

    HEMERLY, Elder Moreira. Controle por computador de sistemas dinâmicos. 2. ed. São Paulo: Blucher, 2011. 249 p. ISBN 978-85-212-0266-0.
    }

    \dataatualizacao{12/12/23} % Luciano   
    \competencias{
        cg-aprender/{ce-ap-1, ce-ap-2},
        cg-empreender/{ce-emp-3, ce-emp-4, ce-emp-5}, 
        cg-atuar/{ce-atuar-3, ce-atuar-4, ce-atuar-5}
    }

}
%     \disciplina{controle2}{
    \titulo      {6}{Controle 2}
    \objetivo    {Desenvolver habilidades de modelagem, análise e projeto de sistemas de controle para ambientes de natureza dinâmica com característica linear em que as grandezas físicas devem evoluir de acordo com restrições ou requisitos desejados, baseada na teoria de controle clássico para sistemas em tempo discreto SISO (Single Input Single Output). Desenvolver habilidades de modelagem, análise e projeto de sistemas de controle para ambientes de natureza dinâmica com característica linear em que as grandezas físicas devem evoluir de acordo com restrições ou requisitos desejados; baseada na teoria de controle baseada na abordagem de espaço de estados.}
    \requisitos  {Controle 1} % % xxx?)
    \recomendadas{N/A}
    \ementa      {O problema de controle em sistemas amostrados, sistemas de controle digital: (equações de diferença / teorema de Shannon) aplicação de conversor A/D e conversor D/A junto ao processo. Mapeamento do plano s no plano z. Estabilidade de sistemas em tempo discreto: critérios de routh e Jury; aproximações de tempo discreto. Projeto de controlador discreto a partir de projeto de controlador de tempo contínuo. Erro em regime permanente. Resposta transiente no plano z: influência do período de amostragem em transitórios; controlador PID discreto, projeto no domínio da frequência; controlador dead beat. Introdução a espaço de estados: conceito sobre variável de estado, representação de sistemas dinâmicos no espaço de estados. Análise das equações de estado: controlabilidade e observabilidade. Projeto de lei de controle. Projeto de estimador. Projeto do compensador.}
    \creditos    {6 total (4 teórico, 2 práticos)}
    %    \extra       {x horas}
    \codigo      {DC}{1001534}
    \bibliografia {
        KUO, Benjamin C. Digital control systems. 2. ed. Ft. Worth: Saunders College Publishing, c1992. ISBN 0-03-012884-6.

        FRANKLIN, Gene F.; POWELL, J. David; WORKMAN, Michael L. Digital control of dynamic systems. 2. ed. Reading: Addison-Wesley, 1990. ISBN 0-201-11938-2.

        CASTRUCCI, Plínio de Lauro; BITTAR, Anselmo; SALES, Roberto Moura. Controle automático. Rio de Janeiro: LTC, 2011. ISBN 978-85-216-1786-0.

        NISE, Norman S. Control systems engineering. 2. ed. Redwood City: The Benjamin/Cummings, c1995. ISBN 0-8053-5424-7.

        FRANKLIN, Gene F.; POWELL, J. David; EMANI-NAEINI, Abbas. Feedback control of dynamic systems. 2. ed. Reading: Addison-Wesley, 1991.
    }
    {OGATA, Katsuhiko. Engenharia de controle moderno. 5. ed. São Paulo: Pearson, 2011. ISBN 978-85-7605-810-6.

    NISE, Norman S. Control systems engineering. 2. ed. Redwood City: The Benjamin/Cummings, c1995. ISBN 0-8053-5424-7.

    ISERMANN, Rolf. Digital control systems. 2. ed. Berlin: Springer-Verlag, c1991. ISBN 3-540-50997-6.

    HEMERLY, Elder Moreira. Controle por computador de sistemas dinâmicos. 2. ed. São Paulo: Blucher, 2011. ISBN 978-85-212-0266-0.

    NISE, Norman S. Control systems engineering. 2. ed. Redwood City: The Benjamin/Cummings, c1995. ISBN 0-8053-5424-7.

    CRUZ, José Jaime da. Controle robusto multivariável. São Paulo: Edusp, 1996. ISBN 9788531403413.

    GOODWIN, Graham Clifford; GRAEBE, Stefan F.; SALGADO, Mario E. Control system design. Upper Saddle River: Prentice Hall, c2001. ISBN 0-13-958653-9.}

    \dataatualizacao{30/10/23} % Kelen, Luciano, Fedy, Alexandre, Matias     
    \competencias{
        cg-aprender/{ce-ap-1, ce-ap-2},
        cg-empreender/{ce-emp-3, ce-emp-4, ce-emp-5}, 
        cg-atuar/{ce-atuar-3, ce-atuar-4, ce-atuar-5}
    }
}
%     \disciplina{om}{
    \titulo      {8}{Otimização Matemática}
    \objetivo    {Desenvolver competências nos seguintes tópicos da área de otimização: modelagem e análise e resolução de problemas de otimização lineares e não lineares; gerar capacitação para resolução de tais problemas de forma analítica e computacional. Abordagem a partir de versões aproximadas das estratégias exatas.}
    \requisitos  {Cálculo 2}  % % xxx?)
    \recomendadas{N/A}
    \ementa      {Programação Linear: Método simplex, Dual do Problema, Dualidade. Programação Inteira. Método Branch-and-Bound. Programação não linear: com e sem restrições. Método gradiente conjugado e Hessiano. Multiplicadores de Lagrange, Fluxo em Redes: algoritmos Ford-Fulkerson e Edmonds-Karp (teorema min cut/max flow); Emparelhamento máximo em grafos bipartidos: algoritmo húngaro. Métodos dos mínimos quadrados e regressão linear. Teoria das filas. Simulação de Eventos Discretos.}
    \creditos    {4 total (4 teóricos)}
    %    \extra       {x horas}
    \codigo      {DC}{1001346}
    \bibliografia {
        Jon Kleinberg, Eva Tardos. Algorithm Design: Pearson New International Edition. Pearson Education Limited, 2013. 832 p. ISBN 9781292037042

        T.H. Cormen, C.E. Leiserson, R.L. Rivest, C. Stein. Introduction to Algorithms, 3rd ed., McGraw-Hill, 2009.

        LACHTERMACHER, Gerson. Pesquisa operacional na tomada de decisões. 4. ed. São Paulo: Pearson Prenttice Hall, 2013. 223 p. ISBN 9788576050933

        ARENALES, Marcos Nereu; ARMENTANO, Vinícius Amaral; MORABITO, Reinaldo; YANASSE, Horácio. Pesquisa operacional. Rio de Janeiro: Elsevier, 2007. 524 p. (Coleção CAMPUS-ABREPO Engenharia de Produção). ISBN 85-352-1454-3

        TAHA, Hamdy A. Pesquisa operacional. 8. ed. São Paulo: Pearson, 2008. xiii, 359 ISBN 9788576051503
    }{
        HILLIER, Frederick S; LIEBERMAN, Gerald J. Introdução à pesquisa operacional. 8. ed. São Paulo: McGraw Hill, 2006. 828 p. ISBN 8586804681

        ANDRADE, Eduardo Leopoldino de. Introdução à pesquisa operacional: métodos e modelos para análise de decisões. 3. ed. Rio de Janeiro: LTC, 2004. xiii, 192 p. ISBN 8521614128

        ELLENRIEDER, Alberto Von. Pesquisa operacional. Rio de Janeiro: Almeida Neves-Editores, 1971. 261 p.
    }
    \dataatualizacao{06/11/23} % Kelen, Luciano, Fedy, Alexandre, Kato,    Helio, Jander, Menotti, Orides
    \competencias{
        % Otimização Matemática
        % Inserido por Murillo R. P. Homem, em 09/02/2023
        % Compilado a partir do formulário preenchido por Murillo R. P. Homem
        %cg-produzir/{ce-pro-1, ce-pro-2, ce-pro-5},
        %cg-atuar/{ce-atuar-1, ce-atuar-3, ce-atuar-4},
        %cg-gerenciar/{ce-ger-1, ce-ger-2}
        cg-empreender/{ce-emp-1, ce-emp-2},
        cg-atuar/{ce-atuar-1, ce-atuar-2},
        cg-pautar/{ce-paut-4}
    }
}
%     \input{shared/7/7protsisdigan.tex}
%     \disciplina{sistdist}{
    \titulo      {7}{Sistemas Distribuídos}
    \objetivo    {Familiarizar o estudante com aspectos inerentes à interligação lógica de sistemas computacionais fracamente acoplados. Familiarizar o estudante com as dificuldades e técnicas para prover comunicação, sincronização e coordenação entre múltiplos sistemas de computação distribuídos. Capacitar o estudante a tratar do compartilhamento ordenado e seguro de recursos computacionais distribuídos. Capacitar o estudante a tratar do desenvolvimento de técnicas e infraestruturas de software para ambientes computacionais distribuídos. Habilitar o estudante a criar aplicações que usem de maneira eficiente múltiplos recursos computacionais distribuídos.}
    \requisitos  {Sistemas Operacionais} % % xxx?)
    \recomendadas{N/A}
    \ementa      {Motivações, objetivos e caracterização de Sistemas Distribuídos. Arquiteturas de sistemas distribuídos; middleware. Processos, threads e unidades de execução de código; modelos cliente / servidor e peer-to-peer; virtualização. Comunicação em rede, protocolos e APIs. Invocação de códigos remotos. Comunicação orientada a mensagens, a fluxos e multicast. Nomeação: identificadores e localização. Sincronização. Relógios físicos e lógicos. Ordenação. Exclusão mútua. Eleição; coordenação. Consistência e replicação: modelos de consistência; gerenciamento de réplicas; protocolos de consistência. Tolerância a faltas: modelos; redundância; resiliência de processos e de comunicação. Comunicação confiável. Acordos distribuídos e consenso; recuperação. Segurança: ameaças, políticas e mecanismos; criptografia; canais seguros; controle de acesso. Gerenciamento de segurança. Estudo de casos em Sistemas Distribuídos.}
    \creditos    {4 total (4 teóricos)}
    %    \extra       {x horas}
    \codigo      {DC}{1001503}
    \bibliografia {TANENBAUM, A. S., Steen, M. V. Sistemas distribuídos: princípios e paradigmas.. 2. ed. São Paulo. Pearson Prentice Hall, 2007. %

    COULOURIS, G.; DOLLIMORE, J.; KINDBERG, T.; and BLAIR, G. Distributed systems: concepts and design. 5th. ed. Addison-Wesley, 2012. % 

    TANENBAUM, A. S. Distributed operating systems. Prentice Hall, c1995. 614 p. (disponível na BCo)}
    {Addison-Wesley, 2009. Ghosh, Sukumar. Distributed systems: an algorithmic approach. Chapman \& Hall/CRC, c2007.
    % 
    BIRMAN, K. P. Reliable distributed systems: technologies, web services, and applications. New York: Springer, 2010.
    % 
    ANTONOPOULOS, N.; Gilliam L. Cloud computing: principles, systems and applications. New York. Springer, 2010.    % 
    SINHA, P. K. Distributed operating systems: concepts and design. New York: IEEE Computer Society Press, 1997.
    % 
    TEL, G. Introduction to distributed algorithms. 2nd. ed. Cambridge University Press, 2000.
    }
    % Fredy Valente 06/02/2023
    \competencias{
        cg-aprender/{ce-ap-1, ce-ap-2, ce-ap-4},
        cg-produzir/{ce-pro-2, ce-pro-4, ce-pro-5},
        cg-atuar/{ce-atuar-1, ce-atuar-2, ce-atuar-3, ce-atuar-4}
    }
}
%     % \input{shared/7/7str.tex} % Optativa
%     \disciplina{teccom}{
    \titulo      {7}{Tecnologia de Comunicação}
    \objetivo    {Capacitar o estudante nas tecnologias de comunicação para redes de computadores, abordando suas operações, funcionalidades e serviços, incluindo aspectos de hardware dos módulos de transmissão e recepção, em meios guiados (cabo, fibra ótica) e não guiados (ar, água, espaço sideral). Habilitar o conhecimento das tecnologias de transmissão digital de dados por rádio, incluindo sinais, sinalização por ondas de rádio e óticas, os conceitos teóricos para transmissão analógica e digital, os padrões e protocolos industriais existentes e emergentes, necessários a execução de projetos de sistemas de comunicação para redes de computadores sem fio.}
    \requisitos  {Sistemas Operacionais} % % xxx?)
    \recomendadas{N/A}
    \ementa      {Propagação de sinal em diferentes meios físicos (guiados e não guiados). Estratégias de codificação. Modulação, Multiplexação. Transmissão Analógica e Digital. Telefonia e Comutação. Topologias de Rede de Computadores. Controle de Acesso ao meio físico. Controle do Enlace de Dados. Delimitações, Endereçamento, Tratamento de Erros e Encapsulamento. Tecnologias e Padrões de Comunicação da Indústria e Emergentes. Padrões de Interoperabilidade e Segurança; Radio definido por Software. Projeto e Implementação de Sistema de Comunicação.}
    \creditos    {4 total (2 teóricos, 2 práticos)}
    %    \extra       {x horas}
    \codigo      {DC}{1001505}
    \bibliografia {NICOLAIDIS, I.; BARBEAU, M.; KRANAKIS, E.; Ad-Hoc, Mobile and Wireless Networks. Heildelberg Springer-Verlag, 2004, ISBN 3-540-22543-9.

    STALLINGS, W.; Data and Computer Communications. 6th Edition, Upper Saddle River : Prentice Hall, c2000, ISBN 0-13-084370-9.
    }
    {
        JOHNSON JR, C.R.; SETHARES, W.A.; Telecommunication Breakdown - Concepts of Communication Transmitted via Software Defined Radio Pearson-Prentice Hall, 2003, ISBN 0-13-143047-5.

    RAPPORT, T.S.; Comunicações sem Fio, Princípios e Práticas, 2a Edição, Pearson-Prentice Hall, 2009, ISBN 978-85-7605-198-5
    }
% Fredy Valente 06/03/2023
    \competencias{
        cg-aprender/{ce-ap-1, ce-ap-2, ce-ap-4},
        cg-produzir/{ce-pro-2, ce-pro-4, ce-pro-5},
        cg-atuar/{ce-atuar-1, ce-atuar-2, ce-atuar-3, ce-atuar-4}
    }
}

%     \disciplina{engsis}{
    \titulo      {6}{Engenharia de Sistemas}
    \objetivo    {Capacitar o estudante para que o mesmo defina de maneira precoce no ciclo de desenvolvimento de um sistema as necessidades do usuário, bem como as funcionalidades requeridas, realizando a documentação sistemática dos requisitos, e abordando a síntese de projeto e a etapa de validação de forma a considerar o problema completo: operação; custos e cronogramas; performance; treinamento e suporte; teste; instalação e fabricação de sistemas computacionais físicos.}
    \requisitos  {Engenharia de Software 1} % %Engenharia de Software 1)
    \recomendadas{Arquitetura de Computadores 1}
    \ementa      {Engenharia de sistemas (design, síntese, análise, avaliação, manutenção). Detalhamento do design e síntese (design conceitual, preliminar e detalhado). Decomposição lógica (Functional packing). Stakeholders. Work Breakdown Structure (WBS). Matriz de responsabilidades. Requisitos técnicos. Aplicação de CADs. Padronização e normativas para o design de sistemas de engenharia. Detalhamento de análise e avaliação de sistemas computacionais físicos. Gerenciamento de Configurações. Revisão Técnica e Auditorias. Trade Studies. Modelagem e Métricas de Simulação. Gerenciamento de riscos. Otimização. Confiabilidade. Sustentabilidade. Análise de Tolerância a Falhas. Detalhamento da manutenção de sistemas computacionais físicos. Análise de Tarefa de Manutenção (Maintenance Task Analysis - MTA). Predição.}
    \creditos    {4 total (4 teóricos)}
    %    \extra       {x horas}
    \codigo      {DC}{1001545}
    \bibliografia { NASA Systems Engineering Handbook: NASA/SP-2016-6105 Rev2 - NASA - National Aeronautics and Space Administration (Author), Space Science Library;
    % 
    Systems engineering handbook : a guide for system life cycle processes and activities / prepared by International Council on Systems. Engineering (INCOSE) ; compiled and edited by, David D. Walden, ESEP, Garry J. Roedler, ESEP, Kevin J. Forsberg, ESEP,. R. Douglas Hamelin, Thomas M. Shortell, CSEP., 4. ed. 2015.
    % 
    BLANCHARD, B. S; FABRYCKY, W. J. Systems Engineering and Analysis 5th Ed. Prentice Hall International Series in Industrial \& Systems Engineering, 2011;}
    {
        Systems Engineering Fundamentals (2001). The Defense acquisition University Press Fort BElvoir, Virgin The Defense acquisition University Press Fort BElvoiria 22060-5565.

    SAGE, A. P. Introduction to Systems Engineering, John Wiley \& Sons, 2000.}
    
    % Fredy 06/03/2023
    
    \dataatualizacao{30/10/23} % Kelen, Luciano, Fedy, Alexandre, Matias 
    \competencias{
        cg-aprender/{ce-ap-2},
        cg-gerenciar/{ce-ger-1, ce-ger-2, ce-ger-3},
        cg-produzir/{ce-pro-1, ce-pro-2, ce-pro-3, ce-pro-4, ce-pro-5},
        cg-empreender/{ce-emp-1, ce-emp-2, ce-emp-4}
    }
}
%     \disciplina{projsisemb}{
    \titulo{7}{Projeto de Sistemas Computacionais Embarcados}
    \objetivo{Ao final da disciplina o estudante deve ser capaz de entender os conceitos, elementos, problemas e soluções típicas no desenvolvimento de sistemas computacionais embarcados. Entender o princípio de operação, configuração, vantagens e desvantagens dos periféricos mais utilizados em sistemas computacionais. Projetar, analisar e testar o hardware e o software de sistemas computacionais embarcados e de aplicar técnicas para solução de problemas inerentes a estes sistemas.}
    \requisitos  {Arquitetura e Organização de Computadores 2 e Engenharia de Sistemas} % % xxx?) 
    \recomendadas{N/A}
    \ementa      {Conceitos e aplicações de sistemas computacionais embarcados. Metodologias para o desenvolvimento de Sistemas Embarcados: engenharia dirigida por modelos, AADL, SysML. Co-projeto de hardware e software. Ciclo de desenvolvimento de software: diagramas de fluxo de dados, statecharts, redes de petri temporizadas. Sensores, conversores, atuadores e outros componentes típicos. Microkernels: multitarefa, escalonamento e sincronização. Sistemas Críticos: RTOS, tolerância a falhas, redundância, certificação. Geração automática de código. Testes e simulação: hardware e software in the loop. Exemplos práticos de projeto de sistemas embarcados. Prototipação.}
    \creditos    {4 total (4 práticos)}
    %    \extra       {x horas}
    \codigo      {DC}{1001538}
    \bibliografia {WOLF, Wayne. Computers as components: principles of embedded computing system design. San Francisco: Morgan Kaufmann, c2005. 6556 p. ISBN 0-12-369459-0.
    % 
    BALL, Stuart R. Analog interfacing to embedded microprocessor systems. 2. ed. Boston: Newnes, c2004. 322 p. (Embedded Technology Series). ISBN 978-0-7506-7723-3.
    % 
    BRÄUNL, Thomas. Embedded robotics: mobile robot design and applications with embedded systems. 2. ed. Berlin: Springer- Verlag, c2006. 458 p. ISBN 3-540-34318-0.
    % 
    QING, Li; CAROLINE, Yao. Real-time concepts for embedded systems. San Frascisco: CMP Books, c2003. 294 p. ISBN 978-1-57820-124-2. QING, Li; CAROLINE, Yao. Real-time concepts for embedded systems. San Francisco: CMP Books, c2003. 294 p. ISBN 978-1-57820-124-2.
    % 
    HOLT, Jon; PERRY, Simon. SysML for systems engineering: a model-based approach. 2. ed. Stevenage: Institution of Engineering and Technology, 2013. 930 p. (Professional Applications of Computing Series; 7). ISBN 978-1-84919-651-2.}
    {Peter Marwedel, Embedded System Design: Embedded Systems Foundations of Cyber-Physical Systems, and the Internet of Things 3rd ed. 2018 Edition;

    KORDON, F., HUGUES, J. CANALS, A. ; DOHET, A. Embedded Systems: Analysis and Modeling with SysML, UML and AADL 1st Edition, 2013;

    Frank Vahid e Tony Givargis. Embedded System Design: A Unified Hardware/Software Introduction. Wiley. 2002.}

    \dataatualizacao{12/12/23} % Luciano   
    \competencias{
        % Fredy Valente 10/03/2023
        %cg-aprender/{ce-ap-1, ce-ap-2, ce-ap-4},
        %cg-produzir/{ce-pro-2, ce-pro-4, ce-pro-5},
        %cg-atuar/{ce-atuar-1, ce-atuar-2, ce-atuar-3, ce-atuar-4}
        cg-aprender/{ce-ap-1, ce-ap-2, ce-ap-4},
        cg-produzir/{ce-pro-1, ce-pro-2, ce-pro-4, ce-pro-5},
        cg-atuar/{ce-atuar-1, ce-atuar-2, ce-atuar-3, ce-atuar-4}        
    }
}

% \disciplina{
%     \titulo      {7}{Projeto de Sistemas Embarcados I }  
%     \objetivo    {Ao final da disciplina o estudante deve ser capaz de projetar, analisar e testar o hardware e o software de Sistemas Embarcados; e de aplicar técnicas para solução de problemas inerentes a estes sistemas. Deve também propor e especificar um sistema completo.}
%     \requisitos  {XX.XXX-X} % % xxx?) Sistemas Digitais (PCB), Arquiteturas de Alto Desempenho (SoCs), Circuitos Eletronicos (Sensores e Atuadores)
%     \recomendadas{N/A}
%     \ementa      {Metodologias para Projeto e Verificação de Sistemas Sistemas Embarcados; Diferenças entre Microcontroladores e Microprocessadores; Co-projeto de hardware e software; Interface com sensores, atuadores, conversores e outros componentes típicos.} % Base: Microcontroladores e Aplicações, mas com ênfase nos requisitos de um projeto com o objetivo de compatibilizar as necessidades das aplicações aos recursos. 
%     % Garantir que se consiga ter o projeto da placa! Mauricio
%     \creditos    {4 total (2 teóricos, 2 práticos)} % 4 práticos?
% %    \extra       {x horas}
%     \codigo      {DC}{XX.XXX-X}
% }

% \disciplina{
%     \titulo      {8}{Projeto de Sistemas Embarcados II}
%     \objetivo    {Ao final da disciplina o estudante deve ter implementado e testado o hardware e o software de um Sistema Embarcado a partir de seu projeto.}
%     \requisitos  {XX.XXX-X} % % xxx?)
%     \recomendadas{N/A}
%     \ementa      {Cronograma; Acompanhamento das fases de implementação e testes; Apresentação.}
%     \creditos    {4 total (4 práticos)}
% %    \extra       {0 horas}
%     \codigo      {DC}{XX.XXX-X}
% }
% \section{Eixo:Formação Humanas}
%     \disciplina{sem1}{
    \titulo      {3}{Seminários 1}
    \objetivo    {Assegurar a formação de profissionais dotados de conhecimento das questões sociais, profissionais, legais, éticas, políticas e humanísticas; da compreensão do impacto da computação e suas tecnologias na sociedade; de utilizar racionalmente os recursos disponíveis de forma transdisciplinar; de competências para entender a dinâmica social segundo as perspectivas econômicas e assumir decisões que levem em conta tais perspectivas; e de capacidade para atuar segundo as tendências profissionais atuais.}
    \requisitos  {N/A} % xxxxxxx
    \recomendadas{N/A}
    \ementa      {A disciplina será baseada em seminários dos mais diversos temas, de acordo com o seu objetivo. Os temas a seguir serão obrigatoriamente abordados em todas as ofertas da disciplina. Outros temas serão apresentados aos estudantes de acordo com as necessidades mais prementes associadas ao perfil do egresso:
        \begin{itemize}
            \itemsep0em
            \item A propriedade intelectual e suas implicações no contexto da computação;
            \item A ética, a moral e o direito: pilares de sustentação da dinâmica social;
            \item Estratégias de ensino e aprendizagem;
            \item A importância da matemática;
            \item Comportamentos individuais dirigidos à manutenção da saúde;
            \item As relações interpessoais no ambiente de trabalho e as competências profissionais especificas no trabalho em equipe;
            \item A importância da ergonomia no ambiente de trabalho;
            \item A necessidade da reforma política no Brasil.
        \end{itemize}
    }
    \creditos    {2 total (2 práticos)}
    %    \extra      {0 horas}
    \codigo      {DC}{1001344}
    \bibliografia {
        LEMOS, Ronaldo. Direito, tecnologia e cultura. Rio de Janeiro: Ed. FGV, 2005. 211 p. ISBN 85-225-0516-0.

        COLBARI, Antônia de Lourdes. Ética do trabalho: a vida familiar na construção da identidade profissional. São Paulo: Letras \& Letras, 1995. 278 p. ISBN 85-85387-53-X

        DIAZ BORDENAVE, Juan E.; PEREIRA, Adair Martins. Estratégias de ensino-aprendizagem. 10. ed. Petropolis: Vozes, 1988. 312 p.

        PRADO, Shirley Donizete (Org.) et al. Alimentação, consumo e cultura. Curitiba: CRV, 2013. 240 p. (Série Sabor Metrópole ; v. 1). ISBN 9788580427790.

        GUERIN, Bernard. Analyzing social behavior: behavior analysis and the social sciences. Reno: Context Press, c1994. 382 p. ISBN 1-87978-13-6.

        COUTO, Hudson de Araujo. Ergonomia aplicada ao trabalho: o manual tecnico da maquina humana. Belo Horizonte: Ergo, 1995.

        BOBBIO, Norberto. A teoria das formas de governo. Brasília: UnB, 1980. Não paginado (Colecao Pensamento Politico; v.17).
    }{
        SCIENTIFIC authorship: credit and intellectual property in science. New York: Routledge, 2003. 384 p. ISBN 0-415-94293-4.

        SENNETT, Richard. A corrosao do carater. 4. ed. Rio de Janeiro: Record, 2000. 204 p. ISBN 85-01-0561-5.

        MATURANA, Humberto Romesin; VARELA GARCIA, Francisco J. A árvore do conhecimento: as bases biológicas da compreensão humana. São Paulo: Palas Athena, 2001. 283 p. ISBN 85-72420-32-0.

        RODRIGUES, Rosicler Martins. Alimentacao e saude. Sao Paulo: Moderna, 1994. 48 p. (Colecao Desafios Serie Teen). ISBN 85-16-01140-2.

        PICKERING, Peg. Como administrar conflitos profissionais: técnicas para transformar conflitos em resultados. 10. ed. São Paulo: Market Books, c1999. 114 p. ISBN 85-87393-28-6.

        WISNER, Alain. A inteligencia no trabalho: textos selecionados de ergonomia. Sao Paulo: FUNDACENTRO, 2003.

        SENADO FEDERAL; SECRETARIA DE DOCUMENTAÇÃO E INFORMAÇÃO; SUBSECRETARIA DE BIBLIOTECA. Formas e sistemas de governo: bibliografia. Brasília , 1991.
    %TODO: Algum comentário que sobre os temas variáveis.
    }
    % Fredy Valente 13/03/2023
    \competencias{
        cg-aprender/{ce-ap-1, ce-ap-2},
        cg-atuar/{ce-atuar-3, ce-atuar-4},
        cg-pautar/{ce-paut-1, ce-paut-2, ce-paut-3},
    }
}
%     \disciplina{metcient}{
    \titulo      {8}{Metodologia Científica}
    \objetivo    {Habilitar o estudante a compreender e dominar os mecanismos do processo de investigação científica tanto para o desenvolvimento do Trabalho de Conclusão de Curso (TCC) quanto para sua atuação profissional. Familiarizar o estudante com a metodologia do trabalho científico caracterizando procedimentos básicos, pesquisa bibliográfica, projetos e relatórios; publicações e trabalhos científicos; e os princípios e práticas para a elaboração do TCC.}
    \requisitos  {N/A} % Podemos abrir mão do requisito "Texto técnico"? Esta disciplina não está na grade da EnC e se não fizermos isso, teremos duas metodologias exatamente iguais com códigos diferentes (provavelmente equivalentes) apenas por isso.
    \recomendadas{N/A}
    \ementa      {Caracterização do que é pesquisa, sua motivação e metodologia de desenvolvimento. Apresentação dos tipos de pesquisa (iniciação científica, trabalho de conclusão de curso, etc.) e seus objetivos. Introdução aos principais conceitos relacionados à pesquisa (como objetivo, tema, problema, hipótese e justificativa). Descrição detalhada das etapas da pesquisa: determinação do tema-problema de trabalho, revisão bibliográfica, construção lógica do trabalho, desenvolvimento do trabalho e redação do texto. Conceituação de aspectos da ética na pesquisa científica: definição, princípios, plágio, conduta ética na pesquisa científica. Aprofundamento da organização da escrita científica: estrutura formal do trabalho, suas partes e conteúdo esperado, tipos de publicações científicas e suas peculiaridades. Orientação sobre a elaboração de referências e citações bibliográficas e a apresentação da pesquisa.}
    \creditos    {4 total (4 teóricos)}
    %    \extra       {3 horas}
    \codigo      {DC}{1001343}
    \bibliografia {
        WAZLAWICK, Raul Sidnei. Metodologia de pesquisa para ciência da computação, Rio de Janeiro: Elsevier, 2009. 159 p.

        CERVO, Amado Luiz.; BERVIAN, Pedro Alcino; da Silva, R. Metodologia científica, São Paulo: Makron, 4ª ed., 1996. 209 p.

        LAKATOS, Eva Maria; MARCONI, Marina de Andrade. Fundamentos de metodologia científica. 7. ed. Sao Paulo: Atlas, 2010. 297 p. ISBN 978-85-224-5758-8.
    }{
        CASELI, Helena; UFSCAR. SEAD. Metodologia científica. São Carlos, SP: EdUFSCar, 2013.

        BARROS, Aidil Jesus da Silveira; LEHFELD, Neide Aparecida de Souza. Fundamentos de metodologia científica. 3. ed. São Paulo: Pearson Prentice Hall, 2007. 158 p. ISBN 978-85-7605-156-5.

        BASTOS, Cleverson Leite; KELLER, Vicente. Aprendendo a aprender: introducao a metodologia científica. 18. ed. Petropolis: Vozes, 1998. 111 p. ISBN 85-326-0586-9.

        PARRA FILHO, Domingos; SANTOS, João Almeida. Metodologia científica. 4. ed. São Paulo: Ed. Futura, 2001. 277 p. ISBN 85-86082-81-3.

        KOCHE, José Carlos. Fundamentos de metodologia científica: teoria da ciência e iniciação à pesquisa. 29. ed. Petrópolis, RJ: Vozes, 2011. 182 p. ISBN 9788532618047.

        MATTAR, João. Metodologia científica na era da informática. 3. ed. São Paulo: Saraiva, 2011. 308 p. ISBN 9788502064478.

        SANTOS, João Almeida; PARRA FILHO, Domingos. Metodologia científica. 2. ed. São Paulo: Cengage Learning, 2012. 251 p. ISBN 9788522112142.
    }
    
    \dataatualizacao{30/10/23} % Kelen, Luciano, Fedy, Alexandre, Matias     
    \competencias{
        cg-aprender/{ce-ap-1, ce-ap-2, ce-ap-3},
        cg-produzir/{ce-pro-1, ce-pro-2, ce-pro-3}
    }

}
%     \disciplina{sem2}{
    \titulo      {8}{Seminários 2}
    \objetivo    {Assegurar a formação de profissionais dotados de conhecimento das questões sociais, profissionais, legais, éticas, políticas e humanísticas; da compreensão do impacto da computação e suas tecnologias na sociedade; de utilizar racionalmente os recursos disponíveis de forma transdisciplinar; de competências para entender a dinâmica social segundo as perspectivas econômicas e assumir decisões que levem em conta tais perspectivas; e de capacidade para atuar segundo as tendências profissionais atuais.}
    \requisitos  {Seminários 1}
    \recomendadas{N/A}
    \ementa      {A disciplina será baseada em seminários dos mais diversos temas, de acordo com o seu objetivo. Os temas a seguir serão obrigatoriamente abordados em todas as ofertas da disciplina. Outros temas serão apresentados aos estudantes de acordo com as necessidades mais prementes associadas ao perfil do egresso:
        \begin{itemize}
            \itemsep0em
            \item A dinâmica do progresso social e econômico a partir da perspectiva tecnológica;
            \item O mercado e sua influência nas estratégias empresariais;
            \item Estratégias econômicas no contexto da inovação;
            \item Estratégias viáveis para ampliar características sustentáveis em projetos e em ciclos produtivos;
            \item Os componentes formadores do empreendedorismo;
            \item Segurança no trabalho;
            \item Cultura africana e indígena na formação da sociedade brasileira;
            \item Meio ambiente como fundamento necessário para a qualidade de vida;
            \item Estratégias para minimização das desigualdades sociais no Brasil.
        \end{itemize}
    }
    \creditos    {2 total (2 práticos)}
    %    \extra      {0 horas}
    \codigo      {DC}{1001369}
    \bibliografia {
        TIAGO SEVERINO (ORG.). Desenvolvimento social integrado: uma análise a partir da produção cultural, da tecnologia da informação e da saúde. Rio de Janeiro: Letra e Imagem, 2013. 238 p. ISBN 978-85-61012-13-7.

        CASTRO, Antonio Barros De. Estrategias empresariais na industria brasileira: discutindo mudancas. Rio de Janeiro: Forense Universitaria, c1996. 288 p. ISBN 85-218-0172-6.

        FLEURY, Afonso Carlos Correa; FLEURY, Maria Tereza Leme. Estratégias empresariais e formação de competências: um quebra-cabeça caleidoscópico da indústria brasileira. 3. ed. São Paulo: Atlas, 2004. ISBN 85-224-3807-2.

        RAMAL, Silvina Ana. Como transformar seu talento em um negócio de sucesso: gestão de negócios para pequenos empreendimentos. Rio de Janeiro: Elsevier, c2006. 196 p. ISBN 85-352-2111-5.

        SATO, Michele; SANTOS, José Eduardo dos. Agenda 21: em sinopse. São Carlos, SP: EdUFSCar, 1999. 60 p. ISBN 85-85173-39-4.

        COUTO, Hudson de Araujo. Ergonomia aplicada ao trabalho: o manual tecnico da maquina humana. Belo Horizonte: Ergo, 1995. 353 p.

        ALFABETIZAÇÃO ecológica: a educação das crianças para um mundo sustentável. São Paulo: Cultrix, 2006. ISBN 9788531609602.

        A MATRIZ africana no mundo. São Paulo: Selo Negro, 2008.  (Sankofa Matrizes Africanas da Cultura Brasileira 1). ISBN 978-85-87478-32-0.

        BRASIL. MINISTÉRIO DA EDUCAÇÃO. SECRETARIA DA EDUCAÇÃO CONTINUADA, ALFABETIZAÇÃO E DIVERSIDADE. Orientações e ações para a educação das relações étnico-raciais. Brasília: SECAD, 2006. ISBN 85-296-0042-8.
    }
    {
        VESENTINI, Jose William; VLACH, Vania Rubia Farias. Geografia critica: geografia do mundo industrializado. 5. ed. Sao Paulo: Atica, 1994. 190 p. ISBN 85-08-04665-0.

    WICK, Calhoun W.; LEÓN, Lu Stanton. O desafio do aprendizado: como fazer sua empresa estar sempre à frente do mercado. São Paulo: Nobel, 1997. 222 p. ISBN 85-312-0902-3.

    MELLO NETO, Francisco Paulo de; FROES, Cesar. Empreendedorismo social: a transição para a sociedade sustentável. Rio de Janeiro: Qualitymark, 2002. 208 p. ISBN 208857303372X.

    ANDRADE, Renato Fonseca de. Conexões empreendedoras: entenda por que você precisa usar as redes sociais para se destacar no mercado e alcançar resultados. São Paulo: Gente, 2010. 129 p. ISBN 978-85-7312-701-0.

    INSTITUTO SOCIOAMBIENTAL. Almanaque Brasil socioambiental: uma nova perspectiva para entender o pais e melhorar nossa qualidade de vida. Sao Paulo: ISA, 2005. 479 p. ISBN 85-85994-30-4.

    WISNER, Alain. A inteligencia no trabalho: textos selecionados de ergonomia. Sao Paulo: FUNDACENTRO, 2003.

    NEIMAN, Zysman; MOTTA, Cristiane Pires Da. O ambiente construido. Sao Paulo: Atual, [s.d.]. 58 p. (Educacao Ambiental). ISBN 85-7056-371-X.

    BRASIL. MINISTÉRIO DA EDUCAÇÃO. SECRETARIA DA EDUCAÇÃO CONTINUADA, ALFABETIZAÇÃO E DIVERSIDADE. Orientações e ações para a educação das relações étnico-raciais. Brasília: SECAD, 2006. ISBN 85-296-0042-8.
    }
    % Fredy Valente 13/03/2023
    \dataatualizacao{06/11/23} % Kelen, Luciano, Fedy, Alexandre, Kato, Helio, Jander, Menotti, Orides      
    \competencias
    {
        cg-aprender/{ce-ap-1, ce-ap-2},
        cg-atuar/{ce-atuar-3, ce-atuar-4},
        cg-pautar/{ce-paut-1, ce-paut-2, ce-paut-3}
    }
}
%     \input{shared/8/8estagio.tex}
%     \disciplina{tcc1}{
    \titulo      {9}{Trabalho de Conclusão de Curso 1}
    \objetivo    {Contribuição pessoal do estudante para a sistematização do conhecimento em Engenharia de Computação apresentando uma contribuição para o desenvolvimento tecnológico da Computação.}
    \requisitos  {Metodologia Científica} % xxxxxxx
    \recomendadas{N/A}
    \ementa      {Elaboração de um projeto para o trabalho de conclusão de curso sob a orientação de um docente.} %TODO: 
    \creditos    {2 total (2 teóricos)}
    %    \extra       {2 horas}
    \codigo      {DC}{1001506}
    \bibliografia {
        KNUTH, Donald Ervin. The art computer programming. 3. ed. Reading: Addison - Wesley, 1997. 650 p. ISBN 0-201-89683-4.

        DIJKSTRA, Edsger Wybe; FEIJEN, W.h.j. A method of programming. Wokingham: Addison-Weley, 1988. 188 p.

        SOUZA, Marco Antonio Furlan de. Algoritmos e lógica de programação: um texto introdutório para engenharia. 2.ed. São Paulo: Cengage Learning, 2014. 234 p. ISBN 9788522111299.
    }{
        PARRA FILHO, Domingos; SANTOS, João Almeida. Apresentação de trabalhos científicos: monografia, TCC, teses, dissertações. 5. ed. São Paulo: Ed. Futura, 2000. 140 p. ISBN 85-7413-027-3.

        VOLPATO, Gilson Luiz. Bases teóricas para redação científica: ...por que seu artigo foi negado? São Paulo: Cultura Acadêmica, 2010. 125 p. ISBN 978-85-98605-15-9.

        CASTRO, Cláudio de Moura. A prática da pesquisa. 2. ed. São Paulo: Pearson, 2014. 190 p. ISBN 9788576050858.

        BASTOS, Cleverson Leite; KELLER, Vicente. Aprendendo a aprender: introducao a metodologia cientifica. 11. ed. Petropolis: Vozes, 1998. 104 p. ISBN 8532605869.

        BOOTH, Wayne C.; COLOMB, Gregory G.; WILLIAMS, Joseph M. A arte da pesquisa. 2. ed. Sao Paulo: Martins Fontes, 2005. 351 p. (Colecao Ferramentas). ISBN 85-336-2157-4.

        Bibliografia complementar de acordo com o projeto estabelecido junto ao orientador.
    }
    
    % Fredy Valente 13/03/2023
    \dataatualizacao{06/11/23} % Kelen, Luciano, Fedy, Alexandre, Kato, Helio, Jander, Menotti, Orides    
    \competencias{
        %cg-aprender/{ce-ap-1, ce-ap-2, ce-ap-3, ce-ap-4},
        %cg-produzir/{ce-pro-1, ce-pro-2, ce-pro-4, ce-pro-5},
        %cg-empreender/{ce-emp-1, ce-emp-2},
        cg-aprender/{ce-ap-1, ce-ap-2},
        cg-produzir/{ce-pro-1, ce-pro-2, ce-pro-5},
        cg-atuar/{ce-atuar-1, ce-atuar-5}       
    }
}

\disciplina{tcc2}{
    \titulo      {10}{Trabalho de Conclusão de Curso 2}
    \objetivo    {Contribuição pessoal do estudante para a sistematização do conhecimento em Engenharia de Computação apresentando uma contribuição para o desenvolvimento tecnológico da Computação.}
    \requisitos  {Trabalho de Conclusão de Curso 1} % xxxxx
    \recomendadas{N/A}
    \ementa      {Desenvolvimento e apresentação do trabalho de conclusão de curso sob a orientação de um docente.} %TODO: 
    \creditos    {6 total (6 práticos)}
    %    \extra       {2 horas}
    \codigo      {DC}{1001485}
    \bibliografia {
        KNUTH, Donald Ervin. The art computer programming. 3. ed. Reading: Addison - Wesley, 1997. 650 p. ISBN 0-201-89683-4.

        DIJKSTRA, Edsger Wybe; FEIJEN, W.h.j. A method of programming. Wokingham: Addison-Weley, 1988. 188 p.

        SOUZA, Marco Antonio Furlan de. Algoritmos e lógica de programação: um texto introdutório para engenharia. 2.ed. São Paulo: Cengage Learning, 2014. 234 p. ISBN 9788522111299.

    }{
        PARRA FILHO, Domingos; SANTOS, João Almeida. Apresentação de trabalhos científicos: monografia, TCC, teses, dissertações. 5. ed. São Paulo: Ed. Futura, 2000. 140 p. ISBN 85-7413-027-3.

        VOLPATO, Gilson Luiz. Bases teóricas para redação científica: ...por que seu artigo foi negado? São Paulo: Cultura Acadêmica, 2010. 125 p. ISBN 978-85-98605-15-9.

        CASTRO, Cláudio de Moura. A prática da pesquisa. 2. ed. São Paulo: Pearson, 2014. 190 p. ISBN 9788576050858.

        BASTOS, Cleverson Leite; KELLER, Vicente. Aprendendo a aprender: introducao a metodologia cientifica. 11. ed. Petropolis: Vozes, 1998. 104 p. ISBN 8532605869.

        BOOTH, Wayne C.; COLOMB, Gregory G.; WILLIAMS, Joseph M. A arte da pesquisa. 2. ed. Sao Paulo: Martins Fontes, 2005. 351 p. (Colecao Ferramentas). ISBN 85-336-2157-4.

        Bibliografia complementar de acordo com o projeto estabelecido junto ao orientador.
    }
          % Fredy Valente 13/03/2023
    \dataatualizacao{12/12/23} % Luciano  
    \competencias{
        % Fredy Valente 13/03/2023
        %cg-aprender/{ce-ap-1, ce-ap-2, ce-ap-3, ce-ap-4},
        %cg-produzir/{ce-pro-1, ce-pro-2, ce-pro-4, ce-pro-5},
        %cg-empreender/{ce-emp-1, ce-emp-2},
        cg-aprender/{ce-ap-2},
        cg-produzir/{ce-pro-1, ce-pro-2, ce-pro-5},
        cg-atuar/{ce-atuar-1, ce-atuar-5}       
    }
}
% \section{Eixo:Optativas} % TODO: incluir as disciplinas no BCC e as nossas
% TODO: resolver se haverá agrupamentos (humanas, profissionalizantes, eletivas)
%     \disciplina{empreend}{
    \titulo      {7-9}{Empreendedores em Informática}
    \objetivo    {Desenvolver a capacidade empreendedora dos estudantes, estimulando e oferecendo ferramentas àqueles cuja vocação e/ou vontade profissional estiver direcionada à geração de negócios. Estimular os estudantes a desenvolver postura empreendedora; levar cada estudante a elaborar o planejamento de um negócio como trabalho acadêmico da disciplina; motivar os estudantes a desenvolver empreendimentos no decorrer de sua formação acadêmica, de modo a enriquecê-la.}
    % \requisitos  {XX.XXX-X} % xxxxx
    \requisitos{N/A}
    \recomendadas{N/A}
    \ementa      {Postura empreendedora. Teoria visionária. Inovação. Processo de desenvolvimento de negócios. Princípios do Reconhecimento de Oportunidades e de Modelagem de Negócios. Prototipação Rápida / Canvas. Validação de Soluções. Financiamento de negócios tecnológicos. Planos de negócios. Tópicos em negócios: propriedade intelectual, marketing, planejamento financeiro. Elaboração de planos de negócios pelos estudantes. Orientação à elaboração de planos de negócios.}
    \creditos    {4 total (4 teóricos)}
    %    \extra       {x horas}
    \codigo      {DC}{02.709-0}
    \bibliografia {
        FERRARI, Roberto. Empreendedorismo para computação: criando negócios em tecnologia. Rio de Janeiro: Elsevier, 2010. 164 p. (Série SBC). ISBN 978-85-352-3417-6. No SIBI UFSCar: B 658.421 F375e (BCo). Download PDF do Science Direct (gratuito de dentro da UFSCar ou com Proxy): \url{http://www.sciencedirect.com/science/book/9788535234176.}

        Afonso Cozzi, Valéria Judice, Fernando Dolabela, Louis Jacques Filion (orgs); EMPREENDEDORISMO de base tecnológica: spin-off: criação de novos negócios a partir de empresas constituídas, universidades e centros de pesquisa. Rio de Janeiro: Elsevier, 2008. 138 p. ISBN 978-85-352-2668-3. No SIBI UFSCar: B 658.11 E55b (BCo).

        SARKAR, Soumodip. O empreendedor inovador: faça diferente e conquiste seus espaço no mercado. Rio de Janeiro: Elsevier : Campus, 2008. 265 p. : il., grafs., tabs. ISBN 9788535230857.No SIBI UFSCar: 658.421 S245e (B-So).
    }{
        VALERIO NETTO, Antonio. Gestão das pequenas e médias empresas de base tecnológica. Barueri: Minha editora, 2006. 236 p. ISBN 85-98416-31-2. No SIBI UFSCar: B 658.022 V164g (BCo).

        DORNELAS, José Carlos Assis. Empreendedorismo: transformando ideias em negócios. 4. ed. Rio de Janeiro: Elsevier, 2012. 260 p. ISBN 978-85-352-4758-9. No SIBI UFSCar: B 658.421 D713e.4 (BCo).

        ELISABETH, Sandra; CALADO, Robisom D. Transformando ideias em negócios lucrativos: aplicando a metodologia Lean Startup. Rockville: Global South, 2015. ISBN 9781943350070. no SIBI UFSCar: 658.11 E43t (B-So).

        RIES, Eric. A startup enxuta: como os empreendedores atuais utilizam a inovação continua para criar empresas extremamente bem-sucedidas. São Paulo: Leya, 2012. 274 p. ISBN 978-85-8178-004-7. No SIBI UFSCar: G 658.421 R559s (BCo)
    }
   \dataatualizacao{12/12/23} % Luciano  
    \competencias{
        cg-empreender/{ce-emp-1, ce-emp-2, ce-emp-3, ce-emp-4, ce-emp-5},
        cg-atuar/{ce-atuar-1, ce-atuar-4, ce-atuar-5}
    }
}
%     \disciplina{robotica}{
    \titulo      {7-9}{Manipuladores Robóticos}
    \objetivo    {O objetivo da disciplina é capacitar os estudantes para entender e desenvolver modelos e simulação de sistemas robóticos utilizando conceitos que representam o estado da arte. Mesmo que o material se aplique a uma variedade de sistemas robóticos, a aplicação mais natural deste conteúdo se destina a robôs manipuladores operando em ambientes reais. A disciplina se concentra principalmente na mecânica da manipulação, no controle e planejamento de movimentos de sistemas robóticos isolados ou conjuntos. A disciplina serve como base de teórica e prática permitindo que os estudantes participem ativamente em projetos voltados para a área de robótica, automação e sistemas inteligentes. O curso foi projetado para balancear o conteúdo teórico com as suas respectivas aplicações.}
    \requisitos  {Álgebra Linear 1,  Geometria Analítica e Cálculo Diferencial e Integral 1} % xxxxxxx
    \recomendadas{N/A}
    \ementa      {Introdução. Descrições Espaciais e transformações. Cinemática de Manipuladores. Cinemática Inversa de Manipuladores. Cinemática Espacial – Jacobiana. Modelagem Dinâmica. Planejamento de Trajetórias. Projeto programação de manipuladores.}
    \creditos    {4 total (2 teóricos, 2 práticos)}
    %    \extra       {3 horas}
    \codigo      {DC}{1001522}
    \bibliografia {
        MURRAY, Richard M.; LI, Zexiang; SASTRY, S. Shankar. A mathematical introduction to robotic manipulation. Boca Raton, Fla.: CRC Press, c1994. 456 p. ISBN 0-8493-7981-4.

        CRAIG, J.J.; Introduction to Robotics: Mechanics and Control - Addison-Wesley Pub. Co. pp464 3ªedição (ISBN 0201095289) (2005);

        NIKU, Saeed B. Introdução à robótica: análise, controle, aplicações. 2. ed. Rio de Janeiro: LTC, 2013. 382 p. ISBN 9788521622376.

        YOUNG, John F. Robotics. London: Butterworths, 1973. 303 p.
    }{
        MATTHEW T. MASON, Mechanics of Robotic Manipulation, MIT Press August 2001, ISBN-10: 0-262-13396-2 ISBN-13: 978-0-262-13396-8

        LAVALLE, S., Planning Algorithms. Cambridge University Press, 2006.

        THRUN, S., BURGARD, W., e FOX, D., Probabilistic Robotics., The MIT Press (ISBN-10: 0-262-20162-3), 2005}
        
     \competencias{
%        cg-aprender/{ce-ap-3, ce-ap-4},
        cg-produzir/{ce-pro-1, ce-pro-2},
        cg-empreender/{ce-emp-1, ce-emp-2, ce-emp-4}, 
        cg-atuar/{ce-atuar-1, ce-atuar-3, ce-atuar-5},
%       cg-gerenciar/{ce-ger-1, ce-ger-1, ce-ger-3},
%        cg-pautar/{ce-paut-1, ce-paut-3, ce-paut-4}, 
%        cg-buscar/{ce-busc-1, ce-busc-3}
    }
}
%     \disciplina{musical}{
    \titulo      {7-9}{Introdução à Computação Musical}
    \objetivo    {Familiarizar o estudante com a temática da computação musical, abordando as relações formais entre teoria musical, matemática e computação. Habilitar o estudante a compreender, projetar e implementar algoritmos para síntese, análise e processamento de estruturas musicais.}
    \requisitos  {N/A} % % xxx?)
    \recomendadas{N/A}
    \ementa      {Apresentação das relações entre matemática, computação e música. Introdução à matemática do tom puro; parâmetros físicos do som: frequência, amplitude e fase; parâmetros perceptuais do som: intensidade (loudness), altura (pitch) e timbre (envoltória da onda); o tom complexo: harmônicos e formantes; representação digital da informação sonora: amostragem, pseudonímia (aliasing), formatos de arquivos de áudio e algoritmos de compressão. O padrão MIDI. Síntese de sons: ondas fixas, granular, aditiva, subtrativa e técnicas não lineares; análise de sons: decomposição em frequências (análise de Fourier), ruído e filtros lineares digitais; linguagens e ambientes de programação para computação musical; composição algorítmica.}
    \creditos    {4 total (2 teóricos, 2 práticos)}
    %    \extra       {x horas}
    \codigo      {DC}{1001499}
    \bibliografia {
        Roederer, J. G. "Introdução à física e a psicofísica da música". São Paulo: EdUSP, 1998.

        Loy, G. "Musimathics: the mathematical foundations of music". Vol. 1, 2. Cambridge, MA: The MIT Press, c2006.

        Beauchamp, J. W. "Analysis, synthesis, and perception of musical sounds: the sound of music". New York: Springer Science, c2007. Modern Acoustic and Signal Processing.
    }{
        Moore, F. R. "Elements of Computer Music". Upper Saddle River, NJ: Prentice Hall, 1990.

        Road, C. "The Computer Music Tutorial". Cambridge, MA: The MIT Press, 1996.

        Rowe, R. "Machine Musicianship". Cambridge, MA: The MIT Press, 2001.
    }
    \competencias{
    % Inserido por Murillo R. P. Homem, em 08/03/2023
         cg-aprender/{ce-ap-1, ce-ap-2, ce-ap-3, ce-ap-4},
         cg-produzir/{ce-pro-1, ce-pro-2, ce-pro-3, ce-pro-4},
         cg-atuar/{ce-atuar-1, ce-atuar-2, ce-atuar-3, ce-atuar-4}
         }
}

%     \disciplina{sc}{
    \titulo      {7-9}{Segurança Cibernética}
    \objetivo    {Gerar capacitação para entender, analisar e projetar técnicas de exploração de falhas de segurança de sistemas cibernéticos. Gerar competências para abordagem e proteção de sistemas computacionais, utilizando técnicas de exploração mais comuns. Capacitar para projeto e análise de sistemas computacionais seguros.}
    \requisitos  {Sistemas Operacionais e Arquitetura e Organização de Computadores 1} % xxxxx
    \recomendadas{N/A}
    \ementa      {Introdução à segurança de sistemas. Arquiteturas para segurança: segurança para aplicativos, sistemas operacionais e códigos legados, isolamento, controle de acesso. Criptografia: encriptação, identificação, autenticação, integridade, não repudiação, infraestrutura de chaves públicas (PKI). Segurança web: modelo de segurança de serviços e de navegadores web; Vulnerabilidades comuns (Top OWASP), tais como: SQL Injection, XSS e CSRF. Segurança de software: compilação e semântica de execução, ataques de controle de fluxo, defesas contra ataques de controle de fluxo, ROP, integridade de controle de fluxo (CFI). Segurança de rede: monitoramento, detecção de intrusão (IDS) e arquitetura de redes seguras. Tópicos avançados, tais como: segurança de aplicações móveis, módulos SAM, SIM, JavaCard e Contactless Smart Cards, e-Wallets, EMV e sistemas de bilhetagem eletrônica.}
    \creditos    {4 total (2 teóricos, 2 práticos)}
    %    \extra       {x horas}
    \codigo      {DC}{1001520}
    \bibliografia {
        STALLINGS, William. Criptografia e segurança de redes: princípios e práticas. 4. ed. São Paulo: Pearson, 2010. ISBN 9788576051190.

        ERICKSON, Jon Mark. Hacking: the art of exploration. San Frascisco: No Starch Press, 2003. ISBN 1-59327-007-0.

        NICHOLS, Randall K. ICSA guide to cryptography. New York: McGraw-Hill, 1999. ISBN 0-07-913759-8.

        NORTHCUTT, S.; NOVAK, J.; MCLACHLAN, D.; Network instrusion detection: an analyst's handbook. 2. ed. Indianapolis: New Riders, 2000. ISBN 0-7357-1008-2.

        STAJANO, Frank. Security for ubiquitous computing. Chichester: John Wiley \& Sons, c2002. ISBN 0-470-84493-0.
    }{
        Cyber Security Engineering – A Practical Approach for Systems and Software Assurance, Nancy R. Mead \& Carol C. Woody, Addison-Wesley, 2017.

        Applied Cryptography, Protocols, Algorithms and Source Code in C, Bruce Schneier, Published by John Wiley and Sons, 1996 – Reprinted in 2016.

        Introduction to Modern Cryptography, Johnatan Katz, CRC Press 2015.

        Criptografia Essencial, A Jornada do Criptógrafo, Sean Michael Wykes, Elsevier, 2016.

        Criptografia e Segurança de Redes, William Stallings, Editora Pearson, 2007

        Applied Cryptography for Cyber Security and Defense: Information Encryption and Cyphering, Hamid R. Nemati and Li Yang, Premier Reference Source, 2010.
    }
    % Fredy Valente 06/03/2023
    \dataatualizacao{06/11/23} % Kelen, Luciano, Fedy, Alexandre, Kato, Helio, Jander, Menotti, Orides
    \competencias
    {
        cg-aprender/{ce-ap-1, ce-ap-2, ce-ap-4},
        cg-atuar/{ce-atuar-1, ce-atuar-2, ce-atuar-3, ce-atuar-4}
    }
}
%     \input{shared/4/4tc.tex}


%%%%%%%%%%%%%%%%%%%%%%%%%%%%%%%%%%%%%%%%%%%%%%%%%%%%%%%%%%%%%%%%%%%


% \chapter{Convênios}
% \section{Convênios com Empresas/Organizações}
% O curso pode proporcionar treinamento em tecnologias/empresas em cooperação com Empresas que se dispõe em colaborar com o mesmo. A regulamentação dos convênios segue as normas estabelecidas na Pró-Reitoria de Graduação da UFSCar.

