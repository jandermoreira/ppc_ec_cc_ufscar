%! Author = Jander Moreira
%! Date = 27/02/2023

\section {Infraestrutura Necessária para o Funcionamento do Curso}

Para a execucão deste Projeto Pedagógico, o curso usufrui da infraestrutura e compromisso com a qualidade na formação da UFSCar. De maneira específica, apresenta-se aqui a qualificação de parte do corpo docente e técnico-administrativo atuante no curso, bem como de parte das instalações utilizadas.

\subsection{Corpo Docente e Técnico}

O Curso de Engenharia de Computação é atendido principalmente pelo Departamento de Computação (DC), que conta atualmente com 43 docentes em tempo integral e dedicação exclusiva listados no Quadro~\ref{table:CorpoCocente}.

%\begin{table}[!htb]
%\centering
\begin{singlespace}
    \begin{longtable}{lcc}
        \caption{Corpo docente atuante no curso de Engenharia de Computação.}
        \label{table:CorpoCocente} \\
        \sline
        \textbf{Nome}                      & \textbf{Titulação} & \textbf{Vínculo/Dedicação} \\ \hline
        Alexandre Luis Magalhães Levada    & Doutor             & Efetivo/40h DE             \\
        Auri Marcelo Rizzo Vincenzi        & Doutor             & Efetivo/40h DE             \\
        Cesar Augusto Camilo Teixeira      & Doutor             & Efetivo/40h DE             \\
        Cesar Henrique Comin               & Doutor             & Efetivo/40h DE             \\
        Daniel Lucrédio                    & Doutor             & Efetivo/40h DE             \\
        Delano Medeiros Beder              & Doutor             & Efetivo/40h DE             \\
        Diego Furtado Silva                & Doutor             & Efetivo/40h DE             \\
        Ednaldo Brigante Pizzolato         & Doutor             & Efetivo/40h DE             \\
        Edílson Reis Rodrigues Kato        & Doutor             & Efetivo/40h DE             \\
        Emerson Carlos Pedrino             & Doutor             & Efetivo/40h DE             \\
        Estevan Rafael Hruschka Junior     & Doutor             & Efetivo/40h DE             \\
        Fabiano Cutigi Ferrari             & Doutor             & Efetivo/40h DE             \\
        Fredy João Valente                 & Doutor             & Efetivo/40h DE             \\
        Helena de Medeiros Caseli          & Doutor             & Efetivo/40h DE             \\
        Helio Crestana Guardia             & Doutor             & Efetivo/40h DE             \\
        Heloisa de Arruda Camargo          & Doutor             & Efetivo/40h DE             \\
        Hermes Senger                      & Doutor             & Efetivo/40h DE             \\
        Jander Moreira                     & Doutor             & Efetivo/40h DE             \\
        Joice Lee Otsuka                   & Doutor             & Efetivo/40h DE             \\
        Júnia Coutinho Anacleto Silva      & Doutor             & Efetivo/40h DE             \\
        Kelen Cristiane Teixeira Vivaldini & Doutor             & Efetivo/40h DE             \\
        Luciano de Oliveira Neris          & Doutor             & Efetivo/40h DE             \\
        Marcela Xavier Ribeiro             & Doutor             & Efetivo/40h DE             \\
        Marcio Merino Fernandes            & Doutor             & Efetivo/40h DE             \\
        Mário César San Felice             & Doutor             & Efetivo/40h DE             \\
        Mauricio Fernandes Figueiredo      & Doutor             & Efetivo/40h DE             \\
        Marilde Terezinha Prado Santos     & Doutor             & Efetivo/40h DE             \\
        Murilo Coelho Naldi                & Doutor             & Efetivo/40h DE             \\
        Murillo Rodrigo Petrucelli Homem   & Doutor             & Efetivo/40h DE             \\
        Orides Morandin Junior             & Doutor             & Efetivo/40h DE             \\
        Paulo Rogerio Politano             & Doutor             & Efetivo/40h DE             \\
        Paulo Matias                       & Doutor             & Efetivo/40h DE             \\
        Renato Bueno                       & Doutor             & Efetivo/40h DE             \\
        Ricardo Cerri                      & Doutor             & Efetivo/40h DE             \\
        Ricardo José Ferrari               & Doutor             & Efetivo/40h DE             \\
        Ricardo Menotti                    & Doutor             & Efetivo/40h DE             \\
        Ricardo Rodrigues Ciferri          & Doutor             & Efetivo/40h DE             \\
        Roberto Ferrari Junior             & Doutor             & Efetivo/40h DE             \\
        Sandra Abib                        & Doutor             & Efetivo/40h DE             \\
        Sergio Donizetti Zorzo             & Doutor             & Efetivo/40h DE             \\
        Valter Vieira de Camargo           & Doutor             & Efetivo/40h DE             \\
        Vânia Paula de Almeida Neris       & Doutor             & Efetivo/40h DE             \\
        Wanderley Lopes de Souza           & Doutor             & Efetivo/40h DE             \\ \sline
    \end{longtable}
    %\end{table}
\end{singlespace}

O corpo técnico administrativo, listado no Quadro \ref{table:CorpoTecnico},  é composto por 8 servidores lotados no DC.

\begin{table}[h!]
    \centering
    \caption{Corpo técnico administrativo atuante no curso de Bacharelado em Engenharia de Computação.}
    \label{table:CorpoTecnico}
    \begin{tabular}{ll}
        \sline
        \textbf{Nome}               & \textbf{Atividade}                  \\ \hline
        Carlos Alberto Ferro Gobato & Técnico em eletrônica               \\
        Darli José Morcelli         & Assistente administrativo           \\
        Jorgina Vera de Moraes      & Servente de limpeza                 \\
        Mariana Massimino Feres     & Técnica de tecnologia da informação \\
        Paulo Cesar Donizeti Paris  & Técnico de laboratório              \\
        Willian Câmara Corrêa       & Técnico de laboratório              \\
        Ivan Rogério da Silva       & Assistente administrativo           \\
        Nicanor José Costa          & Assistente administrativo           \\ \sline
    \end{tabular}
\end{table}

Os alunos também são atendidos por docentes dos departamentos de Matemática, Estatística, Física etc, bem como por outros técnicos-administrativos lotados em diferentes setores da universidade.

\subsection{Espaço físico}

O Departamento de Computação dispõe de seis laboratórios de ensino para graduação e um laboratório de informática para graduação (LIG). Listados no Quadro \ref{table:Labs}, dois desses laboratórios de ensino são equipados para o ensino e a prática de experiências relativas às disciplinas dos eixos Engenharias e Sistemas e Arquitetura de Computadores. Os demais são específicos para o ensino e a prática de programação e demais atividades relacionadas à Metodologia e Técnicas da Computação, sendo utilizados para aulas práticas e também para a realização dos trabalhos por parte dos estudantes. O Departamento de Computação também conta com um auditório para 80 pessoas.

\begin{table}[h!]
    \centering
    \caption{Laboratórios do DC voltados para o ensino da graduação.}
    \label{table:Labs}
    \begin{tabular}{lll}
        \sline
        \textbf{Laboratório}       & \textbf{Atividade principal}       & \textbf{Capacidade} \\ \hline
        Laboratório de ensino 1    & \textit{Hardware}                  & 30 estudantes       \\
        Laboratório de ensino 2    & Programação e desenvolvimento      & 40 estudantes       \\
        Laboratório de ensino 3    & Programação e desenvolvimento      & 40 estudantes       \\
        Laboratório de ensino 4    & Programação e desenvolvimento      & 40 estudantes       \\
        Laboratório de ensino 5    & \textit{Hardware} e Lógica digital & 30 estudantes       \\
        Laboratório de ensino 6    & Programação e desenvolvimento      & 40 estudantes       \\
        Laboratório de Informática & Uso geral                          & 40 estudantes       \\ \sline
    \end{tabular}
\end{table}

\subsection{Infraestrutura de apoio}

Como infraestrutura de apoio, nos laboratórios temos atualmente a seguintes configuração dos microcomputadores presentes nos laboratórios de ensino (Quadro~\ref{table:LabsSetup}).

\begin{table}[h!]
    \centering
    \caption{Configuração dos Laboratórios do DC voltados para o ensino da graduação.}
    \label{table:LabsSetup}
    \begin{tabular}{llll}
        \sline
        \textbf{Laboratório}    & \textbf{Processador}          & \textbf{Memória} & \textbf{HD}      \\ \hline
        Laboratório de ensino 1 & Intel Core I7 4790 - 3.6 GHz  & 8 GB             & HD Sata 1 T      \\
        Laboratório de ensino 2 & Intel Core I5 6500 - 3.2 GHz  & 8 GB             & HD Sata 500 G    \\
        Laboratório de ensino 3 & Intel Core I5 6600 - 3.33 GHz & 8 GB             & 2 HDs Sata 500 G \\
        Laboratório de ensino 4 & Intel Core I7 4790 - 3.6 GHz  & 16 GB            & HD Sata 1 T      \\
        Laboratório de ensino 5 & Intel Core I7 4790 - 3.6 GHz  & 8 GB             & HD Sata 1 T      \\
        Laboratório de ensino 6 & Intel Core 2 Quad Q8200       & 4 GB             & HD Sata 250 G    \\
        & - 2.3GHz                      &                  & HD 320 G         \\  \sline
    \end{tabular}
\end{table}

Todos os laboratórios de ensino são específicos para aulas e não são utilizados para pesquisas. Esses laboratórios possuem microcomputadores, projetor multimídia e ar-condicionado. As demandas de infraestrutura para as disciplinas de Física para o curso de Engenharia de Computação são supridas pelos Laboratórios de Física Experimental A e B do Departamento de Física.

Para as aulas experimentais em \textit{hardware} é previsto um máximo de 20 (vinte) estudantes por turma. Os laboratórios atendem aos cursos de Bacharelado em Engenharia de Computação, Bacharelado em Ciências da Computação e Bacharelado em Engenharia Física. Os Laboratórios de Ensino 1 e 5 possuem os itens necessários às atividades práticas. Os itens presentes nesses laboratórios estão listados nos Quadros \ref{table:LE1} e  \ref{table:LE5}.

\begin{table}[h!]
    \centering
    \caption{Equipamentos presentes no Laboratório de Ensino 1.}
    \label{table:LE1}
    \begin{tabular}{cl}
        \sline
        \textbf{Quantidade} & \textbf{Item}                                   \\ \hline
        5                   & Osciloscópio Tektronix 100 MHz TDS 1012C        \\
        4                   & Osciloscópio Tektronix 2225 50 MHz              \\
        3                   & Osciloscópio Minipa MO-1250S                    \\
        11                  & Fonte de alimentação simétrica Minipa MPC 3003D \\
        12                  & Gerador de função Minipa MFG-4200               \\
        6                   & Kit DVC25 Tes equipamentos                      \\
        5                   & Kit Arduino                                     \\
        19                  & Kit Intel Galileo                               \\
        10                  & Kit Grove Intel IOT edition                     \\ \sline
    \end{tabular}
\end{table}


\begin{table}[!h]
    \centering
    \caption{Equipamentos presentes no Laboratório de Ensino 5.}
    \label{table:LE5}
    \begin{tabular}{cl}
        \sline
        \textbf{Quantidade} & \textbf{Item}                                      \\ \hline
        10                  & Kit FPGA DE1 Altera                                \\
        1                   & Fonte de alimentação simétrica Instrutherm FA-3050 \\
        5                   & Gerador de função Politerm VC 2002                 \\
        2                   & Gerador de função Instrutherm GF 220               \\
        5                   & Osciloscópio digital Instrutherm 70 MHz            \\
        4                   & Osciloscópio Minipa MO-125 50 MHz                  \\         \sline
    \end{tabular}
\end{table}

A UFSCar oferece o Ambiente Virtual de Aprendizagem (AVA) Moodle que é utilizado por professores e alunos em diferentes níveis de aplicação, sendo uma ferramenta para gerenciamento de cursos utilizada para cobrir três eixos básicos do processo de ensino-aprendizagem:
\begin{itemize}
    \item Gerenciamento de conteúdos: organização de conteúdos a serem disponibilizados aos estudantes no contexto de disciplinas/turmas;
    \item Interação entre usuários: diversas ferramentas para interação com e entre estudantes e professores: fórum, bate-papo, mensagens, etc.
    \item Acompanhamento e avaliação: definição, recepção e avaliação de tarefas, questionários e enquetes, atribuição de notas, cálculo de médias, etc.
\end{itemize}

%%%%%%%%%%%%%%%%%%%%%%%%%%%%%%%%%%%%%%%%%%%%%%%%%%%%%%%%%%%%%%%%%%%%
% \chapter{Gerenciamento do Curso}~\label{cha:gerenciamento}


% \section{Composição e Funcionamento do Conselho do Curso}

% O Curso de Engenharia de Computação, assim como todos os demais cursos da Universidade Federal de São Carlos tem sua administração acadêmica regulamentada pela Portaria GR nº 662/03 (Regulamento Geral das Coordenações de Cursos de Graduação da UFSCar), que estabelece em seus Artigos 1º e 2º:


% \begin{itemize}
% \item Art. 1º - A Coordenação de Curso, prevista no Art. 43 do Estatuto da UFSCar, é um órgão colegiado responsável pela organização didática e pelo funcionamento de um determinado curso, do qual recebe a denominação.

% \item Art. 2º - As Coordenações de Curso de Graduação serão constituídas por:
% \begin{itemize}
% \item[] I- Coordenador;
% \item[] II- Vice-Coordenador;
% \item[] III- Conselho de Coordenação. (Cf. 1)
% \end{itemize}

% \end{itemize}


% \section{Núcleo Docente Estruturante}

% O Núcleo Docente Estruturante (NDE), instituído pela Resolução CoG nº 035 de 08 de novembro de 2010, é composto por docentes que participaram da criação do curso, por docentes que foram coordenadores do curso, por docentes dos departamentos que oferecem disciplinas constituintes da matriz curricular do curso, neste caso, coincidindo com os representantes departamentais do Conselho de Coordenação de Curso, ou seja, Art. 4º. O Núcleo Docente Estruturante será constituído:

% \begin{itemize}
% \item[] I- Pelo Coordenador do Curso;
% \item[] II- Por um mínimo de cinco professores pertencentes ao corpo docente do curso há pelo menos dois anos, salvo em caso de cursos novos.
% \end{itemize}

% § 1º. A indicação dos representantes de que trata o caput deste artigo será feita pelo Conselho de Coordenação do Curso, para um mandato de dois anos.

% § 2º. A renovação do NDE será feita de forma parcial, garantindo-se a permanência de pelo menos 50\% de seus membros em cada ciclo avaliativo do Sistema Nacional de Avaliação da Educação Superior (SINAES) (...). (Cf.2)

% Nesta instância o Projeto pedagógico do Curso será permanentemente avaliado, com base em análise relacionada ao desenvolvimento e consolidação do mesmo.

% \section{Eixos de Formação}

% Conforme abordado no Capítulo~\ref{cha:MarcoConceitual}, o curso possui XX eixos de formação:
% \begin{compenum}
%     \item ...
% \end{compenum}

% \section{Coordenador de Eixo de Formação}

% Cada eixo de formação possui um \textbf{Coordenador de Eixo}, que é um professor do curso responsável por promover a integração das disciplinas e atividades didático-pedagógicas no seu eixo de responsabilidade. O Coordenador de Eixo é membro representante do eixo (área) junto ao NDE. Atribuições e Responsabilidades do Coordenador de Eixo:

% Atividades Interdisciplinares
% \begin{itemize}
% \item Zelar por não replicar conteúdo em disciplinas distintas;
% \item Integrar projetos de diferentes disciplinas;
% \item Integrar docentes e conteúdos de diferentes disciplinas;
% \item Criar um ambiente de harmonia, colaboração e cooperação entre os docentes do eixo;
% \end{itemize}

% Exemplos de atividades de integração de projeto. Disciplinas em semestres consecutivos desenvolver partes de um mesmo projeto. Disciplinas no mesmo semestre podem compartilhar o mesmo projeto e aproveitar a avaliação para as mesmas.

% Atividades Intereixos
% \begin{itemize}
% \item Auxiliar a integração de diferentes eixos;
% \item Indicar formas de incentivo ao desenvolvimento de linhas de pesquisa e extensão;
% \end{itemize}

% As principais atribuições do Coordenador de Eixo são promover incrementos e atividades ligadas a seu eixo de representação para a melhoria do curso como um todo. Cabe também ao Coordenador de Eixo certificar-se que conteúdos, metodologias de ensino, avaliações e carga horária efetiva de cada disciplina estão sendo cumpridos de acordo com o que estabelece o projeto pedagógico. Os coordenadores de eixo deverão trabalhar em conjunto com os demais professores dos respectivos eixos de formação, recomendando-se que ao menos duas reuniões semestrais ocorram: uma de planejamento do semestre a ser iniciado, e outra para avaliação ao final do semestre letivo. O Coordenador de Eixo deverá se reportar semestralmente ao NDE e ao Conselho de Curso.

% \section{Coordenação de Disciplina}

% O Coordenador da Disciplina é um dos docentes da disciplina que possui experiência com o conteúdo da disciplina. O NDE do curso indica para o Conselho de Curso o coordenador de cada disciplina sempre antes do início de cada semestre. O Coordenador da Disciplina é responsável pelas seguintes atividades:

% Atividades Intradisciplinares
% \begin{itemize}
% \item Uniformizar conteúdo das várias turmas de uma disciplina;
% \item Zelar para que as ementas estejam completas, atualizadas e escritas em um nível de detalhe que diminua possíveis variações;
% \item Reportar-se ao Coordenador de Eixos para eventuais alterações que impactem em outras disciplinas.
% \end{itemize}

% É responsabilidade do Coordenador da Disciplina trabalhar para a uniformização dos planos de ensino das  várias turmas de uma mesma disciplina nos itens a seguir:
% \begin{compenum}
%     \item conteúdo programático: itens e carga horária por item;
%     \item bibliografia;
%     \item avaliação.
% \end{compenum}


\section{Administração e Condução do Curso}

O curso de graduação Bacharelado em Engenharia de Computação é formado por professores, servidores técnico-administrativos e alunos e conta com a infra-estrutura disponibilizada pela Pró-reitoria de Graduação da UFSCar e pelas instalações do CCET - Centro de Ciência e Tecnologia da UFSCar.

Para que o curso realize sua missão de formar alunos com excelência, é preciso o empenho mútuo de alunos, docentes e servidores técnico-administrativos (TAs).
É imprescindível que todo docente do curso conheça em profundidade o Projeto Pedagógico e zele pelo seu cumprimento na íntegra. Com essa atitude o docente terá conhecimento dos princípios pedagógicos que regem o curso. Fica a cargo da chefia do Departamento, o estímulo dessa prática dentre seus pares.

O NDE e os Coordenadores de Disciplinas devem trabalhar em conjunto, realizando, obrigatoriamente, o mínimo de uma reunião por semestre. A pauta de convocação da reunião deve ser pública e feita com, no mínimo, 48 horas de antecedência. Fica a critério da Coordenação de Curso estabelecer data e horário para que as reuniões ocorram.

O Coordenador de Disciplina deve reportar-se semestralmente ao NDE sobre suas atividades relativas à sua área de representação. Os dados fornecidos por esses membros devem ficar públicos a todos os envolvidos no curso na forma de ata a ser divulgada em no máximo 15 dias úteis após a realização da reunião.


\section{Processo para Autoavaliação do Curso}

A autoavaliação dos cursos se faz com base no Plano de Desenvolvimento Institucional da UFSCar (PDI/UFSCar), no perfil estabelecido pela UFSCar para o profissional/cidadão a ser formado por todos os cursos, bem como nos princípios e concepções estabelecidos no Regimento Geral dos Cursos de Graduação, instituído pela UFSCar em 2016~\cite{RGCG}.

Desde a publicação da Lei 10.861 de 14 de abril de 2004, que instituiu o Sistema de Avaliação da Educação Superior (SINAES), a UFSCar vem estudando forma para a realização da autoavaliação dos seus cursos e em 2011 a Pró-Reitoria de Graduação implantou uma comissão de avaliação de cursos de graduação chamada de Comissão Própria de Avaliação (CPA), a qual coordena os processos internos de autoavaliação institucional nos moldes propostos pela atual legislação para os processos de avaliação dos cursos.

Segundo o Regimento Geral dos Cursos de Graduação da UFSCar a avaliação das especificidades de cada curso fica sob responsabilidade de sua Coordenação, composta pelo Coordenador do Curso, Conselho de Curso e do Núcleo Docente Estruturante. Especificamente, os Artigos 93, 94 e 98 do Regimento Geral de Cursos de Graduação da UFSCar definem as competências do Conselho de curso, da Coordenação do curso e do Núcleo Docente Estruturante (NDE), respectivamente.

Segundo o Regimento Geral de Cursos da UFSCar, cabe ao Núcleo Docente Estruturante (NDE) de cada curso analisar os resultados das autoavaliações a fim de propor melhorias ao Conselho de Coordenação no sentido do aperfeiçoamento do Projeto Pedagógico de Curso, respeitando os prazos para reformulações curriculares estabelecidos.

A avaliação é realizada por meio de formulários de avaliação, os quais são respondidos pelos docentes da área majoritária de cada curso, pelos discentes e, eventualmente, pelos técnico-administrativos e egressos. Esses formulários abordam questões sobre as dimensões do Perfil Profissional a ser formado pela UFSCar; da formação recebida nos cursos; do estágio supervisionado; da participação em pesquisa, extensão e outras atividades; das condições didático-pedagógicas dos professores; do trabalho das coordenações de curso; do grau de satisfação com o curso realizado; das condições e serviços proporcionados pela UFSCar; e das condições de trabalho para docentes e técnico-administrativos.

A primeira autoavaliação do curso de Engenharia de Computação coordenada pela Comissão Própria de Avaliação (CPA) realizada pela CPA foi em 2014, como relatada no item 2.4 do capítulo 2 (Marco Referencial do Curso). Em~\cite{CPA}, encontra-se a avaliação realizada pela CPA segundo a visão do estudante.

Também deve ser ressaltado que a Coordenação do Curso de Bacharelado em Engenharia de Computação sempre atuou fortemente não apenas na promoção do curso junto à comunidade externa, mas também no acompanhamento dos egressos. O contato contínuo e intenso com os egressos fornece valiosas informações sobre a colocação dos mesmos no mercado de trabalho e provê informações importantes sobre a formação profissional recebida durante o curso e sua efetividade perante o mercado profissional.

A autoavaliação realizada pela CPA, a avaliação das disciplinas/atividade curriculares e o acompanhamento dos egressos em sua colocação no mercado de trabalho, visam, além de uma busca contínua de melhorias do projeto pedagógico, também a sua implantação e execução com sucesso de acordo com as exigências necessárias para um curso de Bacharelado em Engenharia de Computação de qualidade reconhecida.

