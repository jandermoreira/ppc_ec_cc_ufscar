%%%%%%%%%%%%%%%%%%%%%%%%%%%
% Marco Conceitual do Curso

O profissional formado em Engenharia de Computação (EC) deverá possuir a competência (conhecimento, habilidade e atitude) para atuar em diferentes áreas, incluindo Engenharia Elétrica, Engenharia de Automação, Engenharia de Produção, Tecnologia da Informação, Ciências da Computação e Física, entre outras áreas agregadas ao curso de Engenharia de Computação. Desta forma, habilita o egresso ter capacidade de entender, mapear, propor e implementar soluções integradoras que atendam os requisitos do problema através de execução de projeto de engenharia que atenda à demanda.

% "deverá possuir a competência" ? Será que é melhor dizer que "terá a competência para..." ? (helio)
% ^^^-----------Concordo (jander)

% Será que é certo dizer que o EC pode atuar em EE, EA, EP, TI, CC, F, etc.? Será que não é preciso mencionar "em áreas releacionadas a essas formações?  (helio)

\textcolor{red}{No ano de 2000, em um trabalho conjunto elaborado por coordenadores e 
representantes das Comissões de reformulaçao
 Curricular dos Cursos de Graduação e a
Pró-Reitoria de Graduação da UFSCar, foi estabelecido o perfil do profissional a ser formado pelos cursos de graduação estabelecendo-se as competências necessárias a serem adquiridas pelo egresso, documento este que foi revisado em 2008 (UFSCar, 2008). Este documento estabelece as competências gerais a serem adquiridas durante a graduação da UFSCar. Sendo estabelecidos um conjunto de oito competências gerais, descritas a seguir:}
\textcolor{red}{\begin{itemize}
    \item Aprender;
    \item Produzir;
    \item Empreender;
    \item Atuar;
    \item Comprometer;
    \item Gerenciar;
    \item Pautar;
    \item Buscar
\end{itemize}}

\textcolor{red}{Neste documento, do Perfil do Profissional a ser Formado na UFSCar (UFSCar, 2008), também formam definidas as competências específicas do profissional a ser formado, isto é, estabelece o detalhamento das competências gerais, de acordo um uma interpretação geral para todos os cursos de graduação da UFSCar.}

\textcolor{red}{A partir das competências gerais e das competências específicas estabelecidas, o NDE do curso de EC, realizou a revisão das competências específicas, adaptando-o para o curso de Bacharelado em Engenharia de Computação.}

%Assim foi estabelecida uma nova leitura das competências específicas, agora aptadas para o curso de EC, como descrito no anexo A deste PPC.

% Desta forma, o curso se caracteriza pela agregação de conhecimento de diferentes áreas, particularmente de Engenharia Elétrica e Ciências da Computação, o que confere ao profissional a capacidade de analisar, especificar, projetar, implementar, integrar, testar e manter sistemas computacionais completos, modulares, autônomos, integrados, móveis, distribuídos, inteligentes, complexos ou dedicados, baseados em tecnologias de hardware e software que automatizem a execução de processos dos problemas de engenharia de computação.}


\section{Competências Gerais da Formação do Engenheiro de Computação}\label{sec:competencias-gerais-da-formacao-do-engenheiro-de-computacao}

\textcolor{red}{As competências gerais da formação do Engenheiro de Computação deverão tratar dos conhecimentos, das habilidades e da atitude profissional adquiridos pelo egresso durante o curso classificadas de acordo com o Perfil do Profissional a ser Formado na UFSCar (UFSCar, 2018), isto é, Aprender, Produzir, Empreender, Atuar, Comprometer, Gerenciar, Pautar e Buscar. Os conhecimentos devem ser necessários e suficientes para que o egresso possa atuar com segurança e domínio do assunto em todas as atividades relacionadas ao Engenheiro de Computação, enquanto deverá, durante o curso, desenvolver habilidades na busca de soluções de problemas. Deverá também possuir uma compreensão adequada do mundo e da sociedade, levando em consideração aspectos humanísticos, estando capacitado a atuar de forma proativa, ética e profissional para solucionar os problemas relacionados.}
% "... deverão tratar..." ?  ou "tratam, estão relacionadas, ...?  (helio)
\textcolor{red}{Para tal, poderá fazer uso de diferentes tecnologias e estratégias, propondo soluções criativas e inovadoras para a sociedade, implementadas como sistemas computacionais e contribuindo diretamente para o desenvolvimento sustentável do país e para a geração de riqueza, dentro dos princípios da ética profissional.}
% Para tal, o egresso do curso poderá fazer uso... (?) (helio)

O egresso possuirá, assim, o conhecimento e as habilidades para atuar em diferentes indústrias, no ensino, na pesquisa e na extensão ou, ainda, empreender novos negócios.

Adicionalmente, o engenheiro de computação estará capacitado a entender e considerar aspectos de negócios no processo de desenvolvimento, como o gerenciamento de projetos de engenharia de computação, e possuirá a habilidade de adaptação à constante e rápida evolução da área, aprendendo de forma autônoma e contínua. Também estará apto à produção e à divulgação de novos conhecimentos, tecnologias, serviços e produtos.


\section{Competências Específicas da Formação do Engenheiro de Computação}\label{sec:competencias-especificas-da-formacao-do-engenheiro-de-computacao}

O curso de Engenharia de Computação deverá proporcionar ao estudante a capacidade analítica para o entendimento e a resolução de problemas de engenharia de computação, capacidade de interpretação e compreensão de conteúdos de especificações de processos e de tecnologias, e as necessárias competências para o desenvolvimento e condução de projetos. As competências específicas relacionadas a cada uma das competências gerais (UFSCar. 2008) apresentadas irão tratar diretamente da formação do perfil do egresso no mercado de trabalho. Assim, as competências específicas estabelecem o conhecimento, as habilidades e a atitude do egresso para o curso de Bacharelado em Engenharia de Computação e estão distribuídas nas disciplinas oferecidas pelo curso ao longo da graduação do egresso.
% que tal: ".. ao longo da graduação." ? parece estranho falar da graduação do egresso... (h)

\subsection*{Competências específicas para \textsc{Aprender} de forma autônoma e contínua}\label{sec:competencias-especificas-para-aprender-de-forma-autonoma-e-continua}

O curso de Bacharelado em Engenharia de Computação fornecerá ao egresso a competência de aprender relacionada a:

\begin{compitem}
    \item Analisar o problema, processo de execução, entender e aplicar uma metodologia de solução de problemas de engenharia, utilizando técnicas e métodos multidisciplinares da Ciência da Computação e Engenharia Elétrica;
    \item Analisar o desempenho das soluções propostas ou implementadas, através de modelos analíticos, de simulação ou de experimentação;
    \item Analisar documentos de especificação de requisitos para o problema;
    \item Compreender as demandas que devem ser atendidas incluindo os requisitos de usuários, a legislação vigente, as restrições e limites aplicáveis para escolha das tecnologias e soluções;
    \item Compreender especificações técnicas de módulos de software e hardware;
    \item Compreender, analisar e entender normas e padrões documentados para os módulos e tecnologias as serem empregadas na solução;
    \item Aplicar técnicas e restrições para adequação da solução para as normas e padrões técnicos, além das restrições governamentais vigentes no país de uso da solução;
\end{compitem}

\subsection*{Competências específicas para \textsc{Produzir} e divulgar novos conhecimentos, tecnologias e produtos}

Utilizando os conhecimentos e habilidades adquiridas durante o curso de Bacharelado em Engenharia de Computação o egresso será capaz de:

\begin{compitem}
    \item Projetar uma arquitetura de solução prototipada, planejar desenvolvimento de solução definitiva incluindo revisão de módulos de hardware e software e interfaces de integração Inter módulos e com o ambiente destino;
    \item Resolver problemas que demandam conhecimento das tecnologias de automação e controle, para diferentes problemas em diversas áreas e campos de aplicação;
    \item Propor e implementar soluções para problemas que exijam conhecimentos de programação de computadores, conhecimentos matemáticos e físicos dentro dos limites da engenharia;
    \item Propor e implementar soluções que envolvam decisão sobre o design, a estrutura e a arquitetura do software, uso de padrões de projeto, estruturas de desenvolvimento e componentes de software.
    % substituir "design" por projeto? (h)
    \item Propor e implementar soluções para problemas que impliquem no uso de técnicas e programação concorrente, paralelismo, gestão de eventos, comunicação e programação distribuída, controle de execução, manuseio de exceções e erros, sistemas interativos, persistência e coerência de dados;
    \item Propor e implementar soluções para problemas que requeiram o desenvolvimento de software suficientemente complexo para exigir a aplicação de conhecimentos instrumentais às áreas de automação e controle, engenharia de software, e redes de computadores;
    \item Propor, selecionar componentes de hardware e implementar soluções complexas para o sensoriamento e captura de dados com métricas e grandezas, para monitoramento, atuação e controle de ambientes, máquinas, equipamentos, pessoas, objetos e entidades;
    \item Propor, desenvolver e implementar soluções de hardware customizadas para a solução de problemas de engenharia, inclusive com capacidade de reconfiguração flexível através de técnicas baseadas em sistemas em chip;
    \item Produzir, documentar e manter documentação de projeto de forma segura, confiável e aderente à solução implementada, inclusive apontando as restrições legislativas, normativas, éticas e padrões atendidos, com versionamento e identificação de origem;
\end{compitem}

\subsection*{Competências específicas para \textsc{Empreender} formas diversificadas de atuação profissional}

O curso de Bacharelado em Engenharia de Computação fornecerá ao egresso a competência de empreender relacionada a:

\begin{compitem}
    \item Estabelecer novos conhecimentos, tecnologias, serviços e produtos;
    \item Inovar nas soluções de problemas, através da compilação do conhecimento e experiências adquiridas;
    \item Gerar novos negócios em tecnologias nas diferentes áreas do conhecimento; %luciano
    \item Realizar o projeto de produtos e soluções de forma a extrapolar os conhecimentos adquiridos.
\end{compitem}

\subsection*{Competências específicas para \textsc{Atuar} multi, inter e transdiciplinarmente}

No que diz respeito à competência de atuar, o curso de Bacharelado em Engenharia de Computação será capaz de fornecer ao egresso as competências de atuar relacionadas à:

\begin{compitem}
    \item Realização de tarefas práticas em grupo, o que o leva a adquirir a capacidade de liderar e ser liderado;
    \item Habilidade de usar de forma correta a língua portuguesa na forma escrita e falada, através da leitura de materiais bibliográficos, preparação de documentos técnicos, elaboração de relatórios e apresentação de trabalhos de forma oral.
    \item Habilidade de cumprimento de prazos;
    \item Aprendizado e transmissão de conhecimento aos membros da equipe e a conciliação entre teoria e prática;
    \item Capacidade de atuar e trabalhar com equipes multidisciplinares;
    \item Desenvolver e implementar problemas multidisciplinares inovadores relacionados às engenharias e as tecnologias de computação;
    % desenvolver e implementar "soluções para" problemas ... (h)
\end{compitem}

\subsection*{Competências específicas para \textsc{comprometer-se} com a preservação da biodiversidade no ambiente natural e construído, com sustentabilidade e melhoria da qualidade de vida}

O curso de Bacharelado em Engenharia de Computação fornecerá ao egresso a competência de comprometer-se relacionada a:

\begin{compitem}
    \item Entender a necessidade de analisar os impactos de soluções de engenharia em um contexto global, ambiental e social;%luciano
    \item Propor soluções tecnológicas para o desenvolvimento sustentável da sociedade; %luciano
    % será que as soluções propostas levam ao desenvolvimento sustentável, ou que elas devem estar cientes e levar em consideração essas questões? (h)
    \item Propor projetos que estabelecem relações entre o ambiente, as tecnologias e a sociedade;
    \item Propor formas de melhoria da qualidade de vida da sociedade;
\end{compitem}

\subsection*{Competências específicas para \textsc{Gerenciar} processos participativos de organização pública e/ou privada e/ou incluir-se neles.}

No que diz respeito à competência de gerenciar, o egresso do curso de Bacharelado em Engenharia de Computação será capaz de:

\begin{compitem}
    \item Elaborar estratégias para o gerenciamento e controle de desenvolvimento de projetos de sistemas (hardware e software) e soluções em grau de complexidade que demandem o uso e técnicas e modelos de qualidade.
    \item Construir estratégias de desenvolvimento de software e hardware que garantam o funcionamento da solução conforme especificado, através da combinação de técnicas de prototipagem, codificação, validação, testes e homologação dos módulos e conjuntos.
    \item Coordenar a estruturação e execução do projeto de software e hardware para uma plataforma determinada, de forma a atender os requisitos do sistema, documentando as decisões tomadas de forma clara e concisa.
    \item Gerenciar e solucionar problemas que surjam na fase de desenvolvimento de projetos de software e/ou hardware através do uso de estratégias de simulação de interfaces, modelagem de uso em processos de negócio, reutilização de módulos, padronização de interfaceamento e uso de ferramentas e estratégias de gerenciamento do desenvolvimento, com a correspondente documentação de todo o processo.
\end{compitem}

\subsection*{Competências específicas para \textsc{Pautar-se} na ética e na solidariedade enquanto ser humano, cidadão e profissional.}

O curso de Bacharelado em Engenharia de Computação fornecerá ao egresso a competência de pautar-se relacionada a:
% parece estranho dizer "competência de pautar-se...". Será que ficaria melhor algo como "competência para atuar de forma técnica, mas também levando em consideração fatores sociais, como: "   ... ou algo assim.... (h)
\begin{compitem}
    \item Reconhecer, entender e aplicar os limites éticos na solução proposta e ser capaz de descartar soluções que recaiam fora destes limites;
    \item Capacidade de compreender e aceitar as diferenças existentes em uma sociedade na busca de soluções tecnológicas para a melhora da qualidade de vida; % luciano
    \item Analisar os relacionamentos pessoais internos e externos individuais e em grupo;
    \item Reconhecer os limites éticos profissionais e a sua importância na sociedade;
\end{compitem}

\subsection*{Competências específicas para \textsc{buscar} maturidade, sensibilidade e equilíbrio ao agir profissionalmente.}

O curso de Bacharelado em Engenharia de Computação fornecerá ao egresso a competência de buscar relacionada a:
% novamente, talvez seja o caso de não usar "competência de buscar" simplesmente. Talvez, ... a competência de considerar aspectos e conhecimentos de maturidade, respeito e equilíbrio em suas ações relacionadas a:"   ou algo nesse sentido... (h)
\begin{compitem}
    \item Gerenciar ou integrar equipes de trabalho multiculturais, diversas ou plurais;%luciano
    \item Participar como liderança em projetos e na sua participação social;
    \item No desenvolvimento e implantação de projetos relacionados à Engenharia de Computação;
\end{compitem}


\textcolor{red}{Essas competências gerais e especificas, estabelecidas em diálogo com todos os docentes e o NDE do curso de EC, foram sistematizadas de forma a possibilitar um melhor entendimento, ou um entendimento comum entre NDE e os docentes do curso. Assim foi estabelecida uma leitura direcionada às competências específicas, aptadas para o curso de EC de acordo com o Perfil do Profissional a ser Formado na UFSCar (UFSCar, 2008), como descrito no apêndice A deste PPC. O objetivo do apêndice A é o de se estabelecer uma forma mais concisa e objetiva do entendimento das competências e permitir que o NDE possa identificar e avaliar, através dos planos de ensinos formulados pelos docentes de cada disciplina a sua aplicação.}

%%%%%%%%%%%%%%%%%%%%%%%%%%%%%%%%%%%%%%%%%%%%%%%%%%%%%%%%%%%%%%%%%%%%


\section{Estratégias e metodologias de ensino e avaliação}\label{sec:estrategias-e-metodologias-de-ensino-e-avaliacao}

\subsection{Atividades em disciplinas}\label{sec:atividades-em-disciplinas}

% Planilha com estas informações: \url{https://docs.google.com/spreadsheets/d/1bBWnGmKhBWGH3HnDPU2ffqRCjNfU20VQ-vxeyX3qj1A/edit#gid=0}

As atividades listadas abaixo são as possíveis formas de se implementar as competências nas disciplinas do curso de Bacharelado em Engenharia de Computação, sendo que, são aplicadas de forma apropriada de acordo com cada disciplina. Abaixo são apresentas as atividades de acordo com cada competência geral (Anexo~\ref{cha:competencias}).
% essa lista é exaustiva e exclusiva? talvez possa-se dizer "... listadas são exemplos de possíveis ...". 
% também parece estranho falar "formas de implementar as competências...". talvez dizer algo como "formas de avaliar e prover as competências..."   (h)
                         
% achei as listas um pouco redundantes entre si, com itens que aparecem para as diferentes competências... há também características com poucos itens, com bastante desbalanceamento... (h)

\ExecuteLista{\Competencia}{cg-aprender, cg-produzir, cg-empreender,
    cg-atuar, cg-comprometer, cg-gerenciar, cg-pautar, cg-buscar}{
    \subsubsection{\Atributo{\Competencia}{aspecto}}

    \begin{itemize}[noitemsep]
        \ExecuteLista{\Atividade}{\Atributo{\Competencia}{atividades}}{
            \item \Atributo{\Atividade}{descricao}
        }
    \end{itemize}
}


\subsection{Metodologias}\label{sec:metodologias}
Algumas metodologias, ativas ou não, podem ser utilizadas para implementar as competências nas disciplinas, sendo umas mais apropriadas que outras, de acordo com a competência que se deseja transmitir. São elas:

% esta seção está estranha, parece que sem um propósito definido: essas metodologias são usadas no curso? Só listá-las aqui parece não ser muito objetivo (h)

\begin{itemize}
    \item Apresentação de seminários e discussões em grupos;
    \item Metodologia Design Thinking;
    \item Metodologia PBL – Aprendizagem Baseada em Problemas;
    \item Metodologia PjBL - Aprendizagem Baseada em Projetos;
    \item Metodologia TBL - Aprendizado baseado em Times (Grupos);
    \item Metodologias baseada em Estudo de Casos;
    \item Metodologias que possibilitam ao discente maior tempo e profundidade no desenvolvimento de tarefas de laboratório. Poderia ser utilizado o Espaço Maker do DC;
    \item Sala de aula invertida;
    \item Trabalho em grupo.
\end{itemize}

\subsection{Avaliação}\label{sec:avaliacao}

Por outra parte, torna-se necessário proporcionar aos estudantes vários momentos de avaliação, multiplicando as suas oportunidades de aprendizagem e diversificando os métodos utilizados. Assim, permite-se que os estudantes apliquem os conhecimentos que adquirem, exercitem e controlem eles próprios a aprendizagem e o desenvolvimento das competências, recebendo feedback frequente sobre as dificuldades e progressos alcançados.

O Regimento prevê ainda a realização de procedimentos e/ou aplicação de instrumentos de avaliações em, pelo menos, três datas distribuídas no período letivo para cada disciplina/atividade curricular. Serão considerados aprovados os estudantes que obtiverem frequência igual ou superior a setenta e cinco por cento das aulas e desempenho mínimo equivalente à nota final igual ou superior a seis.
% Qual regimento é este? Do que se trata? (h)

% parece estranho falar em "recomendação" neste documento. Afinal, ele é uma apresentação do curso e não um guia para os docentes que forem preparar suas disciplinas, certo? 
% essa estória de nota e fre
A utilização de diferentes métodos e instrumentos de avaliação é recomendada. A escolha dos métodos e instrumentos de avaliação depende de vários fatores: das finalidades, do objeto de avaliação, da área disciplinar e nível de grau de conhecimento dos estudantes a que se aplicam, do tipo de atividade, do contexto, e dos próprios avaliadores. Portanto, propõe-se que, além da tradicional prova individual e trabalho em grupo, outras formas de avaliação das atividades elencadas para se implementar as competências podem ser sugeridas abaixo:

\begin{itemize}
    \item Apresentação de projetos e avaliação em grupo;
    \item Apresentação de relatórios em seminários;
    \item Apresentações, relatórios, exercícios periódicos;
    \item Avaliação por pares;
    \item Compartilhamento das soluções entre grupos distintos;
    \item Emprego de metas (com soluções esperadas) a serem alcançadas;
    \item Identificação de evolução do material trazido e compartilhado, em uma abordagem coletiva e em pares;
    \item Observação e apresentação dos resultados de desenvolvimento prático;
    \item Realização de avaliações formativas e somativas;
    \item Verificação da funcionalidade e o desempenho de programas desenvolvidos;
    \item Verificação se as soluções desenvolvidas atendem às especificações.
\end{itemize}
                                           