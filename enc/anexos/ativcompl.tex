\documentclass{regulamento}
 
\usepackage{enumitem}
\usepackage{amsmath}
\usepackage{url}
\usepackage{float}
\restylefloat{table}
\usepackage{multicol}
\usepackage{setspace}
\usepackage{wasysym} % Para a caixa de seleção do formulário
\usepackage{booktabs}


\centro{Centro de Ciências Exatas e de Tecnologia}
\curso{Bacharelado em Engenharia de Computação}
\coordenacao{Coordenação do Curso de Bacharelado em Engenharia de Computação}
\tituloDocumento{Regulamento de Atividades Complementares}

\begin{document}
 
\imprimirTituloDocumento

\titulo{Disposições Gerais}

\artigo{As Atividades Complementares são todas e quaisquer atividades de caráter acadêmico, científico e cultural realizadas pelo estudante ao longo de seu curso de graduação.}

\artigo{Este Regulamento estabelece uma relação de atividades complementares a serem consideradas para fim de integralização curricular, de acordo com os objetivos do curso.}

\titulo{Carga Horária Exigida}

\artigo{O estudante deve cumprir no mínimo 60 (sessenta) horas em atividades complementares para concluir o curso.}

\artigo{Na Tabela~\ref{tab:complementares} são apresentadas as atividades complementares possíveis de serem realizadas bem como as respectivas cargas horárias e documentos de comprovação a serem apresentados. Naquelas em que não há carga horária estipulada, o conselho do curso irá deliberar de acordo com as seguintes referências:} 

\inciso{Nas atividades passíveis de avaliação/supervisão ou naquelas cujo resultado seja verificável será adotado como referência um quarto das horas cumpridas.}

\inciso{Nas demais atividades em que a participação seja passiva será adotado como referência um oitavo das horas cumpridas.}

\begin{table}[H]
    \centering
    \begin{tabular}{lcp{2cm}p{8cm}}
    \textbf{Atividade} & \textbf{Horas} & \textbf{Caráter} & \textbf{Comprovante} \\
    \sline
Iniciação Científica & 30 & pesquisa & Relatório de finalização da IC/Declaração do Orientador/Certificado de conclusão da IC \\ \hline
PET & 30 & ensino, pesquisa e extensão & Declaração do Tutor/Certificado de Participação no PET emitido pela Pró-Reitoria \\ \hline
Projeto Integrador Extensionista & 30 & extensão & Declaração de Finalização do Projeto/Declaração do Orientador \\ \hline
Empresa Jr. & 30 & extensão & Declaração emitida pelo docente responsável \\ \hline
Monitoria & 30 & ensino & Relatório de monitoria preenchido pelo docente da disciplina atestando a participação e dedicação do monitor. \\ \hline
Projeto Extensão & 30 & extensão & Certificado emitido pelo professor coordenador da atividade. \\ \hline
ACIEPE & 30 & ensino, pesquisa e extensão & Automático pelo sistema acadêmico \\ \hline
Participação/Organização de Eventos & * & extensão & Certificado emitido pela coordenação da atividade. \\ \hline
Estágio não-obrigatório & -- & extensão & Contrato de estágio assinado pelo coordenador de estágios. \\
\sline
    \end{tabular}
    \caption{Atividades Complementares}
    \label{tab:complementares}
\end{table}

\artigo{As horas atribuídas a cada atividade são correspondentes a um semestre, ou seja, atividades com duração de um ano receberão o dobro da carga horária indicada na tabela.}

\titulo{Projeto Integrador Extensionista (PIE)}

\artigo{O Projeto Integrador Extensionista (PIE) é uma atividade de extensão desenvolvida no Departamento de Computação que visa atender a Lei Nº 13.005 de 25 de junho de 2014, a qual aprova o Plano Nacional de Educação (PNE) para o período de 2014 a 2024 e afirma que devem-se intensificar as atividades de extensão nos cursos de graduação, sendo recomendado que 10\% da carga horária do curso seja destinada a atividades de extensão.}

\capitulo{Requisitos}

\artigo{O PIE deve ser desenvolvido em grupo, por estudantes que tenham cursado as disciplinas Construção de Algoritmos e Programação, Algoritmos e Estruturas de Dados 1 e Programação Orientada a Objetos.}

\capitulo{Objetivos}

\artigo{O objetivo do PIE é propiciar aos estudantes um embasamento prático dos conceitos teóricos adquiridos por meio dos conteúdos programáticos ministrados em sala de aula. Tais projetos devem, obrigatoriamente, atenderem demandas externas ao departamento, inclusive atendendo a demandas de empresas, caracterizando-se como projetos de extensão, sendo supervisionados por um professor-orientador da Universidade.}

\artigo{A intenção é que o PIE aproxime-se da forma como os estudantes atuarão na vida profissional: agindo positivamente, na solução de problemas técnicos, sociais, políticos e econômicos, objetivando o desenvolvimento socioeconômico nas perspectivas local, regional, nacional e/ou internacional.}

\artigo{Os objetivos específicos do PIE são:}

\inciso{Propiciar aos estudantes identificar com mais clareza a relação existente entre as disciplinas cursadas, além de promover cada vez mais a interação dos conteúdos apresentados;}
    
\inciso{Propiciar aos estudantes compreender quais conhecimentos e tecnologias podem ser combinadas e adequadas para a resolução de cada problema;}
    
\inciso{Possibilitar aos estudantes fundamentos e aspectos metodológicos iniciais para realização de trabalhos profissionais, estimulando o espírito cooperativo e sensibilizando-o para a importância do trabalho em equipe;}
    
\inciso{Incentivar aos estudantes na identificação de problemas que afetem a comunidade externa ao DC e que possam ser resolvidos por meio do uso de técnicas computacionais;}
        
\inciso{Possibilitar aos estudantes a troca de experiências e o desenvolvimento da capacidade de organização para o desenvolvimento de trabalho em equipe;}
    
\inciso{Incentivar aos estudantes a busca por inovação e o registro de propriedade intelectual e/ou patente no Instituto Nacional de Proteção Intelectual (INPI), com apoio da Agência de Inovação da Fundação de Apoio Institucional (FAI) da UFSCar;}
    
\inciso{Propiciar aos estudantes o desenvolvimento de habilidades de comunicação, escrita e apresentação por meio da defesa do PIE para uma banca avaliadora.}

\capitulo{Oferta}

\artigo{Serão lançados editais com periodicidade mínima anual contendo:}

\inciso{Texto caracterizando PIE e diretrizes gerais para o desenvolvimento do PIE;}

\inciso{Chamada e formato para a submissão de propostas PIE;}

\inciso{Datas para submissão, julgamento, divulgação e homologação dos projetos habilitados;}

\inciso{Período para a inscrição das equipes nos projetos habilitados;}

\inciso{Divulgação dos projetos e equipes a serem desenvolvidos no período.}

\artigo{Os PIEs podem ser propostos por docentes, estudantes e empresas, sendo obrigatório que um docente da UFSCar atue como professor-orientador do projeto.}

\capitulo{Atividades}

\artigo{Os PIEs devem, obrigatoriamente, empregar conhecimentos de 3 (três) ou mais disciplinas e se enquadrarem como extensão, ou seja, possuírem potencial de atingir a comunidade externa ao Departamento de Computação.}

\artigo{Os PIEs poderão contemplar práticas e/ou atividades como:}

\inciso{Projetos de pesquisa aplicada;}

\inciso{Elaboração de diagnósticos empresariais;}

\inciso{Projetos técnicos;}

\inciso{Desenvolvimento de materiais didáticos e instrucionais;}

\inciso{Desenvolvimento de protótipos;}

\inciso{Desenvolvimento de aplicativos e de produtos;}

\inciso{Projetos de inovação tecnológica;}

\inciso{Outras modalidades reconhecidas como relevantes pela Coordenação de Curso.}

\artigo{Para os projetos com potencial de inovação tecnológica, sugere-se que a equipe do projeto avalie a possibilidade junto Agência de Inovação da UFSCar de:  %pode orientar as equipes em como proceder para efetuar o registro.
}

\inciso{registrar o mesmo como registro de software no INPI ou divulgar como código-fonte aberto (repositórios), quando se tratar de desenvolvimento de software;}

\inciso{registrar como patente, quando de tratar de dispositivos de hardware.}

\capitulo{Visão Geral do Processo}

\artigo{O processo de submissão, avaliação e acompanhamento de PIEs contém as seguintes atividades:}

\inciso{Submissão da proposta conforme cronograma previsto em edital específico;}

\inciso{Caracterização da proposta como integrador e extensão: haverá uma comissão definida em edital que avaliará se o projeto se caracteriza como integrador e extensão, emitindo um parecer no prazo estipulado no edital;}

\inciso{Cadastramento dos PIEs aprovados como atividade de extensão: para os projetos aprovados, o professor-orientador deve cadastrar o projeto submetido como atividade de extensão, dentro de programa de extensão específico, previamente cadastrado pelo coordenador do curso;}

\inciso{Acompanhamento da execução do projeto: o professor-orientador deve acompanhar a execução do projeto e realizar avaliação individual e em grupo dos estudantes participantes;}

\inciso{Elaboração, sob a orientação do professor, de um relatório final, conforme modelo disponibilizado pela Coordenação de Curso;}

\inciso{Apresentação dos resultados do PIE para banca examinadora, que poderá aprovar ou reprovar o resultado final obtido;} 

\inciso{Validação dos créditos pelos órgãos competentes.}

\capitulo{Propostas}

\artigo{A proposta para o PIE deve conter:}

\inciso{Capa}

    \alinea{Título;}

    \alinea{Áreas do Conhecimento/Disciplinas Contempladas;}

    \alinea{Sugestão de orientadores(as) ou indicar o orientador;}

    \alinea{Sugestão do tamanho da equipe necessária para o projeto (limite mínimo e máximo) ou indicar a equipe;}

\inciso{Contextualização;}

\inciso{Caracterização do problema;}

\inciso{Justificativa;}

\inciso{Objetivos;}

\inciso{Fundamentação Teórica (explicitando o vínculo com os conteúdos das disciplinas envolvidas);}

\inciso{Metodologia;}

\inciso{Cronograma, incluindo:}

    \alinea{Atividades previstas, considerando a dedicação de 12 horas semanais por estudante;}
    
    \alinea{Previsão de entrega dos produtos do projeto;}
    
    \alinea{Datas de todas as reuniões presenciais e virtuais.}
    
\inciso{Bibliografia.}

\capitulo{Obrigações do Orientador}

\artigo{As atividades relativas ao PIE serão supervisionadas pelo professor-orientador do Projeto Integrador que possui as seguintes obrigações:}

\inciso{Cadastrar o projeto como atividade de extensão junto à Pró-Reitoria de Extensão (ProEx);}

\inciso{Verificar o andamento das atividades de acordo com o cronograma submetido e aprovado;}

\inciso{Orientar os estudantes na condução das atividades;}

\inciso{Registrar os encontros presenciais e virtuais.}

\capitulo{Obrigações dos estudantes}

\artigo{Aos estudantes cabe a realização das atividades do projeto, de acordo com o cronograma submetido e aprovado. Além disso, os estudantes devem:}

\inciso{Comparecer às reuniões presenciais e virtuais de acordo com o cronograma submetido e aprovado;}

\inciso{Dedicar pelo menos 6 horas semanais ao projeto.}

\capitulo{Avaliação}

\artigo{A avaliação será composta de duas etapas:}

\inciso{A primeira etapa consiste em uma avaliação individual e contínua, e ficará a cargo do orientador. Nesta etapa, serão considerados assiduidade e desempenho individual de cada estudante. Os estudantes reprovados nesta etapa serão desligados do projeto e não poderão ter os créditos convalidados;}

\inciso{A segunda etapa consiste em uma avaliação do projeto como um todo, que deve ser apresentado em forma textual (relatório final) e apresentação oral mediante uma banca examinadora. Nesta etapa, a banca examinadora irá avaliar o cumprimento da proposta aprovada, com atenção especial para o enfoque obrigatório de projeto integrador e extensionista.}

\artigo{Em termos de assiduidade, o aluno deve cumprir no mínimo 75\% de frequência nas atividades do projeto.}

\artigo{A banca será composta por um mínimo de três integrantes e um máximo de quatro, sendo pelo menos dois professores da UFSCar. }

\artigo{O professor-orientador é membro natural da banca examinadora e irá presidir a sessão. }

\artigo{A indicação de nomes de membro da banca, bem como a definição da data e reserva de sala é de responsabilidade do professor-orientador, respeitando o cronograma pré-estabelecido.}

\artigo{Em caso de reprovação, o projeto poderá ser reapresentado, mediante solicitação por meio de formulário próprio, para a mesma banca examinadora.}

\artigo{O estudante será reprovado automaticamente no Projeto Integrador quando
ocorrer pelo menos um dos itens abaixo:} 

\inciso{O trabalho não cumprir o objetivo proposto;}

\inciso{O trabalho for plágio;}

\inciso{O trabalho não for desenvolvido pelos estudantes;}

\inciso{O trabalho estiver fora das normas técnicas exigidas pela 
Instituição;}

\inciso{O trabalho não for entregue no prazo estabelecido;}

\inciso{Não for comprovada a presença de pelo menos 75\% (setenta e cindo por cento) nas atividades do projeto.}

\artigo{A ocorrência de qualquer dos itens anteriores deve ser comunicada
pelo professor orientador à Coordenação de Curso, que após avaliar a situação emitirá um parecer final.}


\capitulo{Obrigações da Coordenação de Curso}

\artigo{Para garantir a oferta contínua de projetos em andamento, a coordenação de curso irá, a cada ano letivo, indicar dez docentes do Departamento de Computação que deverão submeter ao menos uma proposta de PIE naquele ano.}

\artigo{A coordenação também será responsável por organizar e divulgar os editais de candidatura, aprovar as bancas de avaliação e validar os créditos.}

\titulo{Disposições Finais}

\artigo{O presente Regulamento de Atividades Complementares entra em vigor a partir da data de sua aprovação pelo Conselho de Coordenação de Curso.}

\artigo{Os casos omissos serão resolvidos pela Coordenação de Curso, cabendo ao Conselho do Curso estipular a carga horária a ser considerada em cada um dos casos e analisar a necessidade de atualização deste Regulamento.}

\end{document}