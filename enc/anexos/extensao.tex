
%%%%%%%%%%%%%%%%%%%%%%%%%%%%%%%%%%%%%%%%%%%%%%%%%%%
%! Criador = Luciano Neris
%! Data = 09/02/2024
%%%%%%%%%%%%%%%%%%%%%%%%%%%%%%%%%%%%%%%%%%%%%%%%%%%

%\documentclass{regulamento}
%\usepackage{enumitem}
%\usepackage{amsmath}
%\usepackage{url}
%\usepackage{float}
%\restylefloat{table}
%\usepackage{multicol}
%\usepackage{setspace}
%\usepackage{wasysym} % Para a caixa de seleção do formulário
%\usepackage{booktabs}
%\usepackage{graphicx}


%\centro{Centro de Ciências Exatas e de Tecnologia}
%\curso{Bacharelado em Engenharia de Computação}
%\coordenacao{Coordenação do Curso de Bacharelado em Engenharia de Computação}
%\tituloDocumento{Regulamento da Inserção Curricular das atividades de Extensão (Minuta)}

%\begin{document}
 
%\imprimirTituloDocumento

%\titulo{Disposições Gerais}

%\bigskip

%\bigskip

\bigskip

\textcolor{black}{O Ministério da Educação através do Conselho Nacional de Educação estabeleceu através da Resolução
N}\textcolor{black}{\textsuperscript{o}}\textcolor{black}{ 7, de 18 de Dezembro de 2018, as “Diretrizes para a Extensão
na Educação Superior Brasileira”, a qual define os princípios, os fundamentos e os procedimentos que devem ser
observados no planejamento nas políticas, na gestão e na avaliação das instituições de educação superior de todos os
sistemas de ensino do país.}


%\bigskip

\textcolor{black}{\ A UFSCar, seguindo as orientações da Resolução Nº 7 de 18 de Dezembro de 2018, estabeleceu a
RESOLUÇÃO CONJUNTA COG/COEX Nº 2 DE }21 de novembro de \textcolor{black} { 20}23, que trata da\textcolor{black}{s
diretrizes a serem utilizadas na UFSCar para a implantação das Atividades Complementares Extensionistas (ACEs)}
\textcolor{black}{nos projetos pedagógicos dos cursos de graduação.}

\textcolor{black}{\ \ Este documento estabelece a obrigatoriedade das ACEs integrarem os currículos de todos os cursos
de graduação da UFSCar e }de forma\textcolor{black}{ prevista no respectivo Projeto Pedagógico do Curso (PPC),
perfazendo um }\textbf{\textcolor{black}{percentual mínimo de 10 (dez) por cento Atividades curriculares}} dos cursos de
graduação.\textcolor{black}{ }


\bigskip

\textbf{\textcolor{black}{Definição:}}\textcolor{black}{ Extensão Universitária constitui-se em processo
interdisciplinar, político educacional, cultural, científico, tecnológico, que promove a interação transformadora entre
as instituições de ensino superior e os outros setores da sociedade, por meio da produção e da aplicação do
conhecimento, em articulação permanente com o ensino e a pesquisa. (Artigo 3º da Resolução CNE/CES 7/2018)}


\bigskip

Nos termos desta resolução, são denominadas Atividades Curriculares de Extensão (ACEs) as atividades extensionistas
passíveis de inserção curricular na graduação. %
%Reescrever?
%Helio Crestana Guardia
%2024 Feb 2 11:52
Para que sejam reconhecidas como ACEs, o discente poderá realizar sua integralização das seguintes categorias (Resolução
COG.COEX):

I – Atividades Curriculares Obrigatórias, Optativas ou Eletivas com carga horária integral ou parcial voltada à
abordagem extensionista;

II – Atividades Curriculares de Integração entre Ensino, Pesquisa e Extensão (ACIEPEs) %
%Dá para mencionar no PPC que qualquer ACIEPE, por sua natureza, valha como ACE?
%Helio Crestana Guardia
%2024 Feb 2 11:53
previstas nos PPCs; e

III – Atividades Complementares de Extensão: Ações de extensão, com ou sem bolsa, com aprovação registrada na
Pró-Reitoria de Extensão nas modalidades de projetos, cursos, oficinas, eventos, prestação de serviços e %
%.
%Helio Crestana Guardia
%2024 Feb 2 11:54
ACIEPEs não previstas nos PPCs.


\bigskip

A categoria de participação de cada discente deve estar indicada no registro de uma ACE. Há duas categorias de
participação discente possíveis para as \textbf{ACEs do tipo III}:

\ (a) equipe de trabalho da ACE ou 

\ (b) público-alvo (participantes inscritos). 


\bigskip

No caso das ACIEPEs, pela natureza da sua concepção, todos os inscritos têm participação categorizada de forma
equivalente à da equipe de trabalho.


\bigskip

Atividades derivadas de iniciativas da UFSCar, tais como, Empresas Juniores, Cursinhos Pré-Vestibulares, Programa de
Educação Tutorial (PET), Programa Institucional de Bolsas de Iniciação à Docência (PIBID), poderão ser consideradas
atividades curriculares de extensão do tipo III, desde que estejam registradas como ações de extensão.


\bigskip

A carga horária contabilizada não poderá ser duplamente contabilizada como atividade de outra natureza.


\bigskip

Os estágios obrigatório e não obrigatório seguem normativas próprias e não podem ser considerados como atividade
curricular de extensão.


\bigskip

%\textcolor{red}{%
%Talvez seja o caso de reescrever...
%De forma a não ampliar a carga horária do curso...
%houve uma reorganização de atividades e conteúdos ...
%incorporando ACEs em disciplinas e substituindo atividades por outras com caráter extensionista...
%(ou algo nesse sentido...)
%Helio Crestana Guardia
%2024 Feb 2 12:02
%Para}\textcolor{red}{ o curso de Engenharia de Computação foi estabelecido que a carga horária original do curso não
%seria alterada com a inclusão das ACEs, ou seja, disciplinas obrigatórias e do tipo extensionistas são propostas na
%forma de uma ACE definida pelo curso de Engenharia de Computação.}

%\bigskip

A estratégia utilizada será a de inclusão/adaptação de Disciplinas Obrigatórias com caráter extensionista contabilizando
195 créditos, a inclusão de ACEs, num total de 180 créditos, e a diminuição das disciplinas optativas para 120
créditos, conforme as Tabelas 1 e 2. As ACEs podem ser compostas pelas atividades I, II e III descritas anteriormente, sendo
que para o caso I, deverão ser propostas pelo menos três disciplinas com caráter extensionista, conforme estabelecido
pelo curso de Engenharia de Computação.


\bigskip

%{\centering
%\textbf{Tabela 1. Distribuição de créditos e carga horária antiga. (b) Nova distribuição de créditos e carga
%horária. }\ \ \ 
%\par}

\begin{table}[H]
    \centering
    \caption{Distribuição de créditos e carga horária antiga}
    \label{tab:integralizacao}
    \begin{tabular}{lcc}
        \sline
        \textbf{Atividades Curriculares} & \textbf{Créditos} & \textbf{Carga Horária} \\
        \hline
        Disciplinas Obrigatórias         & 200               & Normal = 3.000                  \\
        Disciplinas Optativas            & 12                & Normal = 180                    \\
        Disciplinas Eletivas             & 8                 & Normal = 120                    \\
        Estágio Supervisionado           & 12                & Normal = 180                    \\
        Trabalho de Conclusão de Curso   & 8                 & Normal = 120                    \\
        Atividades Complementares        & 4                 & Normal = 60
        \\
        \hline
        \textbf{Total}                   & \textbf{244}      & \textbf{3.660}         \\
        \sline
    \end{tabular}
\end{table}


\begin{table}[H]
    \centering
    \caption{Nova distribuição de créditos e carga horária}
    \label{tab:integralizacao}
    \begin{tabular}{lcc}
        \sline
        \textbf{Atividades Curriculares} & \textbf{Créditos} & \textbf{Carga Horária} \\
        \hline
        Disciplinas Obrigatórias         & 200               & Normal (P+T) = 2.805           \\
                                         &                   & Ext (P+T) = 195                 \\        
        ACEs                             & 12                & Ext (P+T) = 180                    \\
        Disciplinas Optativas            & 8                 & Normal = 120                    \\
        Estágio Supervisionado           & 12                & Normal = 180                    \\
        Trabalho de Conclusão de Curso   & 8                 & Normal = 120                    \\
        Atividades Complementares        & 4                 & Normal = 60
        \\
        \hline
        \textbf{Total}                   & \textbf{244}      & \textbf{3.660}         \\
        \sline
    \end{tabular}
\end{table}

Para a contabilização das ACE$\text{\textgreek{’}}$s, ficam estabelecidos os limites para cada possível atividade:

ACE$\text{\textgreek{’}}$s = $\Sigma $ Aciepes*(60h cada), Projetos de Extensão (120h/ano), Cati Jr (120h/ano), PET%
%Será que vale a pena restringir? Pode apenas gerar mais trabalho no mecanismo de controle.
%Helio Crestana Guardia
%2024 Feb 2 13:13
* (120h/ano), disciplinas optativas extensionistas (ACE$\text{\textgreek{’}}$s novas 60h cada) == 180 h  (*precisam ser formalizadas na Proex)


\bigskip

Para o curso de Engenharia de Computação, o discente terá que cursar o equivalente mínimo de \textbf{366h} (10\% da
carga horária total do curso), para isto, existem as seguintes opções: 


\bigskip

\textbf{Disciplinas extensionistas: }

\textbf{Obrigatórias}:

{}- Engenharia de Sistemas (carga horária: 60h extensionistas);

{}- Sistemas Embarcados (carga horária: 60h extensionistas);

{}- Seminários I e II (carga horária: 30h extensionistas x 2 = total 60 h)

{}- Sistemas distribuídos (carga horária: 45h normais 15h extensionistas)

Total: 195h.


\bigskip

\textbf{Optativas Extensionistas}:

{}- Sistemas de Integração e Automação Industrial (carga horária: 60h);

{}- Robótica Móvel (carga horária: 60h);

{}- Projetos com CLP (carga horária: 60h);

%...

Total: 180h. (devem ser ofertadas pelo curso e obrigatoriamente cursadas caso o discente não realize nenhuma atividade
ACE das categorias II e III)


\bigskip

%ACIEPES Registradas no PPC da Engenharia de Computação (carga horária máxima: 60h cada):

%{}- Robótica na Escola (Luciano); 

%{}- Internet das Coisas (Ricardo Menotti);


%\bigskip

Projetos de Extensão e ACIEPES (desde que o aluno esteja devidamente registrado na PROEX com sua carga horária, de
acordo com os limites estabelecidos anteriormente).


\bigskip


\textbf{Caracterização de uma Disciplina com caráter extensionista (ACE Tipo I)}


\bigskip

De acordo com a Resolução conjunta COG COEX, estabelecidas pela UFSCar para a caracterização de uma atividade
extensionista, uma ACE deve explorar principalmente as competências gerais listadas abaixo (vide Anexo A): \ 

{}- \textbf{Comprometer}{}-se com a preservação da biodiversidade no ambiente natural e construído e com
sustentabilidade e melhoria da qualidade da vida, 

{}- \textbf{Gerenciar} processos participativos de organização pública ou privada ou incluir-se neles, 

{}- \textbf{Pautar}{}-se na ética e na solidariedade enquanto ser humano, cidadão e profissional, 

{}- \textbf{Buscar} maturidade, sensibilidade e equilíbrio ao agir profissionalmente

{}- \textbf{Empreender }\ formas diversificadas de atuação profissional;

{}- \textbf{Produzir }e \textbf{divulgar }novos conhecimentos, tecnologias, serviços e produtos. 


\bigskip

O Art. 3º da Resolução COG.COEX estabelece que para serem reconhecidas como Atividades Curriculares de Extensão, as
propostas deverão atender aos princípios:


\bigskip

I - contribuição para a formação integral do estudante estimulando sua formação como cidadão crítico e responsável.

Competências estabelecidas para esse item pelo NDE:

{}- \textbf{Buscar} maturidade, sensibilidade e equilíbrio ao agir profissionalmente, 

{}- \textbf{Pautar}{}-se na ética e na solidariedade enquanto ser humano, cidadão e profissional, 


\bigskip

II - estabelecimento de diálogo construtivo e transformador com os demais setores da sociedade brasileira e/ou
internacional;

Competências estabelecidas para esse item pelo NDE:

{}- \textbf{Produzir }e \textbf{divulgar }novos conhecimentos, tecnologias, serviços e produtos;

criando repositórios de software aberto à comunidade, ex.: github, produzindo material de interesse e disponibilização
para comunidade, ex.: Softwares, Readme, etc. 


\bigskip

III - envolvimento proativo dos estudantes na promoção de iniciativas que expressam o compromisso social das
instituições de ensino superior com todas as áreas e prioritariamente as de comunicação, artes, cultura, direitos
humanos e justiça, educação, meio ambiente, saúde, tecnologia e produção, e trabalho, em consonância com as políticas
ligadas às diretrizes para a educação ambiental, relações educação étnico-raciais, direitos humanos e educação
indígena; 

Competências estabelecidas para esse item pelo NDE:

{}- \textbf{Comprometer}{}-se com a preservação da biodiversidade no ambiente natural e construído e com
sustentabilidade e melhoria da qualidade de vida. Na criação de produtos sustentáveis e que melhorem a qualidade de
vida da população.


\bigskip

IV – contribuição ao enfrentamento de questões no contexto local, regional, nacional ou internacional. Recomenda-se a
referência aos objetivos de desenvolvimento sustentável (ODS) definidos pela ONU; 

Competências estabelecidas para esse item pelo NDE:

{}- \textbf{Comprometer}{}-se com a preservação da biodiversidade no ambiente natural e construído e com
sustentabilidade e melhoria da qualidade de vida.

{}- \textbf{Gerenciar }processos participativos de organização pública ou privada ou incluir-se neles.

Os temas explorado devem ser encaixar nas ODS da ONU:

\textbf{ODS ONU - temas de projetos para Optativas Extensionistas}



\begin{figure}[H]
    \centering
    \caption{ODS ONU - Temas de projetos para Optativas Extensionistas.}
    \label{fig:ODS}
    \includegraphics[width=\textwidth]{enc/imagens/ODS-ONU.pdf}
    %Fonte:~\textcite{ARM2013}
\end{figure}



\bigskip

\textbf{Plano de Ensino de uma Disciplina Extensionista (ACE Tipo I)}


\bigskip

Para ser considerada uma ACE a disciplina deve relacionar as competências, estabelecidas pelo NDE do curso de Engenharia
de Computação para um atividade extensionista, de acordo com a Resolução CoEx/CoG. 

A disciplina candidata a ser considerada uma ACE deve apresentar ao menos uma das seguintes características: 

\begin{itemize}[series=listWWNumi,label=\textstyleListLabeli{${\bullet}$}]
\item Projeto e/ou implementação real (hardware e/ou software);
\item Disponibilização em repositório de software público;
\item Interação direta com público externo ao curso;
\item Interação indireta com o público externo ao curso;
\item Disponibilização de software para teste público;
%\item …
\end{itemize}
\bigskip

As atividades propostas nas disciplinas deverão explorar, além dos conhecimentos e habilidades, as competências de uma
ACE. Tais competências devem ser implementadas e avaliadas de acordo com o estabelecido no plano de ensino da
disciplina.


\bigskip

Na disciplina extensionista, ACE Tipo I, há a obrigatoriedade do conteúdo produzido ser externado à comunidade, o que
pode ser realizado na forma de workshops, palestras, apresentação de pôsteres e/ou com as tecnologias de comunicação
existentes, como Redes Sociais (%
%será que vale a pena mencioar? Podem deixar de ser populares e/ou surgirem outras. (TikTok vale?:{}-)
%Helio Crestana Guardia
%2024 Feb 2 13:41
whatsapp, instagram, facebook, linkedin, etc), Fóruns de discussão, vídeos publicados na WEB de caráter “público”,
Webconferência on line, salas virtuais (ex.: google classroom), repositórios públicos de software, etc. garantindo
assim a interação com o público externo (comunidade interna e externa à UFSCar).


\bigskip

O conteúdo da disciplina extensionista (ACE do tipo I) a ser externado ao público interno e externo à UFSCar pode ser
parcial, durante a execução da disciplina, ou total, incluindo a apresentação de projetos, que podem ser divulgados
prévia e amplamente nos meios de comunicação digital%
%Pensar se vale a pena listar ou deixar aberto.
%Helio Crestana Guardia
%2024 Feb 2 13:45
, inclusive nos oficiais, em pelo menos um deles, como o “site” do Departamento de Computação - DC da UFSCar, \ Boletim
informativo da UFSCar, Boletim de oportunidades da UFSCar, etc.

%\end{document}