%%%%%%%%%%%%%%%%%%%%%%%%%%%%%%%%%%%%%%%%%%%%%%%%%%%
%! Criador = Jander Moreira
%! Data = 26/11/2022


%%%%%%%%%%%%%%%%%%%%%%%%%%%%%%%%%%%%%%%%%%%%%%%%%%%
As Diretrizes Curriculares Nacionais (DCNs) para as Engenharias, instituídas pela Câmara de Educação Superior do Conselho Nacional de Educação (CES/CNE) através da Resolução nº 02/2019, definem os princípios, os fundamentos, as condições e as finalidades, para aplicação, em âmbito nacional, na organização, no desenvolvimento e na avaliação do curso de graduação em Engenharia das Instituições de Educação Superior (IES). Elas trazem, entre outros aspectos, a ênfase no desenvolvimento de competências técnicas e sócio emocionais dos estudantes ao longo da sua trajetória de formação, buscando criar um ambiente propício para o desenvolvimento do pensamento criativo, com sólida base teórica, da capacidade de inovação e de empreendedorismo dos graduandos em engenharia. Neste contexto, este apêndice busca adequar o PPC do curso de Bacharelado em Engenharia de Computação para atender às referidas diretrizes segundo o perfil do profissional a ser formado pela UFSCar.


% \section{Caracterização das competências da UFSCar}

%\textcolor{red}{\textbf{Melhorar essa parte aqui!}}

Os aspectos do perfil do profissional a ser formado pela UFSCar e suas respectivas competências estão descritas no documento Perfil do profissional a ser formado na UFSCar \cite{ufscar2008perfil} referendado pela Resolução CEPE/UFSCar no 776/2001. Esse documento define os seguintes aspectos e suas competências básicas a serem desenvolvidas pelos egressos:

%A UFSCar definiu, de acordo com o documento YYYYYYY, as competências gerais para os egressos segundo os seguintes aspectos:
\begin{itemize}
    \singlespacing
    \setlength{\itemsep}{0pt}
    \ExecuteLista{\Compet}{cg-aprender, cg-produzir, cg-empreender, cg-atuar, cg-comprometer, cg-gerenciar, cg-pautar}{
        \item \Atributo{\Compet}{aspecto};
    }
    \item \Atributo{cg-buscar}{aspecto}.  % usa ponto final no último item
\end{itemize}

A seguir estão caracterizados cada um destes aspectos, juntamente com as competências específicas envolvidas e siglas utilizadas, devidamente adaptadas ao Curso de Engenharia de Computação.

\ExecuteLista{\CompGeral}{cg-aprender, cg-produzir, cg-empreender, cg-atuar, cg-comprometer, cg-gerenciar, cg-pautar, cg-buscar}{%
    \TabelaCompetencias{\CompGeral}
}

% Jander: TODO ESTE CONTEÚDO VAI PARA JUNTO DA DESCRIÇÃO DAS EMENTAS
%
% \section{Disciplinas e competências associadas}
% \label{sec:disc-comp}
% Esta seção apresenta a relação das atividades obrigatórias e suas respectivas competências. O agrupamento é feito pelos eixos \textit{Algoritmos e programação}, \textit{Arquiteturas de computadores}, \textit{Metolodolias e técnicas de computação}, \textit{Humanas}, \textit{Eletrônica} e \textit{Engenharia e sistemas}.

% \NewDocumentCommand{\ReportaEixo}{ m }{%
%     Disciplinas:\par
%     \begin{itemize}[noitemsep, topsep = 0pt]
%         \ExecuteLista{\Discipl}{#1}{
%             \item \Atributo{\Discipl}{nome} (\Atributo{\Discipl}{semestre}\textdegree~sem.)
%         }
%     \end{itemize}

%     \ExecuteLista{\Discipl}{#1}{
%         \noindent%
%         \begin{minipage}{\textwidth}
%             \MostraCompetencias{\Discipl}
%         \end{minipage}
%     }
% }


% \subsection{Eixo: Algoritmos e programação}
% \ReportaEixo{cap, ipa, aed1, ori, aed2, poo, paa, ppd}

% \subsection{Eixo: Arquiteturas de computadores}
% \ReportaEixo{ld, sd, arq1, arq2, arqad}

% \subsection{Eixo: Metodologias e técnicas de computação}
% \ReportaEixo{es1, so, bd, om, ia, rc, sistdist, ihc}

% \subsection{Eixo: Humanas}
% \ReportaEixo{sem1, sem2, metcient, tcc1, tcc2}
% % Jander: O QUE FAZER COM "ATIVIDADE COMPLEMENTAR"?

% \subsection{Eixo: Eletrônica}
% \ReportaEixo{circeletricos, circeletronicos1, circeletronicos2}

% \subsection{Eixo: Engenharia e sistemas}
% \ReportaEixo{sistdin, controle1, controle2, psd, engsis, teccom, projsisemb}

% \subsection{Disciplinas optativas}
% % fazer: analisar como apresentar as optativas!!!
% \begin{itemize}[noitemsep]
%     \ExecuteLista{\Discipl}{
%         am1,am2,asp,bdcd,cg,compiladores,controle,dw1,dw2,
%         dm,devops,empreend,es2,musical,libras,robosautonomos,
%         lm,robotica,md,plp,pde,pdi,pdi3dv,pvd,
%         pooa,pmac,pibd,protsisdigan,sc,siai,tc
%     }{
%         \item \Atributo{\Discipl}{nome}
%     }
% \end{itemize}

