\documentclass{regulamento}
 
\usepackage{enumitem}
\usepackage{amsmath}
\usepackage{url}
\usepackage{float}
\restylefloat{table}
\usepackage{multicol}
\usepackage{setspace}
\usepackage{wasysym} % Para a caixa de seleção do formulário

\centro{Centro de Ciências Exatas e de Tecnologia}
\curso{Bacharelado em Engenharia de Computação}
\coordenacao{Coordenação do Curso de Bacharelado em Engenharia de Computação}
\tituloDocumento{Regulamento do Estágio Curricular Obrigatório e Não-obrigatório}

\begin{document}
 
\imprimirTituloDocumento

\titulo{Disposições Gerais}

\artigo{No Curso de Bacharelado em Engenharia de Computação, o Estágio Curricular é estruturado conforme o estabelecido na Lei nº. 11.788/2008, de 25 de setembro de 2008 da Presidência da República que regulamenta os estágios, e pelo no Regimento Geral dos Cursos de Graduação da Universidade, estabelecido em setembro de 2016 que dispõe sobre a realização de estágios de estudantes dos Cursos de Graduação da Universidade Federal de São Carlos. De acordo com o no Regimento Geral dos Cursos de Graduação da Universidade, os estágios na Universidade serão curriculares, podendo ser obrigatórios ou não obrigatórios.}

\titulo{Objetivos do Estágio Curricular}

\artigo{Observando o Perfil do Profissional previsto para o Curso de Bacharelado em Engenharia de Computação e o previsto no Art. 1º da Lei nº. 11.788/2008 - O Estágio é um ato educativo escolar supervisionado, desenvolvido no ambiente de trabalho, que visa à preparação para o trabalho produtivo de educandos que estejam frequentando o ensino regular em instituições de educação superior. Foram definidos para o Estágio Curricular os seguintes objetivos:}

\inciso{Consolidar o processo de formação do profissional bacharel em Engenharia de Computação para o exercício da atividade profissional de forma integrada e autônoma;}

\inciso{Possibilitar oportunidades de interação dos estudantes com institutos de pesquisa, laboratórios e empresas que atuam nas diversas áreas da Engenharia de Computação;}

\inciso{Desenvolver a integração Universidade-Comunidade, estreitando os laços de cooperação.}

\titulo{Caracterização do Estágio Curricular}

\artigo{O Estágio Curricular deve ser desenvolvido nas áreas de conhecimento no âmbito da Engenharia de Computação, mediante um Plano de Trabalho, elaborado em comum acordo entre as partes envolvidas.}

\artigo{O Estágio não poderá ser realizado no âmbito de atividades de monitoria ou iniciação científica.}

\artigo{A integralização da carga horária exigida para a realização de estágios obrigatórios se
concretizará mediante a frequência e aprovação na disciplina Estágio em Engenharia de Computação.}

\paragrafounico{A disciplina estabelece um pré-requisito de 200 (duzentos) créditos aprovados no curso.}

\artigo{As atividades de estágio poderão ser desenvolvidas durante as férias escolares ou durante o período letivo, embora a oferta da disciplina seja de acordo com os semestres letivos da Universidade.}

\titulo{Jornada de Atividade em Estágio}

\artigo{De acordo com a Lei 11.788/08, a jornada de atividade em estágio será definida de comum acordo entre a Instituição de Ensino, a parte concedente e o estudante estagiário, devendo constar do termo de compromisso e ser compatível com as atividades acadêmicas.}

\artigo{A jornada não poderá ultrapassar 6 (seis) horas diárias e 30 (trinta) horas semanais durante o período letivo.}

\inciso{Não poderá haverá haver conflito de horário entre a jornada de estágio e as atividades acadêmicas.}

\artigo{Durante as férias, ou se tiver completado os créditos em disciplinas necessários para a conclusão do curso, o estagiário poderá ter jornada de até 40 (quarenta) horas semanais.}

\artigo{O aluno que já exerce atividade profissional compatível com a sua área de atuação, pode solicitar diminuição de até 50\% da carga horária exigida para o estágio. A solicitação deve ser encaminhada à coordenação do curso que irá analisar o caso e decidirá a porcentagem a ser reduzida.}

%TODO: adicionei esta possibilidade de adiantar disciplinas para fazer estágio, caso contrário os estudantes adiam ao invés de adiantar. O BCC também fez assim, conferir

\titulo{Modalidades de Estágio}

\capitulo{Estágio Obrigatório}

\artigo{De acordo com a Lei 11.788/08, Estágio Curricular Obrigatório é aquele cujo cumprimento da carga horária é requisito para aprovação e obtenção de diploma. Esta obrigatoriedade atende o estabelecido no Art. 7º da Resolução CNE/CES Nº. 5/2016, de 16 de novembro de 2016 que institui as Diretrizes Curriculares Nacionais do Curso de Graduação em Engenharia de Computação o qual define que a formação do Engenheiro de Computação incluirá, como etapa integrante da graduação, estágios curriculares obrigatórios sob supervisão direta da instituição de ensino, através de relatórios técnicos e acompanhamento individualizado durante o período de realização da atividade.}

% \artigo{O Estágio Curricular obrigatório tem por objetivo que o estudante adquira experiência na área profissional bem como coloque em prática os conhecimentos teóricos adquiridos no decorrer do curso, preparando-o para o exercício futuro da profissão. Para isso, a inserção na empresa é necessária possibitando-lhe o contato com situações, problemas, processos reais, bem como com processos de tomada de decisão e realização de tarefas, complementando a sua formação teórica.}

\artigo{Na realização do estágio obrigatório o estudante receberá orientação de um professor do curso, o qual o auxiliará em questões não previstas em sua grade curricular sempre que as partes julgarem necessário.}

\artigo{Durante o período de estágio, o estudante deverá relatar o trabalho realizado na empresa através de um relatório final, entregue ao Coordenador de Estágio, do trabalho que realizou, a fim de possibilitar a avaliação sobre o currículo oferecido aos estudantes do referido curso.}

\artigo{A carga horária mínima do estágio curricular deverá atingir 180 (centro e oitenta) horas a serem realizadas no 10º semestre do curso.}

\paragrafounico{O estudante poderá adiantar o início do estágio para o 9º semestre do curso caso tenha pretensão de realizá-lo durante um ano.}

\artigo{O Estágio obrigatório será desenvolvido obedecendo as etapas de: }

\inciso{Planejamento o qual se efetivará com a elaboração do plano de trabalho e formalização do termo de compromisso;}

\inciso{Supervisão e acompanhamento, que se concretizarão em três níveis: Profissional, Didático-pedagógica e administrativa desenvolvidos pelo supervisor local de estágio, pelo professor orientador e pelo coordenador de estágio, respectivamente;}

\inciso{Avaliação, realizada em dois níveis: profissional e didática desenvolvidos pelo supervisor local de estágio e professor orientador, respectivamente.}

\capitulo{Estágio Não-Obrigatório}

\artigo{O Estágio Não-Obrigatório é aquele desenvolvido como atividade opcional. Para realizá-lo o estudante deve ter sido aprovado em, no mínimo, 120 créditos e a jornada deve ser compatível com as atividades acadêmicas. A carga horária desenvolvida no estágio não-obrigatório será computada na grade do estudante como Atividade Complementar.}

\titulo{Coordenação de Estágio}

A Atividade de Estágio é regulamentada pela Coordenação de Estágio, composta por um Coordenador de Estágio e um Secretário da Coordenação de Estágio.

O Coordenador de estágio é professor do curso responsável pela disciplina Estágio Supervisionado.
As atribuições da Coordenação de Estágio são:

\inciso{Estar em contato com empresas interessadas em contratar estagiários;}

\inciso{Informar o estudante sobre as regras para a realização do estágio;}

\inciso{Direcionar os estudantes quanto ao preenchimento correto do Termo de Compromisso de Estágio;}

\inciso{ Avaliar o plano de trabalho de estágio;}

\inciso{ Designar Orientador do Estágio;}

\inciso{ Coordenar a tramitação de todos os instrumentos jurídicos: convênios, termos de compromisso, requerimentos, cartas de apresentação, cartas de autorização ou outros documentos necessários para que o estágio seja oficializado, bem como a guarda destes;}

\inciso{ Coordenar as atividades de avaliações do Estágio obrigatório.}


\titulo{Estágio Internacional}

\artigo{O estágio em empresas estrangeiras é permitido desde que estas sigam a legislação brasileira. }

\titulo{Condições para realização do Estágio Curricular Obrigatório}~\label{sec:CECO}

\artigo{Para realização do Estágio Curricular Obrigatório o estudante deve atender os seguintes requisitos:}

\inciso{Estar matriculado regularmente no Curso de Bacharelado em Engenharia de Computação;}

\inciso{	Ter concluído 200 créditos do seu curso;}

\inciso{	Possuir um supervisor da parte concedente, para orientação, acompanhamento e avaliação do estágio;}

\inciso{Celebrar um termo de compromisso entre o estudante, a parte concedente do estágio e a Universidade; }

\inciso{Elaborar um plano de atividades a serem desenvolvidas no estágio, compatíveis com este projeto pedagógico, o horário e o calendário escolar, de modo a contribuir para a efetiva formação profissional do estudante;}

\inciso{Acompanhamento efetivo do estágio por professor orientador designado pela coordenação de estágio e por supervisor da parte concedente, sendo ambos responsáveis por examinar e aprovar os relatórios periódicos e final, elaborados pelo estagiário.}


\titulo{Orientação e supervisão de estágio}

\artigo{O professor responsável pela orientação do estudante durante o Estágio Curricular será um professor do Curso de Bacharelado em Engenharia de Computação, sendo este responsável pelo acompanhamento e avaliação das atividades dos estagiários e terá as seguintes atribuições:}

\inciso{Orientar os estudantes na elaboração dos relatórios e na condução de seu Projeto de Estágio;}

\inciso{Orientar o estagiário quanto aos aspectos técnicos, científicos e éticos;}

\inciso{Supervisionar o desenvolvimento do programa pré-estabelecido, controlar frequências, analisar relatórios, interpretar informações e propor melhorias para que o resultado esteja de acordo com a proposta inicial, mantendo sempre que possível contato com o supervisor local do estágio;}

\inciso{Estabelecer datas para entrevista(s) com o estagiário e para a entrega de relatório(s) das atividades realizadas na empresa;}

\inciso{Avaliar o estágio, especialmente o(s) relatório(s), e encaminhar ao colegiado o seu parecer, inclusive quanto ao número de horas que considera válidas.}


\artigo{O supervisor do estagiário deverá ser um profissional que atue no local no qual o estudante desenvolverá suas atividades de estágio e terá as seguintes atribuições:}


\inciso{Garantir o acompanhamento contínuo e sistemático do estagiário, desenvolvendo a sua orientação e assessoramento dentro do local de estágio;}

\inciso{Informar à Coordenação de Estágio as ocorrências relativas ao estagiário, buscando assim estabelecer um intercâmbio permanente entre a Universidade e a Empresa;}

\inciso{Apresentar um relatório de avaliação do estagiário à Coordenação de Estágio, em caráter confidencial.}


\titulo{Obrigações do estagiário}

\artigo{O estagiário, durante o desenvolvimento das atividades de estágio, terá as seguintes obrigações:}

\inciso{Apresentar documentos exigidos pela Universidade e pela concedente;}

\inciso{Seguir as determinações do Termo de compromisso de estagio;}

\inciso{Cumprir integralmente o horário estabelecido pela concedente, observando assiduidade e pontualidade;}

\inciso{Manter sigilo sobre conteúdo de documentos e de informações confidenciais referentes ao local de estágio;}

\inciso{Acatar orientações e decisões do supervisor local de estágio, quanto às normas internas da concedente;}

\inciso{Efetuar registro de sua frequência no estágio;}

\inciso{Elaborar e entregar relatório das atividades de estágio e outros documentos nas datas estabelecidas;}

\inciso{Respeitar as orientações e sugestões do supervisor local de estagio;}

\inciso{Manter contato com o professor orientador de estágio, sempre que julgar necessário;}

\inciso{Assumir o estágio com responsabilidade, zelando pelo bom nome da Instituição Concedente e do Curso de Bacharelado em Engenharia de Computação.}

\titulo{Formalização do termos de compromisso}

\artigo{Deverá ser celebrado Termo de Compromisso de estágio entre o estudante, a parte concedente do estágio e a Universidade, estabelecendo:}

\inciso{O plano de atividades a serem realizadas, que figurará em anexo ao respectivo termo de compromisso;}

\inciso{As condições de realização do estágio, em especial, a duração e a jornada de atividades, respeitada a legislação vigente;}

\inciso{As obrigações do Estagiário, da Concedente e da Universidade;}

\inciso{O valor da bolsa ou outra forma de contra prestação devida ao Estagiário, e o 
auxílio-transporte, a cargo da Concedente, quando for o caso; }

\inciso{O direito do estagiário ao recesso das atividades na forma da legislação vigente;}

\inciso{A empresa contratante deverá segurar o estagiário contra acidente pessoal, sendo que uma cópia da mesma deverá ser anexada ao termo após sua realização.}
















% \artigo{O Regulamento de Estágios do Curso de Bacharelado em Engenharia de Computação
% baseia-se nas disposições contidas na Portaria GR no. 1882//92, que aprova a Resolução CEPE
% no.146/92, que dispõe sobre a regulamentação referente à estágio de alunos na Universidade
% Federal de São Carlos, que considera a Lei Federal N. 11.788, de 25 de Setembro de 2008, que
% disciplina no país a realização de estágios por estudantes de ensino superior.}

% \artigo{O objetivo do Regulamento de Estágios do Curso de Bacharelado em Engenharia de Computação é disciplinar o planejamento, implementação e avaliação das atividades de estágio dos
% alunos do curso de Bacharelado em Engenharia de Computação.}

% \artigo{O presente Regulamento deve ser aprovado pelo Conselho de Coordenação de
% Curso, podendo ser revisto periodicamente, no todo ou em parte, para seu aperfeiçoamento ou
% atualização, face às necessidades da aprendizagem aplicada em complementação às atividades
% teóricas do curso.}

% \titulo{Dos Estágios}

% \artigo{A grade curricular do curso de Bacharelado em Engenharia de Computação
% estabelece a realização de 300 (trezentas) horas de estágio curricular obrigatório.}

% \artigo{A integralização da carga horária exigida para a realização de estágios obrigatórios se
% concretizará mediante a frequência e aprovação na disciplina Estágio Supervisionado.}

% \paragrafounico{A disciplina estabelece um pré-requisito de 200 (duzentos) créditos aprovados no curso.}

% \artigo{As ementas da referida disciplina estabelece a observação e realização de
% atividades de estágio em Centros de Pesquisa, Empresa ou qualquer órgão que necessite de um
% profissional da área de Computação fora do próprio departamento.}

% \paragrafounico{A amplitude e diversidade das necessidades informacionais da
% comunidade usuária são os critérios de caracterização dos centros supra-citados.}

% \artigo{Fica também estabelecida a realização de estágios complementares pelos alunos,
% com duração acertada diretamente entre o estagiário e a instituição concedente, desde que
% estabelecidos os instrumentos jurídicos necessários.}

% \titulo{Da Organização Administrativa}

% \artigo{Fica criada a Coordenação de Estágios do Curso de Bacharelado em Engenharia de
% Computação, subordinada à Coordenação de Curso, com as seguintes atribuições:}

% \inciso{coordenar e supervisionar o planejamento, implementação e avaliação das atividades de
% estágio do Curso de Bacharelado em Engenharia de Computação, de acordo com as
% disposições legais da Universidade e do presente regulamento;}

% \inciso{rever e propor modificações no Regulamento de Estágios, a partir de sugestões da
% comunidade externa e interna e da Coordenação de Curso;}

% \inciso{manter contato com setor competente da Pró-Reitoria de Graduação para acompanhar
% as mudanças nos dispositivos legais, receber orientações e atender solicitações;}

% \inciso{manter contato com as instituições externas ou setores internos para fins de realização
% de estágios;}

% \inciso{promover palestras por parte das instituições e empresas para recrutamento de
% estagiários;}

% \inciso{organizar e manter um cadastro das instituições concedentes de estágio;}

% \inciso{encaminhar à Coordenação de Curso minutas de Acordos de Cooperação para
% Realização de Estágio e termos aditivos para tramitação e aprovação, mantendo uma cópia
% em arquivo;}

% \inciso{elaborar e assinar termos de compromisso de estágio;}

% \inciso{definir o professor-orientador de cada estágio, entregando o Termo de Compromisso
% correspondente;}

% \inciso{orientar os professores orientadores nos procedimentos de planejamento,
% implementação e avaliação dos estágios;}

% \inciso{coordenar as visitas de acompanhamento dos professores orientadores;}

% \inciso{expedir correspondências e declarações referentes à estágio;}

% \inciso{receber dos professores-orientadores documentação comprobatória dos estágios
% realizados;}

% \inciso{promover seminários dos estagiários concluintes aos candidatos a estágio nos
% semestres subseqüentes;}

% \inciso{manter um arquivo dos estágios realizados, com prontuários individuais por aluno;}

% \inciso{elaborar relatório anual de atividades;}

% \inciso{exercer as demais funções inerentes à coordenação e supervisão de estágios, além
% daquelas que lhe forem conferidas pela Coordenação de Curso.}

% \artigo{A Coordenação de Estágios será exercida por um docente do Departamento de
% Computação, com a devida aprovação, substituição e recondução por deliberação do Conselho de
% Coordenação de Curso.}

% \artigo{De acordo com o Artigo 1o. da Portaria GR no. 1882/92, a realização de estágios
% exige o estabelecimento de Acordo de Cooperação para Realização de Estágio entre a Universidade
% e instituição concedente, que contenha, no mínimo : objetivo do convênio, contrapartida da Universidade,
% cobertura do aluno por seguro obrigatório, áreas abrangidas e vigência.}

% \paragrafo{Quando já existir um convênio firmado de caráter geral, será necessário a realização
% de um Termo Aditivo que trate especificamente de estágio.}

% \paragrafo{Quando já existir um Acordo de Cooperação para a Realização de Estágio já firmado
% entre a Universidade e a instituição concedente, bastará somente o Termo de Compromisso
% relativo ao aluno.}

% \artigo{A celebração do Acordo de Cooperação para a Realização de Estágio, quando
% não existir, terá o início de sua tramitação pela Coordenação de Estágios, de acordo com orientação
% geral da Pró-Reitoria de Graduação, encaminhada à Coordenação de Curso para tramitação no
% Conselho Departamental e Conselho Interdepartamental e envio à Reitoria para assinatura.}

% \paragrafo{O início da tramitação do Acordo de Cooperação para a Realização de Estágio
% ocorrerá nas seguintes condições:}

% \alinea{quando um aluno estiver interessado em estagiar na instituição e a mesma concordar em
% ser concedente de campo de estágio;}

% \alinea{quando um docente solicitar e a instituição concordar em ser concedente de campo de
% estágio;}

% \alinea{quando o aluno, por conta própria, conseguir o estágio;}

% \alinea{quando a instituição estiver interessada.}

% \paragrafo{A Coordenação de Estágios deverá solicitar à Procuradoria Jurídica a lista dos
% convênios firmados para fins de arquivo próprio e consulta geral.}

% \artigo{Após a tramitação do Acordo de Cooperação para Realização de Estágio, com a
% devida formalização das responsabilidades da Universidade e da instituição concedente poderá ser
% assinado o Termo de Compromisso específico para cada estudante, onde necessariamente deverá
% constar a data do Acordo de Cooperação que lhe deu origem.}

% \artigo{Conforme Artigo 2o. da Portaria GR 1882/92, cada Termo de Compromisso deverá
% conter as seguintes informações básicas: nome do estagiário, período de duração do estágio, as
% obrigações da Universidade, as obrigações da instituição concedente, as obrigações do estagiário, o
% número da apólice de seguro e a remuneração do estagiário, quando for o caso, assinado pelo
% responsável da instituição concedente, pela Coordenação de Estágios e pelo estudante.}
% \label{artigo:termo}

% \paragrafounico{O mesmo dispositivo legal dispõe que o Termo de Compromisso seja
% acompanhado do Plano de Trabalho do estagiário, em que conste o nome do estagiário, o
% nome do orientador da instituição concedente e o nome do professor orientador da
% Universidade e suas respectivas assinaturas.}

% \artigo{Com base no Artigo 4o. da citada Portaria, a realização de estágios na própria
% Universidade exige manifestação do setor interessado e realização de um Termo de Compromisso
% que contenha: nome do estagiário, período de duração do estágio, as obrigações das partes
% envolvidas, as obrigações do estagiário e a designação de responsabilidade da remuneração sob
% forma de bolsa, quando for o caso, assinado pelo Chefe da unidade que recebe o estagiário, pela da
% Coordenação de Estágios e pelo estudante.}

% \paragrafounico{Aplicam-se, também neste caso, as disposições contidas no parágrafo
% único do Artigo~\ref{artigo:termo} do presente regulamento.}

% \titulo{Da Organização Didático-Pedagógica}

% \artigo{Para o acompanhamento de cada estágio, a Coordenação de Estágios distribuirá
% os docentes do Departamento de Computação, através dos critérios de capacitação e eqüidade, para
% que cada um exerça a função de professor-orientador.}

% \paragrafounico{Cada professor-orientador fixará um horário de atendimento aos
% estagiários sob sua responsabilidade.}

% \artigo{O estágio curricular exige a existência de um orientador na instituição concedente
% ou unidade interna da Universidade concedente, cujo critério de aceite é sua capacitação profissional
% teórico-prática na área de informação, para garantir a qualidade de aprendizagem dos alunos.}

% \artigo{O orientador externo é o responsável pela elaboração do Plano de Trabalho do
% Estagiário, segundo estrutura básica estabelecida, com a devida ciência e aceite do professor
% orientador.}

% \paragrafo{O cronograma do Plano de Trabalho do Estagiário deverá conter obrigatoriamente
% um período para conhecimento da instituição e da unidade de informação, as atividades a
% serem desenvolvidas, um período para a elaboração do relatório final de estágio e a
% previsão de pelo menos uma visita de acompanhamento do professor-orientador.}

% \paragrafo{Ao aluno com vínculo empregatício em instituições concedentes de estágio ou na
% própria Universidade, alocados em unidades de informação, fica autorizada a realização de
% estágio curricular, desde que o Plano de Trabalho do Estagiário seja mais diversificado que
% suas funções regulares, respeitadas as demais condições estabelecidas no presente
% regulamento.}

% \artigo{Cada Plano de Trabalho corresponderá a 300 (trezentas) horas de atividades de estágio,
% conforme ementa da disciplina Estágio.}

% \artigo{Cada aluno deverá entregar ao professor-orientador, ao final de cada mês de
% realização do estágio, o Relatório Parcial de Estágio, devidamente preenchido e assinado, para fins
% de controle de freqüência e execução do Plano de Trabalho.}

% \paragrafo{professor orientador deverá registrar no Relatório Parcial de Estágio a data da
% visita de acompanhamento realizada, conforme previsão no Plano de Trabalho.}

% \paragrafo{O professor-orientador estabelecerá a primeira nota ou menção pela avaliação dos
% Relatórios Parciais de Estágio.}

% \artigo{Ao final da duração do estágio, cada aluno deverá entregar ao professor-orientador
% o Relatório Final de Estágio, realizado e desenvolvido conforme estrutura básica
% estabelecida pelo Coordenador de Estágios.}

% \paragrafounico{O professor-orientador determinará a segunda nota ou menção pela
% avaliação do Relatório Final de Estágio.}

% \artigo{A terceira nota ou menção será atribuída pelo orientador externo, através do
% preenchimento do formulário Avaliação Externa de Estagiário.}

% \artigo{A média final da disciplina Estágio será resultado da média aritmética das três
% menções anteriores, respeitando-se as condições de aprovação do regime escolar vigente na
% Universidade.}

% \artigo{Ao final do semestre, a documentação comprobatória do estágio realizado por
% cada aluno será encaminhada pelo professor-orientador à Coordenação de Estágios, que manterá
% arquivo específico.}

% \paragrafounico{Cada prontuário de aluno entregue à Coordenação de Estágios será
% composto do Termo de Compromisso, do Plano de Trabalho, dos Relatórios Parciais de
% Estágio, do Relatório Final de Estágio, da Avaliação Externa do Estagiário, antecedido de
% uma folha-síntese, em que constem as menções parciais e média final, além da apreciação
% da Coordenação de Estágios, que se responsabilizará pelo seu arquivamento.}

\titulo{Disposições Finais}

\artigo{O presente Regulamento de Estágios entra em vigor a partir da data de sua
aprovação pelo Conselho de Coordenação de Curso.}

\artigo{Os casos omissos serão resolvidos pela Coordenação de Estágios e, em última
instância, pela Coordenação de Curso.}


\end{document} 