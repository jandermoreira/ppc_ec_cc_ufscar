\documentclass{regulamento}
 
\usepackage{enumitem}
\usepackage{amsmath}
\usepackage{url}
\usepackage{float}
\restylefloat{table}
\usepackage{multicol}
\usepackage{setspace}
\usepackage{wasysym} % Para a caixa de seleção do formulário
\usepackage{booktabs}


\centro{Centro de Ciências Exatas e de Tecnologia}
\curso{Bacharelado em Engenharia de Computação}
\coordenacao{Coordenação do Curso de Bacharelado em Engenharia de Computação}
\tituloDocumento{Regulamento do Trabalho de Conclusão de Curso}

\begin{document}
 
\imprimirTituloDocumento

\titulo{Disposições Gerais}

\artigo{O Trabalho de Conclusão de Curso (TCC) é um componente curricular obrigatório e se constitui em um trabalho de produção orientada: (i) acadêmico, de cunho mais científico; ou (ii) desenvolvimento de um projeto de engenharia, de cunho mais aplicado; que sintetiza e integra conhecimentos, competências e habilidades adquiridos durante o curso.}

\artigo{O TCC deverá propiciar aos estudantes de graduação a oportunidade de reflexão, análise e crítica, articulando a teoria e a prática, resguardando o nível adequado de autonomia intelectual dos estudantes.} 

\titulo{Condições para Realização}

\artigo{A realização dessa atividade deverá versar sobre qualquer área do conhecimento da Engenharia da Computação.}

\artigo{Essa atividade deverá ser desenvolvida mediante a orientação de um docente vinculado ao Curso de Bacharelado em Engenharia de Computação da UFSCar, com titulação de doutor e reconhecida experiência acadêmica.}

\paragrafounico{Será permitida a co-orientação com a participação de profissionais externos à UFSCar.}

\artigo{A integralização da carga horária exigida para a realização do TCC se concretizará mediante a frequência e aprovação nas disciplinas Trabalho de Conclusão de Curso 1 (TCC-1) e Trabalho de Conclusão de Curso 2 (TCC-2).}

\paragrafo{Caberá aos docentes encarregados dessas Disciplinas em cada oferta estabelecerem cronogramas e critérios de avaliação.}

\paragrafo{Para cursar a disciplina Trabalho de Conclusão de Curso 1 o estudante deve ter sido aprovado na disciplina Metodologia Científica.}

\paragrafo{Para cursar a disciplina Trabalho de Conclusão de Curso 2 o estudante deve ter sido aprovado na disciplina Trabalho de Conclusão de Curso 1.}

\titulo{Desenvolvimento do TCC}

\artigo{No Curso de Bacharelado em Engenharia de Computação estão previstos dois momentos para se realizar o TCC:}

\inciso{Na disciplina TCC-1, oferecida no 9º semestre, o estudante realizará o projeto de seu TCC, seguindo orientações recebidas na disciplina Metodologia Científica, e finalizará com uma proposta sua pesquisa, descrita em um plano;}


\inciso{Na disciplina TCC-2, oferecida no 10º semestre, após ter sido aprovado na disciplina TCC-1, o estudante continuará em sua pesquisa e realizará a sua implementação.}

\artigo{O responsável principal pelo acompanhamento do estudante no desenvolvimento do trabalho de monografia é o professor-orientador. }

\artigo{O professor-orientador é escolhido pelo aluno de acordo com a maior proximidade do tema a ser desenvolvido, ou seja, devem ser orientadores que possuam a expertise do tema na forma de sua concepção e modelagem e que tenham conhecimento das técnicas para fazê-lo. }

\artigo{O professor-orientador deverá acompanhar o desenvolvimento do trabalho durante todo o seu período de realização, orientando constantemente o estudante em sua execução.}

\titulo{Avaliação}

\capitulo{TCC-1}

\artigo{A qualidade da proposta apresentada ao orientador e ao docente responsável, bem como o  cumprimento dos prazos, serão os objetos de avaliação para aprovação na disciplina TCC-1.}

\capitulo{TCC-2}

\artigo{O TCC-2 será avaliado por uma banca examinadora.}

\paragrafo{O trabalho final deverá ser apresentado em forma de monografia e realizada uma exposição oral a membros de uma banca de avaliação.}

\paragrafo{A monografia deverá seguir o rigor acadêmico de autenticidade (caso contrário é considerado plágio), o formalismo e os critérios de qualidade, de acordo com as normas atuais.}

\paragrafo{No texto escrito serão avaliadas a redação, a qualidade do trabalho realizado e as contribuições para a formação do estudante, bem como sua adequação às normas da Associação Brasileira de Normas Técnicas (ABNT).}

\paragrafo{Na apresentação oral serão avaliadas a exposição do trabalho realizado e a arguição pelos examinadores.}

\paragrafo{A apresentação da monografia deverá ser realizada em sessão pública dentro das datas estabelecidas previamente no início de cada semestre.}

\titulo{Composição da Banca}

\artigo{A banca será composta por um mínimo de três integrantes e um máximo de quatro, sendo pelo menos dois professores da UFSCar. }

\artigo{O professor-orientador é membro natural da banca examinadora e irá presidir a sessão. }

\artigo{A indicação de nomes de membro da banca, bem como a definição da data e reserva de sala é de responsabilidade do professor-orientador e do estudante, respeitando o cronograma pré-estabelecido.}

\titulo{Disposições Finais}

\artigo{O presente Regulamento do Trabalho de Conclusão de Curso entra em vigor a partir da data de sua aprovação pelo Conselho de Coordenação de Curso.}

\artigo{Os casos omissos serão resolvidos pelos docentes responsáveis pelas respectivas Disciplinas, e em última instância pela Coordenação de Curso.}

\end{document}