
%\chapter{Integração Ensino, Pesquisa e Extensão}

No âmbito deste projeto pedagógico, o termo ``ensino'' envolve atividades relacionadas ao processo de ensino/aprendizagem, fazendo com que o aluno adquira novos conhecimentos. O termo ``pesquisa'' se refere a atividades que produzem avanço no conhecimento científico, incluindo inovação. O termo ``extensão'' caracteriza atividades que transmitem o conhecimento gerado na UFSCar para a sociedade.

Em um curso de excelência, a integração entre ensino, pesquisa e extensão é fundamental para que a universidade cumpra seu papel na sociedade. Com o ensino, a universidade desenvolve competências e habilidades no aluno, que quando formado, as utilizará na sociedade, melhorando-a. O ensino também fornece ao aluno a fundamentação teórica necessária para a realização de pesquisa, formando assim pesquisadores competentes.

Com a pesquisa, promove-se um constante aperfeiçoamento dos professores, fazendo que seu conhecimento sobre um assunto seja aprofundado, culminando em um ensino de melhor qualidade. A pesquisa também gera conhecimento que pode ser transferido para resolução de problemas reais ou promover melhorias na sociedade.

Com a extensão, obtém-se um contato mais próximo entre a universidade e a sociedade. O ensino é beneficiado, pois com a extensão é possível trazer para a universidade demandas reais da sociedade, que podem ser incorporadas aos componentes curriculares. Também beneficia a pesquisa, pois as demandas e oportunidades estimulam a produção de conhecimento que pode ser mais diretamente aplicado na sociedade, produzindo inovação.

Cada componente curricular do Bacharelado em Ciência da Computação proporciona um nível diferente de integração entre ensino, pesquisa e extensão, como pode ser visto na Tabela abaixo. As atividades complementares são uma exceção, visto que depende da atividade. Maiores detalhes sobre o caráter das atividades complementares podem ser obtidos na Seção \ref{sec:atividadesComplementares}.

\begin{center}
\begin{tabular}{|c|c|c|c|} \hline
\textbf{Componente curricular} & \textbf{Ensino} & \textbf{Pesquisa} & \textbf{Extensão} \\ \hline
Disciplinas & X & &  \\ \hline
Áreas de conhecimento  & X & &  \\ \hline
Projeto integrador & X & X & X \\ \hline
TCC & X & X & X \\ \hline
Estágio & X & & X \\ \hline
\multicolumn{1}{|l|}{Atividades complementares} & X & X & X  \\ \hline
% As Atividades complementares são atividades curriculares que não estão compreendidas no desenvolvimento regular das disciplinas do Curso. São todas e quaisquer atividades de caráter acadêmico, científico e cultural realizadas pelo estudante ao longo de seu curso de graduação, que contribuem para o enriquecimento científico, profissional e cultural e para o desenvolvimento de valores e hábitos de colaboração e de trabalho em equipe. 

\hline
\end{tabular}
\end{center}

%De modo geral, a articulação entre as disciplinas é mediada pelo sistema de requisitos implantado na UFSCar, cuja concepção de construção de conhecimentos, competências e habilidades se pauta pela evolução gradativa e embasada também no desempenho dos alunos.

%A grande maioria das disciplinas que constituem o curso de Ciência da Computação é formativa, com um pequeno número de disciplinas informativas. Essas disciplinas formativas têm o objetivo de estimular o aluno a estudar e aprender independentemente, incitando-o a ser mais autodidata, capacitando-o a \lq\lq aprender a aprender\rq\rq, condição indispensável para que enfrente depois de formado problemas envolvendo ciência e/ou tecnologia . Neste sentido, o curso de Ciência da Computação está em sintonia com os novos rumos do ensino de computação, que preconizam a importância de acabar com o conceito de formatura; o egresso do curso será conscientizado, desde seu primeiro dia na universidade, a ter disposição e atitudes que levem ao \textit{aprendizado contínuo} ao longo da sua vida. 


Para a consecução do perfil do egresso idealizado, a abordagem multi/interdisciplinar figura como fundamental para a geração integrada de conhecimento, cuja prática e contato com os problemas reais, serão vivenciados pelo egresso no exercício profissional ou pelo estudante durante seu Estágio. Neste sentido, o Estágio Supervisionado, regulamentado pela Lei nº 11.788, de 25 de setembro de 2008, prevê uma carga horária de no mínimo de 360 horas, realizado no 8º período do curso. 
Outro importante momento de vivência prática são as aulas em laboratório, posto que nelas são propiciadas as simulações de situações reais em ambiente acadêmico, instigando o estudante à observação e compreensão dos diversos fenômenos reproduzidos.

Outro aspecto relevante e vinculado ao desenvolvimento das competências e habilidades delineadas anteriormente se refere ao estímulo para a realização de trabalhos em equipe, na medida em que estes promovem a troca de informações, implicando na organização do trabalho a partir da divisão de tarefas e compartilhamento de responsabilidades.
O Cientista da Computação formado na UFSCar tem uma forte base em algumas ciências básicas, como a matemática. Porém, esta formação básica, essencial para o seu sucesso profissional, está sempre associada a seu lado aplicado. Assim, um dos princípios desse curso é estimular a criatividade e, sobretudo, dotar o estudante/egresso de capacidade para o desenvolvimento e aplicação de ferramentas de ponta para ``criar soluções''. Na medida do possível, as disciplinas com alto conteúdo tecnológico são e serão desenvolvidas com base na discussão e resolução de problemas reais das empresas.

O curso de Ciência da Computação é direcionado ao estudante com uma forte aptidão para ciências exatas e que deseja aplicar esses conhecimentos básicos na investigação e na resolução de problemas tecnológicos. Assim, não se considera, aqui,  a histórica divisão entre as especializações e ataca-se o problema proposto por meio de uma estratégia multidisciplinar, tornando a formação mais que generalista: uma formação de multiespecialista.

A possibilidade que o estudante de Ciência da Computação tem, ao frequentar e participar de atividades relacionadas aos diferentes grupos de pesquisa do Departamento de Computação, ou de outros departamentos da UFSCar, consiste em uma oportunidade ímpar na sua formação. Neste sentido, algumas das experiências relacionadas às disciplinas práticas são realizadas no âmbito desses grupos.


\section{Articulação Ensino, Pesquisa e Extensão}

A UFSCar, ao longo de sua história, preocupa-se em promover ativamente a integração entre as atividades de ensino, pesquisa e extensão, reconhecendo que essas atividades, quando adequadamente articuladas e executadas de forma balanceada, potencializam-se umas às outras. Esta diretriz acadêmica também está fundamentada neste projeto, já que os estudantes poderão se envolver com atividades de ensino, pesquisa e extensão, vinculadas diretamente ao curso ou ofertadas pelos Departamentos com ele comprometidos. 


\section{Atividades de Pesquisa}

Uma primeira estratégia para desenvolver no estudante as habilidades de pesquisa no curso de Ciência da Computação é a oportunidade de desenvolver a Iniciação Científica (IC). Para tanto, a IC deve ser formalizada pelos canais internos da universidade, conforme sua afinidade temática, em conjunto com o docente orientador. Uma iniciação científica é uma atividade desenvolvida pelo estudante sob a supervisão de um professor orientador. O objetivo é que o aluno desenvolva algum projeto que, normalmente, demanda 1 ano e que equivale a 165 horas de atividade complementar. A IC permite que o estudante tenha contato com temas inovadores da computação e que tenha a experiência do que signfica fazer pesquisa em computação. Em muitos casos, o estudante acaba tendo contato direto com os alunos de mestrado e doutorado do departamento e desenvolve seu projeto em laboratórios de pesquisa. Uma iniciação científica é bastante relevante para estudantes que pretendem seguir carreira acadêmica e de pesquisa, geralmente fazendo um mestrado e doutorado. Mas atualmente também há muitas empresas que valorizam egressos com competências de pesquisa. 

Outra forma de articulação com pesquisa é Projeto Integrador Extensionista. Esse projeto visa a conciliar as habilidades já adquiridas pelos estudantes para a resolução de problemas de interesse da sociedade e que, ao mesmo tempo, apresentem características de inovação, exigindo dos mesmos uma dedicação à pesquisa do estada da arte para a resolução de tais problemas.

Outra forma ainda de contato com pesquisa são os grupos PET (Programa de Educação Tutorial). O Bacharelado em Ciência da Computação possui um grupo PET que tem o objetivo de desenvolver atividades de ensino, pesquisa e extensão. Diferentemente de uma iniciação científica, estudantes petianos tem contato com os três pilares da universidade (ensino, pesquisa e extensão). As pesquisas conduzidas dentro do grupo PET são menos aprofundadas do que pesquisas que são feitas na forma de iniciações científicas, já que essas possuem a pesquisa como norte prioritário. 

\section{Atividades de Extensão}

As atividades de extensão são importantes não apenas como meio de difusão do conhecimento gerado na universidade, mas também como mecanismo de aproximação da realidade. De maneira mais explícita, os estudantes terão a oportunidade de participar de atividades de extensão organizadas pelos diversos canais internos da universidade. A Universidade Federal de São Carlos valoriza estas atividades e tem, na Pró-Reitoria de Extensão, um órgão da sua administração central totalmente devotado à organização e ao desenvolvimento de atividades de extensão, inclusive financiando parcialmente estas iniciativas.

A diversidade das atividades de pesquisa e extensão beneficia os estudantes de graduação que se envolvem diretamente com elas em projetos de iniciação científica e de extensão, alargando sua formação com atividades complementares. Dentre as várias iniciativas presentes na Universidade Federal de São Carlos, destaca-se o movimento das empresas juniores (Empresa Jr.) e os grupos de competição em olimpíadas de informática e maratona de programação.

A Ciência da Computação possui sua empresa júnior (compartilhada com o Curso Engenharia de Computação), sediada no Departamento de Computação da UFSCar e que desenvolve diversos projetos junto a empresas da região e a comunidade acadêmica.

Visando incentivar ainda mais a participação dos estudantes na solução de problemas reais e que afetam a sociedade, o presente projeto pedagógico institui no curso de Ciência da Computação o Projeto Integrador Extensionista. Esse tipo de atividade de extensão, detalhado no Anexo~\ref{anexo: PIE}, caracteriza-se por permitir aos estudantes aplicarem conceitos adquiridos em várias disciplinas do curso na solução de problemas reais demandados pela comunidade externa.

%\helenacomentario{Falar do caráter de extensão do Projeto Integrador? Auri, acho que foi você quem escreveu o texto sobre o Projeto Integrador, não? Será que poderia inserir um parágrafo aqui sobre ele?}

%\helenacomentario{De novo, acham pertinente explicar brevemente outros componentes curriculares marcados como ``extensão'' no início dessa seção? TCC?, Estágio, ACIEPE, PET}

% Valter: Sim, eu acho relevante !