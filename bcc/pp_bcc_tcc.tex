

O Trabalho de Conclusão de Curso (TCC) é um componente curricular não obrigatório e se constitui em um trabalho acadêmico de produção orientada, que sintetiza e integra conhecimentos, competências e habilidades adquiridos durante o curso. No curso de Bacharelado em Ciências da Computação, o TCC pode substituir o Estágio Obrigatório. O TCC deverá propiciar aos estudantes de graduação a oportunidade de reflexão, análise e crítica, articulando a teoria e a prática, resguardando o nível adequado de autonomia intelectual dos estudantes. A realização dessa atividade deverá versar sobre qualquer área do conhecimento da Ciência da Computação que vise à inovação tecnológica.

Essa atividade deverá ser desenvolvida mediante a orientação de um docente da UFSCar, com titulação de doutor e reconhecida experiência profissional (professor-orientador), aceita-se a co-orientação com a participação de profissionais externos à UFSCar, e um docente responsável pela disciplina Trabalho de Conclusão de Curso, responsável belo bom andamento de todos os TCCs de maneira geral.

O produto final do TCC será apresentado na forma de uma monografia com uma exposição oral. A monografia deverá seguir o rigor acadêmico de autenticidade (caso contrário é considerado plágio), o formalismo e os critérios de qualidade, de acordo com as normas atuais. No texto escrito serão avaliadas a redação, a qualidade do trabalho realizado e as contribuições para a formação do estudante. Na apresentação oral serão avaliadas a exposição do trabalho realizado e a arguição pelos examinadores.

Além disso, a sua documentação deve seguir padrões acadêmicos, respeitadas as devidas normas para isso, tais como as da ABNT – Associação Brasileira de Normas Técnicas, para a escrita do documento final.



\subsection{A disciplina Trabalho de Conclusão do Curso}

No curso de Ciência da Computação o TCC é realizado no 8º (oitavo) semestre em carater optativo com o Estágio Obrigatório, onde a disciplina é nomeada como Trabalho de Conclusão do Curso. 

O TCC é um trabalho acadêmico de caráter de pesquisa ou técnico, dessa forma poderá ser substituído por um trabalho de Iniciação Científica (IC). No caso do estudante desejar substituir a IC pelo TCC, ele deverá entregar um requerimento na secretaria da graduação solicitando a equivalência da IC com o TCC e submeter o relatório da IC para uma comissão de professores. Essa comissão irá atestar se a IC realizada equivale ao TCC.

Caso seja aprovada a equivalência da IC com o TCC, deverá também haver a exposição oral do trabalho para uma banca examinadora e o documento a ser entregue para avaliação pode ser o prórpio relatório da IC, desde que este siga os modelo Pibic ou Fapesp. As regras de formação dessa banca examinadora e a avaliação são as mesmas do TCC.

Obs.: A IC, caso utilizada para substituir o TCC, não poderá ser computada como Atividade Complementar também.


\subsection{As Atribuições do Professor-Orientador}

O responsável principal pelo acompanhamento do estudante no desenvolvimento do trabalho de monografia é o professor-orientador. O professor-orientador é escolhido de acordo com a maior proximidade do tema a ser desenvolvido, ou seja, devem ser orientadores que possuam a expertise do tema na forma de sua concepção e modelagem e que tenham conhecimento das técnicas para fazê-lo. O professor-orientador deverá acompanhar o desenvolvimento do trabalho durante todo o seu período de realização, orientanto constantemente o aluno em sua execução.

\subsection{As Atribuições do Professor Responsável pela Disciplina}

O professor responsável pela disciplina de Trabalho de Conclusão de Curso irá fazer o acompanhamento do desenvolvimento de todos os TCCs incritos na disciplina, junto aos estudantes e professores orientadores. O professor responsável pela disciplina irá acompanhar o bom andamento de todos os TCCs e garantir um nível qualidade adequado de cada um, de acordo com as necessidades e formalismos acadêmicos existentes.


\subsection{Apresentação do Trabalho}
A apresentação da monografia deverá ser realizada em sessão pública dentro das datas estabelecidas previamente no início de cada semestre. O estudante deverá apresentar o trabalho para uma banca examinadora. A banca será composta por um mínimo de três integrantes e um máximo de quatro, sendo pelo menos dois professores da UFSCar. O professor-orientador é membro natural da banca examinadora e irá presidir a sessão. A indicação de nomes de membro da banca, bem como a definição da data e reserva de sala é de responsabilidade do professor-orientador e do aluno, respeitando o cronograma pré-estabelecido.

A avaliação será feita através da defesa e da entrega da monografia final corrigida pelo estudante após aplicada as alterções propostas pelos membros da banca em sua defesa. 

