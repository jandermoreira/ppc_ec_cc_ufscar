
Esta seção dispõe sobre o gerenciamento do curso, que tem como principal objetivo a coordenação didático-pedagógica, visando a elaboração e a condução do projeto pedagógico do curso em concordância com a política de ensino, pesquisa e extensão da Universidade. A gestão do curso é conduzida pelo Coordenador do Curso com o apoio do Conselho de Coordenação de Curso (que possui a função deliberativa), Núcleo Docente Estruturante e Coordenadores de Núcleo de Conhecimento. 


\section{Composição e Funcionamento do Conselho de Coordenação do Curso}

O Curso de Ciência da Computação, assim como todos os demais cursos da Universidade Federal de São Carlos, tem sua administração acadêmica regulamentada pelo Regimento Geral dos Cursos de Graduação, que estabelece em seu artigo 87:

\begin{quote}
Art. 87. A gestão do Curso de Graduação é realizada pelos seguintes órgãos:\\
I - Conselho de Coordenação;\\
II - Coordenação do Curso.
\end{quote}

O Conselho de Coordenação deve ser composto por docentes, servidores técnico-administrativos e estudantes, além dos seus respectivos suplentes. Tal composição é determinada pelo Artigo 89 do Regimento Geral dos Cursos de Graduação, conform segue:

\begin{quote}
Art. 89. Cabe ao Conselho de Coordenação do Curso, na definição de seu Regimento Interno, estabelecer os critérios para participação e procedimentos para eleição de seus membros, respeitando a legislação vigente, garantindo, no mínimo:\\
I - O Coordenador do Curso como presidente;\\
II - O Vice-Coordenador do Curso como vice-presidente;\\
III - Representação docente dos diversos núcleos de conhecimento ou campos de atuação que compõem o currículo do curso para mandato de dois anos, permitida uma recondução;\\
IV - Representação discente para mandato de um ano, permitida uma recondução.\\
\textsection 1\textordmasculine. No impedimento do Coordenador e do Vice-Coordenador, a presidência do Conselho de Coordenação de Curso de Graduação é exercida por um docente membro do Conselho de Coordenação, previamente designado pelo Coordenador.\\
\textsection 2\textordmasculine. Os representantes dos docentes e dos discentes são indicados por seus pares. 
\end{quote}

Por sua vez, a composição da coordenação do curso é definida pelo Artigo 90 do  Regimento Geral das Coordenações de Cursos de Graduação, como segue:

\begin{quote}
Art. 90. A Coordenação de Curso de Graduação é composta por:\\
I - Coordenador de Curso;\\
II - Vice-Coordenador de Curso;\\
III - Secretário de Curso.\\
\textsection 1\textordmasculine. Cabe ao Coordenador superintender e coordenar as atividades do Curso de Graduação, de acordo com as diretrizes do Conselho de Coordenação.\\
\textsection 2\textordmasculine. Cabe ao Vice-Coordenador substituir o Coordenador do Curso de Graduação em suas faltas e impedimentos.\\
\textsection 3\textordmasculine. No impedimento do Coordenador e/ou do Vice-Coordenador, as funções da Coordenação de Curso de Graduação são atribuídas a um docente membro do Conselho de Coordenação, previamente designado pelo Coordenador.
\end{quote}

O Conselho de Coordenação de Curso é o responsável por deliberar ou não as ações relativas ao Curso, seguindo as normas estabelecidas pelo Regulamento Geral das Coordenações de Cursos de Graduação da UFSCar. Especificamente, o Artigos 93 e 94 do Regimento definem as competências do Conselho e da Coordenação do curso, respectivamente.




\section{Núcleo Docente Estruturante}

O Núcleo Docente Estruturante (NDE), regido pelos Artigos 98 a 110 do Regimento Geral dos Cursos de Graduação da UFSCar, é um órgão consultivo e propositivo responsável pelo processo de concepção, avaliação e atualização do Projeto Pedagógico do Curso. Para isso, o NDE é reponsável pela análise dos resultados de avaliações internas e externas, a fim de propor melhorias ao Conselho de Coordenação do Curso para que o Projeto Pedagógico do Curso possa ser aperfeiçoado.

Além disso, cabe ao NDE zelar pela qualidade da formação do profissional e Diretrizes Curriculares Nacionais para o curso ou legislação correspondente. Ainda, é obrigação do NDE a proposta de atividades de pesquisa e extensão para atender às necessidades da graduação em relação a novas demandas sociais.

O NDE, em conformidade com o Artigo 100 do Regimento Geral dos Cursos de Graduação, é composto por, pelo menos, seis docentes, sendo:

\begin{itemize}
\item[] I - Coordenador do Curso;
\item[] II - Um mínimo de cinco docentes pertencentes ao corpo docente do curso há pelo menos dois anos.
\end{itemize}




\section{O Coordenador de Núcleo de Conhecimento}
 

Conforme abordado na Seção~\ref{sec:marco_estrutural_areas_conhecimento}, o curso possui sete núcleos de conhecimento:
\begin{enumerate}
\item Fundamentos de Matemática e Estatística;
\item Fundamentos de Ciência da Computação;
\item Algoritmos e Programação;
\item Metodologia e Técnicas da Computação;
\item Sistemas de Computação
\item Formação Multidisciplinar e Humanística;
\item Orientações;
\end{enumerate}

Cada núcleo de conhecimento possui um \textbf{Coordenador de Núcleo de Conhecimento}, que é um professor do curso responsável por promover a integração das disciplinas e atividades didático-pedagógicas na seu núcleo de responsabilidade. O Coordenador de Núcleo é membro representante de núcleo junto ao NDE. 
A principal atribuição do Coordenador de Núcleo é zelar pela constante evolução da qualidade do núcleo de conhecimento que ele representa. 
Para isso, atribuições mais específicas são:

\begin{itemize}

\item Promover incrementos e atividades ligadas à seu núcleo de representação para a melhoria do curso como um todo;

\item Certificar-se que conteúdos, metodologias de ensino, avaliações e carga horária efetiva de cada disciplina estão sendo cumpridos de acordo com o que estabelece o projeto pedagógico. 

\item Trabalhar em conjunto com os demais professores dos respectivos núcleos de conhecimento, recomendando-se que ao menos duas reuniões semestrais ocorram: uma de planejamento do semestre a ser iniciado, e outra para avaliação ao final do semestre letivo;

\item Reportar semestralmente ao NDE e ao Conselho de Curso sobre as ações realizadas no semestre;

\item Zelar por não replicar conteúdo em disciplinas distintas;

\item Integrar projetos de diferentes disciplinas da mesma áre. Por exemplo, disciplinas em semestres consecutivos desenvolver partes de um mesmo projeto. Disciplinas no mesmo semestre podem compartilhar o mesmo projeto e aproveitar a avaliação para as mesmas;

\item Integrar docentes e conteúdos de diferentes disciplinas;
    
\item Criar um ambiente de harmonia, colaboração e cooperação entre os docentes do núcleo;

\item Promover Atividades Inter-núcleos.
    As atividades inter-núcleos exigem a comunicação efetiva entre os coordenadores de núcleo para averiguar a viabilidade de desenvolver projetos que envolvam disciplinas de núcleos diferentes. Esse tipo de projeto é o que mais caracteriza a multidisciplinaridade pois poderia, por exemplo, integrar disciplinas do Núcleo de Matemática e Estatística com disciplinas do Núcleo de Metodologia e Técnicas da Computação.  

\item Auxiliar a integração de diferentes núcleos;

\item Indicar formas de incentivo ao desenvolvimento de linhas de pesquisa e extensão;

\end{itemize}



\section{Administração e Condução do Curso}

O curso de graduação Bacharelado em Ciência da Computação é formado por professores, servidores técnico-administrativos e alunos e conta com a infra-estrutura disponibilizada pela Pró-reitoria de Graduação da UFSCar e pelas instalações do CCET - Centro de Ciência e Tecnologia da UFSCar.

Para que o curso realize sua missão de formar alunos com excelência, é preciso o empenho mútuo de alunos, docentes e servidores técnico-administrativos (TAs).
É imprescindível que todo docente do curso conheça em profundidade o Projeto Pedagógico e zele pelo seu cumprimento na íntegra. Com essa atitude o docente terá conhecimento dos princípios pedagógicos que regem o curso.  Fica a cargo da chefia do Departamento, o estímulo dessa prática dentre seus pares.

O NDE e os Coordenadores de Disciplinas devem trabalhar em conjunto, realizando, obrigatoriamente, o mínimo de uma reunião por semestre. A pauta de convocação da reunião deve ser pública e feita com, no mínimo, 48 horas de antecedência. Fica a critério da Coordenação de Curso estabelecer data e horário para que as reuniões ocorram. 

\section{Processo para Autoavaliação do Curso}

A avaliação dos cursos de graduação da UFSCar é uma preocupação presente na Instituição e considerada de fundamental importância para o aperfeiçoamento dos projetos pedagógicos dos cursos e a melhoria dos processos de ensino e aprendizagem. Desde a publicação da Lei 10.861 de 14 de abril de 2004, que instituiu o Sistema de Avaliação da Educação Superior (SINAES), a Comissão Própria de Avaliação/UFSCar tem coordenado os processos internos de autoavaliação institucional nos moldes propostos pela atual legislação e contribuído com os processos de avaliação dos cursos. 

O sistema de avaliação dos cursos de graduação da UFSCar, implantado em 2011, foi concebido pela Pró-Reitoria de Graduação (ProGrad) em colaboração com a Comissão Própria de Avaliação (CPA) com base em experiências institucionais anteriores, quais sejam: o Programa de Avaliação Institucional das Universidades Brasileiras (PAIUB)	 e o Programa de Consolidação	das Licenciaturas (PRODOCÊNCIA). O PAIUB, iniciado em 1994, realizou uma ampla avaliação de todos os cursos de graduação da UFSCar existentes até aquele momento e o projeto PRODOCÊNCIA/UFSCar, desenvolvido entre os anos de 2007 e 2008, realizou uma avaliação dos cursos de licenciatura dos campi de São Carlos e de Sorocaba. 

A avaliação dos cursos de graduação é feita atualmente por meio de formulários de avaliação, os quais são respondidos pelos docentes da área majoritária de cada curso, pelos discentes e, eventualmente, pelos técnico-administrativos e egressos. Esses formulários abordam questões sobre as dimensões do Perfil Profissional a ser formado pela UFSCar; da formação recebida nos cursos; do estágio; da participação em pesquisa, extensão e outras atividades; das condições didático-pedagógicas dos professores; do trabalho das coordenações de curso; do grau de satisfação com o curso realizado; das condições e serviços proporcionados pela UFSCar; e das condições de trabalho para docentes e técnico-administrativos.

A ProGrad, juntamente com a CPA, são responsáveis pela concepção dos instrumentos de avaliação, bem como da divulgação do processo e do encaminhamento dos resultados às respectivas coordenações de curso. A operacionalização desse processo ocorre por meio da plataforma eletrônica Sistema de Avaliação Online (SAO), desenvolvida pelo Centro de Estudos de Risco (CER) do Departamento de Estatística.

Cada Conselho de Coordenação de Curso, bem como seu Núcleo Docente Estruturante (NDE), após o recebimento dos resultados da avaliação, analisam esses resultados para o planejamento de ações necessárias, visando a melhoria do curso.

Deve ser ressaltado que a Coordenação do Curso de Ciência da Computação sempre atuou fortemente não apenas na promoção do curso junto à comunidade externa, mas também no acompanhamento dos egressos. O contato contínuo e intenso com os egressos fornece valiosas informações sobre a colocação dos mesmos no mercado de trabalho e provê informações importantes sobre a formação profissional recebida durante o curso e sua efetividade perante o mercado profissional.

