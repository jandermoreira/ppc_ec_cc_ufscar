
\subsection{Projeto Integrador Extensionista}\label{sec:pie-introducao}
\cerricomentario{Aqui não tinha que ser uma nova seção?}


Conforme a Lei Nº 13.005 de 25 de junho de 2014, a qual aprova o Plano Nacional de Educação (PNE) para o período de 2014 a 2024, devem-se intensificar as atividades de extensão nos cursos de graduação, sendo recomendado que 10\% da carga horária do curso seja destinada a atividades de extensão. Com essa motivação, este PPC incentiva aos(às) discentes a participação em projetos que permitam aos(às) mesmos(as) aplicar os conhecimentos adquiridos no curso e, ao mesmo tempo, atenda demandas da comunidade externa, caracterizando-se como projetos extensionistas~\cite{Brasil2014}.\cerricomentario{Essa referencia não está no .bib}

O Projeto Integrador Extensionista (PIE) consiste de uma atividade interdisciplinar, proposta aos discentes que tenham cursado no mínimo 48 créditos {EU ACHO EMLHOR FALAR EM TERMOS DAS DISCIPLINAS QUE O ALUNO DEVE TER CURSADO DO QUE EM NÚMERO DE CRÉDITOS} do curso de Bacharelado em Ciência da Computação do Departamento de Computação da Universidade Federal de São Carlos. O projeto deve ser desenvolvido em grupo durante o período de, no mínimo, 1 (um) ano e contabiliza 300 horas em atividades de extensão para seus participantes, exigindo dos mesmos uma dedicação de 12 horas semanais {NAO SEI SE FAZ SENTIDO FALAR ISSO. SE O ALUNO RESOLVER FAZER EM 3 ANOS, ISSO LHE EXIGIRÁ UM NÚMERO BEM MENOR DE HORAS DURANTE A SEMANA} nas atividades do projeto.

O objetivo do PIE é propiciar aos discentes um embasamento prático dos conceitos teóricos adquiridos por meio dos conteúdos programáticos ministrados em sala de aula.  Tais projetos devem, obrigatoriamente, atenderem demandas externas ao departamento, inclusive atendendo a demandas de empresas, caracterizando-se como projetos de extensão, sendo supervisionados por um docente orientador da UFSCar.

Conforme ressaltado em~\cite{SENAC}, a Metodologia de Projetos consiste em uma alternativa pedagógica que privilegia a relação dialógica e aprendizagem coletiva. Leva em consideração o princípio de que se aprende em comunhão, em experiências e vivências de construção colaborativa, ao assumir responsabilidades em ações conjuntas e promover o protagonismo do discente diante de situações problematizadoras. O aprendizado se dá pela experiência proporcionada durante o desenvolvimento do projeto, ou seja, aprende-se problematizando, pesquisando, testando hipóteses, tomando decisões e agindo em equipe para atingir os objetivos.

A intenção é que o PIE aproxime-se da forma como os discentes atuarão na vida profissional: agindo positivamente, na solução de problemas técnicos, sociais, políticos e econômicos, objetivando o desenvolvimento socioeconômico nas perspectivas local, regional, nacional e/ou internacional.

Além disso, o PIE também pretende tornar os processos de ensino e de
aprendizagem mais dinâmicos, interessantes, significativos, reais e atrativos para os discentes, englobando conteúdos e conceitos essenciais à compreensão da realidade social em geral e, em particular, do mundo do trabalho, assim como, suas inter-relações, sem a imposição de conteúdos e conceitos, de forma fragmentada e autoritária.

\subsection{Objetivos}

\subsubsection{Objetivo Geral}

Proporcionar aos discentes a oportunidade de desenvolver um projeto que integre o conhecimento das diferentes disciplinas ministradas no curso de Ciência da Computação do DC/UFSCar e atenda as demandas da comunidade externa ao DC.

\subsubsection{Objetivos Específicos}

\begin{itemize}
    \item Propiciar aos discentes identificar com mais clareza a relação existente entre as disciplinas cursadas, além de promover cada vez mais a interação dos conteúdos apresentados;
    
    \item Propiciar aos discentes compreender quais conhecimentos e tecnologias podem ser combinadas e adequadas para a resolução de cada problema;
    
    \item Possibilitar aos discentes fundamentos e aspectos metodológicos iniciais para realização de trabalhos profissionais, estimulando o espírito cooperativo e sensibilizando-o para a importância do trabalho em equipe;
    
    \item Incentivar aos discentes na identificação de problemas que afetem a comunidade externa ao DC e que possam ser resolvidos por meio do uso de técnicas computacionais;
        
    \item Possibilitar aos discentes a troca de experiências e o desenvolvimento da capacidade de organização para o desenvolvimento de trabalho em equipe;
    
    \item Incentivar aos discentes a busca por inovação e o registro de propriedade intelectual no Instituto Nacional de Proteção Intelectual (INPI), com apoio da Agência de Inovação da FAI;
    
    \item Propiciar aos discentes o desenvolvimento de habilidades de comunicação, escrita e apresentação por meio da defesa do projeto para uma banca avaliadora.
\end{itemize}


\subsection{Da Oferta}

Serão lançados editais com periodicidade mínima anual contendo:

\begin{itemize}
    \item Texto caracterizando PIE e diretrizes gerais para o desenvolvimento do PIE;
    \item Chamada e formato para a submissão de propostas PIE;
    \item Datas para submissão, julgamento, divulgação e homologação dos projetos habilitados;
    \item Período para a inscrição das equipes nos projetos habilitados;
    \item Divulgação dos projetos e equipes a serem desenvolvidos no período.
\end{itemize}

Os projetos integradores devem ser realizados em grupos de discentes. Os projetos podem ser propostos por professores, discentes e empresas, sendo obrigatório que um docente da UFSCar atue como orientador do projeto.

%Existirá a preocupação em submeter e habilitar uma quantidade mínima de projetos para atender a demanda de discentes.

\subsection{Das Atividades}

Os Projetos Integradores Extensionistas devem, obrigatoriamente, empregar conhecimentos de 3 (três) ou mais disciplinas e se enquadrarem como extensão, ou seja, possuírem potencial de atingir a comunidade externa ao Departamento de Computação.

Os PIEs poderão contemplar práticas e/ou atividades como:

\begin{itemize}
    \item Projetos de pesquisa aplicada;
    \item Elaboração de diagnósticos empresariais;
    \item Projetos técnicos;
    \item Desenvolvimento de materiais didáticos e instrucionais;
    \item Desenvolvimento de protótipos;
    \item Desenvolvimento de aplicativos e de produtos;
    \item Projetos de inovação tecnológica;
    \item Outras modalidades reconhecidas como relevantes pela Coordenação de Curso.
\end{itemize}

Para os projetos com potencial de inovação tecnológica, sugere-se que a equipe do projeto avalie a possibilidade de registrar o mesmo como registro de software no INPI ou divulgar como código-fonte aberto (repositórios). A Agência de Inovação da UFSCar pode orientar as equipes em como proceder para efetuar o registro.

\subsubsection{Da Visão Geral do Processo}

O processo de submissão, avaliação e acompanhamento de PIEs contém as seguintes atividades:

\begin{enumerate}
    
\item Submissão da proposta conforme cronograma previsto em edital específico;

\item Caracterização da proposta como integrador e extensão. Haverá uma comissão definida em edital que avaliará se o projeto se caracteriza como integrador e extensão. A comissão irá emitir parecer sobre os projetos submetidos no semestre;

\item Cadastramento dos projetos aprovados como atividade de extensão. Para os projetos aprovados, o orientador deve cadastrar o projeto submetido como atividade de extensão, dentro de programa de extensão específico, previamente cadastrado pelo coordenador do curso;

\item Acompanhamento da execução do projeto. O orientador deve acompanhar a execução do projeto e realizar avaliação individual e em grupo dos discentes participantes;

\item Elaboração, sob a orientação do professor, de um relatório final, conforme modelo disponibilizado pela Coordenação de Curso.

\item Apresentação dos resultados do Projeto Integrador para banca examinadora, que poderá aprovar ou reprovar o resultado final obtido. 

\item Validação dos créditos pelos órgãos competentes.

\end{enumerate}

\subsubsection{Das Propostas}

Poderão participar do projeto integrador os discentes devidamente matriculados e/ou rematriculados no curso de Bacharelado em Ciência da Computação, cumpridos os requisitos estabelecidos de terem cursado no mínimo 48 créditos.

A proposta para o PIE deve conter:

\begin{enumerate}
    \item Capa
    \begin{itemize}
        \item Título
        \item Núcleos de Conhecimento/Disciplinas Contempladas
        \item Sugestão de orientadores(as) ou indicar o orientador
        \item Sugestão do tamanho da equipe necessária para o projeto (limite mínimo e máximo) ou indicar a equipe
    \end{itemize}

    \item Contextualização;
    \item Caracterização do problema;
    \item Justificativa;
    \item Objetivos;
    \item Fundamentação Teórica (explicitando o vínculo com os conteúdos das disciplinas envolvidas);
    \item Metodologia;
    \item Cronograma, incluindo:
    \begin{itemize}
        \item Atividades previstas, considerando a dedicação de 12 horas semanais por discente;
        \item Previsão de entrega dos produtos do projeto;
        \item Datas de todas as reuniões presenciais e virtuais.
    \end{itemize}
    \item Bibliografia.
\end{enumerate}

\subsection{Das Obrigações do Orientador}

As atividades relativas ao Projeto Integrador serão supervisionadas pelo professor orientador do Projeto Integrador. Ele deve:

\begin{enumerate}
    \item Cadastrar o projeto como atividade de extensão;
    \item Verificar o andamento das atividades de acordo com o cronograma submetido e aprovado;
    \item Orientar os discentes na condução das atividades;
    \item Registrar os encontros presenciais e virtuais.
\end{enumerate}


%As orientações presenciais dos trabalhos acontecerão com periodicidade estabelecida em calendário, proposto semestralmente e aprovado pela Coordenação de Curso.

%Os discentes são responsáveis por agendar com o professor orientador as datas para orientação, dentro do período estabelecido no calendário do período letivo e cronograma do projeto. 

As orientações presenciais devem ser registradas em formulário próprio, fornecido pela supervisão de Projeto Integrador e assinadas pelo professor orientador e pelos discentes presentes.

\subsection{Das Obrigações dos Discentes}

Aos discentes cabe a realização das atividades do projeto, de acordo com o cronograma submetido e aprovado.

Além disso, os discentes devem:

\begin{itemize}
    \item Comparecer às reuniões presenciais e virtuais de acordo com o cronograma submetido e aprovado;
    \item Dedicar 12 horas semanais ao projeto.

\end{itemize}

\subsection{Da Avaliação}

A avaliação será composta de duas etapas:

\begin{enumerate}

\item A primeira etapa consiste em uma avaliação individual e contínua, e ficará a cargo do orientador. Nesta etapa, serão considerados assiduidade e desempenho individual de cada discente. Os discentes reprovados nesta etapa serão desligados do projeto e não poderão ter os créditos convalidados;

\item A segunda etapa consiste em uma avaliação do projeto como um todo, que deve ser apresentado em forma textual (relatório final) e apresentação oral mediante uma banca examinadora. Nesta etapa, a banca examinadora irá avaliar o cumprimento da proposta aprovada, com atenção especial para o enfoque obrigatório de projeto integrador e extensionista.

\end{enumerate}

Em termos de assiduidade, o aluno deve cumprir no mínimo 75\% de frequência nas atividades do projeto.

A banca examinadora deve ser composta por três membros, sendo pelo menos um docente do DC e outros dois membros com capacidade para avaliar projeto. Cabe à Coordenação de Curso a aprovação da bancas.

Ao final da segunda etapa, a banca examinadora emitirá um parecer indicando se o projeto foi aprovado ou reprovado. Em caso de aprovação, os créditos equivalentes a 300 horas em atividades de extensão serão atribuídos a cada discente membro da equipe.

Em caso de reprovação, o projeto poderá ser reapresentado, mediante solicitação por meio de formulário próprio, para a mesma banca examinadora.




O discente será reprovado automaticamente no Projeto Integrador quando
ocorrer pelo menos um dos itens abaixo:
\begin{enumerate}
    \item O trabalho não cumprir o objetivo proposto;
    \item O trabalho for plágio;
    \item O trabalho não for desenvolvido pelos discentes;
    \item O trabalho estiver fora das normas técnicas exigidas pela Instituição;
    \item O trabalho não for entregue no prazo estabelecido;
    \item Não for comprovada a presença de pelo menos 75\% (setenta e cindo por cento) nas atividades do projeto.
\end{enumerate}

A ocorrência de qualquer dos itens anteriores deve ser comunicada
pelo professor orientador à Coordenação de Curso, que após avaliar a situação emitirá um parecer final.


\subsubsection{Das Obrigações da Coordenação de Curso}

Para garantir a oferta contínua de projetos em andamento, a coordenação de curso irá, a cada ano letivo, indicar dez docentes do Departamento de Computação que deverão submeter ao menos uma proposta de PIE naquele ano.

A coordenação também será responsável por organizar e divulgar os editais de candidatura, aprovar as bancas de avaliação e validar os créditos.