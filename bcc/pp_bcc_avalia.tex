\section{Princípios gerais de avaliação da aprendizagem}

%\cerricomentario{Essa seção foi copiada do PPC antigo}


A avaliação da aprendizagem, concebida como um processo contínuo de acompanhamento do
desempenho dos estudantes, é feita por meio de procedimentos, instrumentos e critérios adequados aos
objetivos, conteúdos e metodologias relativas a cada atividade curricular. É um elemento essencial de
reordenação da prática pedagógica, pois permite um diagnóstico da situação e indica formas de
intervenção no processo, com vistas à aquisição do conhecimento, à aprendizagem e à reflexão
sobre a própria prática, tanto para os estudantes como para os professores.

Compreender a avaliação como diagnóstico significa ter o cuidado constante de observar, nas
produções e manifestações dos estudantes, os sinais ou indicadores de sua situação de aprendizagem.
Na base desta avaliação está o caráter contínuo de diagnóstico e acompanhamento, sempre tendo
em vista o progresso dos estudantes e sua aproximação aos alvos pretendidos a partir de sua situação~real.

A avaliação presente no curso de Bacharelado em Ciência da Computação da UFSCar está
fundamentada na concepção de que o que se pretende não é simplesmente medir aprendizagem
segundo escalas ou valores, mas interpretar a caminhada dos estudantes com base nos registros e
apreciações sobre seu trabalho. Além disso, segue normas internas sem, no entanto, limitar a
liberdade de cada professor. As avaliações são realizadas em vários momentos e não se restringem
somente a uma avaliação de conteúdos. Há avaliações em grupo e individuais, trabalhos, listas de
exercícios, participação, interesse, pontualidade e assiduidade.

Entendida desta maneira, a avaliação só tem sentido quando articulada ao projeto
pedagógico institucional, que lhe confere significado, e enquanto elemento constituinte do processo
educativo, como instrumento que objetiva determinar novos rumos ou corrigir o rumo atual.
No que se refere aos aspectos administrativos presentes na sistemática de avaliação de
rendimento dos estudantes, o  Bacharelado em Ciência da Computação da UFSCar segue os
preceitos dos artigos de 18 a 28 do  Regimento Geral dos Cursos de Graduação da UFSCar~\cite{Regimento-Geral-CursosGraduacao-UFSCar}, homologado pela Resolução ConsUni n\textordmasculine~867, de 27 de outubro de 2016.  

%\diegocomentario{Eu acho que eu deixaria esse parágrafo para a parte de formas de avaliação da aprendizagem (próxima seção).}

Em síntese, a avaliação presente no curso de Bacharelado em
Ciência da Computação da UFSCar tem as seguintes funções: 

\begin{itemize}
    \item acompanhar o desenvolvimento
das disciplinas do curso e diagnosticar aspectos que devem ser mantidos ou reformulados em cada uma delas;

    \item desenvolver, entre os docentes e discentes, uma postura favorável à avaliação, enquanto instrumento das práticas educativas;
    
    \item focalizar a produção do conhecimento crítico e transformador;
    
    \item avaliar não apenas o conhecimento adquirido, mas também as competências profissionais, por meio do desenvolvimento de trabalhos, projetos, estágios, etc.
\end{itemize}


\section{Formas de Avaliação da Aprendizagem}


A UFSCar, por meio do Regimento Geral dos Cursos de Graduação de setembro de 2016, estabeleceu normas para a sistemática de avaliação do desempenho dos estudantes, estabelecendo que a avaliação é parte integrante e indissociável do ato educativo e deve vincular-se, necessariamente, ao processo de \lq\lq ação-reflexão-ação\rq\rq, que compreende o ensinar e o aprender nas disciplinas e atividades curriculares dos cursos, na perspectiva de \textit{formar profissionais cidadãos capazes de uma ação interativa e responsável na sociedade atual}, caracterizada por sua constante transformação. 

A avaliação contínua propicia o acompanhamento da evolução do estudante, bem como através desta se torna possível diagnosticar o conhecimento prévio dos estudantes, refletir sobre os resultados obtidos e construir estratégias de ensino individuais ou coletivas de superação das dificuldades apresentadas.

De acordo com o Artigo 19 do Regimento Geral dos Cursos de Gradução, os Planos de Ensino das disciplinas devem descrever, de forma minuciosa, os procedimentos, instrumentos e critérios de avaliação,
diferenciados e adequados aos objetivos, conteúdos e metodologias relativas a cada disciplina. 
É necessário proporcionar aos estudantes instrumentos de avaliação diferenciados e adequados aos objetivos, multiplicando as suas oportunidades de aprendizagem e diversificando os métodos utilizados. Assim, permite-se que os estudantes apliquem os conhecimentos que adquirem, exercitem e controlem eles próprios a aprendizagem e o desenvolvimento das competências, recebendo comentários e opiniões frequentes sobre as dificuldades e progressos alcançados. 

O Regimento Geral dos Cursos de Gradução prevê, no item II do Artigo 19, a aplicação de procedimentos/instrumentos de avaliações em, pelo menos, três datas distribuídas no período letivo para cada atividade curricular, cabendo ao professor divulgar dois terços dos resultados dos instrumentos aplicados até trinta dias antes do final do período letivo.

A escolha dos métodos e instrumentos de avaliação depende de vários fatores: das finalidades, do objeto de avaliação, da área disciplinar, do tipo de atividade, do contexto, e dos próprios avaliadores. Propõe-se que, além da prova individual com questões dissertativas, outras formas de avaliação sejam consideradas, tais como:

\begin{itemize}
\item Trabalhos individuais ou coletivos;
\item Atividades de culminância (projetos, monografias, seminários, exposições etc).
\end{itemize}

Tais instrumentos são utilizados como critério de aprovação do estudante, de acordo com o Artigo 20 do Regimento Geral dos Cursos de Gradução, como segue:

\begin{quote}
Art. 20. O estudante regularmente inscrito em atividades curriculares é considerado aprovado quando obtiver, simultaneamente:\\
I - Frequência igual ou superior a setenta e cinco por cento das aulas e/ou das atividades acadêmicas curriculares efetivamente realizadas;\\
II - Desempenho mínimo equivalente à nota final igual ou superior a 6 (seis) ou conceito equivalente. 
\end{quote}

Outro aspecto relevante do Regimento Geral dos Cursos de Gradução trata sobre o Processo de Avaliação Complementar (PAC), estabelecido pelo Artigo 22 do regimento. Caso o estudante não obtenha nota final suficiente para sua aprovação, o PAC poderá ser utilizado como recurso para recuperação de conteúdos. Para isso, é necessário que a o estudante obtenha frequência igual ou superior a setenta e cinco por cento e nota final igual ou superior a cinco.

O Regimento Geral dos Cursos de Gradução ainda define os prazos para realização do PAC, conforme segue:

\begin{quote}
Art. 24. O Processo de Avaliação Complementar (PAC) deve ser realizado em período subsequente ao término do período regular de oferecimento da atividade curricular. \\
Parágrafo Único. A realização do processo de que trata o caput pode prolongar-se até o 35\textordmasculine~(trigésimo quinto) dia letivo do período subsequente para atividades curriculares de duração semestral e até 70\textordmasculine~(septuagésimo) dia letivo do período subsequente para atividades curriculares de duração anual, não devendo incluir atividades em horários coincidentes com outras atividades curriculares realizadas pelo estudante.
\end{quote}


% \diegocomentario{Isso parece um pouco desencaixado, embora interessante. Ficaria na próxima seção?} 

% \valtercomentario{Sim, interessante, mas alguém teria que pegar e explicar melhro cada ponto deste. Estou achando mais viável remover.}

% Neste projeto, propõem-se ainda ações e procedimentos que contribuam para a avaliação geral do Curso de Bacharelado em Ciência da Computação:

% \begin{itemize}
% \item Certificar a capacidade profissional de forma coletiva, além da individual;
% \item Avaliar não apenas o conhecimento adquirido, mas também as competências profissionais;
% \item Diagnosticar o uso funcional e contextualizado dos conhecimentos;
% \item \cerri{}{Acompanhar o desenvolvimento
% das disciplinas do curso e diagnosticar aspectos que devem ser mantidos ou reformulados em cada
% uma delas;}
% \item \cerri{}{Desenvolver, entre os docentes e discentes, uma postura favorável à avaliação, enquanto instrumento das práticas educativas.}
% \end{itemize}

% \subsection{Adiantamento de conhecimento}

% Um dos objetivos deste curso, conforme descrito no Capítulo \ref{chap:intro}, é proporcionar experiências de aprendizagem onde o estudante desenvolve atividades complementares fora da sala de aula, seja em forma de Iniciação Científica, Projeto Integrador, ou mesmo projetos pessoais. Nessas experiências, é esperado que o aprendizado do estudante extrapole o domínio de conhecimento de um único projeto ou disciplina. Como resultado disso, é comum que o estudante desenvolva competências, por conta própria ou sob orientação docente, que são cobertas, parcialmente ou totalmente, em disciplinas que ainda não tenha cursado.

% Ao mesmo tempo, a Portaria GR/UFSCar n\textordmasculine~522/06, em seu Art. 2\textordmasculine~, define que as avaliações devem, entre outras funções, ``diagnosticar o conhecimento prévio dos estudantes''. Neste sentido, o curso de Bacharelado em Ciência da Computação prevê a possibilidade do adiantamento do conhecimento, para que as competências adquiridas pelos estudantes nessas experiências de aprendizagem fora de sala de aula possam ser convertidas em frequência, nota ou créditos, totais ou parciais, estimulando assim uma maior busca por um aprendizado ativo e focado em atividades de cunho prático.

% \alterar{Nesta seção}{Na próxima seção} são descritos os termos para o adiantamento de conhecimento para o curso de Bacharelado em Ciência da Computação.

% \subsection{Condições para solicitação de exame de adiantamento de conhecimento}

% Se o estudante comprovar domínio de conhecimento de conteúdo relativo a uma parte ou à totalidade de qualquer das disciplinas disponíveis para exame de adiantamento de conhecimento (listadas na próxima seção), poderá requerê-lo a qualquer momento do curso. Caso a requisição seja aprovada, o estudante poderá ser dispensado parcialmente ou totalmente da disciplina.

% O exame de adiantamento de conhecimento é individual, e deve ser solicitado pelo estudante à Coordenação do Curso. A solicitação deve ser circunstanciada e indicar o domínio de conhecimento de conteúdo. É imprescindível que esse domínio de conhecimento tenha sido adquirido como resultado de alguma atividade de cunho prático realizada antes (sem limite de tempo??? \cerricomentario{Sugestão do Diego: 2 anos, assim como é na pós mesmo}) ou durante o curso de graduação, e que se enquadre nas seguintes situações:

% \begin{itemize}
%     \item Projeto de iniciação científica, com ou sem bolsa;
%     \item Desempenho de atividade profissional, remunerada ou não remunerada;
%     \item Projeto acadêmico, realizado em disciplina da graduação;
%     \item Projeto integrador, conforme descrito na Seção \ref{sec:pie-introducao}.
%     \item Outras situações???? \cerricomentario{Sugestão do Diego: alguma coisa com grupos acadêmicos (PET, CATi, etc), depois de ter completado os créditos complementares?}
% \end{itemize}

% É também imprescindível que os resultados da atividade realizada estejam disponíveis para análise e comprovação.

% Ao realizar a solicitação, o estudante deve indicar claramente para quais disciplinas e quais conteúdos e competências dentro das disciplinas o exame deve cobrir. O projeto pedagógico do curso ficará disponível para consulta do estudante e deve ser utilizado como referência para essa indicação.

% \subsection{Disciplinas disponíveis para adiantamento de conhecimento}

% Somente podem ser solicitados exames de adiantamento de conhecimento para as disciplinas listadas a seguir:

% \begin{itemize}
%     \item Construção de compiladores
%     \item Computação gráfica
%     \item Banco de dados
%     \item ...
% \end{itemize}

% \subsection{Sobre o exame de adiantamento de conhecimento}

% Uma vez aprovado o pedido de adiantamento de conhecimento, por parte da Coordenação do Curso, o processo será encaminhado ao coordenador da(s) disciplina(s) citadas (agora temos essa figura no curso, não?), que providenciará todos os atos e procedimentos para a convocação do estudante, elaboração e aplicação do exame, e emissão de parecer justificado.

% O formato e conteúdo do exame serão definidos pelo coordenador da(s) disciplina(s), ou por uma comissão de um ou mais docentes por ele indicados. O formato e conteúdo irão depender da natureza da solicitação, podendo versar sobre os tópicos indicados pelo estudante, mas também outros tópicos que possam ser julgados correlatos. O exame deve obrigatoriamente incluir uma demonstração de cunho prático, mas pode também contemplar atividades avaliativas teóricas, por escrito e/ou em formato oral.

% O parecer emitido após o exame deve indicar e justificar claramente e de forma objetiva se a solicitação foi aceita, total ou parcialmente. 

% \subsection{Sobre os resultados do exame de adiantamento de conhecimento}

% São admitidas as seguintes possibilidades de resultado a serem indicados no parecer:

% \begin{itemize}

% \item O estudante é completamente dispensado da disciplina. Neste caso, a disciplina será lançada no histórico como APROVEITAMENTO POR EQUIVALÊNCIA (não sei como isso pode ser feito);

% \item O estudante é dispensado de cursar parte da disciplina. Neste caso, o parecer deve indicar claramente de quais conteúdos o estudante está dispensado. O estudante deve posteriormente se matricular na disciplina e entregar o parecer ao professor responsável no início do semestre letivo. O professor responsável pela oferta irá decidir e comunicar ao estudante de quais aulas ou provas o estudante estará dispensado, com base no parecer;

% \item O estudante é reprovado no exame. Neste caso, ele não poderá solicitar novamente adiantamento de conhecimento para as disciplinas envolvidas (para evitar abusos?).

% \end{itemize}

% Uma vez emitido, o parecer deve ser aprovado pelo conselho de curso, caso contrário não terá validade.

% A ausência não justificada ou a não realização pelo estudante do exame de adiantamento de conhecimento implicará em reprovação.