
\chapter {Plano de Implantação}\label{sec:implantacao}

\section {Infraestrutura necessária para o funcionamento do curso}

\subsection{Corpo docente e técnico}

O Curso de Bacharelado em Ciência da Computação é atendido principalmente pelo Departamento de Computação (DC), que conta atualmente com 43 docentes (listados na Tabela \ref{table:CorpoCocente}) e 8 técnicos administrativos (listados na Tabela \ref{table:CorpoTecnico}). 

\begin{table}[!htb]
\centering
\begin{tabular}{|l|l|l|} \hline
\textbf{Nome} & \textbf{Titulação} & \textbf{Vínculo/Dedicação} \\ \hline
Alexandre Luis Magalhães Levada & Doutor & Efetivo/40h DE \\ \hline
Auri Marcelo Rizzo Vincenzi & Doutor & Efetivo/40h DE \\ \hline
Cesar Augusto Camilo Teixeira & Doutor & Efetivo/40h DE \\ \hline
Cesar Henrique Comin & Doutor & Efetivo/40h DE \\ \hline
Daniel Lucrédio & Doutor & Efetivo/40h DE \\ \hline
Delano Medeiros Beder &Doutor & Efetivo/40h DE \\ \hline
Diego Furtado Silva & Doutor & Efetivo/40h DE \\ \hline
Ednaldo Brigante Pizzolato & Doutor & Efetivo/40h DE \\ \hline
Edílson Reis Rodrigues Kato & Doutor & Efetivo/40h DE \\ \hline
Emerson Carlos Pedrino & Doutor & Efetivo/40h DE \\ \hline
Estevan Rafael Hruschka Junior & Doutor & Efetivo/40h DE \\ \hline
Fabiano Cutigi Ferrari & Doutor & Efetivo/40h DE \\ \hline
Fredy João Valente & Doutor & Efetivo/40h DE \\ \hline
Helena de Medeiros Caseli & Doutor & Efetivo/40h DE \\ \hline
Helio Crestana Guardia & Doutor & Efetivo/40h DE \\ \hline
Heloisa de Arruda Camargo & Doutor & Efetivo/40h DE \\ \hline
Hermes Senger & Doutor & Efetivo/40h DE \\ \hline
Jander Moreira & Doutor & Efetivo/40h DE \\ \hline
Joice Lee Otsuka & Doutor & Efetivo/40h DE \\ \hline
Junia Coutinho Anacleto Silva & Doutor & Efetivo/40h DE \\ \hline
Kelen Cristiane Teixeira Vivaldini & Doutor & Efetivo/40h DE \\ \hline
Luciano de Oliveira Neris & Doutor & Efetivo/40h DE \\ \hline
Marcela Xavier Ribeiro & Doutor & Efetivo/40h DE \\ \hline
Marcio Merino Fernandes & Doutor & Efetivo/40h DE \\ \hline
Mário César San Felice & Doutor & Efetivo/40h DE \\ \hline
Mauricio Fernandes Figueiredo & Doutor & Efetivo/40h DE \\ \hline
Marilde Terezinha Prado Santos & Doutor & Efetivo/40h DE \\ \hline
Murillo Coelho Naldi & Doutor & Efetivo/40h DE \\ \hline
Murillo Rodrigo Petrucelli & Doutor & Efetivo/40h DE \\ \hline
Orides Morandin Junior & Doutor & Efetivo/40h DE \\ \hline
Paulo Rogerio Politano & Doutor & Efetivo/40h DE \\ \hline
Paulo Matias & Doutor & Efetivo/40h DE \\ \hline
Renato Bueno & Doutor & Efetivo/40h DE \\ \hline
Ricardo Cerri & Doutor & Efetivo/40h DE \\ \hline
Ricardo José Ferrari & Doutor & Efetivo/40h DE \\ \hline
Ricardo Menotti & Doutor & Efetivo/40h DE \\ \hline
Ricardo Rodrigues Ciferri & Doutor & Efetivo/40h DE \\ \hline
Roberto Ferrari Junior & Doutor & Efetivo/40h DE \\ \hline
Sandra Abib & Doutor & Efetivo/40h DE \\ \hline
Sergio Donizetti Zorzo & Doutor & Efetivo/40h DE \\ \hline
Valter Vieira de Camargo & Doutor & Efetivo/40h DE \\ \hline
Vânia Paula de Almeida Neris & Doutor & Efetivo/40h DE \\ \hline
Wanderley Lopes de Souza & Doutor & Efetivo/40h DE \\ \hline
\end{tabular}
\caption{Corpo docente atuante no curso de Bacharelado em Ciência da Computação.}
\label{table:CorpoCocente}
\end{table}

\begin{table}[htb!]
\centering
\begin{tabular}{|l|l|} \hline
\textbf{Nome} & \textbf{Atividade} \\ \hline
Carlos Alberto Ferro Gobato & Técnico em eletrônica \\ \hline
Darli José Morcelli & Assistente administrativo \\ \hline
Jorgina Vera de Moraes & Servente de limpeza \\ \hline
Mariana Massimino Feres & Técnica de tecnologia da informação \\ \hline
Paulo Cesar Donizeti Paris & Técnico de laboratório \\ \hline
Willian Câmara Corrêa & Técnico de laboratório \\ \hline
Ivan Rogério da Silva & Assistente administrativo \\ \hline
Nicanor José Costa & Assistente administrativo \\ \hline
\end{tabular}
\caption{Corpo técnico-administrativo atuante no curso de Bacharelado em Ciência da Computação.}
\label{table:CorpoTecnico}
\end{table}



Ao longo do período de quatro anos, os departamentos de Matemática, Estatística e Letras colaboram na formação do egresso com o oferecimento de disciplinas obrigarórias.


\subsection{Espaço físico}

O Departamento de Computação dispõe de 6 laboratórios de ensino para graduação e 1 laboratório de informática para graduação. Dois desses laboratórios de ensino são equipados para o ensino relacionado às áreas de arquiterura de computadores, microprocessadores, microcontroladores e lógica digital, contando com equipamentos como osciloscópios, geradores de função e plataformas de prototipagem . Todos os laboratórios de ensino contam microcomputadores, projetor multimídia e ar-condicionado.  

\begin{table}[htb!]
\centering
\begin{tabular}{|l|l|l|} \hline
\textbf{Laboratório} & \textbf{Atividade principal}& \textbf{Capacidade} \\ \hline
Laboratório de ensino 1    & Hardware &  30 alunos\\ \hline
Laboratório de ensino 2    & Programação e desenvolvimento &  40 alunos\\ \hline
Laboratório de ensino 3    & Programação e desenvolvimento &  40 alunos\\ \hline
Laboratório de ensino 4    & Programação e desenvolvimento &  40 alunos\\ \hline
Laboratório de ensino 5    & Hardware e Lógica digital &  30 alunos\\ \hline
Laboratório de ensino 6    & Programação e desenvolvimento &  40 alunos\\ \hline
Laboratório de Informática & Uso geral &  40 alunos \\ \hline
\end{tabular}
\caption{Laboratórios do Departamento de Computação voltados para o ensino da graduação.}
\label{table:Labs}
\end{table}
 
 As configurações de hardware e software dos laboratórios é atualizada constantemente, e estão disponíveis em http://www.dc.ufscar.br/suporte/laboratorios-de-ensino-le.
 
 O Departamento de Computação também conta com um auditório para 80 pessoas.


%\subsection{Software}
%\begin{itemize}
%\item Estabelecer metodologia baseada em máquina virtuais
%\item Bom p/ professores, alunos e manutenção dos labs
%\item Moodle:  Padronização NOVO Moodle UFSCar
%\item EAD: Suporte para implantação no curso em até 20\%:
%\begin{itemize}
%\item  Disciplinas totalmente em EAD
%\item  Disciplinas parcialmente em EAD
%\end{itemize}
%\end{itemize}
