

O Estágio tem por objetivo que o aluno adquira experiência na área profissional bem como coloque em prática os conhecimentos teóricos adquiridos no decorrer do curso, preparando-o para o exercício futuro da profissão. Para isso, a inserção na empresa é necessária possibitando-lhe o contato com situações, problemas, processos reais, bem como com processos de tomada de decisão e realização de tarefas, complementando a sua formação teórica.

No Curso de Ciência da Computação, o Estágio é estruturado conforme o estabelecido na Lei nº. 11.788/2008, de 25 de setembro de 2008 da Presidência da República que regulamenta os estágios, e pelo no Regimento Geral dos Cursos de Graduação da UFSCar, estabelecido em setembro de 2016 que dispõe sobre a realização de estágios de estudantes dos Cursos de Graduação da Universidade Federal de São Carlos. De acordo com o no Regimento Geral dos Cursos de Graduação da UFSCar, os estágios na UFSCar podem ser obrigatórios ou não obrigatórios. 

Na realização do estágio o estudante receberá orientação de um professor do Departamento de Computação o qual o auxiliará em questões não previstas em sua grade curricular sempre que as partes julgarem necessário. O orientador de estágio é responsável também por acompanhar as atividades que o aluno vem desenvolvendo e mostrar para ele caminhos que possam alavancar sua carreira e aumentar suas chances de sucesso na empresa. 

\subsection{Modalidades de Estágio}


\subsubsection{Estágio Obrigatório}
De acordo com a Lei 11.788/08, Estágio Obrigatório é aquele cujo cumprimento da carga horária é requisito para aprovação e obtenção de diploma. 

O Estágio Obrigatório será desenvolvido obedecendo as etapas de: 

\begin{itemize}
\item Planejamento o qual se efetivará com a elaboração do plano de trabalho e formalização do termo de compromisso;
\item Supervisão e acompanhamento, que se concretizarão em três níveis: Profissional, Didático-pedagógica e administrativa desenvolvidos pelo supervisor local de estágio, pelo professor orientador e pelo coordenador de estágio, respectivamente;
\item Avaliação, realizada em dois níveis: profissional e didática desenvolvidos pelo supervisor local de estágio e professor orientador, respectivamente 
\end{itemize}

A carga horária mínima do estágio obrigatório é de 360 (trezentas e sessenta) horas a serem realizadas, preferencialmente, no 8º semestre do curso. Também é importante enfatizar que o aluno só pode se matricular na disciplina de estágio (Estágio Obrigatório) se o mesmo já tiver sido aprovado em 122 créditos de disciplinas da matriz curricular. Esses 122 créditos incluem apenas disciplinas obrigatórias, eletivas e optativas da matriz curricular do curso.


\subsubsection{Estágio Não Obrigatório}
É aquele desenvolvido como atividade opcional. Para realizá-lo o aluno deve ter sido aprovado em, no mínimo, 98 créditos de disciplinas da matriz curricular e a jornada deve ser compatível com as atividades acadêmicas. A carga horária desenvolvida no estágio não-obrigatório não será computada na grade do aluno.


\subsection{Objetivos do Estágio}

Observando o Perfil do Profissional previsto para o Curso de Ciência da Computação e o previsto no Art. 1º da Lei nº. 11.788/2008, e sabendo-se que o Estágio é um ato educativo escolar supervisionado, desenvolvido no ambiente de trabalho, que visa à preparação para o trabalho produtivo de educandos que estejam frequentando o ensino regular em instituições de educação superior, foram definidos para o Estágio os seguintes objetivos:

\begin{itemize}
\item Consolidar o processo de formação do profissional bacharel em Ciência da Computação para o exercício da atividade profissional de forma integrada e autônoma.
\item Possibilitar oportunidades de interação dos alunos com institutos de pesquisa, laboratórios e empresas que atuam nas diversas áreas da Ciência da Computação. 
\item Desenvolver a integração Universidade-Comunidade, estreitando os laços de cooperação.
\end{itemize}

\subsection{Caracterização do Estágio}
O Estágio deve ser desenvolvido nas áreas de conhecimento no âmbito da Ciência da Computação, mediante um Plano de Trabalho, elaborado em comum acordo entre as partes envolvidas. O Estágio não poderá ser realizado no âmbito de atividades de monitoria ou iniciação científica.

As atividades de estágio poderão ser desenvolvidas durante as férias escolares ou durante o período letivo, embora a oferta da disciplina seja de acordo com os semestres letivos da UFSCar.


\subsection{Jornada de Atividade em Estágio}

De acordo com a Lei 11.788/08, a jornada de atividade em estágio será definida de comum acordo entre a Instituição de Ensino, a parte concedente e o aluno estagiário, devendo constar do termo de compromisso e ser compatível com as atividades escolares. Essa compatibilidade se faz possível quando se somando a carga horária semanal e o número de créditos que o estudante fará no semestre não ultrapassem o total de 40 créditos, ficando limitado a jornada de estágio a no máximo 30 (trinta) horas semanais. No entanto, o aluno poderá fazer 40 horas desde que ele tenha integralizado os créditos e desde que a única dependência de sua formação seja o estágio.

O aluno que já exerce atividade profissional compatível com a sua área de atuação, pode solicitar diminuição de até 50\% da carga horária exigida para o estágio \cite{Regimento-Geral-CursosGraduacao-UFSCar}. A solicitação deve ser encaminhada à coordenação do curso que irá analisar o caso e decidirá a porcentagem a ser reduzida.  


\subsection{Coordenação de Estágio}
A Atividade de Estágio é regulamentada pela Coordenação de Estágio, composta pelo Coordenador de Estágio e o Secretário da Coordenação de Estágio.

O Coordenador de Estágio é professor do Departamento de Computação responsável pela disciplina Estágio em Computação.
As atribuições da Coordenação de Estágio são:

\begin{itemize}
\item Estar em contato com empresas interessadas em contratar estagiários;
\item Informar o aluno sobre as regras brasileiras e institucionais do estágio;
\item Direcionar os alunos quanto ao preenchimento correto do Termo de Compromisso de Estágio;
\item  Avaliar o plano de trabalho de estágio;
\item  Designar Orientador do Estágio Curricular;
\item  Coordenar a tramitação de todos os instrumentos jurídicos: convênios, termos de compromisso, requerimentos, cartas de apresentação, cartas de autorização ou outros documentos necessários para que o estágio seja oficializado, bem como a guarda destes.
\item  Coordenar as atividades de avaliações do Estágio obrigatório.
\end{itemize}

 
\subsection{Estágio Internacional}
O estágio em empresas estrangeiras é permitido desde que estas sigam a legislação brasileira. 


\subsection{Condições para realização do Estágio Obrigatório}

Para realização do estágio o estudante deve atender os seguintes requisitos:

\begin{itemize}
\item Estar matriculado regularmente no curso de Bacharelado em Ciência da Computação;
\item 	Ter concluído 160 créditos do seu curso;
\item 	Possuir um supervisor da parte concedente, para orientação, acompanhamento e avaliação do estágio.
\item Celebração de termo de compromisso entre o estudante, a parte concedente do estágio e a UFSCar; 
\item Elaboração de plano de atividades a serem desenvolvidas no estágio, compatíveis com este projeto pedagógico, o horário e o calendário escolar, de modo a contribuir para a efetiva formação profissional do estudante;
\item Acompanhamento efetivo do estágio por professor orientador designado pela coordenação de estágio e por supervisor da parte concedente, sendo ambos responsáveis por examinar e aprovar os relatórios periódicos e final, elaborados pelo estagiário.
\end{itemize}


\subsection{Orientação e Supervisão do Estágio}


O professor responsável pela orientação do estudante durante o Estágio será um professor do curso de Ciência da Computação, sendo este responsável pelo acompanhamento e avaliação das atividades dos estagiários e terá as seguintes atribuições:

\begin{itemize}
\item Orientar os alunos na elaboração dos relatórios e na condução de seu Projeto de Estágio;
\item Orientar o estagiário quanto aos aspectos técnicos, científicos e éticos;
\item Supervisionar o desenvolvimento do programa pré-estabelecido, controlar frequências, analisar relatórios, interpretar informações e propor melhorias para que o resultado esteja de acordo com a proposta inicial, mantendo sempre que possível contato com o supervisor local do estágio;
\item Estabelecer datas para entrevista(s) com o estagiário e para a entrega de relatório(s) das atividades realizadas na empresa;
\item Avaliar o estágio, especialmente o(s) relatório(s), e encaminhar ao colegiado o seu parecer, inclusive quanto ao número de horas que considera válidas.
\end{itemize}


O Supervisor do estagiário deverá ser um profissional que atue no local no qual o estudante desenvolverá suas atividades de estágio e terá as seguintes atribuições:

\begin{itemize}
\item Garantir o acompanhamento contínuo e sistemático do estagiário, desenvolvendo a sua orientação e assessoramento dentro do local de estágio. Não é necessário que o supervisor seja Cientista da Computação, mas deve ser um profissional que tenha extensa experiência na área de atuação;
\item Informar à Coordenação de Estágio as ocorrências relativas ao estagiário, buscando assim estabelecer um intercâmbio permanente entre a Universidade e a Empresa;
\item Apresentar um relatório de avaliação do estagiário à Coordenação de Estágio, em caráter confidencial.
\end{itemize}

\subsection{Obrigações do estagiário}
O estagiário, durante o desenvolvimento das atividades de estágio, terá as seguintes obrigações:

\begin{itemize}
\item Apresentar documentos exigidos pela UFSCar e pela concedente;
\item Seguir as determinações do Termo de compromisso de estagio;
\item Cumprir integralmente o horário estabelecido pela concedente, observando assiduidade e pontualidade;
\item Manter sigilo sobre conteúdo de documentos e de informações confidenciais referentes ao local de estágio;
\item Acatar orientações e decisões do supervisor local de estágio, quanto às normas internas da concedente;
\item Efetuar registro de sua frequência no estágio;
\item Elaborar e entregar relatório das atividades de estágio e outros documentos nas datas estabelecidas;
\item Respeitar as orientações e sugestões do supervisor local de estagio
\item Manter contato com o professor orientador de estágio, sempre que julgar necessário.
\item Assumir o estágio com responsabilidade, zelando pelo bom nome da Instituição Concedente e do curso de Ciência da Computação.
\end{itemize}

\subsection{Formalização do termos de compromisso}
Deverá ser celebrado Termo de Compromisso de estágio entre o estudante, a parte concedente do estágio e a UFSCar, estabelecendo:

\begin{itemize}
\item O plano de atividades a serem realizadas, que figurará em anexo ao respectivo termo de compromisso;
\item As condições de realização do estágio, em especial, a duração e a jornada de atividades, respeitada a legislação vigente;
\item As obrigações do Estagiário, da Concedente e da UFSCar;
\item O valor da bolsa ou outra forma de contraprestação devida ao Estagiário, e o auxílio-transporte, a cargo da Concedente, quando for o caso; 
\item O direito do estagiário ao recesso das atividades na forma da legislação vigente;
\item A empresa contratante deverá segurar o estagiário contra acidente pessoal, sendo que uma cópia da mesma deverá ser anexada ao termo após sua realização.
\end{itemize}

