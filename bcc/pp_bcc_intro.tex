\chapter{Introdução}
\label{chap:intro}

A sociedade passa por profundas transformações econômicas, sociais e culturais. Vive-se um grande progresso tecnológico que induz mudanças no papel exercido pela computação em praticamente todos os setores da sociedade. Nesse contexto, o ensino em computação cumpre papel fundamental no mundo atual, ao mesmo tempo em que é constantemente desafiado a adaptar-se com qualidade e excelência em um cenário de mudanças constantes.

A Ciência da Computação engloba aspectos teóricos e práticos relacionados com a utilização de dispositivos computacionais (hardware) para a execução de tarefas descritas na forma de um programa de computador (software). Ela figura no centro de muitos dos avanços tecnológicos alcançados no século XX, em praticamente todas as áreas do conhecimento. Atualmente, a Ciência da Computação tem papel fundamental em muitos ramos da ciência, como física, química, biologia e medicina. Além disso, ela é essencial no desenvolvimento de praticamente todos os ramos da indústria atual.

Pode-se considerar o emprego da Ciência da Computação como atividade fim ou meio. Seu emprego como atividade fim visa evoluir a computação em si, promovendo a criação, melhoria e inovação em hardware e software. Já seu emprego como atividade meio consiste na sua utilização para apoiar o desenvolvimento de outras áreas.

Este documento constitui-se no Projeto Pedagógico do Curso(PPC) de Ciência da Computação modalidade Bacharelado~(BCC), da Universidade Federal de São Carlos~(UFSCar), campus de São Carlos. Trata-se de uma proposta diferenciada de formação de profissionais na área de computação, sendo uma resposta aos desafios que o progresso tecnológico impõe à sociedade e às instituições de ensino superior. Os diferenciais incluem a existência de linhas de formação, exposição do aluno a uma grande quantidade de atividades complementares, inserção do aluno em projetos integradores, maior incentivo à pesquisa, etc. 



\section{Objetivos}
\label{sec:objetivos}

O objetivo geral do Bacharelado em Ciência da Computação da UFSCar, campus de São Carlos, é formar profissionais de excelência que empreguem a computação como atividade fim ou meio de modo a contribuir para o crescimento social, cultural e econômico do país. Esses objetivos são alcançados por meio da exposição do estudante a um conjunto de componente curriculares diversificado, que lhe tragam conhecimentos atualizados do mercado e da área acadêmica.



\section{Organização deste Documento}

Este documento está organizado em 9 capítulos e 1 anexo. No Capítulo~\ref{sec:principios} são apresentados os princípios sobre os quais foi projetado este curso; o Capítulo~\ref{sec:marco_referencial} apresenta a discussão sobre o marco referencial do curso; o Capítulo~\ref{sec:marco_conceitual} apresenta a discussão sobre o marco conceitual do curso; o Capítulo~\ref{sec:marco_estrutural} apresenta os detalhes da constituição do curso e de suas disciplinas; o Capítulo~\ref{sec:integracao} trata da relação entre ensino, pesquisa e extensão; o Capítulo~\ref{sec:avaliacao_aprendizagem} aborda as formas de avaliar a aprendizagem dos estudantes; o Capítulo~\ref{sec:avaliacao_gerenciamento} aborda os mecanismos pelos quais os docentes podem avaliar e manter o bom funcionamento do curso; o Capítulo~\ref{sec:implantacao} discute como esta reformulação curricular será colocada em prática. Finalmente, o Anexo~\ref{anexo: PIE} detalha o Projeto Integrador Extensionista.



\chapter{Princípios Norteadores desta Reformulação} \label{sec:principios}


Com base no cenário observado, os seguintes princípios foram adotados para esta reformulação curricular do Bacharelado em Ciência da Computação:

\begin{itemize}

\item \textbf{Oferecimento de Linhas de  Formação:} Este princípio é contemplado por meio de um conjunto maior de disciplinas optativas da área de computação, que cobrem conteúdos avançados para complementar a formação do estudante. É natural que ao longo do curso cada estudante se identifique mais com determinadas áreas da computação, sejam elas básicas ou aplicadas. Por esse motivo, este curso possui o conceito de \textit{Linhas de Formação}, que consiste em um conjunto de disciplinas optativas que podem ser selecionadas pelo estudante a fim de aprofundar seus conhecimentos em áreas específicas. Isso permite que o estudante possa configurar sua matriz curricular a partir do quinto período e finalizar o curso com mais aptidão em determinadas áreas. 

\item \textbf{Maior Engajamento em Atividades Complementares:} Reconhecendo que a formação do estudante depende grandemente de sua exposição a experiências diversificadas durante o curso, esta proposta incentiva os estudantes a cumprirem um número alto de horas em atividades complementares, por exemplo: Iniciações Científicas; Programas de Educação Tutorial (PET); Projeto Integrador Extensionista; Organização de Eventos como a Semana de Computação; Participação em Empresa Júnior; Treinamento e Competição na Maratona de Programação.

\item \textbf{Melhor Caracterização da Ciência da Computação:} Este curso é dividido em núcleos de conhecimento, agrupando atividades curriculares relacionadas. Dois núcleos foram cuidadosamente definidas no sentido de caracterizar melhor as bases da Ciência da Computação: \textit{Fundamentos da Ciência da Computação} e \textit{Algoritmos e Programação}. Esses dois núcleos foram concebidos de modo a capacitar o estudante no entendimento da base teórica na qual se assenta a computação, a prepará-lo para solucionar problemas computacionais complexos, além de qualificá-lo para o eventual engajamento em atividades de pesquisa científica na área.


\item \textbf{Menor Concentração de Disciplinas de Matemática:} Grande parte dos cursos de Ciência da Computação em âmbito nacional concentra as disciplinas de matemática nos primeiros semestres do curso. Isso tem gerado muitos problemas em consequência de alta carga teórica que essas disciplinas demandam, levando os estudantes dos primeiros semestres a reprovarem. Nesta nova proposta, as disciplinas de matemática foram distribuídas ao longo do curso, exigindo que o aluno curse apenas uma disciplina de matemática em cada semestre. Assim, ele terá mais tempo para se dedicar a esses conteúdos que são imprescindíveis para um bachareal em Ciência da Computação.


\item \textbf{Aperfeiçoamento de competências de desenvolvimento de projetos:} Este princípio é contemplado por meio de uma atividade curricular denominada "Projeto Integrador Extensionista". Essa atividade deve ser desenvolvida por grupos de estudantes e possui o objetivo de integrar o conhecimento de várias disciplinas do curso, além de fornecer ao estudante uma experiência holística de todas as fases do desenvolvimento de um sistema.

\item \textbf{Maior Disponibilidade para Estagiar:} Nesta nova proposta, o último semestre do curso não possui atividades curriculares presenciais. Assim, caso o estudante opte por fazer estágio, ele poderá realizá-lo em empresas geograficamente distantes de São Carlos, inclusive internacionais, pois não necessitará estar presente na Universidade neste semestre.

\item \textbf{Incentivo à Pesquisa:} Nesta proposta, o estudante que possui perfil para a carreira científica em computação pode optar por fazer  Iniciação Científica durante o curso e solicitar equivalência desta com o Trabalho de Conclusão do Curso. Desta forma, o estudante pode agilizar o início de sua pós-graduação.  

\item \textbf{Limitação da quantidade de disciplinas por semestre:} Reconhecendo que o estudante precisa de tempo para exercitar, refletir e compreender os conceitos apresentados nas aulas, a matriz curricular recomendada foi projetada para que o estudante curse no máximo seis disciplinas por semestre. 

\item \textbf{Unificação de disciplinas teóricas e de laboratório:} A matriz curricular não contém mais pares de disciplinas independentes abordando um mesmo conteúdo, sendo uma delas teórica e a outra de laboratório. Optou-se por criar disciplinas mistas, visando uma maior integração entre atividades teóricas e práticas.

%Sem prejuízo ao ensino dos conteúdos estabelecidos por diretrizes curriculares e Ministério da Educação, buscou-se induzir o estabelecimento de práticas de ensino baseadas no desenvolvimento de projetos. Isso se dará em particular nas disciplinas dos eixos de Metodologia e Técnicas da Computação e Especializações, quando espera-se que a maturidade acadêmica dos alunos resulte em um melhor processo de aprendizagem usando-se esta abordagem.

%\item \textbf{Substituição de carga horária teórica por prática:} Em diversas disciplinas buscou-se uma substituição de horas-aula teóricas por atividades práticas e/ou desenvolvimento de projetos.

\item \textbf{Formação multidisciplinar:} Dada a relação complementar da computação com diversas áreas do conhecimento, como: engenharias, matemática, física, estatística, biologia, linguística, etc, a matriz curricular prevê três disciplinas eletivas. Desta forma o estudante poderá buscar uma qualificação complementar em outras áreas do conhecimento disponíveis na universidade.


\item \textbf{Melhoria Contínua e Multidisciplinaridade:} Outro diferencial do curso é a existência de um Coordenador de Núcleo de Conhecimento. Esse coordeandor é um professor do curso com larga experiência em uma determinado núcleo de conhecimento e possui a responsabilidade de identificar oportunidades de colaboração entre as disciplinas de um determinado núcleo de conhecimento e até mesmo entre disciplinas de núcleos de conhecimento distintos. Além disso, esse coordenador também procura garantir que os conteúdos ministrados estejam sempre coerentes e que novas práticas pedagógicas sejam sempre incentivadas entre os docentes.


\item \textbf{Preparação para o aprendizado contínuo:} Acima de tudo, reconhece-se ser impossível ensinar e aprender tudo o que seria desejado em um curso de Ciência da Computação no período de 4 anos. Por esse motivo, buscou-se oferecer uma base sólida de conhecimentos e práticas de estudo que permitam ao estudante atualizar-se e aprimorar-se ao longo de toda sua carreira profissional.

\item \textbf{Consciência do papel do Egresso na Sociedade:} Além da formação específica em computação, o curso também provê um conjunto diversificado de disciplinas optativas do núcleo de humanas no sentido de fornecer ao estudante consciência sobre seu papel social e ético na sociedade. 


\end{itemize}


%A adoção de disciplinas convênio que facilitem o reconhecimento de atividades desenvolvidas em programas de mobilidade estudantil é parte do reconhecimento da importância que esta vivência em diferentes realidades traz para o estudante, reforçando sua adaptabilidade, espírito crítico e tolerância com o diferente. Da mesma forma, busca-se incentivar que o estudante usufrua de uma vida universitária plena, participando de projetos de iniciação científica, empresa júnior, projetos de extensão, competições como Olimpíada Brasileira de Informática, Maratona de Programação, Futebol de Robôs, Campeonato de Desenvolvimento de Jogos Digitais, entre outros, além de intensa socialização com toda a comunidade universitária. Tais experiências enriquecem o indivíduo enquanto pessoa e cidadão, além de atuar, em alguns casos, como atividades integradoras e aprofundadoras de conhecimento, trazendo forte motivação para o estudante para seguir seus estudos.















