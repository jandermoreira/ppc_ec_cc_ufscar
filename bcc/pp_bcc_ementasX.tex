

Este capítulo apresenta com mais detalhes as disciplinas do curso, destacando seus objetivos, ementas, créditos, etc.

Como já comentado anteriormente, até o quarto semestre, o curso é formado exclusivamente de disciplinas obrigatórias. A partir do quarto semestre há várias disciplinas optativas, permitindo que o aluno tenha flexibilidade para escolher as disciplinas que pretende fazer. 

As próximas seções apresentam tanto as disciplinas obrigatórias quanto as optativas.


\subsection{Disciplinas do Primeiro Semestre}

\disciplina{ld}{
    \titulo      {1}{Lógica Digital}
    \objetivo    {Ao final da disciplina o estudante deve ser capaz de projetar e analisar circuitos digitais combinatórios e sequenciais e executar sua implementação usando circuitos integrados e linguagem de descrição de hardware.}
    \requisitos  {N/A}
    \recomendadas{N/A}
    \ementa      {Conceitos fundamentais de Eletrônica Digital. Representação digital da informação. Álgebra Booleana. Tabelas verdade e portas lógicas. Expressões lógicas e formas canônicas. Estratégias de minimização de circuitos. Elementos de memória. Máquinas de estado (Mealy e Moore). Circuitos funcionais típicos (combinacionais e sequenciais).}
    \creditos    {6 total (4 teóricos, 2 práticos)}
    %    \extra       {3 horas}
    \codigo      {DC}{1001351}
    \bibliografia {
        TOCCI, Ronald J.; WIDMER, Neil S. MOSS, Gregory L. Sistemas digitais: princípios e aplicações. 11. ed. São Paulo: Pearson Prentice Hall, 2011. xx, 817 p. : il. ISBN 9788576059226.

        WAKERLY, John F. Digital design: principles and practices. 4. ed. Upper Saddle River: Pearson Prentice Hall, 2006. 895 p. ISBN 0-13-186389-4.

        FLOYD, Thomas L. Sistemas digitais: fundamentos e aplicações. 9. ed. Porto Alegre, RS: Artmed, 2007. xiii, 888 p. ISBN 9788560031931.
    }{
        Stephen Brown, Zvonko Vranesic, and Brown Stephen; Fundamentals of Digital Logic with Verilog Design; McGraw-Hill Companies,Inc.; Edition 2; 2007.

        PEDRONI, Volnei Antonio. Eletrônica digital moderna e VHDL: princípios digitais, eletrônica digital, projeto digital, microeletrônica e VHDL. Rio de Janeiro: Elsevier, 2010. 619 p. ISBN 9788535234657.

        ERCEGOVAC, Milos D.; LANG, Tomás. Digital arithmetic. San Frascisco: Morgan Kaufmann, c2004. 709 p. ISBN 1-55860-798-6.

        Victor P. Nelson, H. Troy Nagle, Bill D. Carroll, David Irwin; Digital Logic Circuit Analysis and Design; Edition 1
        Prentice Hall; 1995.

        Norman Balabanian e Bradley Carlson; Digital Logic Design Principles; Edition 1; Wiley; 2000.
    }
    \dataatualizacao{23/10/23} % Edilson, Márcio, Luciano, Menotti, Helio, Jander, 
    \competencias
    {
        cg-aprender/{ce-ap-1, ce-ap-2, ce-ap-3},
        cg-produzir/{ce-pro-1, ce-pro-2, ce-pro-4, ce-pro-5},
        cg-atuar/{ce-atuar-1, ce-atuar-2, ce-atuar-3, ce-atuar-4}
    }
}

\begin{tabular}{|p{4.5cm}|p{10.0cm}|} \hline
Título & Lógica Digital \\ \hline
Objetivo & Ao final da disciplina o aluno deve ser capaz de projetar e analisar circuitos digitais combinatórios e sequenciais e executar sua implementação usando circuitos integrados e linguagem de descrição de hardware. \\ \hline
Pré-requisitos & não há \\ \hline
Disciplinas recomendadas & não há \\ \hline

Ementa & 1 - Álgebra de Boole, diagrama de Venn e mapas de Veitch-Karnaugh.
2 - Sistemas de numeração e aritmética computacional; 
3 - Circuitos combinatórios; 
4 - Circuitos sequenciais; 
5 - Máquinas de estado finito; 
6 - Elementos básicos de computação: circuitos e unidades lógica e aritméticas, registradores, barramentos, controladores, memórias RAM e ROM e demais elementos típicos; 
7 - Projeto e implementação em laboratório dos sistemas lógicos utilizando CIs (circuitos integrados) e ferramentas EDA (Electronic Design Automation) em FPGAs com linguagens de descrição de hardware;

\\ \hline
Créditos & 6 (4t,2p) \\ \hline
Carga extra-classe & 3 horas \\ \hline
Responsável pela oferta & DC \\ \hline
%\textbf{Ref:} http://courses.cs.washington.edu/courses/cse370/06au/admin/syllabus.html
\end{tabular}
\\
\\


\begin{tabular}{|p{4.5cm}|p{10.0cm}|} \hline

Título & Construção de Algoritmos e Programação \\ \hline
Objetivo & Os alunos devem desenvolver duas competências principais com base na disciplina: (1) a capacidade do pensamento computacional (algorítmico) para proposição de soluções de problemas e (2) capacidade de mapear tais soluções em programas em usando linguagem de programação.\\ \hline
Pré-requisitos &  \\ \hline
Disciplinas recomendadas &  \\ \hline
Ementa &  Aspectos históricos da computação. Bases numéricas: binária, octal e hexadecimal; conversões. Noções gerais da computação: organização de computadores, programas e linguagens. Algoritmos estruturados e programação: tipos básicos de dados; representação e manipulação de dados; estruturas de controle de fluxo (condicionais e repetições); modularização (sub-rotinas, passagem de parâmetros e escopo); documentação. Estruturação básica de dados: variáveis compostas heterogêneas (registros) e homogêneas (vetores e matrizes). Operações em arquivos e sua manipulação. Noções de ponteiros e alocação dinâmica de memória.\\ \hline
Créditos & 8 \\ \hline
Carga extra-classe &  \\ \hline
Responsável pela oferta &  \\ \hline
\end{tabular}
\\
\\




\begin{tabular}{|p{4.5cm}|p{10.0cm}|} \hline

Título & Fundamentos de Ciência da Computação \\ \hline
Objetivo & Fornecer ao aluno ingressante uma visão abrangente da área de Ciência da Computação, seus assuntos e suas aplicações. \\    \hline
Pré-requisitos &  -  \\ \hline
Disciplinas recomendadas & -  \\ \hline
Ementa & Dados. Memória. Representação de Dados Binária. A integração e uso de Arquitetura de Computadores, Sistemas operacionais, Redes e Internet. Algoritmos simples com implementação em linguagem de programação Python.  Introdução à: Sistemas de Banco de Dados, Inteligência Artificial, Computação Gráfica e Teoria da Computação.
  \\ \hline
Créditos & 2 (2t,2p) \\ \hline
Carga extra-classe & 4 horas \\ \hline
Responsável pela oferta & DC \\ \hline
\end{tabular}

\disciplina{calculo1}{
    \titulo      {1}{Cálculo Diferencial e Integral 1}
    \objetivo    {Propiciar o aprendizado dos conceitos de limite, derivada e integral de funções de uma variável real. Propiciar a compreensão e o domínio dos conceitos e das técnicas de cálculo diferencial e integral. Desenvolver a habilidade de implementação desses conceitos e técnicas em problemas nos quais eles se constituem os modelos mais adequados. Desenvolver a linguagem matemática como forma universal de expressão da ciência.}
    % \requisitos{}
    \recomendadas{N/A}
    \ementa      {Números reais e funções de uma variável real. Limites e continuidade. Cálculo Diferencial e aplicações. Cálculo integral e aplicações.}
    \creditos{6 total (5 teóricos, 1 prático)}
    %     % \horas    {90 total (75 teóricas, 15 práticas)}
    % %    \extra       {6 horas}
    \codigo      {DM}{08.221-0}
    \bibliografia {
        GUIDORIZZI, H. L. Um curso de cálculo v. 1 – 5a. Edição, Rio de Janeiro: Livros Técnicos e Científicos, 2001.

        STEWART, J. Cálculo v. 1 – 5a. Edição, São Paulo: Pioneira Thomson Learning, 2006.

        SWOKOWSKI, E. W. Cálculo com geometria analítica v. 1 – 2a. Edição, São Paulo: McGraw-Hill do Brasil, 1994.
    }{
        ANTON, H. Cálculo v. 1, 10. ed, Porto Alegre, RS: Bookman, 2014.

        ÁVILA, G. Calculo: funções de uma variável v. 1 - 6a. Edição, Rio de Janeiro: Livros Técnicos e Científicos, 1994.

        FLEMMING, D. M.; GONCALVES, M. B. Cálculo A: funções, limite, derivação, integração – 6a. Edição, São Paulo: Prentice Hall, 2006.

        LEITHOLD, L. O cálculo com geometria analítica v. 1 – 3. ed, São Paulo: Harbra, 1991.

        THOMAS, G. B. et al. Cálculo, v.1 - 10a. Edição, Addison Wesley, 2002.
    }

    % Inserido por Murillo R. P. Homem, em 01/04/2023
     \competencias{
        cg-aprender/{ce-ap-1, ce-ap-2, ce-ap-3, ce-ap-4},
        cg-atuar/{ce-atuar-1, ce-atuar-2, ce-atuar-3, ce-atuar-4},
        }
        
}


\begin{tabular}{|p{4.5cm}|p{10.0cm}|} \hline

Título & Cálculo Diferencial e Integral 1 \\ \hline
Objetivo & Propiciar aprendizado dos conceitos de limite, derivada e integral de funções de uma variável real. Propiciar a compreensão e o domínio dos conceitos e das técnicas de Calculo Diferencial e Integral. Desenvolver a habilidade de implementação desses conceitos e técnicas em problemas nos quais eles se constituem os modelos mais adequados. Desenvolver a linguagem Matemática como forma universal de expressão da Ciência. \\ \hline
Pré-requisitos & nenhum \\ \hline
Disciplinas recomendadas & nenhum \\ \hline
Ementa & Números Reais, Valor Absoluto, Desigualdades. Funções Reais.   Funções Trigonométricas. Limite e Continuidade: conceito, definição e propriedades. Derivadas: retas tangentes, coeficiente angular, definição de derivada, diferenciais. Aplicações da Derivada : velocidade, taxa de variação.   Regras de Derivação, Regra da Cadeia, Funções Implícitas, Derivação Implícita.Teorema do Valor Médio , Regra de LÂ’ Hôspital.    Funções crescentes e decrescentes , máximos e mínimos, convexidade, esboço de gráficos de funções. Problemas de máximos e mínimos. Funções Exponenciais e Logarítmicas. Funções Trigonométricas Inversas.   Integrais Indefinidas, Integrais Definidas e Propriedades.    Teorema do Valor Médio para Integrais e Teorema Fundamental do Cálculo.    Métodos de Integração e Aplicações: área, volume. \\ \hline
Créditos & 6 (5t,1p) \\ \hline
Carga extra-classe & 8 horas \\ \hline
Responsável pela oferta & DM \\ \hline
\end{tabular}


\subsection{Disciplinas do Segundo Semestre}

\begin{tabular}{|p{4.5cm}|p{10.0cm}|} \hline

Título & Geometria Analítica \\ \hline

Objetivo &  \\ \hline

Pré-requisitos &  \\ \hline

Disciplinas recomendadas & \\ \hline

Ementa &  \\ \hline

Créditos & 4 (5t,1p) \\ \hline
Carga extra-classe & 8 horas \\ \hline
Responsável pela oferta & DM \\ \hline
\end{tabular}
\\
\\


\begin{tabular}{|p{4.5cm}|p{10.0cm}|} \hline

Título & Introdução à Programação Orientada a Objetos \\ \hline

Objetivo & Aos alunos serão apresentados os conceitos de programação orientada a objetos e suas características principais. Ao final da disciplina, o aluno deverá ser capaz de construir programas utilizando uma linguagem baseada no paradigma de orientação a objetos.  \\ \hline

Pré-requisitos &  \\ \hline

Disciplinas recomendadas & \\ \hline

Ementa & Histórico do paradigma orientado a objetos e comparação com o paradigma estruturado; Conceitos Teóricos e práticos de Orientação a Objetos: abstração, classes, objetos, atributos e métodos, encapsulamento/visibilidade, princípios SOLID (responsabilidade única, aberto-fechado, substituição de Liskov, segregação de interface e inversão de dependência), herança, composição/agregação, sobrecarga, polimorfismo de inclusão e classes abstratas e polimorfismo paramétrico; modularização; Alocação dinâmica; Tratamento de exceções. \\ \hline

Créditos & 4 (5t,1p) \\ \hline
Carga extra-classe & 8 horas \\ \hline
Responsável pela oferta & DC \\ \hline
\end{tabular}
\\
\\


\begin{tabular}{|p{4.5cm}|p{10.0cm}|} \hline

Título & Matemática Discreta \\ \hline

Objetivo &   \\ \hline

Pré-requisitos &  \\ \hline

Disciplinas recomendadas & \\ \hline

Ementa & Teoria dos números; Teoria dos conjuntos; Relações; Funções; Contagem: permutação e combinações; Indução matemática; Recursão; Grafos: teoria \\ \hline

Créditos & 4 (5t,1p) \\ \hline
Carga extra-classe &  \\ \hline
Responsável pela oferta &  \\ \hline
\end{tabular}
\\
\\

\begin{tabular}{|p{4.5cm}|p{10.0cm}|} \hline

Título & Lógica Matemática \\ \hline

Objetivo &   \\ \hline

Pré-requisitos &  \\ \hline

Disciplinas recomendadas & \\ \hline

Ementa & \helenacomentario{Quero conversar com a Heloísa para fechar essa ementa} Lógica proposicional: proposições atômicas, conectivos, proposições compostas, fórmulas bem formadas, linguagem proposicional, semântica (interpretações e modelos), consequência lógica, equivalência lógica, dedução, formas normais, regras de inferência, argumentos, o princípio da resolução; Lógica de primeira ordem (lógica de predicados): alfabetos, termos, fórmulas bem formadas, linguagem de primeira ordem, escopo de quantificadores, variáveis livres e ligadas, semântica (modelos), consequência lógica, equivalência lógica, dedução, skolemização, formas normais, quantificação universal, notação clausal, cláusulas de Horn, substituição e unificação, unificadores mais gerais, o princípio de resolução. %\helenacomentario{O conteúdo de Álgebra de Boole já está em Lógica Digital} Álgebra de Boole: conceitualização, métodos de minimização, minimização de expressões algébricas, mapas de Karnaugh. 
\\ \hline

% Ementa atual (2018): 
% O CÁLCULO PROPOSICIONAL: PROPOSIÇÕES ATÔMICAS, CONECTIVOS, PROPOSIÇÕES COMPOSTAS, FÓRMULAS BEM FORMADAS, LINGUAGEM PROPOSICIONAL, SEMÂNTICA (MODELOS), CONSEQÜÊNCIA LÓGICA, EQUIVALÊNCIA LÓGICA, 
% MÉTODOS DE MINIMIZAÇÃO, MINIMIZAÇÃO DE EXPRESSÕES ALGÉBRICAS, 
% DEDUÇÃO, FORMAS NORMAIS, REGRAS DE INFERÊNCIA, ARGUMENTOS, O PRINCÍPIO DE RESOLUÇÃO; E 
% A LÓGICA DE PRIMEIRA ORDEM: ALFABETOS DE PRIMEIRA ORDEM, TERMOS, FÓRMULAS BEM FORMADAS, LINGUAGEM DE PRIMEIRA ORDEM, ESCOPO DE QUANTIFICADORES, VARIÁVEIS LIVRES E LIGADAS, SEMÂNTICA (MODELOS), CONSEQÜÊNCIA LÓGICA, EQUIVALÊNCIA LÓGICA, DEDUÇÃO, SKOLEMIZAÇÃO, FORMAS NORMAIS, QUANTIFICAÇÃO UNIVERSAL E NOTAÇÃO CLAUSAL, CLÁUSULAS DE HORN, 
%% UNIVERSO DE HERBRAND, DEMONSTRAÇÃO AUTOMÁTICA DE TEOREMAS POR COMPUTADOR, 
% SUBSTITUIÇÃO E UNIFICAÇÃO, UNIFICADORES MAIS GERAIS, O PRINCÍPIO DE RESOLUÇÃO.

Créditos & 4 (3t, 1p) \\ \hline
Carga extra-classe &  \\ \hline
Responsável pela oferta &  \\ \hline
\end{tabular}
\\
\\


\begin{tabular}{|p{4.5cm}|p{10.0cm}|} \hline

Título & Texto Técnico \\ \hline

Objetivo & \helena{}{Tornar o aluno capaz de compreender e produzir textos técnicos, com habilidades para elaborar diferentes tipos de documentos a partir dos conceitos aprendidos nesta disciplina.Desenvolver no aluno a competência para dominar as técnicas de redação e aplicá-las nas diferentes situações do cotidiano acadêmico e profissional.}  \\ \hline

Pré-requisitos &  \\ \hline

Disciplinas recomendadas & \\ \hline

Ementa &  \helena{}{1. Noções básicas de escrita técnica; 2. Planejamento, organização e produção de textos técnicos; 3.  Técnicas de argumentação; 4. Normatização (ABNT e outros); 5. Ferramentas de auxílio à escrita (editores de texto e suas funcionalidades); 6. Elaboração de textos técnicos.} \\ \hline

Créditos & 4 (5t,1p) \\ \hline
Carga extra-classe &  \\ \hline
Responsável pela oferta &  \\ \hline
\end{tabular}


\subsection{Disciplinas do Terceiro Semestre}


\begin{tabular}{|p{4.5cm}|p{10.0cm}|} \hline

Título & Estrutura de Dados \\ \hline

Objetivo &   \\ \hline

Pré-requisitos &  \\ \hline

Disciplinas recomendadas & \\ \hline

Ementa & Noções básicas sobre análise de complexidade; Listas Encadeadas: simples,
dupla, circular; Estruturas básicas: fila, pilha, árvores binárias, grafos;Ordenação e
busca em memória principal: bubblesort, merge-sort, heapsort, quicksort; Ordenação
e busca em memória secundária:árvores-B, tabelas de dispersão (hash) \\ \hline

Créditos & 4 (5t,1p) \\ \hline
Carga extra-classe &  \\ \hline
Responsável pela oferta &  \\ \hline
\end{tabular}
\\
\\



\begin{center}
\begin{tabular}{|p{4.5cm}|p{10.0cm}|} \hline
Título & Interface Humano-Computador \\ \hline
Objetivo &

1. Auxiliar os alunos a considerar requisitos de usuário e aspectos de qualidade de uso na construção de sistemas computacionais interativos;
2. Auxiliar os alunos a fazer design de sistemas computacionais interativos, adotando modelos e técnicas bem estabelecidos; 
3. Auxiliar os alunos a realizar avaliações de sistemas computacionais interativos, adotando modelos e técnicas bem estabelecidos.



\\ \hline
Pré-requisitos & 

\\ \hline
Disciplinas recomendadas & não há \\ 

\hline
Ementa & Introdução à interação humano-computador (IHC): histórico, áreas e disciplinas envolvidas;  Sistemas computacionais interativos; Fundamentos teóricos: fatores humanos e ergonomia, modelos de engenharia, conceitos de qualidade de uso; Design de sistemas computacionais interativos: abordagens ao design, modelagem da interação, apoio a decisões de design, técnicas e estilos de prototipação, documentação de decisões de design;  Avaliação de sistemas computacionais interativos: avaliação analítica e empírica, métodos e técnicas de avaliação de usabilidade e acessibilidade. 



\\ \hline
Créditos & 4 (2t,2p) \\ \hline
Carga extra-classe & ???

\\ \hline
Responsável pela oferta & DC \\ \hline
%\textbf{Ref:} http://www.drps.ed.ac.uk/13-14/dpt/cxinfr09027.htm
\end{tabular}
\end{center}


\begin{tabular}{|p{4.5cm}|p{10.0cm}|} \hline

Título & Teoria da Computação \\ \hline

Objetivo &   \\ \hline

Pré-requisitos &  \\ \hline

Disciplinas recomendadas & \\ \hline

Ementa & Introdução: conceito de computabilidade; Tese de Church-Turing;Máquina de Turing; Decisão e problema da parada; Capacidade e limitações de computadores; Complexidade: tempo, recursos, problemas NP-Completos. \\ \hline

Créditos & 4 (5t,1p) \\ \hline
Carga extra-classe &  \\ \hline
Responsável pela oferta &  \\ \hline
\end{tabular}
\\
\\

\disciplina{arq1}{
    \titulo      {3}{Arquitetura e Organização de Computadores 1}
    \objetivo    {Ao final da disciplina o estudante deve ser capaz de entender os princípios da arquitetura e organização básica de computadores e a relação entre linguagens de alto nível e linguagens de máquina, bem como de criar um computador usando técnicas de implementação de unidades funcionais e analisar seu desempenho.}
    \requisitos  {Lógica Digital} % 02.437-6 -Lógica Digital
    \recomendadas{N/A}
    \ementa      {Conceitos fundamentais de Arquitetura de Computadores. Linguagem de máquina. Aritmética computacional. Organização do computador: monociclo, multiciclo e pipeline. Desempenho de computadores. Hierarquia de memória. Entrada/Saída: barramentos e dispositivos externos. Implementação de um processador completo usando linguagem de descrição de hardware.} %Incluir interrupções?
    \creditos    {6 total (4 teóricos, 2 práticos)}
    %    \extra       {3 horas}
    \codigo      {DC}{1001540} %02.735-9 antiga
    \bibliografia {
        PATTERSON, David A.; HENNESSY, John L. Organização e projeto de computadores: a interface harware/software. 3. ed. Rio de Janeiro: Elsevier, 2005. 484 p. ISBN 8535215212.

        HARRIS, David Money; HARRIS, Sarah L. Digital design and computer architecture. San Frascisco: Elsevier, 2007. 569 p. ISBN 978-0-12-370497-9.

        STALLINGS, William. Arquitetura e organização de computadores. 8. ed. São Paulo: Pearson, 2012. 624 p. ISBN 978-85-7605-564-8.

        SAITO, José Hiroki. Introdução à arquitetura e à organização de computadores: síntese do processador MIPS. São Carlos, SP: EdUFSCar, 2010. 189 p. (Coleção UAB-UFSCar. Sistemas de Informação). ISBN 978-85-7600-207-9.
    }{
        HENNESSY, John L.; PATTERSON, David A. Arquitetura de computadores: uma abordagem quantitativa. 3. ed. Rio de Janeiro: Campus, 2003. 827 p. ISBN 85-352-1110-1.

        STALLINGS, William. Arquitetura e organizacao de computadores: projeto para o desempenho. 5. ed. São Paulo: Prentice Hall, 2002. 786 p. ISBN 85-87918-53-2.
    }
    %\dataatualizacao{16/10/23} % Jander, Alexandre, Edilson, Fredy, Márcio, Alan
    \dataatualizacao{23/10/23} % Marcio, Luciano
    \competencias{
        % cg-aprender/{ce-ap-1, ce-ap-2, ce-ap-3},
        % cg-produzir/{ce-pro-1, ce-pro-2, ce-pro-4, ce-pro-5},
        % cg-atuar/{ce-atuar-1, ce-atuar-2, ce-atuar-3, ce-atuar-4}
       %cg-buscar/{ce-busc-1, ce-busc-4},        
        % Para:
        cg-aprender/{ce-ap-1, ce-ap-2, ce-ap-3},
        cg-atuar/{ce-atuar-1, ce-atuar-2}        
    }
}

% \begin{center}
% \begin{tabular}{|p{4.5cm}|p{10.0cm}|} \hline
% Título & Arquitetura e Organização de Computadores \\ \hline
% Objetivo & Ao final da disciplina o aluno deve ser capaz de entender os princípios da arquitetura e organização básica de computadores e a relação entre linguagens de alto nível e linguagens de máquina, bem como de criar um computador usando técnicas de implementação de unidades funcionais e analisar seu desempenho. \\ \hline
% Pré-requisitos & Lógica Digital \\ \hline
% Disciplinas recomendadas & não há \\ \hline
% Ementa & Conceitos fundamentais de Arquitetura de Computadores; Linguagem de máquina; Aritmética computacional; Organização do computador: monociclo, multiciclo e \emph{pipeline}; Desempenho de computadores; Hierarquia de memória; Entrada/Saída: barramentos e dispositivos externos. \\ \hline
% Créditos & 8 (4t,4p) \\ \hline
% Carga extra-classe & 3 horas \\ \hline
% Responsável pela oferta & DC \\ \hline
% %\textbf{Ref:} Vários, baseados no livro Arquitetura e Organização de Computadores, de Hennessy / Patterson.
% \end{tabular}
% \end{center}

\begin{tabular}{|p{4.5cm}|p{10.0cm}|} \hline
Título & Cálculo Diferencial e Séries \\ \hline
Objetivo &  aluno deverá saber como: Aplicar os critérios de convergência para séries infinitas, bem como expandir funções em série de potências. Interpretar geometricamente os conceitos de funções de duas ou mais variáveis e ter habilidade nos cálculos de derivadas e dos máximos e mínimos de funções. Aplicar os teoremas das funções implícitas e inversas. \\ \hline
Pré-requisitos & Cálculo Diferencial e Integral 1 \\ \hline
Disciplinas recomendadas & nenhum \\ \hline
Ementa & Séries Numéricas: critérios de convergência. Séries de Funções. Funções Reais de várias variáveis. Diferenciabilidade de Funções de várias variáveis. Fórmula de Taylor. Máximos e Mínimos. Transformações. Teorema das Funções Implícitas. Teorema da Função Inversa. \\ \hline
Créditos & 4 (3t,1p) \\ \hline
Carga extra-classe & 6 horas \\ \hline
Responsável pela oferta & DM \\ \hline
\end{tabular}


\subsection{Disciplinas do Quarto Semestre}

\begin{tabular}{|p{4.5cm}|p{10.0cm}|} \hline
Título & Álgebra Linear \\ \hline
Objetivo &  \\ \hline
Pré-requisitos &  \\ \hline
Disciplinas recomendadas &  \\ \hline
Ementa &  \\ \hline
Créditos & 4 (3t,1p) \\ \hline
Carga extra-classe & 6 horas \\ \hline
Responsável pela oferta & DM \\ \hline
\end{tabular}
\\
\\

\begin{tabular}{|p{4.5cm}|p{10.0cm}|} \hline
Título & Organização e Recuperação da Informação \\ \hline
Objetivo &  \\ \hline
Pré-requisitos &  \\ \hline
Disciplinas recomendadas &  \\ \hline
Ementa &  \\ \hline
Créditos & 4 (3t,1p) \\ \hline
Carga extra-classe & 6 horas \\ \hline
Responsável pela oferta & DM \\ \hline
\end{tabular}
\\
\\

\begin{tabular}{|p{4.5cm}|p{10.0cm}|} \hline
Título & Sistemas Operacionais I \\ \hline
Objetivo &  \\ \hline
Pré-requisitos &  \\ \hline
Disciplinas recomendadas &  \\ \hline
Ementa &  \\ \hline
Créditos & 4 (3t,1p) \\ \hline
Carga extra-classe & 6 horas \\ \hline
Responsável pela oferta & DM \\ \hline
\end{tabular}
\\
\\


\begin{center}
\begin{tabular}{|p{4.5cm}|p{10.0cm}|} \hline
Título da Disciplina & Engenharia de Software I \\

\hline Sigla & ES1

\\

\hline
Objetivo &

1.Tornar os alunos aptos a compreenderem processos de desenvolvimento de software;
2.Tornar os alunos aptos a elicitar requisitos e elaborar documentos de requisitos de boa qualidade;
3.Ensinar os alunos a fazer análise e a projetar software de boa qualidade usando a linguagem UML como ferramenta de apoio; 
4.Estimular os alunos a pensar se o projeto criado atende a requisitos de qualidade pré-estabelecidos

\\ \hline
Pré-requisitos & Banco de Dados e Estrutura de Dados

\\ \hline
Disciplinas recomendadas & não há \\ 

\hline
Ementa & Histórico da Engenharia de Software; Visão sobre Ciclo de Vida de Desenvolvimento de Sistemas de Software; Detalhamento do processo de gerenciamento de requisitos com ênfase na elicitação e especificação: documento de requisitos e casos de uso; Apresentação da teoria de Arquitetura de Software; Transformação de Requisitos em Modelos Conceituais (Diagramas de Classes e Diagramas de Sequência do Sistema - DSS); Modelagem comportamental: Diagramas de Estado em nível de análise; Introdução ao Projeto de Software; Transformação dos Modelos de Análise em Modelos de Projeto: Diagrama de Classes e de Pacotes (Subsistemas); Explicitação sobre a forma correta de estruturar as classes para atender requisitos de qualidade declarados no documento de requisitos. Transformação dos Modelos de Análise em Modelos Projeto: Diagramas de Sequência/Colaboração; Diagrama de Estados em nível de projeto. Diagramas de Componentes e de Implantação.



\\ \hline
Créditos & 4 (2t,2p) 

\\ \hline
Carga extra-classe semanal & 4h


\\

\hline 
Bibliografia Básica (3 livros) &

\\

\hline
Bibliografia Complementar (5 livros) &


\\ \hline
Responsável pela oferta & DC \\ \hline
%\textbf{Ref:} http://www.drps.ed.ac.uk/13-14/dpt/cxinfr09027.htm
\end{tabular}
\end{center}



\begin{tabular}{|p{4.5cm}|p{10.0cm}|} \hline
Título & Banco de Dados I \\ \hline
Objetivo &  \\ \hline
Pré-requisitos &  \\ \hline
Disciplinas recomendadas &  \\ \hline
Ementa & Bancos de dados: visão geral, arquitetura, SGBDs, integração com linguagens de programação e sistemas aplicativos; Modelos de dados: relacional, Modelagem de dados: diagrama E-R; Projeto de bancos de dados relacionais; Linguagem
SQL; Transações; Projeto integrador utilizando SGBD de padrão profissional \\ \hline
Créditos & 4 (3t,1p) \\ \hline
Carga extra-classe & 6 horas \\ \hline
Responsável pela oferta & DM \\ \hline
\end{tabular}
\\
\\


\subsection{Disciplinas do Quinto Semestre}

Diferentemente dos semestres anteriores, que haviam apenas disciplinas obrigatórias, o quinto semestre contém também disciplinas optativas. Assim, as próximas subseções apresentam as disciplinas obrigatórias e as optativas. 

\subsubsection{Disciplinas Obrigatórias}


\begin{tabular}{|p{4.5cm}|p{10.0cm}|} \hline
Título & Cálculo Numérico \\ \hline
Objetivo &  O aluno ao sair do curso deverá saber: Aritmética de ponto flutuante. Zeros de funções reais. Sistemas lineares. Interpolação polinomial. Integração numérica. Quadrados mínimos lineares. Tratamento numérico de equações diferenciais ordinárias.  \\ \hline
Pré-requisitos & Cálculo Diferencial e Integral 1 \\ \hline
Disciplinas recomendadas & nenhum \\ \hline
Ementa &   Erros nas representações de números reais. Aritmética de ponto flutuante. Aproximação polinomial de Taylor. Diferenciação numérica.   Zeros reais de funções reais. Métodos: bissecção, Newton e secante.Resolução de sistemas lineares. Métodos diretos: eliminação de Gauss e fatoração LU. Métodos iterativos: Gauss-Jacobi e Gauss-Seidel.  Resolução de sistemas não lineares. Método de Newton. Ajuste de curvas. Método dos quadrados mínimos.Interpolação: o problema; forma de Lagrange; interpolação por partes; erro.   Integração numérica. Fórmulas de Newton-Cotes; erro.  Resolução numérica de equações diferenciais ordinárias. Problemas de Valor Inicial: métodos de Euler, de série de Taylor e de Runge-Kutta. Equações de ordem superior (método de Euler). Problemas de Valor de Contorno: método de diferenças finitas. Erro.
 \\ \hline
Créditos & 4 (3t,1p) \\ \hline
Carga extra-classe & 6 horas \\ \hline
Responsável pela oferta & DM \\ \hline
\end{tabular}
\\
\\



\begin{tabular}{|p{4.5cm}|p{10.0cm}|} \hline
Título & Projeto e Análise de Algoritmos \\ \hline

Objetivo &   \\ \hline
Ementa & Análise de complexidade; Indução matemática, divisão e conquista, algoritmos probabilísticos, algoritmos numéricos, algoritmos geométricos, algoritmos gulosos, programação dinâmica; Algoritmos em grafos: busca em largura e profundidade, Dijkstra,SCC.    \\ \hline
Créditos & 4 (3t,1p) \\ \hline
Carga extra-classe & 6 horas \\ \hline
Responsável pela oferta & DM \\ \hline
\end{tabular}



\subsection{Disciplinas do Sexto Semestre}

\begin{center}
\begin{tabular}{|p{4.5cm}|p{10.0cm}|} \hline
Título & Compiladores \\ \hline
Objetivo & Tornar o aluno não apenas um utilizador de linguagens existentes, mas sim um projetista, com habilidades para criar suas próprias linguagens para situações de diferentes domínios. Desenvolver no aluno a competência para construir um compilador completo utilizando ferramentas de auxílio a construção automática. \\ \hline
Pré-requisitos & Fundamentos de Computação, Lógica Matemática, Matemática Discreta, Teoria da Computação, Paradigmas de Linguagem de Programação, Construção de Algoritmos e Programação \\ \hline
Disciplinas recomendadas & Sistemas Operacionais, Engenharia de Software 1 \\ \hline
Ementa & 1. Apresentação e contextualização; 2. Estrutura de um compilador (etapas de front-end/análise e etapas de back-end/síntese); 3. Análise Léxica; 4. Análise Sintática; 4.1. Análise Sintática Descendente; 4.2. Análise Sintática Ascendente;  5. Análise semântica; 6. Geração e otimização de código; 7. Manipulação de erros; 8. Ferramentas de auxílio à construção de um compilador; 9. Projeto e implementação de um compilador completo, traduzindo uma linguagem de programação simplificada para código executável em arquitetura física ou virtual (50\% dos créditos práticos); 10. Projeto e implementação de um compilador (análise léxica, análise  sintática, análise semântica e geração de código ou interpretação) para um domínio de escolha do aluno (50\% dos créditos práticos). \\ \hline
Créditos & 4 (2t,2p) \\ \hline
Carga extra-classe & 3 horas \helenacomentario{3 créditos segundo o arquivo gradeComExtra.pptx} \\ \hline
Responsável pela oferta & DC \\ \hline
%\textbf{Ref:} http://www.drps.ed.ac.uk/13-14/dpt/cxinfr09007.htm
% Especialistas que escreveram a ementa: Daniel e Helena
\end{tabular}
\end{center}


\begin{center}
\begin{tabular}{|p{4.5cm}|p{10.0cm}|} \hline
Título & Redes de Computadores \\ \hline
Objetivo & \alterar{Tornar o aluno o melhor do mundo no assunto mega-blaster, com habilidades para construir artefatos mega-blaster a partir dos conceitos mega-blaster. Desenvolver no aluno a competência para distinguir o que são artefatos mega-blaster genuínos.} \\ \hline
Pré-requisitos & Circuitos Digitais, Arquitetura e Organização de Computadores, Sistemas Operacionais \\ \hline
Disciplinas recomendadas & não há \\ \hline
Ementa & Comunicação de dados e redes de computadores: visão geral; Redes: estruturas, topologia e meios de transmissão; Modelo ISO/OSI: camadas de aplicação, apresentação, sessão, transporte, rede, enlace, física; sistemas operacionais para redes; Segurança de redes; Projeto em implementação de sistema integrando os  conceitos abordados pela disciplina, preferencialmente em ambiente Linux. \\ \hline
Créditos & 4 (2t,2p) \\ \hline
Carga extra-classe & 6 horas \helenacomentario{3 créditos segundo o arquivo gradeComExtra.pptx} \\ \hline
Responsável pela oferta & DC \\ \hline
%\textbf{Ref:} http://www.drps.ed.ac.uk/13-14/dpt/cxinfr09027.htm
\end{tabular}
\end{center}


\subsection{Disciplinas do Sétimo Semestre}


\begin{center}
\begin{tabular}{|p{4.5cm}|p{10.0cm}|} \hline
Título & Metodologia Científica \\ \hline
Objetivo & Tornar o aluno um melhor cientista, com habilidades para desenvolver projetos de pesquisa empregando a metodologia científica. Desenvolver no aluno a competência para identificar problemas de pesquisa e definir as melhores estratégias a serem aplicadas para solucioná-los. Capacitar o aluno para a produção de material científico relatando as atividades desenvolvidas e os resultados alcançados. \\ \hline
Pré-requisitos & não há \\ \hline
Disciplinas recomendadas & não há \\ \hline

Ementa & 1. Contextualização (o que é a Metodologia Científica, o que é a Pesquisa Científica, seus principais tipos e níveis); 2. Principais conceitos (tema, problema, objetivo, revisão bibliográfica, hipótese, método, técnica, resultados esperados, limitações do trabalho e trabalhos futuros); 3. Etapas de uma pesquisa científica; 4. Ética na pesquisa; 5. Elaboração de um projeto de pesquisa (50\% dos créditos práticos); 6. Escrita científica; 7. Citações e referências bibliográficas; 8. Elaboração de um artigo científico (50\% dos créditos práticos); 9. Apresentação oral de uma pesquisa científica.\\ \hline


Créditos & 4 (2t,2p) \\ \hline
Carga extra-classe & 2 horas \\ \hline
Responsável pela oferta & DC \\ \hline
% PROCURAR
\end{tabular}
\end{center}







NÃO EXISTE MAIS O CONCEITO DE EIXO.... DAQUI PARA BAIXO É DEPRECATED.




\textbf{4.6 - Engenharia de Software II}

\begin{center}
\begin{tabular}{|p{4.5cm}|p{10.0cm}|} \hline
Título & Engenharia de Software II \\ \hline
Objetivo &

1 - Ensinar os alunos a gerenciarem um projeto de software durante todo o seu desenvolvimento;
2 - Tornar os alunos aptos a aplicarem técnicas de garantia de qualidade durante todo o desenvolvimento;


\\ \hline
Pré-requisitos & Engenharia de Software I

\\ \hline
Disciplinas recomendadas & não há \\ 

\hline
Ementa & Aprofundamento sobre Ciclo de Vida de Desenvolvimento e Manutenção de Sistemas; Modelos de Processo e Metodologia Ágil: suas características e diretrizes de escolha; Técnicas de gerenciamento de projetos de software (local e distribuído): Ferramentas de gerenciamento de configuração e de versões; Técnicas de Verificação e Validação de Software: Testes Funcionais e Estruturais; Conceituação e Exemplificação de Tipos de Manutenção de Software; Caracterização de qualidade de software e seu emprego/manutenção ao longo das fases do desenvolvimento.



\\ \hline
Créditos & 4 (2t,2p) \\ \hline
Carga extra-classe & ???

\\ \hline
Responsável pela oferta & DC \\ \hline
%\textbf{Ref:} http://www.drps.ed.ac.uk/13-14/dpt/cxinfr09027.htm
\end{tabular}
\end{center}





\subsection{Eixo Fundamentos da Matemática e Estatística}


\begin{tabular}{|p{4.5cm}|p{10.0cm}|} \hline

Título & Disciplina Exemplo mega-blaster \\ \hline
Objetivo & Tornar o aluno o melhor do mundo no assunto mega-blaster, com habilidades para construir
artefatos mega-blaster a partir dos conceitos mega-blaster. Desenvolver no aluno a competência para distinguir o que são artefatos mega-blaster genuínos. \\ \hline
Pré-requisitos & Disciplina mega-blaster, Disciplina blaster \\ \hline
Disciplinas recomendadas & Programação orientada a blaster, Organização blaster das informações \\ \hline
Ementa & Mega-blaster 1; Mega-blaster 2; Mega-blaster 3; \\ \hline
Créditos & 6 (5t,1p) \\ \hline
Carga extra-classe & 8 horas \\ \hline
Responsável pela oferta & DM \\ \hline
\end{tabular}








\begin{tabular}{|p{4.5cm}|p{10.0cm}|} \hline
Título & Estatística Tecnológica \\ \hline
Objetivo &  Introduzir conceitos fundamentais ao tratamento de dados. Capacitar o aluno a aplicar técnicas estatísticas para a análise de dados na área de computação, e a  apresentar e realizar uma análise crítica dos resultados.
 \\ \hline
Pré-requisitos & Cálculo Diferencial e Integral 1 \\ \hline
Disciplinas recomendadas & nenhum \\ \hline
Ementa &  Conceitos Básicos. Estatística Descritiva. Teoria Elementar de Probabilidade. Variáveis Aleatórias. Distribuição de Probabilidade. Estimação. Intervalo de Confiança. Testes de Hipóteses. Análise de Variância. Análise de Correlação e Regressão. Controle Estatístico de Processo .
 \\ \hline
Créditos & 4 (3t,1p) \\ \hline
Carga extra-classe & 4 horas \\ \hline
Responsável pela oferta & DE\\ \hline
\end{tabular}



\subsection{Eixo Fundamentos de Ciência da Computação}



\begin{quote}
\textbf{2.2- Matemática Discreta}
\\  
\textbf{Ementa:} Teoria dos números; Teoria dos conjuntos; Relações; Funções; Contagem: permutação e combinações; Indução matemática; Recursão; Grafos: teoria.
% Trecho comentado pela Helena em 28/05/2015, pois, pelo o que eu me lembro da reunião que tivemos (Helena, Lucrédio e Heloísa), esse conteúdo não entra na ementa
% e algoritmos ; Exemplos usando uma linguagem de programação funcional.
\\
\textbf{Créditos}: 4:4/0.
\\
\textbf{Ref:} http://people.cs.pitt.edu/~milos/courses/cs441/
\end{quote}



\begin{quote}
%\textbf{2.3-Lógica e Computação}
% Trecho alterado pela Helena em 28/05/2015
\textbf{2.3-Introdução à Lógica}
\\  
% Trecho alterado pela Helena em 28/05/2015
%\textbf{Ementa:} Introdução: modelagem de comportamento e processo computacional;Comportamento: lógica Proposicional;  Computação: máquinas de estado finito (FSM), expressões regulares, linguagens regulares, linguagens livres de contexto. Exemplos práticos do uso de lógica e FSMs na implementação de sistemas; Exemplos usando uma linguagem de programação lógica.
% Sugestão: mesma ementa do PPP atual - copiar de lá!
\textbf{Ementa:} Lógica Proposicional e Lógica de Predicados.
\\
% Trecho alterado pela Helena em 28/05/2015
%\textbf{Créditos}: 6:4/6.
\textbf{Créditos}: 4:4/0.
\\
\textbf{Ref} :http://www.inf.ed.ac.uk/teaching/courses/inf1/cl/
\end{quote}










\begin{quote}
\textbf{2.6- Paradigmas e Linguagens de Programação}
\\  
\textbf{Ementa:} Introdução: paradigmas de linguagens de programação e sua adequação a problemas específicos; Princípios: nomes, tipos, sintaxe, semântica; Desafios para implementação de linguagens e ferramentas de apoio à programação; Programação Imperativa; Programação orientada a objetos; Programação Funcional; Programação Lógica; Programação Concorrente; Linguagens de marcação; Prática de programação comparativa utilizando os paradigmas imperativo, orientado a objetos, funcional e lógico.
\\
\textbf{Créditos}: 4:2/2.
\\
\textbf{Ref:} http://courses.cs.washington.edu/courses/cse341/14wi/syllabus.html
\end{quote}




\subsection{Eixo Algoritmos e Programação}


\begin{quote}
\textbf{3.3- Programação Orientada a Objetos}
\\  
\textbf{Ementa:} Programação orientada a objetos: visão geral e comparação com programação estruturada;Classes, objetos; construtores, métodos; Sobrecarga de operadores; Herança de classes; Gerenciamento de Exceções; Programação orientada a eventos; Projeto integrador utilizando a linguagem Java, preferencialmente exercitando conceitos da disciplina estrutura de dados.
\\
\textbf{Créditos}: 8:4/4.
\\
\textbf{Ref:} http://www.cs.rit.edu/~vcss243/schedule.html
\end{quote}








\subsection{Eixo Metodologias e Técnicas da Computação}

\helena{}{O eixo de Metodologias e Técnicas da Computação totaliza 60 créditos do curso e engloba as disciplinas de: Circuitos Digitais, Arquitetura e Organização de Computadores, Sistemas Operacionais, Redes de Computadores, Computação Gráfica, Computação Concorrente, Interação Humano-Computador, Engenharia de Software 1, Engenharia de Software 2, Software para Web e Dispositivos Móveis, Compiladores, Inteligência Artificial e Processamento de Imagens. A seguir são apresentadas as fichas de caracterização dessas disciplinas.}

\singlespacing


\textbf{Computação Gráfica }


\begin{tabular}{|p{4.5cm}|p{10.0cm}|} \hline

Título & Computação Gráfica \\ \hline
Objetivo & Fornecer uma introdução à síntese de imagens, representação computacional de cenas e programação 2D e 3D. \\ \hline
Pré-requisitos &  Geometria Analítica, Álgebra Linear, Programação de Computadores \\ \hline
Disciplinas recomendadas & - \\ \hline
Ementa & Introdução. Arquitetura Gráfica. API Gráfica OpenGL.Transformações geométricas no plano e no espaço: Transformações Afins; Projeções; Câmera; Transformações geométricas em OpenGL. Modelagem e Representação de Objetos. Cores. Animação. Rendering: Rasterização; Iluminação; Sombras; Textura; Z-Buffer; Scan-Line e Ray-Traycing. \\ \hline
Créditos & 4 (2t,2p) \\ \hline
Carga extra-classe & 4 horas \\ \hline
Responsável pela oferta & DC \\ \hline
\end{tabular}


\textbf{Processamento de Imagens}


\begin{tabular}{|p{4.5cm}|p{10.0cm}|} \hline

Título & Processamento de Imagens \\ \hline
Objetivo & Fornecer uma introdução à teoria e aplicações de processamento digital de imagens. Os tópicos irão incluir fundamentos de aquisição de imagens, realce de imagens, filtros e transformadas, segmentação e aplicações. \\ \hline
Pré-requisitos &  Cálculo I, Geometria Analítica, Álgebra Linear, Introdução à Programação \\ \hline
Disciplinas recomendadas & - \\ \hline
Ementa & Introdução (etapas de um sistema de processamento de imagens); Técnicas de modificação de histogramas; Filtragem espacial de imagens (Filtros Lineares, Teorema da Convolução e Filtros NãoLineares); Filtragem de Imagens no domínio da frequência; Processamento multiresolução; Processamento morfológico; Segmentação de Imagens; Representação e descrição; Classificação de Imagens. \\ \hline
Créditos & 4 (2t,2p) \\ \hline
Carga extra-classe & 4 horas \\ \hline
Responsável pela oferta & DC \\ \hline
\end{tabular}

\textbf{Computação de Alto Desempenho}


\begin{tabular}{|p{4.5cm}|p{10.0cm}|} \hline

Título & Computação de Alto Desempenho \\ \hline
Objetivo & Ensinar princípios de programação distribuída e também paralela usando múltiplas estações e núcleos de processamento. \\ \hline
Pré-requisitos &  Redes,  Sistemas Operacionais, Arquitetura de Computadores, Programação de Computadores \\ \hline
Disciplinas recomendadas & - \\ \hline
Ementa & Programação concorrente: conceitos, modelos e paradigmas; Programação paralela e multithreading usando threads e variáveis compartilhadas: exclusão mútua
e sincronização, locks, barreiras, semáforos, monitores, sincronização de processos, paralelismo em nível de dados, arquiteturas multicore, exemplos usando PThreads, OpenMP e JavaThreads; Programação distribuída usando processos: Conceitos básicos sobre sistemas distribuídos, programação utilizando troca de mensagens, interação entre processos, ambientes de programação distribuída: Java RMI, Web Services; Computação científica  usando MPI \\ \hline
Créditos & 4 (2t,2p) \\ \hline
Carga extra-classe & 4 horas \\ \hline
Responsável pela oferta & DC \\ \hline
\end{tabular}




\textbf{4.1 - Lógica Digital } % Menotti

\normalem

\begin{center}
\end{center}

\textbf{4.2 - Arquitetura e Organização de Computadores} % - Menotti



\textbf{4.3 - Sistemas Operacionais}

\begin{center}
\begin{tabular}{|p{4.5cm}|p{10.0cm}|} \hline
Título & Sistemas Operacionais \\ \hline
Objetivo & Tornar o aluno melhor no entendimento de um sistema operacional, suas funcionalidades, abstrações e peculiaridades, com habilidades para construir programas que usem eficientemente os recursos e serviços providos pelos sistemas operacionais. Desenvolver no aluno a competência para entender e atuar no projeto e na criação de sistemas operacionais. \\
% Versão original enviada pelo Hélio = & \alterar{    Fazer com que os alunos conheçam as funcionalidades providas pelos Sistemas Operacionais como gerenciadores de recursos;     Tornar o aluno ciente dos algoritmos e das abstrações utilizadas em projetos de sistemas operacionais para o gerenciamento de atividades a executar (processos e threads) e para o armazenamento de dados (arquivos);     Habilitar o aluno a identificar os requisitos existentes para diferentes tipos de sistemas computacionais e suas implicações no projeto do sistema operacional (sistemas de tempo-real, servidores, dispositivos com capacidades de software e hardware limitadas);     Tornar os alunos aptos a criar programas que usem eficientemente os recursos e serviços providos por sistemas operacionais;     Tornar os alunos aptos a entender e atuar no projeto e no desenvolvimento de sistemas operacionais.} \\ 
\hline
Pré-requisitos & Circuitos Digitais, Arquitetura e Organização de Computadores, Estrutura de dados \\ \hline
Disciplinas recomendadas & não há \\ \hline
Ementa & 1. Histórico e evolução
2. Interface do usuário com o SO (gráfica, shell e utilitários)
	2.1. Prática com sistema de arquivo, permissões, processos e sinais (10\% dos créditos práticos)
3. Processo, threads e gerenciamento do processador
	3.1. Representação de linhas de execução
	3.2. Contextos e estados de execução
	3.3. Gerenciamento dos processadores
	3.4. Políticas de escalonamento
	3.5. Contabilizações
	3.6. Serviços do SO para gerenciamento do processador (15\% dos créditos práticos)
4. Gerenciamento de memória
	4.1. Histórico e evolução
	4.2. Memória virtual (segmentação e paginação)
	4.3. Áreas de troca
	4.4. Serviços do SO para gerenciamento de memória (10\% dos créditos práticos)
5. Comunicação e sincronização de processos e threads
	5.1. Comunicação entre processos
	5.2. Sincronização entre processos
	5.3. Comunicação e sincronização com threads
	5.4. Serviços do SO para comunicação e sincronização de processos e threads (15\% dos créditos práticos)
6. Gerenciamento de armazenamento
	6.1. Interação com dispositivos de armazenamento
	6.2. Organização de espaços de armazenamento não volátil
	6.3. Serviços do SO para gerenciamento de armazenamento e E/S (10\% dos créditos práticos)
7. Estudo de caso com sistemas operacionais
	7.1. Estudo do núcleo (kernell) de um SO aberto (40\% dos créditos práticos)
. \\ \hline
Créditos & 8 (4t,4p) \\ \hline
Carga extra-classe & 6 horas \\ \hline
Responsável pela oferta & DC \\ \hline
%\textbf{Ref:} http://www.drps.ed.ac.uk/13-14/dpt/cxinfr09015.htm
% Especialista que escreveu a ementa: Hélio
\end{tabular}
\end{center}



\textbf{4.5 - Engenharia de Software I}




\textbf{4.7 - Interface Humano-Computador}








\begin{quote}
\textbf{4.6- Computação Concorrente}
\\  
\textbf{Ementa:} Programação concorrente: conceitos, modelos e paradigmas; Programação paralela e multithreading usando threads e variáveis compartilhadas: exclusão mútua e sincronização, locks, barreiras, semáforos, monitores, sincronização de processos, paralelismo em nível de dados, arquiteturas multicore, exemplos usando PThreads, OpenMP e JavaThreads; Programação distribuída  usando processos: Conceitos básicos sobre sistemas distribuídos, programação utilizando troca de mensagens, interação entre processos, ambientes de programação distribuída:  Java RMI, Web Services; Computação científica usando MPI.  Projetos integradores abordando computação paralela em nível de dados, e computação distribuída.
\\
\textbf{Créditos}: 6:4/2.
\\
\textbf{Ref:} http://www.ict.kth.se/courses/ID1217/
\end{quote}



\textbf{4.7 - Compiladores}



\begin{quote}
\textbf{4.8-  Software para Web e Dispositivos Móveis}
\\  
\textbf{Ementa:} Visão geral tecnológica e comercial abordando sistemas Web, dispositivos e aplicativos móveis. Arquiteturas de hardware e software para sistemas Web; Frameworks de desenvolvimento: MVC (ModelViewControler); Uso de bancos de dados em sistemas web; Interfaces em sistemas web; Arquiteturas de hardware e software para dispositivos móveis; Frameworks para desenvolvimento de aplicativos móveis; Interação entre dispositivos móveis e sistemas de retaguarda via Web; Aspectos de segurança; Modelo de computação na nuvem; Projeto integrador envolvendo aplicação Web e dispositivos móveis.
\\
\textbf{Créditos}: 6:2/4.
\\
\textbf{Ref:} http://www.cs.toronto.edu/~mashiyat/csc309/index.htm
\end{quote}

\doublespacing

\subsection{Eixo Especializações}

\helenacomentario{Inserir texto introdutório explicando o que tem nessa seção}

\textbf{5.1- Inteligência Artificial }

\singlespacing

\begin{center}
\begin{tabular}{|p{4.5cm}|p{10.0cm}|} \hline
Título & Inteligência Artificial \\ \hline
Objetivo & Tornar o aluno o melhor na capacidade de representação de conhecimento, com habilidades para construir algoritmos a partir dos conceitos da IA. Propiciar ao aluno a aquisição dos conceitos relacionados a busca, representação de conhecimento e raciocínio automático. Desenvolver no aluno a competência para saber identificar problemas que podem ser resolvidos com técnicas da IA e quais técnicas podem ser adequadas a cada problema. \\ \hline
Pré-requisitos & Lógica Matemática, Matemática Discreta, Teoria da Computação, Construção de Algoritmos e Programação \\ \hline
Disciplinas recomendadas & Paradigmas de Linguagem de Programação, Sistemas Operacionais, Engenharia de Software 1 \\ \hline
Ementa & 1. Caracterização da área de IA; 2. Métodos de busca para a resolução de problemas; 2.1. Busca desinformada: busca em largura, busca de custo uniforme e busca em profundidade; 2.2. Busca informada: subida da encosta, busca heurística e busca competitiva (jogos); 3. Representação de conhecimento; 3.1. Representação de conhecimento baseada em lógica; 3.2. Representação de conhecimento baseada em categorias e objetos; 3.3. Representação de conhecimento incerto; 4. Métodos de raciocínio e inferência; 4.1. Algoritmo de encadeamento para frente; 4.2. Algoritmo de encadeamento para trás; 4.3. Resolução; 5. Aprendizado de máquina; 5.1. Aprendizado de máquina supervisionado; 5.2. Aprendizado de máquina não supervisionado; 5.3. Modelos conexionistas (redes neurais); 5.1.1. Algoritmos para classificação e regressão (como aprendizagem em árvores de decisão e árvores de regressão); 5.2.1. Algoritmos de agrupamento; 5.2.2. Algoritmos de extração de regras de associação; 6. Aplicações de IA (apoio a decisão, sistemas de recomendação, classificação entre outros); \helenacomentario{Isso vai ser mantido?:} 7. Projeto integrador aplicando os conceitos teóricos da disciplina. \\ \hline
Créditos & 4 (2t,2p) \helenacomentario{Hélio acha que podem ser 6 (3t,3p), mas não vê problema em deixar como está }\\ \hline
Carga extra-classe & 4 horas \helenacomentario{3 créditos segundo o arquivo gradeComExtra.pptx} \\ \hline
Responsável pela oferta & DC \\ \hline
%\textbf{Ref:} http://courses.cs.washington.edu/courses/cse415/14sp/
% Especialista que escreveu a ementa: Heloísa
\end{tabular}
\end{center}




\begin{quote}
\textbf{5.X- OPTATIVA X }
\\  
\textbf{Ementa:} A DEFINIR.
\\
\textbf{Créditos}: 6:4/2.
\\
\textbf{Ref:}
\end{quote}




\subsection{Eixo Humanas e Orientações}


\textbf{6.1- Redação de Textos Técnicos }
\begin{center}
\begin{tabular}{|p{4.5cm}|p{10.0cm}|} \hline
Título & Redação de Textos Técnicos \\ \hline

Objetivo & Tornar o aluno capaz de compreender e produzir textos técnicos, com habilidades para elaborar diferentes tipos de documentos a partir dos conceitos aprendidos nesta disciplina. Desenvolver no aluno a competência para dominar as técnicas de redação e aplicá-las nas diferentes situações do cotidiano acadêmico e profissional.\\ \hline
Pré-requisitos & não há \\ \hline
Disciplinas recomendadas & não há \\ \hline


Ementa & 1. Noções básicas de escrita técnica; 2. Planejamento, organização e produção de textos técnicos (relatórios, projetos, planos de negócio, documentação de software, entre outros exemplos); 3. Técnicas de argumentação; 4. Normatização (ABNT e outros); 5. Ferramentas de auxílio à escrita (editores de texto e suas funcionalidades); 6. Elaboração de textos técnicos.  
\\ \hline
Créditos & 2 (1t,1p) \\ \hline
Carga extra-classe & 2 horas \\ \hline
Responsável pela oferta & DC \\ \hline
% Fonte: Objetivo e Ementa em parte retirado de http://ulbra-to.br/cursos/Pedagogia/2009/2/turmas/0780/impressao-plano
% Ementa em parte retirado de: http://www.ic.unicamp.br/node/418
\end{tabular}
\end{center}



\begin{comment}
\textbf{6.3- Ética na Computação }
\begin{center}
\begin{tabular}{|p{4.5cm}|p{10.0cm}|} \hline
Título & Ética na Computação \\ \hline
Objetivo & Tornar o aluno um profissional ético, com capacidade de atuar de maneira ética nos diversos setores da computação. Desenvolver no aluno a competência para distinguir as falhas éticas e evitá-las em sua vida profissional. \\ \hline
Pré-requisitos & não há \\ \hline
Disciplinas recomendadas & não há \\ \hline
Ementa & 1. Contextualização (o que é ética); 2. Princípios éticos na computação; 3. Códigos de ética na computação; 4. Casos reais de faltas éticas e suas consequências; 5. Legislação correspondente; 6. Diretrizes para uma conduta ética acadêmica e profissional na computação; 7. Elaboração de trabalhos práticos relacionados ao tema.  \\ \hline
Créditos & 4 (2t,2p) \\ \hline
Carga extra-classe & 2 horas\\ \hline
Responsável pela oferta & DC \\ \hline
% PROCURAR
\end{tabular}
\end{center}
\end{comment}

\begin{quote}
\textbf{6.2- Administração }
\\  
\textbf{Ementa:} A DEFINIR.
\\
\textbf{Créditos}: 4:4/0.
\\
\textbf{Ref:}
\end{quote}



\begin{quote}
\textbf{6.3- Trabalho de Conclusão de Curso 1}
\\  
\textbf{Ementa:} A DEFINIR.
\\
\textbf{Créditos}: 10:0/10.
\\
\textbf{Ref:}
\end{quote}




\begin{quote}
\textbf{6.4- Estágio}
\\  
\textbf{Ementa:} A DEFINIR.
\\
\textbf{Créditos}: 24:24/0.
\\
\textbf{Ref:}
\end{quote}



\begin{quote}
\textbf{6.5- Trabalho de Conclusão de Curso 2}
\\  
\textbf{Ementa:} A DEFINIR.
\\
\textbf{Créditos}: 24:0/24.
\\
\textbf{Ref:}
\end{quote}





\textbf{Optativa: Administração de Banco de Dados}


\begin{tabular}{|p{4.5cm}|p{10.0cm}|} \hline
Título & Administração de Banco de Dados\\ \hline
Objetivo & Ensinar princípios de monitoramento de desempenho do banco, segurança, criação de estruturas físicas e lógicas do banco, inicialização de serviços e execução de backups. \\ \hline
Pré-requisitos & Banco de Dados \\ \hline
Disciplinas recomendadas & - \\ \hline
Ementa & Análise de desempenho e ajustes de projeto, índices e consultas. Processamento de transações e log. Recuperação em Banco de Dados: atualização imediata e paginação de sombra. Recuperação no caso de uso de múltiplos bancos e em caso de catástrofe. Segurança: controle de acesso: controle de acesso, controle de fluxo e criptografia. Preservação de privacidade em Banco de dados. \\ \hline
Créditos & 4 (2t,2p) \\ \hline
Carga extra-classe & 4 horas \\ \hline
Responsável pela oferta & DC \\ \hline
\end{tabular}

\textbf{Estatística Tecnológica}
\begin{center}
\begin{tabular}{|p{4.5cm}|p{10.0cm}|} \hline
Título & Estatística Tecnológica \\ \hline
Objetivo & Familiarizar os alunos com metodologia básica para a coleta e tratamento de dados experimentais e medições, proporcionando oportunidade de aplicar o conhecimento estatístico em sua própria área de atuação. \\ \hline
Pré-requisitos & não há \\ \hline
Disciplinas recomendadas & não há \\ \hline
Ementa & 1. Origem e tipos de erros. Independência de dados.2. Histogramas, Probabilidades e Densidades de probabilidades com seus parâmetros.3. Distribuições Binomial, de Poisson, Normal, Qqui-quadrado e suas aplicações. 4. Distribuição da Média Amostral. A distribuição normal como limite de outras distribuições. Propagação de erros. 5. Método de Máxima verossimilhança. Método de Mínimos quadrados. Ajuste de polinômios. Funções lineares e não lineares nos parâmetros.  \\ \hline
Créditos & 4 (3t,1p) \\ \hline
Carga extra-classe & 2 horas\\ \hline
Responsável pela oferta & DES \\ \hline
%fonte: Siga, disciplina ja existente e ofertada pelo DES
\end{tabular}
\end{center}

\textbf{Mineração de Dados}
\begin{center}
\begin{tabular}{|p{4.5cm}|p{10.0cm}|} \hline
Título & Mineração de Dados \\ \hline
Objetivo & Fazer com que o aluno adquira uma compreensão abrangente sobre mineração de dados (\textit{data mining}) e descoberta
de conhecimento em bancos de dados e esteja apto a aplicar mineração de dados para a resolução de problemas práticos. \\ \hline
Pré-requisitos & - \\ \hline
Disciplinas recomendadas & Estatística Tecnológica e Banco de Dados \\ \hline
Ementa & 1. Introdução e visão geral das tarefas de mineração e processo de descoberta de conhecimento. 2. Pré-processamento de dados (limpeza, seleção e transformação de características, normalização e discretização). 3. Classificação, técnicas baseadas em árvores de decisão (ganho de informação), teorema de Bayes, redes neurais=, algorítimo genético e knn. 4. Medidas de qualidade de um classificador. 5.  Agrupamento usando k-means. 6. Associações, técnicas baseadas no algoritmo Apriori. 7. Mineração de Dados e Sociedade. 8. Tendências de emprego na área.\\ \hline
Créditos & 4 (2t,2p) \\ \hline
Carga extra-classe & 3 horas  \\ \hline
Responsável pela oferta & DC \\ \hline
%\textbf{Ref:} http:http://www.kdnuggets.com/data_mining_course/syllabus.html, livro: Data Mining: Concepts and Techniques, Third Edition, by Jiawei Han  (Author), Micheline Kamber  (Author), Jian Pei (Author)
\end{tabular}
\end{center}

\textbf{Programação para Ciência de Dados}
\begin{center}
\begin{tabular}{|p{4.5cm}|p{10.0cm}|} \hline
Título & Programação para Ciência de Dados \\ \hline
Objetivo & Ensinar o aluno a desenvolver seu próprio programa para tratamento e análise de dados usando linguagem R. \\ \hline
Pré-requisitos & - \\ \hline
Disciplinas recomendadas & Estatística Tecnológica,  Banco de Dados e Mineração de Dados\\ \hline
Ementa & 1. Introdução a ciência de dados. 2. Visão geral de um programa. Scripts.  3. Programação exploratória de dados. 4. Importação de dados. 5. Visualização. 6. Transformação de Dados. 7. Aplicação de algoritmos tradicionais de mineração usando R.  ( \\ \hline
Créditos & 4 (2t,2p) \\ \hline
Carga extra-classe & 3 horas  \\ \hline
Responsável pela oferta & DC \\ \hline
%\textbf{Ref:} livro: R for Data Science http://r4ds.had.co.nz/
\end{tabular}
\end{center}

\textbf{Visualização de Dados}
\begin{center}
\begin{tabular}{|p{4.5cm}|p{10.0cm}|} \hline
Título & Visualização de Dados \\ \hline
Objetivo & Ensinar aos alunos princípios e técnicas de visualização de informação e como aproveitar melhor os recursos visuais e hardware gráfico. \\ \hline
Pré-requisitos &  Computação Gráfica \\ \hline
Disciplinas recomendadas &  \\ \hline
Ementa & 1.Dados tabulares. 2.Coordenadas paralelas. 3.Gráficos tridimensionais. 4.Árvores. 5.Projeções multidimensionais. 6.Cores e percepção. 7.Múltiplas visualizações. 8.Visualização de volume e superfície. 
 \\ \hline
Créditos & 4 (2t,2p) \\ \hline
Carga extra-classe & 3 horas  \\ \hline
Responsável pela oferta & DC \\ \hline
%\textbf{Ref:} livro: Information Visualization, Colin Ware
%http://www.sciencedirect.com/science/book/9780123814647
\end{tabular}
\end{center}
