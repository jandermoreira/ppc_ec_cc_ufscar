\chapter{Marco Referencial do Curso}\label{sec:marco_referencial}

% helio
% Será exagero falar em sólida formação em Matemática? 
% Tem sentido a frase abaixo?
% Além de prover formação nas inúmeras áreas da computação, egressos são capacitados para atuar em áreas que têm grande demanda de aplicação de computação em ciência e engenharia e em áreas que requerem especialistas para manutenção de infraestrutura computacional.

%O marco conceitual do curso define o perfil do egresso. Os egressos do curso possuem sólida formação em Ciência da Computação e Matemática, estando preparados para projetar e implementar sistemas computacionais de propósito geral; sistemas de infraestrutura de software, sistemas embarcados, entre outros tipos de sistemas. Além disso, o perfil do egresso a ser formado leva em conta as necessidades locais e regionais. Atualmente, as empresas necessitam de profissionais mais bem habilitados a lidar com a grande quantidade de tecnologias que surgem dia a dia na área da computação. 

\section{Contextualização da Área Ciência da Computação}


Já não é mais possível separar a computação da vida em sociedade. Desde algumas décadas atrás, a tecnologia da computação passou a fazer parte de todos os aspectos da vida humana, chegando ao ponto em que se torna muito difícil encontrar lugares onde não existe ao menos um dispositivo computacional oferecendo suporte às atividades humanas. Para o futuro, a tendência é que essa simbiose entre humanos e computadores se torne cada vez mais presente.

Além da ampla presença, é importante destacar a profundidade e a complexidade com que a computação penetra na sociedade. Diferentemente da eletricidade, por exemplo, que está também presente mas que se limita à atuação no mundo físico, a computação tem potencial para utilizar e transformar a informação, o que é mais abstrato e que pode ser muito mais impactante para a sociedade.

Nesse cenário, a ciência da computação possui papel central. Em muitas atividades, sejam da indústria, comércio, saúde, entre outras, o domínio da computação passou a ser determinante não apenas para a eficiência, mas também para o sucesso das mesmas. Existem inúmeros exemplos de avanços tecnológicos em que a presença de computação tornou possível a realização de atividades que antes eram inviáveis, como: a análise de imensos volumes de dados que permite a busca de determinada informação em meio a trilhões de páginas de texto, ou de rostos em meio a uma multidão; o vôo de quadricópteros, impossível de ser realizado manualmente por um ser humano; e num futuro muito próximo, a condução de automóveis com uma taxa mínima de acidentes.

Assim sendo, a sociedade está e continuará constantemente interessada em novas formas de utilizar computação em inúmeros setores. Isso torna o cenário extremamente dinâmico, o que constitui um desafio para a universidade. O profissional em computação deve possuir habilidades não apenas para aplicar conceitos e tecnologias já existentes, mas também para pesquisar e desenvolver novas tecnologias. Assim, é fundamental que o seu conhecimento não seja restrito àquele desenvolvido e estabelecido. O profissional em computação deve ser capaz de construir e renovar seu conhecimento continuamente, por vezes revendo conceitos desatualizados. Sob este ponto de vista, o profissional de computação deve ser, antes de tudo e ao longo de sua carreira, um cientista.

A complexidade do desenvolvimento tecnológico em computação também traz um outro aspecto e um novo desafio para a universidade. Em função do grande desenvolvimento da área nas últimas décadas, já não é mais possível que um único profissional consiga atuar em qualquer área em que a computação está envolvida. A computação evoluiu em sub-áreas do conhecimento que exigem grande grau de aprofundamento para que o profissional possa trabalhar de forma eficaz e com a competência necessária para o sucesso em sua atuação.

Por fim, a ubiquidade da computação leva o profissional para dentro de todos os setores da sociedade. Ainda que exista espaço para a atuação em computação de maneira pura, na maioria dos casos o cientista da computação precisará trabalhar de maneira muito próxima a profissionais de diversas áreas do conhecimento. Sob este ponto de vista, portanto, a computação não deixa de ter o caráter de atividade meio.

O papel da universidade, nesse contexto, é o de preparar o cientista da computação para atuar na sociedade, empregando suas habilidades para resolver problemas que promovam o bem estar da sociedade. Dado o cenário apresentado, a universidade deve:

\begin{enumerate}
    \item fornecer meios para que o futuro cientista alcance o aprendizado do conhecimento central em computação que já se estabeleceu nas últimas décadas;
    \item fornecer meios para que o estudante possa se aprofundar em uma determinada sub-área da computação, de forma que ele possa dominar com mais profundidade aquela área de conhecimento;
    \item oferecer meios para que o estudante desenvolva habilidades de pesquisa, investigação, desenvolvimento e inovação tecnológica, permitindo que ele possa ir além do conhecimento adquirido e consiga construir seu próprio conhecimento após formado;
    \item estimular, sempre que possível, a proximidade dos estudantes com os diversos setores da sociedade, seja trazendo profissionais para palestras e seminários, ou apresentando desafios e projetos que envolvem a aplicação prática da computação nesses setores.
\end{enumerate}

%\mario{}{Acho que não concordo com o seguinte parágrafo. Pra mim as versões anteriores do curso ainda tentavam ensinar tudo para todos, sem tanta preocupação em separar o que é central e o que é especialização.}
As versões anteriores do curso não faziam uma clara distinção entre o conhecimento central e os aprofundamentos, tratando ambos da mesma forma dentro do projeto pedagógico. A presente reformulação do projeto pedagógico do curso tem uma maior preocupação em separar o conhecimento central de certos aprofundamentos e tratá-los de forma diferenciada, conforme os itens 1 e 2 apresentados acima. Também tem a premissa de aumentar as ações referentes aos itens 3 e 4, alcançando um balanceamento melhor entre as habilidades que serão cada vez mais necessárias para o cientista da computação.

\section{Décadas de Evolução do Bacharelado em Ciência da Computação }

O Bacharelado em Ciência da Computação da Universidade Federal de São Carlos, campus São Carlos, foi
criado em 1973, implantado em 1975 e reconhecido pelo Ministério da Educação – MEC, por meio do
Parecer no. 1522/79, em 11 de novembro de 1979.
Foi implantado no mesmo ano da Licenciatura em Matemática e do
Bacharelado em Ciências Biológicas. Na época, existiam apenas 7 (sete) cursos na Universidade,
sendo 4 (quatro) no Centro de Ciências Exatas e de Tecnologia. Um
desses quatro cursos era o de Processamento de Dados, implantado em 1974, que acabou sendo
extinto em 1986.

Em 1997, o curso foi submetido a um processo de auto-avaliação dentro do Programa de
Avaliação Institucional das Universidades Brasileiras – PAIUB-SESu/MEC, com a participação de
seus alunos, docentes, egressos dos últimos cinco anos e funcionários. O processo desenvolveu-se
com o objetivo de analisar o curso enquanto unidade organizacional, nos seguintes aspectos: perfil
profissional, currículos e programas, condições de funcionamento, desempenho dos docentes e
discentes. Ele também permitiu detectar aspectos positivos e negativos do curso, em nível de
profundidade bastante significativo.
Desde a implantação do curso é realizado um trabalho de acompanhamento que visa avaliar
sua estrutura curricular, assim como seguir as exigências impostas pela evolução natural da área de
computação no país. 

Em 1994 foi realizada uma outra reformulação do curso, e a renovação do reconhecimento foi homologada pela Comissão de
Especialistas e de Verificacação (CEEInf) em 30 de outubro de 2001. A avaliação realizada
atribuiu ao curso o conceito global A e apontou os pontos fortes e fracos em relação aos seguintes
aspectos: corpo docente, plano pedagógico, infra-estrutura e desempenho. 

Em 2006, uma outra reformulação do curso foi realizada no sentido de  sanar problemas detectados nas avaliações comentadas anteriormente bem como adequar o curso à nova
Lei de Diretrizes e Bases da Educação Nacional (Lei no. 9394, de 20 de dezembro de 1996) e às
seguintes determinações do Conselho de Ensino, Pesquisa e Extensão da UFSCar: Parecer CEPE
no. 776/01, que estabelece o perfil geral do profissional a ser formado na UFSCar e Portaria GR no.
771/04, que dispõe sobre normas e procedimentos referentes às atribuições de currículo, criações,
reformulações e adequações curriculares dos cursos de graduação da UFSCar.

Em 2013, novamante novas alterações mais pontuais foram realizadas no sentido da incorporação de disciplinas voltadas ao aprimoramento da capacidade de desenvolvimento de projetos dos alunos. 

Esta atual reformulação do curso está em consonância com o estabelecido na Resolução número 5, de 16 e novembro de 2016, que institui as Diretrizes Curriculares Nacionais para os cursos de graduação na área de Computação, abrangendo os cursos de Bacharelado em Ciência da Computação, Bacharelado em Sistemas de Informação, Engenharia de Computação, Bacharelado em Engenharia de Software e de Licenciatura em Computação ~\cite{DCN-BCC}. Além disso, também está em consonância tanto com o Plano de Desenvolvimento Institucional~(PDI) da UFSCar~\cite{PDI-UFSCar} quanto com o Regimento Geral dos Cursos de Graduação dessa Universidade~\cite{Regimento-Geral-CursosGraduacao-UFSCar}.

A iniciativa de reformulação do curso foi iniciada em 2013 com a Profa. Dra. Marcela Xavier Ribeiro e o Prof. Dr. Márcio Merino Fernandes. Na época, ambos ocupavam as posições de coordenadora do Bacharelado em Ciência da Computação (BCC) e chefe do Departamento de Computação, permanecendo nesses cargos até outubro de 2014.

Após este período, a chefia do Departamento passou pelo Prof. Dr. Jander Moreira e pelo Prof. Dr. Hélio Crestana Guardia, enquanto a coordenação do curso permaneceu com o Prof. Dr. Valter Vieira de Camargo por dois mandatos consecutivos, até outubro de 2018.

Para alcançar a proposta atual do curso, o processo de reestruturação foi longo, repleto de reuniões e troca de ideias, principalmente no contexto do Núcleo Docente Estruturante (NDE) do BCC. Os principais fatores que motivaram a reestruturação do curso foram a inflexibilidade da matriz curricular anterior, alta taxa de retenção dos estudantes, relatos de professores e estudantes, dados analíticos disponibilizados pela Comissão Própria de Avaliação da UFSCar, além de extensa pesquisa em projetos pedagógicos de cursos de Ciência da Computação do Brasil e do exterior. 



