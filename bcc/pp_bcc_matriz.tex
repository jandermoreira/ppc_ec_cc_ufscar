




\section{Integralização de Créditos}

Para integralização do curso de Bacharelado em Ciência da Computação, o aluno deverá ser aprovado em 216 créditos (3240 horas) Os créditos para integralização do curso estão assim distribuídos:

\begin{itemize}
\item Disciplinas Obrigatórias:  118 créditos de disciplinas obrigatórias;
\item Formação Complementar XXXX créditos, assim divididos:
\begin{itemize}
\item Disciplinas Optativas Profissionalizantes - XXXX créditos
\item Disciplinas para o Desenvolvimento Humano e Complementar - XXXX créditos
\item Estágio - 24 créditos.
\end{itemize}
\end{itemize}


\subsubsection{Carga Horária}

\paragraph{Carga Horária Semestral do Curso}

O curso tem a seguinte carga horária semestral:
 XXXXX horas de atividades divididas da seguinte maneira: XXXX horas de atividades em sala de aula (XXXX créditos)+ XXXXX horas de atividades extra-classe;
 
\paragraph{Carga Horária Semanal do Curso}

\textbf{As atividades do curso deverão ser concebidas e planejadas prevendo a seguinte carga horária semanal aos alunos: um total de 46 horas divididas da seguinte maneira: 
- 26 horas em sala de aula (26 créditos) + 4 horas de atividades extra-classe por disciplina (15 horas). Isto não deve ser interpretado como uma regra absoluta, uma vez que é natural que ocorram \lq\lq picos \rq\rq de trabalho ao longo do semestre. Porém, é uma diretriz que deverá ser perseguida o máximo possível.}




\subsubsection{Núcleo de Conteúdos Básicos}
A carga horária mínima do núcleo de conteúdos básicos previsto pela Resolução CNE/CES 11/2002 é de cerca de 30\% do total previsto. Como acreditamos que esta formação básica é fundamental para capacitar o egresso a aprender de forma autônoma e continuada, tornando-o mais versátil e capaz de acompanhar o desenvolvimento tecnológico (a especificidade se torna obsoleta rapidamente, o fundamental persiste), especial atenção foi direcionada para a composição deste núcleo. Essa formação básica será adquirida pelo estudante, através dos seguintes grupos de conhecimentos provenientes das ciências básicas e ciências humanas.

Os conteúdos de Computação Clássica, relacionados à formação geral do Cientista de Computação, darão ao aluno, o aprendizado dos conceitos fundamentais que constituem a base da área computacional, necessários para a sua atuação profissional. 

\subsubsection{Núcleo de Conteúdos Profissionalizantes}

De acordo com o parágrafo 3º, Artigo 6º, a Resolução CNE/CES 11/2002, o Núcleo de Conteúdos Profissionalizantes deve ser composto por cerca de 15\% de carga horária mínima. Neste núcleo se concentram os conteúdos de caráter profissionalizante dos cursos de computação. Os conteúdos oferecidos neste grupo capacitarão o aluno aplicar os conceitos adquiridos ao longo de todo o curso de Ciência da Computação. 

\subsubsection{Núcleo de Formação Específica}
Segundo o parágrafo 4\textordmasculine, Artigo 6\textordmasculine, da Resolução CNE/CES n\textordmasculine 11/2002, os conteúdos abordados nos módulos se caracterizam pela especificidade em relação às extensões e aprofundamentos (...), bem como de outros conteúdos destinados a caracterizar modalidades. Desta forma, os conteúdos concentrados nesse núcleo definem o curso de Ciência da Computação. Sendo o curso de Ciência da Computação multidisciplinar,	incorporando	simultaneamente	formações	específicas	muito particulares com um perfil generalista, a construção de uma estrutura curricular que seja ao mesmo tempo flexível, abrangente e pontualmente profunda é um grande desafio. 

\subsubsection{Núcleo de Práticas Complementares}
No projeto do curso existem três disciplinas, que darão ao aluno a oportunidade de realizar atividades acadêmicas, industriais e/ou sociais, que enriquecerão, profundamente, a sua formação profissional. Estas disciplinas são:

\begin{itemize}
\item Estágio Curricular
\item Trabalho de Final de Curso
\end{itemize}
