\disciplina{lpt}{
    \titulo      {2}{Leitura e Produção de Texto}
    \objetivo    {\diegocomentario{Nicanor vai me mandar} Desenvolver nos estudantes a competência geral como leitor e produtor crítico de textos.  
    Proporcionar ao aluno a prática de leitura e de produção de textos (orais e escritos) envolvam diferentes formas de intertextualidade como paráfrase, paródia, citação do discurso de outro, comentário e resumo.
    Capacitar os alunos a elaborar diferentes tipos de documentos a partir de técnicas de redação e aplicá-las nas diferentes situações do cotidiano acadêmico e profissional. 
    }
    \requisitos  {N/A} % xxxxx
    \recomendadas{N/A}
    \ementa      {
    Considerações sobre a noção de texo: estrutura e inserção cultural. Explicitação do planejamento, da organização e da produção de textos. Prática de leitura, produção, análise e refacção de textos. Ideai geral sobre normas e padrões para elaboração de trabalhos acadêmicos (estrutura, formatação, citações, referências bibliográficas). Aprofundamento em técnicas e conceitos relacionados ao discurso científico oral e escrito.
    }
    \creditos    {4 total (2 teóricos e 2 práticos)}
%    \extra       {0 horas}
    \codigo      {DC}{XX.XXX-X}
    \bibliografia {
        FIORIN, José Luiz; SAVIOLI, Francisco Platão. Lições de texto: leitura e redação. 5. ed. São Paulo: Ática, 2006.
        
        KOCH, Ingedore Grunfeld Villaça. O texto e a construção dos sentidos. 10. ed. São Paulo: Contexto, 2014.
        
        FARACO, Carlos Alberto; TEZZA, Cristovão. Prática de texto para estudantes universitários. 24. ed. Petrópolis: Vozes, 2014. 
        
        KOCH, Ingedore Grunfeld Villaça; TRAVAGLIA, Luiz Carlos. Texto e coerência. 13. ed. São Paulo: Cortez, 2012.
        
        KOCH, Ingedore Grunfeld Villaça. A coesão textual. 22. ed. São Paulo: Contexto, 2016.

        
    }{
    
        MACHADO, A. R.; LOUSADA, E.; ABREU-TARDELLI, L. S. Resumo. Leitura e produção de textos técnicos e acadêmicos. São Paulo: Parábola Editorial, 2004.
        
        MACHADO, A. R.; LOUSADA, E.; ABREU-TARDELLI, L. S. Resenha. Leitura e produção de textos técnicos e acadêmicos. São Paulo: Parábola Editorial, 2004.
        
        KOCH, I. V. O texto e a construção dos sentidos. São Paulo: Contexto, 2005.
        
    }
}