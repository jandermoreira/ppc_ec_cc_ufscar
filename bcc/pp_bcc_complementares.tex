

No contexto deste novo curso de Ciência da Computação, entende-se que as atividades complementares são fundamentais para a formação de um profissional de excelência, repleto de habilidades e competências para sua atuação profissional. Neste sentido, as atividades complementares são todas aquelas que expõem o estudante a experiências inovadoras e enriquecedoras durante seu período acadêmico, complementando a formação recebida em sala de aula. Neste curso, o estudante deverá cumprir 330 horas (22 créditos) em atividades complementares para que possa se formar, como mostrado no Quadro~\ref{table: quadro_integraliza}.

De acordo com o regimento geral dos cursos de graduação da UFSCar \cite{Regimento-Geral-CursosGraduacao-UFSCar}, as Atividades Complementares são todas e quaisquer atividades de caráter acadêmico, científico e cultural realizadas pelo estudante ao longo de seu curso de graduação. A seguir são apresentadas as atividades comlementares previstas para este curso, incluindo, para cada atividade, o nome da atividade, o cômputo em horas, seu caráter (ensino, pesquisa ou extensão) e como deve ser a comprovação. \textcolor{red}{Uma versão atualizada das atividades complementares, aprovada na 65ª Reunião do Conselho do Curso de Ciência da Computação realizada no dia 24/09/2020 e Atualizada na 70ª Reunião realizada no 26/05/2021 está disponível no Anexo B, ao final deste documento.}

% Doc com a tabela: https://docs.google.com/document/d/1j3NyX3fRM9ibYLWjkAAlfUwgDcOP6hDG/edit

%\begin{footnotesize}
%\begin{longtable}{|m{3cm}|m{3cm}|m{4cm}|m{4cm}|m{2cm}|} \hline
%{\bf Nome da Atividade} & {\bf Cômputdo em Horas de Atividades Complementares (HAC)} & {\bf Tipo de comprovação exigida} & {\bf Observação} & {\bf Limitações (considerando o curso)} \\ \hline
%\end{longtable}
%\end{footnotesize}

\begin{itemize}
\item {\bf \sout{Nome da atividade:}}  \sout{Um ano de Iniciação Científica (com ou sem bolsa)}
\begin{itemize}
\item  \sout{{\bf  Cômputo em horas:} 165 horas}
\item  \sout{{\bf Caráter:} pesquisa}
\item  \sout{{\bf Tipo de Comprovação:} Relatório de finalização da IC/Declaração do Orientador/Certificado de conclusão da IC}
\end{itemize}
\end{itemize}

\begin{itemize}
\item \sout{{\bf Nome da atividade:} Um ano de PET}
\begin{itemize}
\item \sout{{\bf Cômputo em horas:} 165 horas}
\item \sout{{\bf Caráter:} extensão, pesquisa, ensino}
\item \sout{{\bf Tipo de Comprovação:} Declaração do Tutor/Certificado de Participação no PET emitido pela Pró-Reitoria}
\end{itemize}
\end{itemize}

\begin{itemize}
\item \sout{{\bf Nome da atividade:} Projeto Integrador}
\begin{itemize}
\item \sout{{\bf Cômputo em horas:} 330 horas}
\item \sout{{\bf Caráter:} extensão}
\item \sout{{\bf Tipo de Comprovação:} Declaração de Finalização do Projeto/Declaração do Orientador}
\end{itemize}
\end{itemize}

\begin{itemize}
\item \sout{{\bf Nome da atividade:} Um ano de Participação em Empresa Jr.}
\begin{itemize}
\item \sout{{\bf Cômputo em horas:} 100 horas (limitado a dois anos)}
\item \sout{{\bf Caráter:} extensão}
\item \sout{{\bf Tipo de Comprovação:} Declaração emitida pelo professor responsável}
\end{itemize}
\end{itemize}

\begin{itemize}
\item \sout{{\bf Nome da atividade:} Monitoria com bolsa}
\begin{itemize}
\item \sout{{\bf Cômputo em horas:} 60 horas/semestre}
\item \sout{{\bf Caráter:} ensino}
\item \sout{{\bf Tipo de Comprovação:} Relatório de monitoria preenchido pelo docente da disciplina atestando a participação e dedicação do monitor}
\end{itemize}
\end{itemize}

\begin{itemize}
\item \sout{{\bf Nome da atividade:} Monitoria Voluntária}
\begin{itemize}
\item \sout{{\bf Cômputo em horas:} 60 horas/semestre (limitado a 3 monitorias - 180 horas)}
\item \sout{{\bf Caráter:} ensino}
\item \sout{{\bf Tipo de Comprovação:} Relatório de monitoria preenchido pelo docente da disciplina atestando a participação e dedicação do monitor}
\end{itemize}
\end{itemize}

\begin{itemize}
\item  \sout{{\bf Nome da atividade:} Participação em monitorias e acompanhamentos de disciplinas de responsabilidade do DC}
\begin{itemize}
\item \sout{{\bf Cômputo em horas:} 20 horas/semestre por disciplina (limitado a 5 disciplinas). Computando apenas para 75\% de frequência.}
\item \sout{{\bf Caráter:} ensino}
\item \sout{{\bf Tipo de Comprovação:} Controle de presença}
\end{itemize}
\end{itemize}

\begin{itemize}
\item \sout{{\bf Nome da atividade:} Bolsa de Extensão}
\begin{itemize}
\item \sout{{\bf Cômputo em horas:} 30 horas/semestre}
\item \sout{{\bf Caráter:} extensão}
\item \sout{{\bf Tipo de Comprovação:} Declaração emitida pelo professor coordenador do projeto de extensão ou outro tipo de comprovante institucional}
\end{itemize}
\end{itemize}

\begin{itemize}
\item \sout{{\bf Nome da atividade:} Participação na Equipe da Maratona de Programação}
\begin{itemize}
\item \sout{{\bf Cômputo em horas:} 04 horas/semana}
\item \sout{{\bf Caráter:} ensino, pesquisa}
\item \sout{{\bf Tipo de Comprovação:} Declaração do professor responsável atestando a participação}
\end{itemize}
\end{itemize}

\begin{itemize}
\item \sout{{\bf Nome da atividade:} ACIEPE}
\begin{itemize}
\item \sout{{\bf Cômputo em horas:} 60 horas/semestre}
\item \sout{{\bf Caráter:} ensino, pesquisa, extensão}
\item \sout{{\bf Tipo de Comprovação:} Conclusão da disciplina com aproveitamento}
\end{itemize}
\end{itemize}

\begin{itemize}
\item \sout{{\bf Nome da atividade:} Semana da Computação (SECOMP)}
\begin{itemize}
\item \sout{{\bf Cômputo em horas:} 60 horas para os organizadores
(limitado a duas edições)}
\item \sout{{\bf Caráter:} extensão}
\item \sout{{\bf Tipo de Comprovação:} Declaração do professor responsável pela SECOMP atestando a atuação dos alunos}
\end{itemize}
\end{itemize}

\begin{itemize}
\item \sout{{\bf Nome da atividade:} Participação em seminários de pesquisa do DC}
\begin{itemize}
\item \sout{{\bf Cômputo em horas:} 02 horas por seminário assistido}
\item \sout{{\bf Caráter:} pesquisa}
\item \sout{{\bf Tipo de Comprovação:} Controle de presença}
\end{itemize}
\end{itemize}

\begin{itemize}
\item \sout{{\bf Nome da atividade:} Representação em órgãos colegiados}
\begin{itemize}
\item \sout{{\bf Cômputo em horas:} Total de horas participadas nas reuniões}
\item \sout{{\bf Caráter:} extensão}
\item \sout{{\bf Tipo de Comprovação:} Atas das reuniões}
\end{itemize}
\end{itemize}

\begin{itemize}
\item \sout{{\bf Nome da atividade:} Participação em intercâmbios internacionais. Exemplo: BEPE FAPESP, estágios no exterior, visitas de curta duração, etc.}
\begin{itemize}
\item \sout{{\bf Cômputo em horas:} A Averiguar pelo conselho de curso, tendo como limitante 160 horas}
\item \sout{{\bf Caráter:} ensino, pesquisa, extensão}
\item \sout{{\bf Tipo de Comprovação:} Certificado}
\end{itemize}
\end{itemize}

\begin{itemize}
\item \sout{{\bf Nome da atividade:} Registro de Software no INPI (Instituto Nacional de Propriedade Intelectual)}
\begin{itemize}
\item \sout{{\bf Cômputo em horas:} 60 horas por registro}
\item \sout{{\bf Caráter:} pesquisa, extensão}
\item \sout{{\bf Tipo de Comprovação:} Certificado de Registro}
\end{itemize}
\end{itemize}


\begin{itemize}
\item \sout{{\bf Nome da atividade:} Outras atividades na área de conhecimento que sejam relevantes para a formação do aluno. Exemplos: Participação em eventos, Participação em grupos acadêmicos, apresentação de trabalhos, ministração de minicursos. Casos omissos serão apreciados pelo Conselho de Curso}
\begin{itemize}
\item \sout{{\bf Cômputo em horas:} A averiguar}
\item \sout{{\bf Caráter:} ensino, pesquisa, extensão}
\item \sout{{\bf Tipo de Comprovação:} A averiguar}
\end{itemize}
\end{itemize}


Cada atividade complementar listada \sout{acima}\textcolor{red}{no Anexo B} possui particularidades em sua execução. Por exemplo, iniciações científicas, participações nos grupos PET e projetos integradores são atividades que possuem a orientação de um professor do departamento. Importante observar que as atividade podem tanto possuir caráter único como múltiplo. Por exemplo, atuações como monitor de disciplinas (monitorias) têm apenas caráter de ensino, enquanto que a participação em grupos PET possui caráter de ensino, pesquisa e extensão. 

Uma atividade que é bastante particular deste curso de Ciência da Computação são os Projetos Integradores Extensionistas (PIEs). Esses projetos envolvem grupos de alunos e têm como objetivo o desenvolvimento de um produto de software que envolva conceitos de várias disciplinas do curso. Maiores detalhes sobre os PIEs podem ser obtidas no Anexo~\ref{anexo: PIE}.

Importante ressaltar que a lista acima não é exaustiva, isto é, quaisquer outras atividades realizadas pelos estudantes e que tenham potencial de enriquecer sua formação, pode ser apresentada e solicitada ao Conselho de Curso, que analisará e deliberará se a mesma poderá ser computada para fins de integralização. 



% \begin{itemize}
% \item 30 horas de Atividades de Acompanhamento Inicial;
% \item 60 horas de Atividades de Desenvolvimento Científico e Tecnológico;
% \item 60 horas de Atividades de Formação;
% \item 60 horas de Atividades de Prática de Conhecimento.
% \end{itemize}

% Cada classe de Atividades Complementares da qual os \auri{alunos}{estudantes} devem obrigatoriamente participar são detalhadas a seguir.

%\subsection{Acompanhamento Inicial}
%\cerricomentario{Existe isso?}

%É uma atividade \auri{semanal de}{com} duração de uma (1) hora semanal, que aproxima o docente (orientador) do estudante (orientado).
%Da parte do professor do curso, essa atividade corresponde em orientar o \auri{aluno}{estudante} e direcioná-lo no estudo. O orientador é um docente do curso que deve promover junto ao \auri{aluno}{estudante} orientado discussão sobre as disciplinas em geral, rotinas de estudo, entendimento do mercado, áreas de pesquisas, interdisciplinaridade e aplicação da ciência da computação, possíveis atividades profissionais, conduta ética, encaminhamento para possíveis orientadores de iniciação científica e áreas de interesse da ciência da computação. A atividade pode ser conduzida com reunião em grupo ou individual, definida de acordo com a necessidade.

%É semestral, e vale 30 horas de atividades complementares para o \auri{aluno}{estudante}, desde que devidamente documentada e avaliada.  O \auri{aluno}{estudante} deve participar dessa atividade em seu primeiro ano de curso. A Atividade também pode ser compatibilizada com Atividades de Iniciação Científica e com o Programa Jovens Talentos para a Ciência realizada nesse período.

%A avaliação da atividade deve ser feita por participação do \auri{aluno}{estudante} e comparecimento \auri{dos mesmos}{} nas reuniões periódicas, \auri{seguida com a apresentação do}{registradas em um} plano de atividades (início das atividades) e relatório de descrição de atividades (na conclusão destas). A Atividade de Acompanhamento Inicial se inicia com a entrega de um plano de atividades com o cronograma previsto das reuniões e atividades a serem realizadas entre orientador/orientado. O relatório de descrição de atividades deve ter o cronograma inicial dos assuntos a serem trabalhados durante a atividade de tutoria e a efetivação/evolução das atividades realizadas.

%A validação dos créditos de atividades complementares é feita pelo Coordenador do Curso, segundo o relatório das atividades feitas e entregue à coordenação de curso. O relatório deve ser assinado pelo orientador e pelo orientado.
% Caberá ao Conselho do Curso juntamente com a Chefia do Departamento definir a forma de atribuição de Professores Orientadores aos respectivos \auri{alunos}{estudantes} no máximo até a segunda semana do início do período letivo de cada semestre. Caberá ao Conselho do Curso aprovar o planejamento das atividades de Acompanhamento Inicial e respectivas avaliações, a serem definidas pelos Professores Orientadores.

% \subsection{Desenvolvimento Científico e Tecnológico}

% Corresponde à apresentação de comprovante de que tenha participado de atividades tais como Atividade Curricular de Integração entre Ensino Pesquisa e Extensão (ACIEPE), Iniciação Científica, Monitoria, Treinamento, Extensão ou Empresa Junior (CATI Jr). A contabilização das horas se fará da seguinte maneira:

% \begin{itemize}

% \item ACIEPEs - 60 horas
% \item Iniciação Científica (com ou sem bolsa) - anual	 - 60 horas
% \item \auri{Bolsa}{Atividade de} Monitoria \auri{}{com Bolsa} - \auri{30 horas}{15 horas semestrais}
% \item \auri{Bolsa}{Atividade de} Monitoria \auri{}{voluntária} - \auri{60 horas}{30 horas semestrais}
% \item Bolsa Treinamento  - 30 horas
% \item Bolsa de Extensão - 30 horas
% \item Participação em Empresa Júnior -  anual - 60 horas
% \item Participação em Grupos de Estudo (PyLadies, por exemplo).

% \end{itemize}

% \subsection{Atividades de Formação}

% \cerricomentario{No caso de palestras do DC, seria o caso de iniciarmos uma lista de presença?}

% Nesta categoria, cada atividade deve ser devidamente comprovada com a declaração do supervisor da atividade.

% \begin{itemize}
% \item	participação na administração da atlética (8 horas por semestre) e em atividades esportivas;
% \item	participação dos seminários da computação e semanas da computação na UFSCar (15 horas por evento);
% \item	participação dos seminários da computação e semanas de computação em outras instituições (8 horas por evento);
% \item	participação de séries de palestras do DC-UFSCar (4 horas por palestra);
% \end{itemize}

%\subsection{Prática de Conhecimento}
%\cerricomentario{Existe isso?}

%Na manhã do segundo sábado após o início de cada período letivo, a Coordenação do Curso, junto com a chefia do Departamento,  aplicará dois simulados de áreas relacionadas ao conhecimento global de formação do Cientista da Computação. Cada simulado feito pelo aluno, com 60\% ou mais de acerto, vale para ele 15 horas da atividade complementar de Prática de Conhecimento. 

