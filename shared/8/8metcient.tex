\disciplina{metcient}{
    \titulo      {8}{Metodologia Científica}
    \objetivo    {Habilitar o estudante a compreender e dominar os mecanismos do processo de investigação científica tanto para o desenvolvimento do Trabalho de Conclusão de Curso (TCC) quanto para sua atuação profissional. Familiarizar o estudante com a metodologia do trabalho científico caracterizando procedimentos básicos, pesquisa bibliográfica, projetos e relatórios; publicações e trabalhos científicos; e os princípios e práticas para a elaboração do TCC.}
    \requisitos  {N/A} % Podemos abrir mão do requisito "Texto técnico"? Esta disciplina não está na grade da EnC e se não fizermos isso, teremos duas metodologias exatamente iguais com códigos diferentes (provavelmente equivalentes) apenas por isso.
    \recomendadas{N/A}
    \ementa      {Caracterização do que é pesquisa, sua motivação e metodologia de desenvolvimento. Apresentação dos tipos de pesquisa (iniciação científica, trabalho de conclusão de curso, etc.) e seus objetivos. Introdução aos principais conceitos relacionados à pesquisa (como objetivo, tema, problema, hipótese e justificativa). Descrição detalhada das etapas da pesquisa: determinação do tema-problema de trabalho, revisão bibliográfica, construção lógica do trabalho, desenvolvimento do trabalho e redação do texto. Conceituação de aspectos da ética na pesquisa científica: definição, princípios, plágio, conduta ética na pesquisa científica. Aprofundamento da organização da escrita científica: estrutura formal do trabalho, suas partes e conteúdo esperado, tipos de publicações científicas e suas peculiaridades. Orientação sobre a elaboração de referências e citações bibliográficas e a apresentação da pesquisa.}
    \creditos    {4 total (4 teóricos)}
    %    \extra       {3 horas}
    \codigo      {DC}{1001343}
    \bibliografia {
        WAZLAWICK, Raul Sidnei. Metodologia de pesquisa para ciência da computação, Rio de Janeiro: Elsevier, 2009. 159 p.

        CERVO, Amado Luiz.; BERVIAN, Pedro Alcino; da Silva, R. Metodologia científica, São Paulo: Makron, 4ª ed., 1996. 209 p.

        LAKATOS, Eva Maria; MARCONI, Marina de Andrade. Fundamentos de metodologia científica. 7. ed. Sao Paulo: Atlas, 2010. 297 p. ISBN 978-85-224-5758-8.
    }{
        CASELI, Helena; UFSCAR. SEAD. Metodologia científica. São Carlos, SP: EdUFSCar, 2013.

        BARROS, Aidil Jesus da Silveira; LEHFELD, Neide Aparecida de Souza. Fundamentos de metodologia científica. 3. ed. São Paulo: Pearson Prentice Hall, 2007. 158 p. ISBN 978-85-7605-156-5.

        BASTOS, Cleverson Leite; KELLER, Vicente. Aprendendo a aprender: introducao a metodologia científica. 18. ed. Petropolis: Vozes, 1998. 111 p. ISBN 85-326-0586-9.

        PARRA FILHO, Domingos; SANTOS, João Almeida. Metodologia científica. 4. ed. São Paulo: Ed. Futura, 2001. 277 p. ISBN 85-86082-81-3.

        KOCHE, José Carlos. Fundamentos de metodologia científica: teoria da ciência e iniciação à pesquisa. 29. ed. Petrópolis, RJ: Vozes, 2011. 182 p. ISBN 9788532618047.

        MATTAR, João. Metodologia científica na era da informática. 3. ed. São Paulo: Saraiva, 2011. 308 p. ISBN 9788502064478.

        SANTOS, João Almeida; PARRA FILHO, Domingos. Metodologia científica. 2. ed. São Paulo: Cengage Learning, 2012. 251 p. ISBN 9788522112142.
    }
    
    \dataatualizacao{30/10/23} % Kelen, Luciano, Fedy, Alexandre, Matias     
    \competencias{
        cg-aprender/{ce-ap-1, ce-ap-2, ce-ap-3},
        cg-produzir/{ce-pro-1, ce-pro-2, ce-pro-3}
    }

}