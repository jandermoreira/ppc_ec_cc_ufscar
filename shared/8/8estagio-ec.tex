\disciplina{estagio}{
    \titulo      {10}{Estágio em Engenharia de Computação}
    \objetivo    {Aplicar os conhecimentos adquiridos no Curso e adquirir novos conhecimentos através de trabalhos práticos desenvolvidos nas empresas.}
    \requisitos  {Ter sido aprovado em no mínimo 200 créditos} % xxxxx
    \recomendadas{N/A}
    \ementa      {Desenvolvimento supervisionado de trabalhos envolvendo assuntos de engenharia da sua área de formação.}
    \creditos    {12 total (12 práticos)}
    %    \extra       {0 horas}
    \codigo      {DC}{1001509}
    \bibliografia {
        Masiero, Paulo Cesar. Ética em computação. Sao Paulo: EDUSP, 2000. 213 p. ISBN 85-314-0575-0

        KNUTH, Donald Ervin. The art computer programming. 3. ed. Reading: Addison - Wesley, 1997. 650 p. ISBN 0-201-89683-4.

        DIJKSTRA, Edsger Wybe; FEIJEN, W.h.j. A method of programming. Wokingham: Addison-Weley, 1988. 188 p.

        SOUZA, Marco Antonio Furlan de. Algoritmos e lógica de programação: um texto introdutório para engenharia. 2.ed. São Paulo: Cengage Learning, 2014. 234 p. ISBN 9788522111299.
    }{
        PARRA FILHO, Domingos; SANTOS, João Almeida. Apresentação de trabalhos científicos: monografia, TCC, teses, dissertações. 5. ed. São Paulo: Ed. Futura, 2000. 140 p. ISBN 85-7413-027-3.

        VOLPATO, Gilson Luiz. Bases teóricas para redação científica: ...por que seu artigo foi negado? São Paulo: Cultura Acadêmica, 2010. 125 p. ISBN 978-85-98605-15-9.

        CASTRO, Cláudio de Moura. A prática da pesquisa. 2. ed. São Paulo: Pearson, 2014. 190 p. ISBN 9788576050858.

        BASTOS, Cleverson Leite; KELLER, Vicente. Aprendendo a aprender: introducao a metodologia cientifica. 11. ed. Petropolis: Vozes, 1998. 104 p. ISBN 8532605869.

        BOOTH, Wayne C.; COLOMB, Gregory G.; WILLIAMS, Joseph M. A arte da pesquisa. 2. ed. Sao Paulo: Martins Fontes, 2005. 351 p. (Colecao Ferramentas). ISBN 85-336-2157-4.

        Material disponibilizado pela empresa, caso seja necessário para a complementação da formação do estudante.
    }
    
    \dataatualizacao{30/10/23} % Kelen, Luciano, Fedy, Alexandre, Matias     
    \competencias{
        cg-aprender/{ce-ap-1, ce-ap-3},
        cg-produzir/{ce-pro-2, ce-pro-4},
        cg-atuar/{ce-atuar-1, ce-atuar-2, ce-atuar-5}
    }

}