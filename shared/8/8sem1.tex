\disciplina{sem1}{
    \titulo      {3}{Seminários 1}
    \objetivo    {Assegurar a formação de profissionais dotados de conhecimento das questões sociais, profissionais, legais, éticas, políticas e humanísticas; da compreensão do impacto da computação e suas tecnologias na sociedade; de utilizar racionalmente os recursos disponíveis de forma transdisciplinar; de competências para entender a dinâmica social segundo as perspectivas econômicas e assumir decisões que levem em conta tais perspectivas; e de capacidade para atuar segundo as tendências profissionais atuais.}
    \requisitos  {N/A} % xxxxxxx
    \recomendadas{N/A}
    \ementa      {A disciplina será baseada em seminários dos mais diversos temas, de acordo com o seu objetivo. Os temas a seguir serão obrigatoriamente abordados em todas as ofertas da disciplina. Outros temas serão apresentados aos estudantes de acordo com as necessidades mais prementes associadas ao perfil do egresso:
        \begin{itemize}
            \itemsep0em
            \item A propriedade intelectual e suas implicações no contexto da computação;
            \item A ética, a moral e o direito: pilares de sustentação da dinâmica social;
            \item Estratégias de ensino e aprendizagem;
            \item A importância da matemática;
            \item Comportamentos individuais dirigidos à manutenção da saúde;
            \item As relações interpessoais no ambiente de trabalho e as competências profissionais especificas no trabalho em equipe;
            \item A importância da ergonomia no ambiente de trabalho;
            \item A necessidade da reforma política no Brasil.
        \end{itemize}
    }
    \creditos    {2 total (2 práticos)}
    %    \extra      {0 horas}
    \codigo      {DC}{1001344}
    \bibliografia {
        LEMOS, Ronaldo. Direito, tecnologia e cultura. Rio de Janeiro: Ed. FGV, 2005. 211 p. ISBN 85-225-0516-0.

        COLBARI, Antônia de Lourdes. Ética do trabalho: a vida familiar na construção da identidade profissional. São Paulo: Letras \& Letras, 1995. 278 p. ISBN 85-85387-53-X

        DIAZ BORDENAVE, Juan E.; PEREIRA, Adair Martins. Estratégias de ensino-aprendizagem. 10. ed. Petropolis: Vozes, 1988. 312 p.

        PRADO, Shirley Donizete (Org.) et al. Alimentação, consumo e cultura. Curitiba: CRV, 2013. 240 p. (Série Sabor Metrópole ; v. 1). ISBN 9788580427790.

        GUERIN, Bernard. Analyzing social behavior: behavior analysis and the social sciences. Reno: Context Press, c1994. 382 p. ISBN 1-87978-13-6.

        COUTO, Hudson de Araujo. Ergonomia aplicada ao trabalho: o manual tecnico da maquina humana. Belo Horizonte: Ergo, 1995.

        BOBBIO, Norberto. A teoria das formas de governo. Brasília: UnB, 1980. Não paginado (Colecao Pensamento Politico; v.17).
    }{
        SCIENTIFIC authorship: credit and intellectual property in science. New York: Routledge, 2003. 384 p. ISBN 0-415-94293-4.

        SENNETT, Richard. A corrosao do carater. 4. ed. Rio de Janeiro: Record, 2000. 204 p. ISBN 85-01-0561-5.

        MATURANA, Humberto Romesin; VARELA GARCIA, Francisco J. A árvore do conhecimento: as bases biológicas da compreensão humana. São Paulo: Palas Athena, 2001. 283 p. ISBN 85-72420-32-0.

        RODRIGUES, Rosicler Martins. Alimentacao e saude. Sao Paulo: Moderna, 1994. 48 p. (Colecao Desafios Serie Teen). ISBN 85-16-01140-2.

        PICKERING, Peg. Como administrar conflitos profissionais: técnicas para transformar conflitos em resultados. 10. ed. São Paulo: Market Books, c1999. 114 p. ISBN 85-87393-28-6.

        WISNER, Alain. A inteligencia no trabalho: textos selecionados de ergonomia. Sao Paulo: FUNDACENTRO, 2003.

        SENADO FEDERAL; SECRETARIA DE DOCUMENTAÇÃO E INFORMAÇÃO; SUBSECRETARIA DE BIBLIOTECA. Formas e sistemas de governo: bibliografia. Brasília , 1991.
    %TODO: Algum comentário que sobre os temas variáveis.
    }
    % Fredy Valente 13/03/2023
    \competencias{
        cg-aprender/{ce-ap-1, ce-ap-2},
        cg-atuar/{ce-atuar-3, ce-atuar-4},
        cg-pautar/{ce-paut-1, ce-paut-2, ce-paut-3},
    }
}