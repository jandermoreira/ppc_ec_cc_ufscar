\disciplina{PrjCLP}{
    \titulo      {8}{Projetos com CLP}
    \objetivo    {Promover o entendimento da modelagem e projeto de sistemas automatizados utilizados no meio industrial, utilizando controladores programáveis e sistemas de supervisão de processos. Caracterizar o Controlador Lógico Programável (CLP) e os Sistemas Supervisórios para uso em Automação de Processos Discretos e Contínuos. Desenvolver habilidades de modelagem, análise e projeto de sistemas Automatizados utilizados nas Indústrias. Apresentar e desenvolver projetos dos principais circuitos funcionais e aplicações}
    \requisitos  {Lógica Digital}
    \recomendadas{N/A}
    \ementa      {Sistemas de Automação Industrial; Elementos de Automação Industrial; Sistemas produtivos de eventos discretos e contínuos; Modelagem de sistemas Discretos; Metodologias e Técnicas de construção de modelos; Controlador Lógico Programável; Sistemas Supervisórios; Linguagens de programação CLP, Ambientes de programação CLP; Ambiente de programação de sistemas Supervisórios; Projeto e Simulação de Sistemas Automáticos reais encontrados na Indústria em geral.}
    \creditos    {4 total (2 teóricos, 2 práticos)}
    %    \extra      {0 horas}
    \codigo      {DC}{xxxxxxx}
    \bibliografia {
    ROQUE, Luiz Alberto Oliveira Lima, Automação de processos com linguagem Ladder e sistemas supervisórios, Rio de Janeiro: LTC, 2014. 440 p. ISBN: 9788521625223.
        
    MORAES, Cícero Couto, Engenharia de automação industrial, 2. ed. Rio de Janeiro: LTC, 2012. 347 p. ISBN: 978-85-216-1532-3.

    SILVEIRA, Paulo Rogério, Automação e controle discreto, 9. ed. São Paulo: Érica, 2012. 230 p. ISBN: 978-85-7194-591-3.
    }{
        KATO, Edilson Reis Rodrigues, Projeto com Controlador Lógico Programável (CLP) - Utilizando Modelagem em Redes de Petri. 1. ed. São Carlos SP: UFSCar, 2023. v. 1. 264p . SBN  978-65-88873-19-9.

    }

    % Edilson Kato 11/03/2024
    \dataatualizacao{11/03/24} % Kato      
    \competencias
    {
        cg-buscar/{ce-busc-3, ce-busc-4},
        cg-empreender/{ce-emp-1, ce-emp-2},
        cg-produzir/{ce-pro-1, ce-pro-2, ce-pro-5},
        cg-gerenciar/{ce-ger-1, ce-ger-2}
    }
}