\disciplina{tcc1}{
    \titulo      {9}{Trabalho de Conclusão de Curso 1}
    \objetivo    {Contribuição pessoal do estudante para a sistematização do conhecimento em Engenharia de Computação apresentando uma contribuição para o desenvolvimento tecnológico da Computação.}
    \requisitos  {Metodologia Científica} % xxxxxxx
    \recomendadas{N/A}
    \ementa      {Elaboração de um projeto para o trabalho de conclusão de curso sob a orientação de um docente.} %TODO: 
    \creditos    {2 total (2 teóricos)}
    %    \extra       {2 horas}
    \codigo      {DC}{1001506}
    \bibliografia {
        KNUTH, Donald Ervin. The art computer programming. 3. ed. Reading: Addison - Wesley, 1997. 650 p. ISBN 0-201-89683-4.

        DIJKSTRA, Edsger Wybe; FEIJEN, W.h.j. A method of programming. Wokingham: Addison-Weley, 1988. 188 p.

        SOUZA, Marco Antonio Furlan de. Algoritmos e lógica de programação: um texto introdutório para engenharia. 2.ed. São Paulo: Cengage Learning, 2014. 234 p. ISBN 9788522111299.
    }{
        PARRA FILHO, Domingos; SANTOS, João Almeida. Apresentação de trabalhos científicos: monografia, TCC, teses, dissertações. 5. ed. São Paulo: Ed. Futura, 2000. 140 p. ISBN 85-7413-027-3.

        VOLPATO, Gilson Luiz. Bases teóricas para redação científica: ...por que seu artigo foi negado? São Paulo: Cultura Acadêmica, 2010. 125 p. ISBN 978-85-98605-15-9.

        CASTRO, Cláudio de Moura. A prática da pesquisa. 2. ed. São Paulo: Pearson, 2014. 190 p. ISBN 9788576050858.

        BASTOS, Cleverson Leite; KELLER, Vicente. Aprendendo a aprender: introducao a metodologia cientifica. 11. ed. Petropolis: Vozes, 1998. 104 p. ISBN 8532605869.

        BOOTH, Wayne C.; COLOMB, Gregory G.; WILLIAMS, Joseph M. A arte da pesquisa. 2. ed. Sao Paulo: Martins Fontes, 2005. 351 p. (Colecao Ferramentas). ISBN 85-336-2157-4.

        Bibliografia complementar de acordo com o projeto estabelecido junto ao orientador.
    }
    
    % Fredy Valente 13/03/2023
    \dataatualizacao{06/11/23} % Kelen, Luciano, Fedy, Alexandre, Kato, Helio, Jander, Menotti, Orides    
    \competencias{
        %cg-aprender/{ce-ap-1, ce-ap-2, ce-ap-3, ce-ap-4},
        %cg-produzir/{ce-pro-1, ce-pro-2, ce-pro-4, ce-pro-5},
        %cg-empreender/{ce-emp-1, ce-emp-2},
        cg-aprender/{ce-ap-1, ce-ap-2},
        cg-produzir/{ce-pro-1, ce-pro-2, ce-pro-5},
        cg-atuar/{ce-atuar-1, ce-atuar-5}       
    }
}

\disciplina{tcc2}{
    \titulo      {10}{Trabalho de Conclusão de Curso 2}
    \objetivo    {Contribuição pessoal do estudante para a sistematização do conhecimento em Engenharia de Computação apresentando uma contribuição para o desenvolvimento tecnológico da Computação.}
    \requisitos  {Trabalho de Conclusão de Curso 1} % xxxxx
    \recomendadas{N/A}
    \ementa      {Desenvolvimento e apresentação do trabalho de conclusão de curso sob a orientação de um docente.} %TODO: 
    \creditos    {6 total (6 práticos)}
    %    \extra       {2 horas}
    \codigo      {DC}{1001485}
    \bibliografia {
        KNUTH, Donald Ervin. The art computer programming. 3. ed. Reading: Addison - Wesley, 1997. 650 p. ISBN 0-201-89683-4.

        DIJKSTRA, Edsger Wybe; FEIJEN, W.h.j. A method of programming. Wokingham: Addison-Weley, 1988. 188 p.

        SOUZA, Marco Antonio Furlan de. Algoritmos e lógica de programação: um texto introdutório para engenharia. 2.ed. São Paulo: Cengage Learning, 2014. 234 p. ISBN 9788522111299.

    }{
        PARRA FILHO, Domingos; SANTOS, João Almeida. Apresentação de trabalhos científicos: monografia, TCC, teses, dissertações. 5. ed. São Paulo: Ed. Futura, 2000. 140 p. ISBN 85-7413-027-3.

        VOLPATO, Gilson Luiz. Bases teóricas para redação científica: ...por que seu artigo foi negado? São Paulo: Cultura Acadêmica, 2010. 125 p. ISBN 978-85-98605-15-9.

        CASTRO, Cláudio de Moura. A prática da pesquisa. 2. ed. São Paulo: Pearson, 2014. 190 p. ISBN 9788576050858.

        BASTOS, Cleverson Leite; KELLER, Vicente. Aprendendo a aprender: introducao a metodologia cientifica. 11. ed. Petropolis: Vozes, 1998. 104 p. ISBN 8532605869.

        BOOTH, Wayne C.; COLOMB, Gregory G.; WILLIAMS, Joseph M. A arte da pesquisa. 2. ed. Sao Paulo: Martins Fontes, 2005. 351 p. (Colecao Ferramentas). ISBN 85-336-2157-4.

        Bibliografia complementar de acordo com o projeto estabelecido junto ao orientador.
    }
          % Fredy Valente 13/03/2023
    \dataatualizacao{12/12/23} % Luciano  
    \competencias{
        % Fredy Valente 13/03/2023
        %cg-aprender/{ce-ap-1, ce-ap-2, ce-ap-3, ce-ap-4},
        %cg-produzir/{ce-pro-1, ce-pro-2, ce-pro-4, ce-pro-5},
        %cg-empreender/{ce-emp-1, ce-emp-2},
        cg-aprender/{ce-ap-2},
        cg-produzir/{ce-pro-1, ce-pro-2, ce-pro-5},
        cg-atuar/{ce-atuar-1, ce-atuar-5}       
    }
}