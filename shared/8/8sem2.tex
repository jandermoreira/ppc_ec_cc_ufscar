\disciplina{sem2}{
    \titulo      {8}{Seminários 2}
    \objetivo    {Assegurar a formação de profissionais dotados de conhecimento das questões sociais, profissionais, legais, éticas, políticas e humanísticas; da compreensão do impacto da computação e suas tecnologias na sociedade; de utilizar racionalmente os recursos disponíveis de forma transdisciplinar; de competências para entender a dinâmica social segundo as perspectivas econômicas e assumir decisões que levem em conta tais perspectivas; e de capacidade para atuar segundo as tendências profissionais atuais.}
    \requisitos  {Seminários 1}
    \recomendadas{N/A}
    \ementa      {A disciplina será baseada em seminários dos mais diversos temas, de acordo com o seu objetivo. Os temas a seguir serão obrigatoriamente abordados em todas as ofertas da disciplina. Outros temas serão apresentados aos estudantes de acordo com as necessidades mais prementes associadas ao perfil do egresso:
        \begin{itemize}
            \itemsep0em
            \item A dinâmica do progresso social e econômico a partir da perspectiva tecnológica;
            \item O mercado e sua influência nas estratégias empresariais;
            \item Estratégias econômicas no contexto da inovação;
            \item Estratégias viáveis para ampliar características sustentáveis em projetos e em ciclos produtivos;
            \item Os componentes formadores do empreendedorismo;
            \item Segurança no trabalho;
            \item Cultura africana e indígena na formação da sociedade brasileira;
            \item Meio ambiente como fundamento necessário para a qualidade de vida;
            \item Estratégias para minimização das desigualdades sociais no Brasil.
        \end{itemize}
    }
    \creditos    {2 total (2 práticos)}
    %    \extra      {0 horas}
    \codigo      {DC}{1001369}
    \bibliografia {
        TIAGO SEVERINO (ORG.). Desenvolvimento social integrado: uma análise a partir da produção cultural, da tecnologia da informação e da saúde. Rio de Janeiro: Letra e Imagem, 2013. 238 p. ISBN 978-85-61012-13-7.

        CASTRO, Antonio Barros De. Estrategias empresariais na industria brasileira: discutindo mudancas. Rio de Janeiro: Forense Universitaria, c1996. 288 p. ISBN 85-218-0172-6.

        FLEURY, Afonso Carlos Correa; FLEURY, Maria Tereza Leme. Estratégias empresariais e formação de competências: um quebra-cabeça caleidoscópico da indústria brasileira. 3. ed. São Paulo: Atlas, 2004. ISBN 85-224-3807-2.

        RAMAL, Silvina Ana. Como transformar seu talento em um negócio de sucesso: gestão de negócios para pequenos empreendimentos. Rio de Janeiro: Elsevier, c2006. 196 p. ISBN 85-352-2111-5.

        SATO, Michele; SANTOS, José Eduardo dos. Agenda 21: em sinopse. São Carlos, SP: EdUFSCar, 1999. 60 p. ISBN 85-85173-39-4.

        COUTO, Hudson de Araujo. Ergonomia aplicada ao trabalho: o manual tecnico da maquina humana. Belo Horizonte: Ergo, 1995. 353 p.

        ALFABETIZAÇÃO ecológica: a educação das crianças para um mundo sustentável. São Paulo: Cultrix, 2006. ISBN 9788531609602.

        A MATRIZ africana no mundo. São Paulo: Selo Negro, 2008.  (Sankofa Matrizes Africanas da Cultura Brasileira 1). ISBN 978-85-87478-32-0.

        BRASIL. MINISTÉRIO DA EDUCAÇÃO. SECRETARIA DA EDUCAÇÃO CONTINUADA, ALFABETIZAÇÃO E DIVERSIDADE. Orientações e ações para a educação das relações étnico-raciais. Brasília: SECAD, 2006. ISBN 85-296-0042-8.
    }
    {
        VESENTINI, Jose William; VLACH, Vania Rubia Farias. Geografia critica: geografia do mundo industrializado. 5. ed. Sao Paulo: Atica, 1994. 190 p. ISBN 85-08-04665-0.

    WICK, Calhoun W.; LEÓN, Lu Stanton. O desafio do aprendizado: como fazer sua empresa estar sempre à frente do mercado. São Paulo: Nobel, 1997. 222 p. ISBN 85-312-0902-3.

    MELLO NETO, Francisco Paulo de; FROES, Cesar. Empreendedorismo social: a transição para a sociedade sustentável. Rio de Janeiro: Qualitymark, 2002. 208 p. ISBN 208857303372X.

    ANDRADE, Renato Fonseca de. Conexões empreendedoras: entenda por que você precisa usar as redes sociais para se destacar no mercado e alcançar resultados. São Paulo: Gente, 2010. 129 p. ISBN 978-85-7312-701-0.

    INSTITUTO SOCIOAMBIENTAL. Almanaque Brasil socioambiental: uma nova perspectiva para entender o pais e melhorar nossa qualidade de vida. Sao Paulo: ISA, 2005. 479 p. ISBN 85-85994-30-4.

    WISNER, Alain. A inteligencia no trabalho: textos selecionados de ergonomia. Sao Paulo: FUNDACENTRO, 2003.

    NEIMAN, Zysman; MOTTA, Cristiane Pires Da. O ambiente construido. Sao Paulo: Atual, [s.d.]. 58 p. (Educacao Ambiental). ISBN 85-7056-371-X.

    BRASIL. MINISTÉRIO DA EDUCAÇÃO. SECRETARIA DA EDUCAÇÃO CONTINUADA, ALFABETIZAÇÃO E DIVERSIDADE. Orientações e ações para a educação das relações étnico-raciais. Brasília: SECAD, 2006. ISBN 85-296-0042-8.
    }
    % Fredy Valente 13/03/2023
    \dataatualizacao{06/11/23} % Kelen, Luciano, Fedy, Alexandre, Kato, Helio, Jander, Menotti, Orides      
    \competencias
    {
        cg-aprender/{ce-ap-1, ce-ap-2},
        cg-atuar/{ce-atuar-3, ce-atuar-4},
        cg-pautar/{ce-paut-1, ce-paut-2, ce-paut-3}
    }
}