\disciplina{arq2}{
    \titulo      {4}{Arquitetura e Organização de Computadores 2}
    \objetivo    {Ao final da disciplina o estudante deve ser capaz de entender a organização das principais arquiteturas modernas, bem como as técnicas de extração de paralelismo para o desenvolvimento visando alto desempenho.} % Destacar a capacitação para o projeto de novas arquiteturas 
    \requisitos  {Arquitetura e Organização de Computadores 1} % Arquitetura e Organização de Computadores I % Co-requisito Sistemas Operacionais
    \recomendadas{N/A}
    \ementa      {Linguagem de máquina de processadores modernos; Níveis de paralelismo: ILP, execução fora de ordem, SIMD, thread. Programação de baixo nível (System Programming) e Suporte ao Sistema Operacional. Interfaces de E/S, interrupções e timers.} % SUGESTÃO: Linguagem de máquina de processadores modernos (?); Níveis de paralelismo: ILP, execução fora de ordem, SIMD, \emph{thread} ; Programação de baixo nível (\emph{System Programming}) e Suporte ao Sistema Operacional.
    \creditos    {4 total (2 teóricos, 2 práticos)} % os dois créditos práticos não podem ser usados apenas para aprofundar os 2 teóricos! 
    %    \extra       {2 horas}
    \codigo      {DC}{1001541}
    \bibliografia {
        HENNESSY, John L.; PATTERSON, David A. Arquitetura de computadores: uma abordagem quantitativa. 3. ed. Rio de Janeiro: Campus, 2003. 827 p. ISBN 85-352-1110-1.

        STALLINGS, William. Arquitetura e organizacao de computadores: projeto para o desempenho. 5. ed. São Paulo: Prentice Hall, 2002. 786 p. ISBN 85-87918-53-2.

        HYDE, Randall. The art of assembly language. San Frascisco: No Starch Press, c2003. 903 p. ISBN 1-886411-97-2.

        IRVINE, Kip R. Assembly language for intel-based computers. 5. ed. Upper Saddle River: Prentice Hall, c2007. 722 p. ISBN 0-13-238310-1.
    }{
        PATTERSON, David A.; HENNESSY, John L. Organização e projeto de computadores: a interface harware/software. 3. ed. Rio de Janeiro: Elsevier, 2005. 484 p. ISBN 8535215212.

        HARRIS, David Money; HARRIS, Sarah L. Digital design and computer architecture. San Frascisco: Elsevier, 2007. 569 p. ISBN 978-0-12-370497-9.

        STALLINGS, William. Arquitetura e organização de computadores. 8. ed. São Paulo: Pearson, 2012. 624 p. ISBN 978-85-7605-564-8.
    }
    \dataatualizacao{23/10/23} % Marcio, Luciano
    \competencias{
        % cg-aprender/{ce-ap-1, ce-ap-2, ce-ap-3},
        % cg-produzir/{ce-pro-1, ce-pro-2, ce-pro-4, ce-pro-5},
        % cg-atuar/{ce-atuar-1, ce-atuar-2, ce-atuar-3, ce-atuar-4}
        cg-aprender/{ce-ap-1, ce-ap-2, ce-ap-3},
        cg-atuar/{ce-atuar-1},
    }
}