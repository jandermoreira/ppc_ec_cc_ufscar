\disciplina{arq1}{
    \titulo      {3}{Arquitetura e Organização de Computadores 1}
    \objetivo    {Ao final da disciplina o estudante deve ser capaz de entender os princípios da arquitetura e organização básica de computadores e a relação entre linguagens de alto nível e linguagens de máquina, bem como de criar um computador usando técnicas de implementação de unidades funcionais e analisar seu desempenho.}
    \requisitos  {Lógica Digital} % 02.437-6 -Lógica Digital
    \recomendadas{N/A}
    \ementa      {Conceitos fundamentais de Arquitetura de Computadores. Linguagem de máquina. Aritmética computacional. Organização do computador: monociclo, multiciclo e pipeline. Desempenho de computadores. Hierarquia de memória. Entrada/Saída: barramentos e dispositivos externos. Implementação de um processador completo usando linguagem de descrição de hardware.} %Incluir interrupções?
    \creditos    {6 total (4 teóricos, 2 práticos)}
    %    \extra       {3 horas}
    \codigo      {DC}{1001540} %02.735-9 antiga
    \bibliografia {
        PATTERSON, David A.; HENNESSY, John L. Organização e projeto de computadores: a interface harware/software. 3. ed. Rio de Janeiro: Elsevier, 2005. 484 p. ISBN 8535215212.

        HARRIS, David Money; HARRIS, Sarah L. Digital design and computer architecture. San Frascisco: Elsevier, 2007. 569 p. ISBN 978-0-12-370497-9.

        STALLINGS, William. Arquitetura e organização de computadores. 8. ed. São Paulo: Pearson, 2012. 624 p. ISBN 978-85-7605-564-8.

        SAITO, José Hiroki. Introdução à arquitetura e à organização de computadores: síntese do processador MIPS. São Carlos, SP: EdUFSCar, 2010. 189 p. (Coleção UAB-UFSCar. Sistemas de Informação). ISBN 978-85-7600-207-9.
    }{
        HENNESSY, John L.; PATTERSON, David A. Arquitetura de computadores: uma abordagem quantitativa. 3. ed. Rio de Janeiro: Campus, 2003. 827 p. ISBN 85-352-1110-1.

        STALLINGS, William. Arquitetura e organizacao de computadores: projeto para o desempenho. 5. ed. São Paulo: Prentice Hall, 2002. 786 p. ISBN 85-87918-53-2.
    }
    %\dataatualizacao{16/10/23} % Jander, Alexandre, Edilson, Fredy, Márcio, Alan
    \dataatualizacao{23/10/23} % Marcio, Luciano
    \competencias{
        % cg-aprender/{ce-ap-1, ce-ap-2, ce-ap-3},
        % cg-produzir/{ce-pro-1, ce-pro-2, ce-pro-4, ce-pro-5},
        % cg-atuar/{ce-atuar-1, ce-atuar-2, ce-atuar-3, ce-atuar-4}
       %cg-buscar/{ce-busc-1, ce-busc-4},        
        % Para:
        cg-aprender/{ce-ap-1, ce-ap-2, ce-ap-3},
        cg-atuar/{ce-atuar-1, ce-atuar-2}        
    }
}