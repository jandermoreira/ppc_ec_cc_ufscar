\disciplina{psd}{
    \titulo      {5}{Processamento de Sinais Digitais}
    \objetivo    {Prover embasamento teórico do ferramental matemático básico para a análise de sinais e sistemas no tempo contínuo e discreto com exemplos de aplicação em problemas de engenharia.}
    \requisitos  {Cálculo Diferencial e Integral 1, Geometria Analítica, Álgebra Linear e Construção de Algoritmos e Programação} % xxxxx
    \recomendadas{N/A}
    \ementa      {Introdução ao processamento de sinais. Fundamentos matemáticos de sinais e sistemas. Convolução de sinais. Análise em frequência de sinais. Série de Fourier, Transformada de Fourier e transformada Z. Amostragem e reconstrução de sinais de tempo contínuo: Teorema de Nyquist e efeito de Aliasing. Filtros digitais: análise, estruturas, técnicas de projeto e aspectos práticos de implementação.}
    \creditos    {6 total (4 teóricos, 2 práticos)}
    %    \extra       {x horas}
    \codigo      {DC}{1001486}
    \bibliografia {
        A.V. Oppenheim, A.S. Willsky e S.H. Nawab, "Signals and Systems", Segunda Edição, Prentice Hall, 1997;

        A.V. Oppenheim e R.W. Schafer, "Discrete Time Signal Processing", Prentice Hall, 1989;

        B.P. Lathi, "Sinais e sistemas lineares", 2. ed. Porto Alegre, RS: Bookman, 2007. 856 p. ISBN 978-85-60031-13-9.
    }{
        S.S. Soliman e M.D. Srninath, "Continuous and Discrete Signals And Systems", Segunda Edição, Prentice Hall, 1998.

        P. Denbigh, "System Analysis \& Signal Processsing", Addison Wesley, 1998.

        J.G.Proakis e D.G.Manolakis, Digital Signal Processing: Principles, Algorithms and Applications, 4a. edição, Pearson Prentice Hall, 2007.
    }
   \competencias{
        cg-aprender/{ce-ap-2, ce-ap-4},
        cg-produzir/{ce-pro-2, ce-pro-4, ce-pro-5},
        cg-atuar/{ce-atuar-1, ce-atuar-2, ce-atuar-3}
    }
}
