\disciplina{projsisemb}{
    \titulo{7}{Projeto de Sistemas Computacionais Embarcados}
    \objetivo{Ao final da disciplina o estudante deve ser capaz de entender os conceitos, elementos, problemas e soluções típicas no desenvolvimento de sistemas computacionais embarcados. Entender o princípio de operação, configuração, vantagens e desvantagens dos periféricos mais utilizados em sistemas computacionais. Projetar, analisar e testar o hardware e o software de sistemas computacionais embarcados e de aplicar técnicas para solução de problemas inerentes a estes sistemas.}
    \requisitos  {Arquitetura e Organização de Computadores 2 e Engenharia de Sistemas} % % xxx?) 
    \recomendadas{N/A}
    \ementa      {Conceitos e aplicações de sistemas computacionais embarcados. Metodologias para o desenvolvimento de Sistemas Embarcados: engenharia dirigida por modelos, AADL, SysML. Co-projeto de hardware e software. Ciclo de desenvolvimento de software: diagramas de fluxo de dados, statecharts, redes de petri temporizadas. Sensores, conversores, atuadores e outros componentes típicos. Microkernels: multitarefa, escalonamento e sincronização. Sistemas Críticos: RTOS, tolerância a falhas, redundância, certificação. Geração automática de código. Testes e simulação: hardware e software in the loop. Exemplos práticos de projeto de sistemas embarcados. Prototipação.}
    \creditos    {4 total (4 práticos)}
    %    \extra       {x horas}
    \codigo      {DC}{1001538}
    \bibliografia {WOLF, Wayne. Computers as components: principles of embedded computing system design. San Francisco: Morgan Kaufmann, c2005. 6556 p. ISBN 0-12-369459-0.
    % 
    BALL, Stuart R. Analog interfacing to embedded microprocessor systems. 2. ed. Boston: Newnes, c2004. 322 p. (Embedded Technology Series). ISBN 978-0-7506-7723-3.
    % 
    BRÄUNL, Thomas. Embedded robotics: mobile robot design and applications with embedded systems. 2. ed. Berlin: Springer- Verlag, c2006. 458 p. ISBN 3-540-34318-0.
    % 
    QING, Li; CAROLINE, Yao. Real-time concepts for embedded systems. San Frascisco: CMP Books, c2003. 294 p. ISBN 978-1-57820-124-2. QING, Li; CAROLINE, Yao. Real-time concepts for embedded systems. San Francisco: CMP Books, c2003. 294 p. ISBN 978-1-57820-124-2.
    % 
    HOLT, Jon; PERRY, Simon. SysML for systems engineering: a model-based approach. 2. ed. Stevenage: Institution of Engineering and Technology, 2013. 930 p. (Professional Applications of Computing Series; 7). ISBN 978-1-84919-651-2.}
    {Peter Marwedel, Embedded System Design: Embedded Systems Foundations of Cyber-Physical Systems, and the Internet of Things 3rd ed. 2018 Edition;

    KORDON, F., HUGUES, J. CANALS, A. ; DOHET, A. Embedded Systems: Analysis and Modeling with SysML, UML and AADL 1st Edition, 2013;

    Frank Vahid e Tony Givargis. Embedded System Design: A Unified Hardware/Software Introduction. Wiley. 2002.}

    \dataatualizacao{12/12/23} % Luciano   
    \competencias{
        % Fredy Valente 10/03/2023
        %cg-aprender/{ce-ap-1, ce-ap-2, ce-ap-4},
        %cg-produzir/{ce-pro-2, ce-pro-4, ce-pro-5},
        %cg-atuar/{ce-atuar-1, ce-atuar-2, ce-atuar-3, ce-atuar-4}
        cg-aprender/{ce-ap-1, ce-ap-2, ce-ap-4},
        cg-produzir/{ce-pro-1, ce-pro-2, ce-pro-4, ce-pro-5},
        cg-atuar/{ce-atuar-1, ce-atuar-2, ce-atuar-3, ce-atuar-4}        
    }
}

% \disciplina{
%     \titulo      {7}{Projeto de Sistemas Embarcados I }  
%     \objetivo    {Ao final da disciplina o estudante deve ser capaz de projetar, analisar e testar o hardware e o software de Sistemas Embarcados; e de aplicar técnicas para solução de problemas inerentes a estes sistemas. Deve também propor e especificar um sistema completo.}
%     \requisitos  {XX.XXX-X} % % xxx?) Sistemas Digitais (PCB), Arquiteturas de Alto Desempenho (SoCs), Circuitos Eletronicos (Sensores e Atuadores)
%     \recomendadas{N/A}
%     \ementa      {Metodologias para Projeto e Verificação de Sistemas Sistemas Embarcados; Diferenças entre Microcontroladores e Microprocessadores; Co-projeto de hardware e software; Interface com sensores, atuadores, conversores e outros componentes típicos.} % Base: Microcontroladores e Aplicações, mas com ênfase nos requisitos de um projeto com o objetivo de compatibilizar as necessidades das aplicações aos recursos. 
%     % Garantir que se consiga ter o projeto da placa! Mauricio
%     \creditos    {4 total (2 teóricos, 2 práticos)} % 4 práticos?
% %    \extra       {x horas}
%     \codigo      {DC}{XX.XXX-X}
% }

% \disciplina{
%     \titulo      {8}{Projeto de Sistemas Embarcados II}
%     \objetivo    {Ao final da disciplina o estudante deve ter implementado e testado o hardware e o software de um Sistema Embarcado a partir de seu projeto.}
%     \requisitos  {XX.XXX-X} % % xxx?)
%     \recomendadas{N/A}
%     \ementa      {Cronograma; Acompanhamento das fases de implementação e testes; Apresentação.}
%     \creditos    {4 total (4 práticos)}
% %    \extra       {0 horas}
%     \codigo      {DC}{XX.XXX-X}
% }