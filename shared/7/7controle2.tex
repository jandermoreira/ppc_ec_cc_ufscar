\disciplina{controle2}{
    \titulo      {6}{Controle 2}
    \objetivo    {Desenvolver habilidades de modelagem, análise e projeto de sistemas de controle para ambientes de natureza dinâmica com característica linear em que as grandezas físicas devem evoluir de acordo com restrições ou requisitos desejados, baseada na teoria de controle clássico para sistemas em tempo discreto SISO (Single Input Single Output). Desenvolver habilidades de modelagem, análise e projeto de sistemas de controle para ambientes de natureza dinâmica com característica linear em que as grandezas físicas devem evoluir de acordo com restrições ou requisitos desejados; baseada na teoria de controle baseada na abordagem de espaço de estados.}
    \requisitos  {Controle 1} % % xxx?)
    \recomendadas{N/A}
    \ementa      {O problema de controle em sistemas amostrados, sistemas de controle digital: (equações de diferença / teorema de Shannon) aplicação de conversor A/D e conversor D/A junto ao processo. Mapeamento do plano s no plano z. Estabilidade de sistemas em tempo discreto: critérios de routh e Jury; aproximações de tempo discreto. Projeto de controlador discreto a partir de projeto de controlador de tempo contínuo. Erro em regime permanente. Resposta transiente no plano z: influência do período de amostragem em transitórios; controlador PID discreto, projeto no domínio da frequência; controlador dead beat. Introdução a espaço de estados: conceito sobre variável de estado, representação de sistemas dinâmicos no espaço de estados. Análise das equações de estado: controlabilidade e observabilidade. Projeto de lei de controle. Projeto de estimador. Projeto do compensador.}
    \creditos    {6 total (4 teórico, 2 práticos)}
    %    \extra       {x horas}
    \codigo      {DC}{1001534}
    \bibliografia {
        KUO, Benjamin C. Digital control systems. 2. ed. Ft. Worth: Saunders College Publishing, c1992. ISBN 0-03-012884-6.

        FRANKLIN, Gene F.; POWELL, J. David; WORKMAN, Michael L. Digital control of dynamic systems. 2. ed. Reading: Addison-Wesley, 1990. ISBN 0-201-11938-2.

        CASTRUCCI, Plínio de Lauro; BITTAR, Anselmo; SALES, Roberto Moura. Controle automático. Rio de Janeiro: LTC, 2011. ISBN 978-85-216-1786-0.

        NISE, Norman S. Control systems engineering. 2. ed. Redwood City: The Benjamin/Cummings, c1995. ISBN 0-8053-5424-7.

        FRANKLIN, Gene F.; POWELL, J. David; EMANI-NAEINI, Abbas. Feedback control of dynamic systems. 2. ed. Reading: Addison-Wesley, 1991.
    }
    {OGATA, Katsuhiko. Engenharia de controle moderno. 5. ed. São Paulo: Pearson, 2011. ISBN 978-85-7605-810-6.

    NISE, Norman S. Control systems engineering. 2. ed. Redwood City: The Benjamin/Cummings, c1995. ISBN 0-8053-5424-7.

    ISERMANN, Rolf. Digital control systems. 2. ed. Berlin: Springer-Verlag, c1991. ISBN 3-540-50997-6.

    HEMERLY, Elder Moreira. Controle por computador de sistemas dinâmicos. 2. ed. São Paulo: Blucher, 2011. ISBN 978-85-212-0266-0.

    NISE, Norman S. Control systems engineering. 2. ed. Redwood City: The Benjamin/Cummings, c1995. ISBN 0-8053-5424-7.

    CRUZ, José Jaime da. Controle robusto multivariável. São Paulo: Edusp, 1996. ISBN 9788531403413.

    GOODWIN, Graham Clifford; GRAEBE, Stefan F.; SALGADO, Mario E. Control system design. Upper Saddle River: Prentice Hall, c2001. ISBN 0-13-958653-9.}

    \dataatualizacao{30/10/23} % Kelen, Luciano, Fedy, Alexandre, Matias     
    \competencias{
        cg-aprender/{ce-ap-1, ce-ap-2},
        cg-empreender/{ce-emp-3, ce-emp-4, ce-emp-5}, 
        cg-atuar/{ce-atuar-3, ce-atuar-4, ce-atuar-5}
    }
}