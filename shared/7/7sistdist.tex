\disciplina{sistdist}{
    \titulo      {7}{Sistemas Distribuídos}
    \objetivo    {Familiarizar o estudante com aspectos inerentes à interligação lógica de sistemas computacionais fracamente acoplados. Familiarizar o estudante com as dificuldades e técnicas para prover comunicação, sincronização e coordenação entre múltiplos sistemas de computação distribuídos. Capacitar o estudante a tratar do compartilhamento ordenado e seguro de recursos computacionais distribuídos. Capacitar o estudante a tratar do desenvolvimento de técnicas e infraestruturas de software para ambientes computacionais distribuídos. Habilitar o estudante a criar aplicações que usem de maneira eficiente múltiplos recursos computacionais distribuídos.}
    \requisitos  {Sistemas Operacionais} % % xxx?)
    \recomendadas{N/A}
    \ementa      {Motivações, objetivos e caracterização de Sistemas Distribuídos. Arquiteturas de sistemas distribuídos; middleware. Processos, threads e unidades de execução de código; modelos cliente / servidor e peer-to-peer; virtualização. Comunicação em rede, protocolos e APIs. Invocação de códigos remotos. Comunicação orientada a mensagens, a fluxos e multicast. Nomeação: identificadores e localização. Sincronização. Relógios físicos e lógicos. Ordenação. Exclusão mútua. Eleição; coordenação. Consistência e replicação: modelos de consistência; gerenciamento de réplicas; protocolos de consistência. Tolerância a faltas: modelos; redundância; resiliência de processos e de comunicação. Comunicação confiável. Acordos distribuídos e consenso; recuperação. Segurança: ameaças, políticas e mecanismos; criptografia; canais seguros; controle de acesso. Gerenciamento de segurança. Estudo de casos em Sistemas Distribuídos.}
    \creditos    {4 total (4 teóricos)}
    %    \extra       {x horas}
    \codigo      {DC}{1001503}
    \bibliografia {TANENBAUM, A. S., Steen, M. V. Sistemas distribuídos: princípios e paradigmas.. 2. ed. São Paulo. Pearson Prentice Hall, 2007. %

    COULOURIS, G.; DOLLIMORE, J.; KINDBERG, T.; and BLAIR, G. Distributed systems: concepts and design. 5th. ed. Addison-Wesley, 2012. % 

    TANENBAUM, A. S. Distributed operating systems. Prentice Hall, c1995. 614 p. (disponível na BCo)}
    {Addison-Wesley, 2009. Ghosh, Sukumar. Distributed systems: an algorithmic approach. Chapman \& Hall/CRC, c2007.
    % 
    BIRMAN, K. P. Reliable distributed systems: technologies, web services, and applications. New York: Springer, 2010.
    % 
    ANTONOPOULOS, N.; Gilliam L. Cloud computing: principles, systems and applications. New York. Springer, 2010.    % 
    SINHA, P. K. Distributed operating systems: concepts and design. New York: IEEE Computer Society Press, 1997.
    % 
    TEL, G. Introduction to distributed algorithms. 2nd. ed. Cambridge University Press, 2000.
    }
    % Fredy Valente 06/02/2023
    \competencias{
        cg-aprender/{ce-ap-1, ce-ap-2, ce-ap-4},
        cg-produzir/{ce-pro-2, ce-pro-4, ce-pro-5},
        cg-atuar/{ce-atuar-1, ce-atuar-2, ce-atuar-3, ce-atuar-4}
    }
}