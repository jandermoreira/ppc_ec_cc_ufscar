\disciplina{sistdin}{
    \titulo      {4}{Sistemas Dinâmicos}
    \objetivo    {Capacitar na elaboração de modelos físico-matemáticos visando possibilitar a análise ou projeto de sistemas. Prover teoria e ferramentas sistemáticas visando concluir sobre características gerais de sistemas em estudo ou visando comportamentos específicos requeridos ao projeto. Oferecer capacitação para seleção ou concepção de simuladores adequados para verificação de comportamentos em atividades de análise ou de síntese. A elaboração de um projeto que satisfaça as exigências de comportamento dinâmico previamente especificado somente se efetiva com a aplicação de conhecimentos técnicos de modelagem de dinâmica de sistemas.}
    \requisitos  {Séries e Equações Diferenciais e Física 1} % TODO: removi Fisica 2, pois não está na nossa grade (verificar)
    \recomendadas {N/A}
    \ementa      {Representação de modelos no domínio do tempo: entrada e saída e matricial (espaço de estados). Representação de modelos no domínio da frequência: Transformada de Laplace. Análise de sistemas e conceitos: modelos, aproximação, validação, protótipos e simuladores. Classificação geral de modelos de sistemas dinâmicos. Modelagem de Sistemas Lineares (Sistemas Elétricos, Sistemas Mecânico, Sistemas Fluídicos e Sistemas Térmicos), considerando as variáveis associadas à energia e fluxo, armazenamento, dissipação e balanço energético. Métricas de desempenho no tempo e na frequência e noções de identificação de parâmetros. Redução de ordem e técnicas de linearização para Sistemas ordem superior e sistemas não lineares. Técnicas computacionais para simulação de sistemas dinâmicos contínuos e discretos no tempo. Aplicações em sistemas diversos: fluídicos, eletro-hidráulicos, eletromecânicos, e termo-hidráulicos.}


    \creditos    {4 total (4 teóricos)}
    %    \extra       {3 horas}
    \codigo      {DC}{1001347}
    \bibliografia { CASTRUCCI, P. L; BITTAR, A. SALES, R. M. Controle automático. Rio de Janeiro: LTC, 2011. ISBN 978-85-216-1786-0.

    OGATA, Katsuhiko. System dynamics. 4. ed. Upper Saddle River, N.J.: Pearson Prentice Hall, 2004. ISBN 0-13-142462-9.

    CLOSE, C. M.; FREDERICK, D. K.; NEWELL, J. C. Modeling and analysis of dynamic systems. 3. ed. New York: John Wiley Sons, 2002. ISBN 0-41-39442-4.
    }
    {
        WELLSTEAD, P.E., Introduction to physical system modeling. Academic Press, New York, 1979;

    DOEBELIN, E.O., System modeling and response: theoretical and experimental approaches, John Wiley, New York, 1980;

    FELÍCIO, L. C.; Modelagem da dinâmica de sistemas e estudo da resposta. Rima 2ed.; 2010.

    AGUIRRE, L.A., Introdução à Identificação de Sistemas, Editora UFMG, 2003. (disponível na BCo)

        KARNOPP, D. C.; MARGOLIS, D. L.; ROSENBERG, R. C.; System Dynamics: Modeling, Simulation, and Control of Mechatronic Systems; Wiley; Edição 5; 2012. ISBN-10: 047088908X

    BROWN F, T,; Engineering System Dynamics: A Unified Graph-Centered Approach; CRC Press; Edição 2; 2001.;

    PALM III, W. J.; System Dynamics; McGraw Hill Education; Edição 3; 2013. ISBN-10: 0073398063;

    BROWN, F. T.; Engineering System Dynamics: A UYnified Graph-Centered Approach, CRC Press; Edição 2;
    2006. ISBN-10: 0849396484;

    SEELER, K. A.; System Dynamics: An Introduction for Mechanical Engineers; Springer; 2014. ISBN-10: 1461491517;

    KLUEVER, C.; Dynamic Systems: Modelin, Simulation, and Control; Wiley, Edição 1, 2015. ISBN-10: 1118289455.
    }
    
    \dataatualizacao{30/10/23} % Kelen, Luciano, Fedy, Alexandre, Matias 
    \competencias{
        cg-aprender/{ce-ap-1, ce-ap-2, ce-ap-4},
        cg-atuar/{ce-atuar-1}
    }
}