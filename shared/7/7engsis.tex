\disciplina{engsis}{
    \titulo      {6}{Engenharia de Sistemas}
    \objetivo    {Capacitar o estudante para que o mesmo defina de maneira precoce no ciclo de desenvolvimento de um sistema as necessidades do usuário, bem como as funcionalidades requeridas, realizando a documentação sistemática dos requisitos, e abordando a síntese de projeto e a etapa de validação de forma a considerar o problema completo: operação; custos e cronogramas; performance; treinamento e suporte; teste; instalação e fabricação de sistemas computacionais físicos.}
    \requisitos  {Engenharia de Software 1} % %Engenharia de Software 1)
    \recomendadas{Arquitetura de Computadores 1}
    \ementa      {Engenharia de sistemas (design, síntese, análise, avaliação, manutenção). Detalhamento do design e síntese (design conceitual, preliminar e detalhado). Decomposição lógica (Functional packing). Stakeholders. Work Breakdown Structure (WBS). Matriz de responsabilidades. Requisitos técnicos. Aplicação de CADs. Padronização e normativas para o design de sistemas de engenharia. Detalhamento de análise e avaliação de sistemas computacionais físicos. Gerenciamento de Configurações. Revisão Técnica e Auditorias. Trade Studies. Modelagem e Métricas de Simulação. Gerenciamento de riscos. Otimização. Confiabilidade. Sustentabilidade. Análise de Tolerância a Falhas. Detalhamento da manutenção de sistemas computacionais físicos. Análise de Tarefa de Manutenção (Maintenance Task Analysis - MTA). Predição.}
    \creditos    {4 total (4 teóricos)}
    %    \extra       {x horas}
    \codigo      {DC}{1001545}
    \bibliografia { NASA Systems Engineering Handbook: NASA/SP-2016-6105 Rev2 - NASA - National Aeronautics and Space Administration (Author), Space Science Library;
    % 
    Systems engineering handbook : a guide for system life cycle processes and activities / prepared by International Council on Systems. Engineering (INCOSE) ; compiled and edited by, David D. Walden, ESEP, Garry J. Roedler, ESEP, Kevin J. Forsberg, ESEP,. R. Douglas Hamelin, Thomas M. Shortell, CSEP., 4. ed. 2015.
    % 
    BLANCHARD, B. S; FABRYCKY, W. J. Systems Engineering and Analysis 5th Ed. Prentice Hall International Series in Industrial \& Systems Engineering, 2011;}
    {
        Systems Engineering Fundamentals (2001). The Defense acquisition University Press Fort BElvoir, Virgin The Defense acquisition University Press Fort BElvoiria 22060-5565.

    SAGE, A. P. Introduction to Systems Engineering, John Wiley \& Sons, 2000.}
    
    % Fredy 06/03/2023
    
    \dataatualizacao{30/10/23} % Kelen, Luciano, Fedy, Alexandre, Matias 
    \competencias{
        cg-aprender/{ce-ap-2},
        cg-gerenciar/{ce-ger-1, ce-ger-2, ce-ger-3},
        cg-produzir/{ce-pro-1, ce-pro-2, ce-pro-3, ce-pro-4, ce-pro-5},
        cg-empreender/{ce-emp-1, ce-emp-2, ce-emp-4}
    }
}