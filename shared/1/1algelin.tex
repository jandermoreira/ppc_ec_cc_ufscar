\disciplina{algelin}{
    \titulo      {2}{Álgebra Linear}
    \objetivo    {Levar o aluno a entender e reconhecer as estruturas da Álgebra Linear,
        que aparecem em diversas áreas da matemática e, trabalhar com estas estruturas, tanto abstrata como concretamente (através de cálculo com representações matriciais).
    }
    \requisitos  {Geometria Analítica} % 81116 OU (215279 E 215384) OU 343510 OU 342017 OU 342190 OU 81515 OU 345083 OU 345970 OU 524182 % 
    \recomendadas{N/A}
    \ementa      {Espaços Vetoriais; Transformações Lineares; Diagonalização de
    Matrizes; Espaços com Produto Interno; Formas Bilineares e
    Quadráticas.}
    \creditos    {4 total (3 teóricos, 1 práticos)}
    %    \extra       {x horas}
    \codigo      {DM}{08.013-6} % Corrigido CoC DC -> DM % Sabrina 
    \bibliografia{
        BOLDRINI, J. L. et al. Álgebra Linear. 3. ed. São Paulo: Harbra, 1986.

        CALLIOLI, C. A.; DOMINGUES, H. H.; Costa, R. C. F. Álgebra linear e aplicações. 6. ed.São Paulo: Atual, 2013.

        COELHO, F. U.; LOURENÇO, M. L. Um curso de álgebra linear. 2. ed. São Paulo: EdUSP, 2010.
    }{
        ANTON, H.; BUSBY, R. C. Álgebra linear contemporânea. Porto Alegre, RS: Bookman, 2008.

        LIMA, E. L. Álgebra linear. 5. ed. Rio de Janeiro: IMPA. (Coleção Matemática Universitária), 2001.

        HOFFMAN, K.; KUNZE, R. Álgebra linear. 2. ed. Rio de Janeiro: Livros Técnicos e Científicos, 1979.

        LANG, S. Álgebra linear. São Paulo: Edgard Blucher, 1971.

        LIPSCHUTZ, S. Álgebra linear. São Paulo: McGraw-Hill do Brasil, 1973.

        MONTEIRO, L. H. J. Álgebra linear. São Paulo: Nobel, 1970.
    }

    % Inserido por Murillo R. P. Homem, em 01/04/2023
     \competencias{
        cg-aprender/{ce-ap-1, ce-ap-2, ce-ap-3, ce-ap-4},
        cg-atuar/{ce-atuar-1, ce-atuar-2, ce-atuar-3, ce-atuar-4},
        }
        
}
