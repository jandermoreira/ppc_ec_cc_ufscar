\disciplina{calculo3}{
    \titulo      {3}{Cálculo 3}
    \objetivo    {Generalizar os conceitos e técnicas do Cálculo Integral de funções de uma variável para funções de várias variáveis. Desenvolver a aplicação desses conceitos e técnicas em problemas correlatos.
    }
    \requisitos  {Cálculo 2} % 82260 OU 89206 OU 342076 OU 342220 OU 342475 OU 524042 OU 82635
    \recomendadas{N/A}
    \ementa      {Integração dupla; Integração tripla; Mudanças de coordenadas; Integral de linha; Diferenciais exatas e independência do caminho; Análise vetorial: Teorema de Gauss, Green e Stokes.}
    \creditos    {4 total (3 teóricos, 1 práticos)}
    %    \extra       {x horas}
    \codigo      {DM}{08.930-3}
    \bibliografia {
        GUIDORIZZI, H. L. Um curso de Cálculo. Volume 3, 5ª edição, Livros Técnicos e Científicos Editora, Rio de Janeiro, 2002.

        THOMAS, G.B. Cálculo. Volume 2, 10ª edição, Addison Wesley, São Paulo, 2003.

        SWOKOWSKI, E. W. Cálculo com Geometria Analítica. Volume 2, 2ª edição, Makron Books, São Paulo, 1995.
    }{
        ÁVILA, G. S. S., Cálculo das funções de múltiplas variáveis. Volume 3, 7ª edição, LTC Editora, Rio de Janeiro, 2006.

        LEITHOLD, L. Cálculo com Geometria Analítica. Volume 2, 2ª edição, Harbra, São Paulo, 1982.

        ANTON, H., Cálculo. Volume 2, 6ª edição, Bookman, Porto Alegre, 2000.

        PINTO, D.; FERREIRA MORGADO, M. Cálculo Diferencial e Integral de Funções de Várias Variáveis, 3a. edição, UFRJ, 2009.

        FLEMMING, D. M.; GONÇALVES, M. B. Cálculo B: funções de várias variáveis, integrais múltiplas, integrais curvilíneas e de superfície - 2. ed. - Pearson Prentice Hall, São Paulo, 2007.
    }

    % Inserido por Murillo R. P. Homem, em 01/04/2023
     \competencias{
        cg-aprender/{ce-ap-1, ce-ap-2, ce-ap-3, ce-ap-4},
        cg-atuar/{ce-atuar-1, ce-atuar-2, ce-atuar-3, ce-atuar-4},
        }
        
}
