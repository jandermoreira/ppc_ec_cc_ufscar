\disciplina{probest}{
    \titulo      {4}{Probabilidade e Estatística}
    \objetivo    {Mostrar aos alunos conceitos de estatística, apresentando uma introdução aos princípios gerais, que serão úteis na área do aluno.}
    \requisitos  {N/A} % xxxxxxx
    \recomendadas{N/A}
    \ementa      {Experimento e Amostragem. Medidas Estatísticas dos Dados. Descrição Estatística dos Dados. Probabilidade. Variável Aleatória. Distribuições de Probabilidades Especiais. Distribuições Amostrais. Estimação de Parâmetros. Testes de Significância. Inferência Tratando-se de Duas Populações. Correlação e Previsão. Teste Qui-Quadrado.}
    \creditos    {4 total (4 teóricos)}
    %    \extra       {x horas}
    \codigo      {DEs}{15.001-0}
    \bibliografia {
        MORETTIN, P.A.; BUSSAB, W.O. Estatística Básica: 5. ed. São Paulo, Editora Saraiva, 2004.

        MONTGOMERY, D.C., RUNGER, G.C. Estatística Aplicada e Probabilidade para Engenheiros; LTC Editora, 2. ed. Rio Janeiro, 2003.

        MAGALHAES, M. N.; LIMA, A. C. P. Noções de Probabilidade e Estatística. 4. ed. São Paulo, EDUSP, 2002.
    }{
        WALPOLE, W.E.;MYERS, R.H.;MYERS, S.L.; YE, K. Prob. e Est. para Engenharia e Ciências, Pearson Prentice-Hall,São Paulo, 2009.

        MOORE, D. A Estatística Básica e Sua Prática, Editora LTC, Rio de Janeiro, 2005.

        COSTA NETO, P.L.O. Estatística, S.Paulo, Ed.Blucher, São Paulo, 1977.

        HOEL, P.G.; PORT, S.C.;STONE, C.J. Introdução à Teoria da Probabilidade. Ed. Interciência, 1978.

        MENDENHALL, W. Probabilidade e estatistica, Rio de Janeiro, RJ, Ed. Campus, 1985.
    }

    % Inserido por Murillo R. P. Homem, em 01/04/2023
     \competencias{
        cg-aprender/{ce-ap-1, ce-ap-2, ce-ap-3, ce-ap-4},
        cg-atuar/{ce-atuar-1, ce-atuar-2, ce-atuar-3, ce-atuar-4},
        }
            
}