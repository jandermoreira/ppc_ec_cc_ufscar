\disciplina{calculo1}{
    \titulo      {1}{Cálculo Diferencial e Integral 1}
    \objetivo    {Propiciar o aprendizado dos conceitos de limite, derivada e integral de funções de uma variável real. Propiciar a compreensão e o domínio dos conceitos e das técnicas de cálculo diferencial e integral. Desenvolver a habilidade de implementação desses conceitos e técnicas em problemas nos quais eles se constituem os modelos mais adequados. Desenvolver a linguagem matemática como forma universal de expressão da ciência.}
    % \requisitos{}
    \recomendadas{N/A}
    \ementa      {Números reais e funções de uma variável real. Limites e continuidade. Cálculo Diferencial e aplicações. Cálculo integral e aplicações.}
    \creditos{6 total (5 teóricos, 1 prático)}
    %     % \horas    {90 total (75 teóricas, 15 práticas)}
    % %    \extra       {6 horas}
    \codigo      {DM}{08.221-0}
    \bibliografia {
        GUIDORIZZI, H. L. Um curso de cálculo v. 1 – 5a. Edição, Rio de Janeiro: Livros Técnicos e Científicos, 2001.

        STEWART, J. Cálculo v. 1 – 5a. Edição, São Paulo: Pioneira Thomson Learning, 2006.

        SWOKOWSKI, E. W. Cálculo com geometria analítica v. 1 – 2a. Edição, São Paulo: McGraw-Hill do Brasil, 1994.
    }{
        ANTON, H. Cálculo v. 1, 10. ed, Porto Alegre, RS: Bookman, 2014.

        ÁVILA, G. Calculo: funções de uma variável v. 1 - 6a. Edição, Rio de Janeiro: Livros Técnicos e Científicos, 1994.

        FLEMMING, D. M.; GONCALVES, M. B. Cálculo A: funções, limite, derivação, integração – 6a. Edição, São Paulo: Prentice Hall, 2006.

        LEITHOLD, L. O cálculo com geometria analítica v. 1 – 3. ed, São Paulo: Harbra, 1991.

        THOMAS, G. B. et al. Cálculo, v.1 - 10a. Edição, Addison Wesley, 2002.
    }

    % Inserido por Murillo R. P. Homem, em 01/04/2023
     \competencias{
        cg-aprender/{ce-ap-1, ce-ap-2, ce-ap-3, ce-ap-4},
        cg-atuar/{ce-atuar-1, ce-atuar-2, ce-atuar-3, ce-atuar-4},
        }
        
}
