\disciplina{calculodifser}{
    \titulo      {2}{Cálculo Diferencial e Séries}
    \objetivo    {O aluno deverá saber como: aplicar os critérios de convergência para séries infinitas, bem como expandir funções em série de potências. Interpretar geometricamente os conceitos de funções de duas ou mais variáveis e ter habilidade nos cálculos de derivadas e dos máximos e mínimos de funções. Aplicar os teoremas das funções implícitas e inversas.}
    % \requisitos  {Cálculo 1} % xxxxxxx
    \recomendadas{N/A}
    \ementa      {1. Séries numéricas: critérios de convergência. 2. Séries de funções. 3. Funções reais de várias variáveis. 4. Diferenciabilidade de funções de várias variáveis. 5. Fórmula de taylor. Máximos e mínimos. 6. Transformações. 7. Teorema das funções implícitas. 8. Teorema da função inversa.}
    \creditos    {4 total (3 teóricos, 1 práticos)}
    %    \extra       {x horas}
    \codigo      {DM}{08.226-0}
    \bibliografia {
        GUIDORIZZI, H. L. Um curso de cálculo. v. 2, Livros Técnicos e Científicos, Rio de Janeiro, 2004

        PINTO, D.; MORGADO, M. C. F. Cálculo diferencial e integral de funções de várias variáveis -  UFRJ/SR-1, 1997.

        THOMAS, G. B. et al.Cálculo, v.2 - 10. ed., Addison Wesley, 2003.
    }{
        ÁVILA, G. S.S. Cálculo: funções de várias variáveis: 3. ed,  Rio de Janeiro: Livros Tecnicos e Cientificos, v.3. 258 p., 1981.

        LIMA, E. L. Curso de análise v.2, Projeto Euclides. Rio de Janeiro, IMPA, 1989.

        RUDIN, W. Principles of mathematical analysis - 3. ed, - McGraw-Hill, 1976.

        STEWART, J. Cálculo v. 2,4. ed,  Pioneira/Thomson Learning, São Paulo, 2001.

        SWOKOWSKI, E. W. Cálculo com geometria analítica, v. 2, 2. ed., – Markron Books, 1991.
    }

    % Inserido por Murillo R. P. Homem, em 01/04/2023
     \competencias{
        cg-aprender/{ce-ap-1, ce-ap-2, ce-ap-3, ce-ap-4},
        cg-atuar/{ce-atuar-1, ce-atuar-2, ce-atuar-3, ce-atuar-4},
        }
        
}