\disciplina{calculo2}{
    \titulo      {2}{Cálculo 2}
    \objetivo    {Aplicar os critérios de convergência para séries infinitas, bem como expandir funções em série de potências. Interpretar geometricamente os conceitos de funções de duas ou mais variáveis e ter habilidade nos cálculos de derivadas e dos máximos e mínimos de funções. Aplicar os teoremas das funções implícitas e inversas}
    \requisitos  {Cálculo Diferencial e Integral 1} % FAVOR NÃO ALTERAR PARA Cálculo 1, pois não é!!! 82210 OU 89109 OU 344001 OU 524034 OU 344001 OU 342009 OU 342211 OU 340456 OU 341258 OU (82619 E 82627) OU (215171 E 215287) OU (215287 E 215384)
    \recomendadas{N/A}
    \ementa      {Curvas e superfícies. Funções reais de várias variáveis. Diferenciabilidade de funções de várias variáveis. Fórmula de Taylor;  Máximos e mínimos; Multiplicadores de Lagrange. Derivação implícita e aplicações.}
    \creditos    {4 total (3 teóricos, 1 práticos)}
%    \extra       {x horas}
    \codigo      {DM}{08.920-6}
     \bibliografia {
        GUIDORIZZI, H. L. Um curso de cálculo. v. 2 – Livros Técnicos e Científicos, Rio de Janeiro, 2004 
        
        PINTO, D.;  MORGADO, M. C. F. Cálculo diferencial e integral de funções de várias variáveis -  UFRJ/SR-1, 1997. 
        
        THOMAS, G. B. et al. Cálculo, v.2 - 10a. Edição, Addison Wesley, 2003. 
    }{
        ÁVILA, G. S.S. Cálculo: funções de várias variáveis - 3a. Ediçao,  Rio de Janeiro: Livros Tecnicos e Cientificos, 1981. v.3. 258 p.  
        
        LIMA, E. L. Curso de análise v.2 - Projeto Euclides. Rio de Janeiro, IMPA, 1989. 
        
        RUDIN, W. Principles of mathematical analysis - 3. ed - McGraw-Hill, c1976. 
        
        STEWART, J.  Cálculo  v. 2, 4. ed,  Pioneira/Thomson Learning, São Paulo, 2001. 
        
        SWOKOWSKI, E. W. Cálculo com geometria analítica, v. 2 - 2. ed. – Markron Books, 1991.
    }

    % Inserido por Murillo R. P. Homem, em 01/04/2023
     \competencias{
        cg-aprender/{ce-ap-1, ce-ap-2, ce-ap-3, ce-ap-4},
        cg-atuar/{ce-atuar-1, ce-atuar-2, ce-atuar-3, ce-atuar-4},
        }
        
}
