\disciplina{robosautonomos}{
    \titulo      {7-9}{Introdução à Programação de Robôs Móveis}
    \objetivo    {Introduzir conceitos básicos sobre hardware e software de robôs móveis. Familiarizar o estudante com os sensores e atuadores mais comuns utilizados na robótica móvel. Estudo de arquiteturas e softwares de controle de robôs móveis. Implementação em laboratório de algoritmos de planejamento de trajetória para a solução de problemas clássicos da robótica móvel.
    Esta disciplina enfoca aspectos computacionais de Robótica Móvel, ilustrados por projetos práticos usando software simulado. A cada tópico será proposto um projeto desafio, em que serão apontadas as dificuldades e limitações das soluções dadas pelos estudantes.}
    \requisitos  {Álgebra Linear 1, Geometria Analítica e Probabilidade e Estatística} % xxxxxxx
    \recomendadas{N/A}
    \ementa      {História e evolução da robótica móvel. Robótica Móvel: definição, aplicações e conceitos básicos. Arquiteturas para Robótica Móvel: reativas, deliberativas e híbridas. Os problemas computacionais da Robótica Móvel: navegação, localização e mapeamento. Descrição e análise de características dos sensores e atuadores mais utilizados na área de robótica móvel. Estudo das arquiteturas de robôs móveis. Estudo de algoritmos de navegação e de cooperação de robôs móveis. Estudo e utilização da ferramenta de simulação para a programação e controle de robôs móveis. Desenvolvimento de projetos em laboratório utilizando simuladores de robôs móveis na solução de problemas.}
    \creditos    {4 total (2 teóricos, 2 práticos)}
    %    \extra       {3 horas}
    \codigo      {DC}{1001501}
    \bibliografia {
        SIEGWART, R.; NOURBAKHSH, I.R. Introduction to Autonomous Mobile Robots, 335 pp. The MIT Press (ISBN-10: 0-262-19502-X), 2004.

        ARKIN, Ronald C. Behavior-based robotics. Cambridge: The MIT Press, 1998. 490 p. (Intelligent Robots and Autonomous Agents). ISBN 978-0-262-01165-5.

        ROMERO, R. A. R; PRESTES, E.; OSÓRIO, F. S.; Wolf, D. F.; Robótica Móvel, LTC, ISBN: 9788521623038, 316 pp.

        BRÄUNL, Thomas. Embedded robotics: mobile robot design and applications with embedded systems. 2. ed. Berlin: Springer- Verlag, c2006. 458 p. ISBN 3-540-34318-0.
    }{
        CHOSET, H., LYNCH, K. M., HUTCHINSON, S., KANTOR, G., BURGARD, W. KAVRAKI, L.E. e THRUN, S. Principles of Robot Motion: Theory, Algorithms, and Implementations, 625 pp., The MIT Press (SBN-10: 0-262-03327-5), 2005.

        BEKEY, G. A., Autonomous Robots – From Biological Inspiration to Implementation and Control, 593 pp. The MIT Press (ISBN-10: 0-262-02578-7), 2005.

        THRUN, S., BURGARD, W., e FOX, D., Probabilistic Robotics., The MIT Press  (ISBN-10:0-262-20162-3), 2005.

    }
    %\dataatualizacao{12/12/23} % Luciano
    \competencias{
        %cg-aprender/{ce-ap-1, ce-ap-2, ce-ap-3},
        %cg-produzir/{ce-pro-1, ce-pro-2},
        %cg-empreender/{ce-emp-1, ce-emp-2, ce-emp-4},
        %cg-atuar/{ce-atuar-1, ce-atuar-3, ce-atuar-5}
        %cg-gerenciar/{ce-ger-1, ce-ger-1, ce-ger-3},
        %cg-pautar/{ce-paut-1, ce-paut-3, ce-paut-4},
        %cg-buscar/{ce-busc-1, ce-busc-3}
        %
        cg-aprender/{ce-ap-1, ce-ap-2, ce-ap-3},
        cg-produzir/{ce-pro-1, ce-pro-2, ce-pro-5},
        cg-empreender/{ce-emp-1, ce-emp-2, ce-emp-4},
        cg-atuar/{ce-atuar-1, ce-atuar-2, ce-atuar-5},
        cg-gerenciar/{ce-ger-1, ce-ger-2, ce-ger-3},
    }
}