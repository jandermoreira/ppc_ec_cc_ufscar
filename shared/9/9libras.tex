\disciplina{libras}{
    \titulo      {7-9}{Introdução à Língua Brasileira de Sinais}
    \objetivo    {Propiciar a aproximação dos falantes do português de uma língua viso-gestual usada pelas comunidades surdas (libras) e uma melhor comunicação entre surdos e ouvintes em todos os âmbitos da sociedade, e especialmente nos espaços educacionais, favorecendo ações de inclusão social oferecendo possibilidades para a quebra de barreiras linguísticas.}
    \requisitos  {N/A} % % xxx?)
    \recomendadas{N/A}
    \ementa      {- surdez e linguagem;
    - papel social da língua brasileira de sinais (libras);
    - libras no contexto da educação inclusiva bilíngue;
    - parâmetros formacionais dos sinais, uso do espaço, relações pronominais, verbos direcionais e de negação, classificadores e expressões faciais em libras;
    - ensino prático da libras.}
    \creditos    {2 total (2 teóricos)}
    %    \extra       {x horas}
    \codigo      {DPsi}{20.100-6}
    \bibliografia { %deixar linhas em branco para separar os livros
        MINISTERIO DA EDUCAÇÃO- MEC. Decreto nº 5626 de 22/12/2005. Regulamenta a Lei nº 10436, de 24 de abril de 2002, que dispõe sobre a Língua Brasileira de Sinais e o art.18 da Lei nº 10098 de 19/12/2000.

        GESSER, Audrei. LIBRAS? Que língua é essa?: crenças e preconceitos em torno da língua de sinais e da realidade surda. São Paulo: Parábola Editorial, 2009.

        LACERDA, C.B, F. de; SANTOS, L.F. dos (orgs). Tenho um aluno surdo, e agora? Introdução à Libras e Educação de surdos. São Carlos: EDUFSCar, 2013.
    }{
        BERGAMASCHI, R.I e MARTINS, R.V.(Org.) Discursos Atuais sobre a surdez. La Salle, 1999.

        BOTELHO, P. Segredos e Silêncios na Educação de Surdos. Autentica, 1998.

        BRITO, L.F. Por uma gramática de Língua de Sinais. Tempo brasileiro, 1995.

        CAPOVILLA, F.C.; RAPHAEL, W.D. Dicionário Enciclopédico Ilustrado Trilingue da Língua Brasileira de Sinais. Volume I: Sinais de A a L (Vol1, PP. 1-834). São Paulo: EDUSP, FABESP, Fundação Vitae, FENEIS, BRASIL TELECOM, 2001a.

        CAPOVILLA, F.C.; RAPHAEL, W.D. Dicionário Enciclopédico Ilustrado Trilingue da Língua Brasileira de Sinais. Volume II: Sinais de M a Z (Vol2, PP. 835-1620). São Paulo: EDUSP, FABESP, Fundação Vitae, FENEIS, BRASIL TELECOM, 2001b.

        FELIPE,T.A; MONTEIRO, M.S. LIBRAS em contexto: curso básico, livro do professor instrutor: Brasília: Programa Nacional de Apoio à Educação dos Surdos, MEC:SEESP, 2001.
        FERNANDES, E. Linguagem e Surdez. Porto Alegre: ARTMED, 2003.

        QUADROS, R.M. e KARNOPP, L.B. Língua de Sinais Brasileira: estudos lingüísticos. Porto Alegre. Artes Médicas, 2004.

        LACERDA, C.B.F. e GOES, M.C.R. (org.). Surdez: Processos Educativos e Subjetividade. Lovise, 2000.

        LODI, A.C.B. Uma leitura enunciativa da Língua Brasileira de Sinais: o gênero contos de fadas. São Paulo, v.20, n.2. p. 281-310, 2004.

        MOURA, M.C. O surdo: caminhos para uma nova identidade. Revinter e FAPESP, 2000.

        MACHADO, P. A política educacional de integração/inclusão: um olhar do egresso surdo. Editora UFSC, 2008.

        QUADROS, R.M. Educação de Surdos: a aquisição da linguagem. Porto Alegre. Artes Médicas, 1997.

        SKLIAR, C. (Org.). Atualidade da Educação Bilingue para Surdos (vol I). Mediação,1999.

        SÁ,N.R.L. Educação de Surdos: a caminho do bilingüismo, EDUF, 1999.

        THOMA, A. e LOPES, M. A invenção da surdez: cultura, alteridade, identidade e diferença no campo da educação. Santa Cruz do Sul: EDUNISC, 2004.

        VASCONCELOS, S.P; SANTOS, F da S; SOUZA, G.R. LIBRAS: Língua de Sinais. Nível 1- AJA- Brasília: Programa Nacional de Direitos Humanos. Ministério da Justiça/ Secretaria de Estado dos Direitos Humanos CORDE.
    }
    % Inserido por Edilson Kato, em 11/03/2024
     \competencias{
        cg-aprender/{ce-ap-1, ce-ap-2, ce-ap-3, ce-ap-4},
        cg-atuar/{ce-atuar-1, ce-atuar-2, ce-atuar-3, ce-atuar-4},
        }
}