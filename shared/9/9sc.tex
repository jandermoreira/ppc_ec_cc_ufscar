\disciplina{sc}{
    \titulo      {7-9}{Segurança Cibernética}
    \objetivo    {Gerar capacitação para entender, analisar e projetar técnicas de exploração de falhas de segurança de sistemas cibernéticos. Gerar competências para abordagem e proteção de sistemas computacionais, utilizando técnicas de exploração mais comuns. Capacitar para projeto e análise de sistemas computacionais seguros.}
    \requisitos  {Sistemas Operacionais e Arquitetura e Organização de Computadores 1} % xxxxx
    \recomendadas{N/A}
    \ementa      {Introdução à segurança de sistemas. Arquiteturas para segurança: segurança para aplicativos, sistemas operacionais e códigos legados, isolamento, controle de acesso. Criptografia: encriptação, identificação, autenticação, integridade, não repudiação, infraestrutura de chaves públicas (PKI). Segurança web: modelo de segurança de serviços e de navegadores web; Vulnerabilidades comuns (Top OWASP), tais como: SQL Injection, XSS e CSRF. Segurança de software: compilação e semântica de execução, ataques de controle de fluxo, defesas contra ataques de controle de fluxo, ROP, integridade de controle de fluxo (CFI). Segurança de rede: monitoramento, detecção de intrusão (IDS) e arquitetura de redes seguras. Tópicos avançados, tais como: segurança de aplicações móveis, módulos SAM, SIM, JavaCard e Contactless Smart Cards, e-Wallets, EMV e sistemas de bilhetagem eletrônica.}
    \creditos    {4 total (2 teóricos, 2 práticos)}
    %    \extra       {x horas}
    \codigo      {DC}{1001520}
    \bibliografia {
        STALLINGS, William. Criptografia e segurança de redes: princípios e práticas. 4. ed. São Paulo: Pearson, 2010. ISBN 9788576051190.

        ERICKSON, Jon Mark. Hacking: the art of exploration. San Frascisco: No Starch Press, 2003. ISBN 1-59327-007-0.

        NICHOLS, Randall K. ICSA guide to cryptography. New York: McGraw-Hill, 1999. ISBN 0-07-913759-8.

        NORTHCUTT, S.; NOVAK, J.; MCLACHLAN, D.; Network instrusion detection: an analyst's handbook. 2. ed. Indianapolis: New Riders, 2000. ISBN 0-7357-1008-2.

        STAJANO, Frank. Security for ubiquitous computing. Chichester: John Wiley \& Sons, c2002. ISBN 0-470-84493-0.
    }{
        Cyber Security Engineering – A Practical Approach for Systems and Software Assurance, Nancy R. Mead \& Carol C. Woody, Addison-Wesley, 2017.

        Applied Cryptography, Protocols, Algorithms and Source Code in C, Bruce Schneier, Published by John Wiley and Sons, 1996 – Reprinted in 2016.

        Introduction to Modern Cryptography, Johnatan Katz, CRC Press 2015.

        Criptografia Essencial, A Jornada do Criptógrafo, Sean Michael Wykes, Elsevier, 2016.

        Criptografia e Segurança de Redes, William Stallings, Editora Pearson, 2007

        Applied Cryptography for Cyber Security and Defense: Information Encryption and Cyphering, Hamid R. Nemati and Li Yang, Premier Reference Source, 2010.
    }
    % Fredy Valente 06/03/2023
    \dataatualizacao{06/11/23} % Kelen, Luciano, Fedy, Alexandre, Kato, Helio, Jander, Menotti, Orides
    \competencias
    {
        cg-aprender/{ce-ap-1, ce-ap-2, ce-ap-4},
        cg-atuar/{ce-atuar-1, ce-atuar-2, ce-atuar-3, ce-atuar-4}
    }
}