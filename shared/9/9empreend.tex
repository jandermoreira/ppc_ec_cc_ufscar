\disciplina{empreend}{
    \titulo      {7-9}{Empreendedores em Informática}
    \objetivo    {Desenvolver a capacidade empreendedora dos estudantes, estimulando e oferecendo ferramentas àqueles cuja vocação e/ou vontade profissional estiver direcionada à geração de negócios. Estimular os estudantes a desenvolver postura empreendedora; levar cada estudante a elaborar o planejamento de um negócio como trabalho acadêmico da disciplina; motivar os estudantes a desenvolver empreendimentos no decorrer de sua formação acadêmica, de modo a enriquecê-la.}
    % \requisitos  {XX.XXX-X} % xxxxx
    \requisitos{N/A}
    \recomendadas{N/A}
    \ementa      {Postura empreendedora. Teoria visionária. Inovação. Processo de desenvolvimento de negócios. Princípios do Reconhecimento de Oportunidades e de Modelagem de Negócios. Prototipação Rápida / Canvas. Validação de Soluções. Financiamento de negócios tecnológicos. Planos de negócios. Tópicos em negócios: propriedade intelectual, marketing, planejamento financeiro. Elaboração de planos de negócios pelos estudantes. Orientação à elaboração de planos de negócios.}
    \creditos    {4 total (4 teóricos)}
    %    \extra       {x horas}
    \codigo      {DC}{02.709-0}
    \bibliografia {
        FERRARI, Roberto. Empreendedorismo para computação: criando negócios em tecnologia. Rio de Janeiro: Elsevier, 2010. 164 p. (Série SBC). ISBN 978-85-352-3417-6. No SIBI UFSCar: B 658.421 F375e (BCo). Download PDF do Science Direct (gratuito de dentro da UFSCar ou com Proxy): \url{http://www.sciencedirect.com/science/book/9788535234176.}

        Afonso Cozzi, Valéria Judice, Fernando Dolabela, Louis Jacques Filion (orgs); EMPREENDEDORISMO de base tecnológica: spin-off: criação de novos negócios a partir de empresas constituídas, universidades e centros de pesquisa. Rio de Janeiro: Elsevier, 2008. 138 p. ISBN 978-85-352-2668-3. No SIBI UFSCar: B 658.11 E55b (BCo).

        SARKAR, Soumodip. O empreendedor inovador: faça diferente e conquiste seus espaço no mercado. Rio de Janeiro: Elsevier : Campus, 2008. 265 p. : il., grafs., tabs. ISBN 9788535230857.No SIBI UFSCar: 658.421 S245e (B-So).
    }{
        VALERIO NETTO, Antonio. Gestão das pequenas e médias empresas de base tecnológica. Barueri: Minha editora, 2006. 236 p. ISBN 85-98416-31-2. No SIBI UFSCar: B 658.022 V164g (BCo).

        DORNELAS, José Carlos Assis. Empreendedorismo: transformando ideias em negócios. 4. ed. Rio de Janeiro: Elsevier, 2012. 260 p. ISBN 978-85-352-4758-9. No SIBI UFSCar: B 658.421 D713e.4 (BCo).

        ELISABETH, Sandra; CALADO, Robisom D. Transformando ideias em negócios lucrativos: aplicando a metodologia Lean Startup. Rockville: Global South, 2015. ISBN 9781943350070. no SIBI UFSCar: 658.11 E43t (B-So).

        RIES, Eric. A startup enxuta: como os empreendedores atuais utilizam a inovação continua para criar empresas extremamente bem-sucedidas. São Paulo: Leya, 2012. 274 p. ISBN 978-85-8178-004-7. No SIBI UFSCar: G 658.421 R559s (BCo)
    }
   \dataatualizacao{12/12/23} % Luciano  
    \competencias{
        cg-empreender/{ce-emp-1, ce-emp-2, ce-emp-3, ce-emp-4, ce-emp-5},
        cg-atuar/{ce-atuar-1, ce-atuar-4, ce-atuar-5}
    }
}