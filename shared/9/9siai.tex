\disciplina{siai}{
    \titulo      {7-9}{Sistemas de Integração e Automação Industrial}
    \objetivo    {Apresentar técnicas, métodos e elementos de automação e sistemas de integração para ambientes produtivos industriais, considerando processos contínuos de fabricação e processos de fabricação por eventos discretos.}
    \requisitos  {Controle 1}
    \recomendadas{N/A}
    \ementa      {Devem-se apresentar técnicas, métodos e elementos de automação e sistemas de integração para ambientes produtivos industriais, considerando processos contínuos de fabricação e processos de fabricação por eventos discretos. Deve-se projetar e construir sistemas integrados de supervisão e controle de modelos de plantas industriais em laboratório. Os tópicos a serem abordados são: 1. Introdução a sistemas de produção (contínuos e de eventos discretos); 2. Modelagem de sistemas e técnicas de análise; 3. Elementos de automação (sensores, atuadores, controladores lógicos programáveis, comandos numéricos computadorizados, sistemas supervisórios e redes industriais); 4. Ambiente integrado de produção; 5. Planejamento e controle da produção; 6. Técnicas inteligentes de planejamento e controle da produção; 7. Gestão do projeto de automação; 8. Projeto e construção de sistema integrado de supervisão e controle de plantas industriais.}
    \creditos    {4 total (1 teórico, 3 práticos)}
    %    \extra       {x horas}
    \codigo      {DC}{1001519}
    \bibliografia {
        GROOVER, Mikell P. Automação industrial e sistemas de manufatura. 3. ed. São Paulo: Pearson, 2011. 581 p. ISBN 978-85-7605-871-7.
        %
        REZENDE, Solange. SISTEMAS inteligentes: fundamentos e aplicações. Barueri: Manole, c2005. 525 p. ISBN 85-204-1683-7.
        %
        DAVID, René; ALLA, Hassane. Discrete, continuous, and hybrid petri nets. Berlin: Springer-Verlag, c2005. 524 p. ISBN 3-540-22480-7.
    }{
        AGUIRRE, Luis Antonio. Encicolpédia de Automática: Controle e Automação. Vol 1. São Paulo: Blucher, c2007. 450 p. ISBN 85-212-0408-4.
        %
        AGUIRRE, Luis Antonio. Encicolpédia de Automática: Controle e Automação. Vol 2. São Paulo: Blucher, c2007. 417 p. ISBN 85-212-0409-1.
        %
        AGUIRRE, Luis Antonio. Encicolpédia de Automática: Controle e Automação. Vol 3. São Paulo: Blucher, c2007. 469 p. ISBN 85-212-0410-7.
    }
 
     \competencias{
%        cg-aprender/{ce-ap-3, ce-ap-4},
        cg-produzir/{ce-pro-1, ce-pro-2},
        cg-empreender/{ce-emp-1, ce-emp-2, ce-emp-4}, 
        cg-atuar/{ce-atuar-1, ce-atuar-3, ce-atuar-5},
        cg-gerenciar/{ce-ger-1, ce-ger-2, ce-ger-3},
%        cg-pautar/{ce-paut-1, ce-paut-3, ce-paut-4}, 
%        cg-buscar/{ce-busc-1, ce-busc-3}
    }
}