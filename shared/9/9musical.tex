\disciplina{musical}{
    \titulo      {7-9}{Introdução à Computação Musical}
    \objetivo    {Familiarizar o estudante com a temática da computação musical, abordando as relações formais entre teoria musical, matemática e computação. Habilitar o estudante a compreender, projetar e implementar algoritmos para síntese, análise e processamento de estruturas musicais.}
    \requisitos  {N/A} 
    \recomendadas{N/A}
    \ementa      {Apresentação das relações entre matemática, computação e música. Introdução à matemática do tom puro; parâmetros físicos do som: frequência, amplitude e fase; parâmetros perceptuais do som: intensidade (\textit{loudness}), altura (\textit{pitch}) e timbre (envoltória da onda); o tom complexo: harmônicos e formantes; representação digital da informação sonora: amostragem, pseudonímia (\textit{aliasing}), formatos de arquivos de áudio e algoritmos de compressão. O padrão MIDI. Síntese de sons: ondas fixas, granular, aditiva, subtrativa e técnicas não lineares; análise de sons: decomposição em frequências (análise de \textit{Fourier}), ruído e filtros lineares digitais; linguagens e ambientes de programação para computação musical; composição algorítmica.}
    \creditos    {4 total (2 teóricos, 2 práticos)}
    \codigo      {DC}{1001499}
    \bibliografia {
        Roederer, J. G. "Introdução à física e a psicofísica da música". São Paulo: EdUSP, 1998.

        Loy, G. "Musimathics: the mathematical foundations of music". Vol. 1, 2. Cambridge, MA: The MIT Press, c2006.

        Beauchamp, J. W. "Analysis, synthesis, and perception of musical sounds: the sound of music". New York: Springer Science, c2007. Modern Acoustic and Signal Processing.
    }{
        Moore, F. R. "Elements of Computer Music". Upper Saddle River, NJ: Prentice Hall, 1990.

        Road, C. "The Computer Music Tutorial". Cambridge, MA: The MIT Press, 1996.

        Rowe, R. "Machine Musicianship". Cambridge, MA: The MIT Press, 2001.
    }
    \competencias{
        % Inserido por Murillo Rodrigo Petrucelli Homem 
        cg-aprender/{ce-ap-1,ce-ap-2},
        cg-atuar/{ce-atuar-1,ce-atuar-2,ce-atuar-3},
        cg-pautar/{ce-paut-4}
    }
}
