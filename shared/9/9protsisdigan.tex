\disciplina{protsisdigan}{
    \titulo      {7-9}{Prototipação de Sistemas Digitais/Analógicos}
    \objetivo    {Capacitação para interpretar, idealizar e projetar sistemas digitais/analógicos de alta complexidade por meio de técnicas de estado da arte da área de hardware. Desenvolver competências em análise de requisitos elétricos/mecânicos e especificação de componentes. Desenvolver habilidades de implementação de protótipos por meio de técnicas de confecção de circuito impresso (do roteamento à confecção), montagem, testes e validação.}
    \requisitos  {Circuitos Eletrônicos 2} % % xxx?)
    \recomendadas{N/A}
    \ementa      {Conceitos introdutórios sobre Prototipação de Circuitos Digitais/Analógicos, Tecnologias e Ferramentas Computacionais relacionadas. Definição de um Projeto: especificação de características elétricas/mecânicas, potência, energia, eficiência, frequência de operação, proteção elétrica, tolerância a falhas, relação sinal/ruído, ciclo de vida do sistema, comunicação e interfaceamento, ambiente de operação, interface homem-máquina, entre outros. Especificação de Componentes: fonte de alimentação, processador, memória, interfaces/drivers de potência, dispositivos de entrada/saída, conversores ad/da, entre outros. Desenvolvimento, simulação e implementação do Protótipo: softwares de simulação, roteamento e prototipação de placas de circuito impresso. Diagramas elétricos, roteamento, layout (automático/manual), dimensionamento das trilhas, ilhas, número de camadas, análise térmica, análise de Compatibilidade Eletromagnética, choque e vibração, plano de terra, linhas de transmissão, máscaras, testes de placas de circuito impresso, soldagem.}
    \creditos    {4 total (4 práticos)}
    %    \extra       {x horas}
    \codigo      {DC}{1001518}
    \bibliografia {
        Khandpur, R. S. Printed Circuit Boards, Design, Fabrication, Assembly and Testing, McGraw-Hill, 2006;

        Wei, Xing-Chang. Modeling and Design of Electromagnetic Compatibility for High-Speed Printed Circuit Boards and Packaging. CRC Press, 2017;

        Sedra, A. S. Microeletrônica, 5ª Edição, Pearson, 2007. (Disponível na BCO);

        Tocci, R. J. Sistemas Digitais, Princípios e Aplicações. Pearson, Prentice Hall, 2011. (Disponível na BCO);

        Pedroni, V. A. Eletrônica Digital Moderna e VHDL. Campus, 2010. (Disponível na BCO);

        Hamblen, J. O.; Hall, T. S.; Furman, M. D. Rapid Prototyping of Digital Systems SOPC Edition, Springer, 2008. (Disponível na BCO);
    }{
        Navabi, Z. Digital Design and Implementation with Field Programmable Devices. Ed. Kap, 2005. (Disponível na BCO);

        Capuano, F. G.; Idoeta, I. V. Elementos de Eletrônica Digital. Editora Érica, 2012. (Disponível na BCO);
    }
    \competencias{
        cg-aprender/{ce-ap-1, ce-ap-2, ce-ap-3},
        cg-produzir/{ce-pro-1, ce-pro-2, ce-pro-4},
        cg-atuar/{ce-atuar-1, ce-atuar-3, ce-atuar-4},
%       cg-gerenciar/{ce-ger-1, ce-ger-1, ce-ger-3},
%        cg-pautar/{ce-paut-1, ce-paut-3, ce-paut-4}, 
%        cg-buscar/{ce-busc-1, ce-busc-3}
    }
}