\disciplina{ipa}{
    \titulo      {1}{Introdução ao Pensamento Algorítmico}
    \objetivo    {Motivar e orientar os estudantes a desenvolver soluções sistemáticas para problemas diversos, contextualizados em situações cotidianas, de modo que estas possam ser implementadas a fim de usar o computador como ferramenta para obtenção de resultados. Desenvolver nos estudantes a habilidade de organizar e analisar dados de um problema, a fim de encontrar soluções utilizando técnicas de abstração, decomposição, reconhecimento de padrões e generalização, além da capacidade de analisar a eficiência de suas soluções.}
    \requisitos  {N/A} % xxxxxxx
    \recomendadas{N/A}
    \ementa      {Introdução ao pensamento algorítmico. Análise e especificação de problemas sob o aspecto de pensamento algorítmico. Técnicas de resolução de problemas: abstração, decomposição, reconhecimento de padrões e generalização. Representação e visualização de dados e soluções, com interpretação de resultados. Noções de paralelização. Noções de eficiência de um algoritmo. Introdução em alto nível de algoritmos de diversas áreas da ciência da computação: ordenação, busca, conectividade em grafos, caminhos mínimos, hashing, k-nn, criptografia.}
    \creditos       {2 total (2 teóricos)}
    %    \extra       {3 horas}
    \codigo      {DC}{1001349}
    \bibliografia {
        BHARGAVA, A. Y. Entendendo Algoritmos: Um guia ilustrado para programadores e outros curiosos. NOVATEC, 2017. 264 p. ISBN 978-85-752-2563-9.

        LOPES, A.; GARCIA, G.. Introdução à programação: 500 algoritmos resolvidos. Rio de Janeiro: Elsevier, 2002. 469 p. ISBN 978-85-352-1019-4.

        SPRAUL, V. A. Think like a programmer: an introduction to creative problem solving. No Starch Press, 2012. 256 p. ISBN 978-1593274245

        SOUZA, M. A. F. Algoritmos e lógica de programação: um texto introdutório para engenharia. 2.ed. São Paulo: Cengage Learning, 2014. 234 p. ISBN 9788522111299.
    }{
        FORBELLONE, A. L. V.; EBERSPACHER, H. F. Lógica de programação: a construção de algoritmos e estruturas de dados. 3. ed. São Paulo: Pearson Prentice Hall, 2008. 218 p. ISBN 978-85-7605-024-7.

        HOLLOWAY, J. P. Introdução a programação para engenharia: resolvendo problemas com algoritmos. Rio de Janeiro: LTC, 2006. 339 p. ISBN 8521614535.

        SKIENA, S. S. The algorithm design manual. New York: Springer-Verlag, c1998. 486 p. ISBN 0-387-94860-0.

        ERWIG, Martin. Once Upon an Algorithm: How Stories Explain Computing. MIT Press, 2017. 336 p. ISBN 978-0262036634

        FILHO, W. F.; PICTET, R. Computer Science Distilled: Learn the Art of Solving Computational Problems. Code Energy, 2017. 183 p. ISBN 0997316004

        TILLMAN, F. A.; CASSONE, D. T. A Professional's Guide to Problem Solving with Decision Science. Pioneering Partnerships LLC, 2018. 298 p. ISBN 978-0999767115
    }
    % Jander, em 15/2/23
    \competencias{
        cg-aprender/{ce-ap-3, ce-ap-4},
        cg-produzir/{ce-pro-2, ce-pro-4},
        cg-atuar/{ce-atuar-1, ce-atuar-2},
        cg-empreender/ce-emp-2,
    }
}