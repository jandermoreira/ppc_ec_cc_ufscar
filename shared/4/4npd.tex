\disciplina{npd}{
    \titulo      {}{Algoritmos para Problemas NP-Difíceis}
    \objetivo    {Tornar os estudantes aptos a tratar problemas NP-Difíceis com diversas estratégias algorítmicas; em particular, programação linear inteira, algoritmos de aproximação e meta-heurísticas.}
    \requisitos  {Projeto e Análise de Algoritmos} % xxxxxxx
    \recomendadas{N/A}
    \ementa      {Problemas NP-difíceis são centrais em ciência da computação, uma vez que estes abrangem uma grande quantidade de problemas importantes tanto do ponto de vista prático quanto teórico. Além disso, a menos que P seja igual a NP, não se espera que existam algoritmos eficientes para resolver tais problemas. Neste curso vamos apresentar três estratégias algorítmicas centrais para tratar Problemas NP-difíceis. A primeira são os algoritmos exatos, que serão apresentados através da modelagem dos problemas em programas lineares inteiros (PLI) com o subsequente uso de resolvedores de PLI. Destacamos que estes resolvedores levam tempo exponencial no pior caso. A segunda estratégia são os algoritmos de aproximação, que correspondem a um compromisso entre eficiência de tempo e de qualidade, uma vez que executam em tempo polinomial e suas soluções estão a no máximo um fator de distância da solução ótima. A terceira abordagem são métodos meta-heurísticos como GRASP e BRKGA, que apresentam maneiras abrangentes de projetar algoritmos para variados problemas encontrando boas soluções na prática, ainda que sem garantia teórica de qualidade.}
    \creditos    {4 total (4 teóricos)}
    %    \extra       {4 horas}
    \codigo      {DC}{XX.XXX-X}
    \bibliografia {
        ARENALES, M. Nereu; ARMENTANO, Vinícius Amaral; MORABITO, Reinaldo; YANASSE, Horácio. Pesquisa operacional. Rio de Janeiro: Elsevier, 2007.

        WOLSEY, L. A. Integer programming. New York: John Wiley & Sons, 1998.

        FERREIRA, C. Eduardo. Uma introdução sucinta a algoritmos de aproximação. Rio de Janeiro: IMPA, c2001.

        George Nemhauser, Laurence Wolsey. Integer and Combinatorial Optimization. 2014.
    }{
        David P. Williamson, David B. Shmoys. The Design of Approximation Algorithms. Cambridge University Press, 2011.

        Michel Gendreau, Jean-Yves Potvin. Handbook of Metaheuristics. 2010.

        Vijay V. Vazirani. Approximation Algorithms. 2003.

        Christos H. Papadimitriou, Kenneth Steiglitz. Combinatorial Optimization: Algorithms and Complexity. Courier Corporation, 1998.

        T.H. Cormen, C.E. Leiserson, R.L. Rivest, C. Stein. Introduction to Algorithms, 3rd ed., McGraw-Hill, 2009.
    }
}