\disciplina{aed2}{
    \titulo      {3}{Algoritmos e Estruturas de Dados 2}
    \objetivo    {Tornar os estudantes aptos a utilizar diversas técnicas de programação em seus projetos; capacitar os estudantes a reconhecer, implementar e modificar algoritmos e estruturas de dados amplamente utilizados; familiarizar os estudantes com o projeto e a análise de algoritmos, através do estudo de uma linguagem algorítmica, exemplos e exercícios práticos; estimular os estudantes a avaliar quais técnicas de programação, algoritmos e estruturas de dados se adequam melhor a cada situação, problema ou aplicação.}
    \requisitos  {Algoritmos e Estruturas de Dados 1} % xxxxxxx
    \recomendadas{N/A}
    \ementa      {Aprofundamento das noções de análise de correção (invariantes e indução matemática) e eficiência (complexidade de tempo e espaço) de algoritmos, incluindo a notação O. Detalhamento dos algoritmos de ordenação não-elementares (heap sort, merge sort e quick sort aleatorizado). Apresentação de algoritmo $O(n \log n)$ para cálculo de inversões entre sequências (adaptação do merge sort). Limitante inferior $\Omega (n \log n)$ para ordenação por comparação. Noções de algoritmos de ordenação não baseados em comparação e com tempo linear (bucket, counting e radix sort). Introdução de tabelas de símbolos com detalhamento de sua implementação usando estruturas de dados como: tabelas de espalhamento (hash tables), skip lists (estrutura probabilística), árvores de busca balanceadas (AVL ou rubro-negras e árvores de busca ótimas). Apresentação do algoritmo de Boyer-Moore e das árvores de prefixos para processamento de cadeias de caracteres. Introdução a grafos com diferentes tipos (simples, dirigido e ponderado) e representações (matrizes, listas de adjacência e listas ortogonais). Detalhamento de diversos algoritmos em grafos como: busca (com aplicação em conectividade), busca em largura (com aplicação em caminhos mínimos não ponderados), busca em profundidade (com aplicações em ordenação topológica e componentes fortemente conexos), caminhos mínimos em grafos sem custos negativos (algoritmo de Dijkstra com e sem heap). Apresentação de exemplos e exercícios práticos, os quais podem envolver estruturas de dados compostas (como heaps ou tabelas hash associados a vetores) e diferentes abordagens algorítmicas (gulosa, divisão e conquista, programação dinâmica, aleatorização etc).}
    \creditos    {4 total (4 teóricos)}
    %    \extra       {4 horas}
    \codigo      {DC}{1001490}
    \bibliografia {
        SEDGEWICK, R. Algorithms in C++, Part 5: graph algorithms. 3rd. ed., Boston: Addison-Wesley, 2001.

        ZIVIANI, N.Projeto de algoritmos: com implementações em Java e C++. 2. ed. São Paulo: Cengage Learning, 2011.

        FEOFILOFF. P. Algoritmos em Linguagem C, Elsevier, 2009.

        CORMEN, T.H. ; LEISERSON, C.E. ; RIVEST, R.L.; STEIN, C. Introduction to Algorithms, 3rd ed., McGraw-Hill, 2009.
    }{
        SEDGEWICK,  R. Algorithms in C++, Parts 1-4: fundamentals, data structures, sorting, searching. 3rd. ed., Boston: Addison - Wesley, 1998.

        BERMAN,  A. M. Data structures via C++: objects by evolution. New York: Oxford University Press, 1997.

        LANGSAM, Y. ;AUGENSTEIN, M. ; TENENBAUM, A. M. Data structures using C and C++. 2. ed. Upper Sadle River: Prentice Hall, 1996.

        ZIVIANI,  N. Projetos de algoritmos: com implementações em Pascal e C. 3. ed. rev. e ampl. São Paulo: Cengage Learning, 2012.

        DROZDEK,  A. Estruturas de dados e algoritmos em C++. São Paulo: Cengage Learning, 2010.
    }
    % Jander, 13/5/23
    \competencias{
        cg-aprender/{ce-ap-1, ce-ap-2},
        cg-produzir/{ce-pro-1, ce-pro-2, ce-pro-5},
        cg-empreender/{ce-emp-1, ce-emp-2}
    }
}