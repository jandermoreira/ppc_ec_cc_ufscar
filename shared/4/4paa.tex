\disciplina{paa}{
    \titulo      {4}{Projeto e Análise de Algoritmos}
    \objetivo    {Tornar os estudantes aptos a aplicar estratégias algorítmicas avançadas a seus projetos; capacitar os estudantes a analisar a correção e o desempenho de algoritmos não-triviais; permitir aos estudantes consolidar os paradigmas de projeto de algoritmos (divisão e conquista, aleatorização, guloso, programação dinâmica), através de diversos exemplos e demonstrações; familiarizar os estudantes com noções da teoria da complexidade computacional; estimular os estudantes a avaliar quais técnicas de projeto, algoritmos e estruturas de dados se adequam melhor a cada situação, problema ou aplicação.}
    \requisitos  {Algoritmos e Estruturas de Dados 1} % xxxxxxx
    \recomendadas{N/A}
    \ementa      {Detalhamento das análises assintóticas (notação O, Omega e Theta). Aprofundamento de divisão-e-conquista: árvore de recorrência e teorema mestre (demonstração, interpretação e exemplos). Apresentação de aplicações em áreas distintas com definição do problema, algoritmo, recorrência, análises de correção e eficiência. Exemplos de aplicações: multiplicação de inteiros e matrizes, ordenação e seleção aleatorizados (Revisão de probabilidade). Revisão de grafos e apresentação da operação de contração de arestas com aplicação no algoritmo probabilístico de Karger para o problema do corte mínimo. Aprofundamento de algoritmos gulosos: aplicações em áreas distintas com definição do problema, algoritmo e invariantes, análises de correção e eficiência. Exemplos de aplicações: escalonamento de tarefas com peso em uma única máquina, coleção disjunta máxima de intervalos, códigos de Huffman, problema da árvore geradora mínima (algoritmo genérico) e abordagens de Prim (com e sem heap) e Kruskal (com detalhamento da estrutura union-find). Aprofundamento de programação dinâmica: princípios de PD (com exemplos); aplicações em áreas distintas com definição do problema, subestrutura ótima com demonstração, algoritmo, implementação eficiente, análises de correção e eficiência. Exemplos de aplicações: conjunto independente ponderado em grafos caminhos, alinhamento de sequências, problema da mochila, caminhos mínimos. Revisão do algoritmo para caminhos mínimos de Dijkstra com apresentação de contra-exemplo para o caso de grafos com custos negativos. Detalhamento dos algoritmos para caminhos mínimos de Bellman-Ford, Floyd-Warshall e Johnson. Introdução de NP-Completude pelo ponto de vista algorítmico: reduções; completude; definição e interpretação de NP-Completude (questão P vs NP). Noções de abordagens para tratar problemas NP-Completos e NP-Difíceis. Algoritmos exatos (Ex: busca exaustiva melhorada para Cobertura por Vértices e programação dinâmica para Caixeiro Viajante); algoritmos de aproximação (Ex: algoritmos guloso e de programação dinâmica para mochila); algoritmos de busca local (Ex: Corte Máximo e 2-SAT).}
    \creditos    {4 total (4 teóricos)}
    %    \extra       {4 horas}
    \codigo      {DC}{1001525}
    \bibliografia {
        S. Dasgupta, C.H. Papadimitriou, U.V. Vazirani. Algoritmos, McGraw-Hill, 2009.

        T.H. Cormen, C.E. Leiserson, R.L. Rivest, C. Stein. Introduction to Algorithms, 3rd ed., McGraw-Hill, 2009.

        R. Sedgewick, K. Wayne. Algorithms, 4th. ed., Addison-Wesley, 2011.
    }{
        J. Kleinberg, É. Tardos. Algorithm Design, Addison-Wesley, 2005.

        SEDGEWICK,  R. Algorithms in C++, Parts 1-4: fundamentals, data structures, sorting, searching. 3rd. ed., Boston: Addison - Wesley, 1998.

        SEDGEWICK, R. Algorithms in C++, Part 5: graph algorithms. 3rd. ed., Boston: Addison-Wesley, 2001.

        Nivio Ziviani. Projeto de algoritmos: com implementações em Java e C++. 2. ed. São Paulo: Cengage Learning, 2011.

        D.E. Knuth. The Art of Computer Programming, vols. 1 e 3, Addison-Wesley, 1973.

        K. H. Rosen. Discrete mathematics and its applications. 7th. ed. New York: McGraw Hill, 2013.
    }
    % Jander, 13/5/23
    \dataatualizacao{4/9/23} % Jander
    \competencias{
        % cg-aprender/{ce-ap-1, ce-ap-2},
        % cg-produzir/{ce-pro-2, ce-pro-4},
        % cg-atuar/{ce-atuar-4, ce-atuar-5}
        cg-aprender/{ce-ap-4},
        cg-produzir/{ce-pro-2, ce-pro-4},
        cg-atuar/{ce-atuar-4}
    }
}