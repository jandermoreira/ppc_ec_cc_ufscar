\disciplina{tc}{
    \titulo      {X}{Teoria da Computação}
    \objetivo    {Familiarizar os estudantes com a teoria de linguagens formais, a teoria de autômatos e a equivalência entre ambas. Capacitar os estudantes a descrever linguagens simples utilizando expressões regulares e gramaticais livres de contexto. Familiarizar os estudantes com noções de representação de problemas e soluções computacionais por meio dessas teorias. Tornar os estudantes aptos a reconhecer problemas indecidíveis e intratáveis por meio dessas teorias.}
    \requisitos  {Matemática Discreta} % Matemática Discreta no BCC, e na enc?
    \recomendadas{N/A}
    \ementa      {Introdução aos conceitos de alfabetos, gramáticas e linguagens; detalhamento das linguagens, expressões e gramáticas regulares. Apresentação dos autômatos finitos (máquinas de estados) e autômatos finitos com saída (máquinas de Mealy e Moore). Detalhamento das gramáticas e linguagens livres de contexto. Apresentação dos autômatos finitos com pilha. Aprofundamento em Máquinas de Turing. Hierarquia das classes de linguagens formais: gramáticas não-restritas e sensíveis ao contexto e linguagens recursivas e recursivamente enumeráveis. Aprofundamento nos limites da computação algorítmica: computabilidade e decidibilidade.}
    \creditos    {4 total (4 teóricos)}
    %    \extra       {3 horas}
    \codigo      {DC}{1001510}
    \bibliografia {
        Hopcroft, J.E.; Motwani R.; Ullman J.D. Introdução à Teoria de Autômatos, Linguagens e Computação. Editora Campus Ltda, 2003.

        Sipser, M.; Introdução à Teoria da Computação - 2a ed. norte-americana. Cengage CTP, 2007. 488p.

        Lewis, Harry R.; Papadimitriou, Christos H.; Elementos de Teoria da Computação. 2.ed. Porto Alegre, Bookman, 2004.
    }{
        Menezes, Paulo Blauth. Linguagens formais e autômatos. 6.ed. Porto Alegre, Bookman, 2010. 256p.
    }

    \dataatualizacao{12/12/23} % Luciano
    \competencias{
        cg-empreender/{ce-emp-1, ce-emp-2},
        cg-atuar/{ce-atuar-1, ce-atuar-4},
        cg-pautar/{ce-paut-4}
    }    
}