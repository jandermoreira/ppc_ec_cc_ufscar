\disciplina{dm}{
    \titulo      {7-9}{Desenvolvimento Móvel}
    \objetivo    {Familiarizar o estudante com os conceitos da programação para dispositivos móveis; familiarizar o estudante com conceitos de programação multiplataforma; capacitar o estudante a desenvolver aplicativos para dispositivos móveis.}
    \requisitos  {Desenvolvimento de Software para Web 2} % % xxx?)
    \recomendadas{N/A}
    \ementa      {Características e evolução dos dispositivos móveis. Versionamento em aplicações móveis. Modelos arquiteturais da programação móvel. Programação da interface para dispositivos móveis. Comunicação e sincronização de dados. Persistência de dados no dispositivo. Utilização dos recursos de hardware do dispositivo móvel. Frameworks de programação para dispositivos móveis.}
    \creditos    {4 total (4 práticos)}
    %    \extra       {x horas}
    \codigo      {DC}{1001337}
    \bibliografia { %deixar linhas em branco para separar os livros
        LAWSON, B. Introdução ao HTML 5. Rio de Janeiro: Alta Books, 2011.

        LEE, V.; SCHENEIDER, H.; SCHELL, R. Aplicações móveis: arquitetura, projeto e desenvolvimento. São Paulo: Pearson Education: Makron Books, 2015. 328 p.

        SILVA, M. S. CSS3: desenvolva aplicações web profissionais com uso dos poderosos recursos de estilização das CSS3. São Paulo: Novatec, 2012.

        SILVA, M. S. HTML 5: a linguagem de marcação que revolucionou a web. São Paulo: Novatec, 2011.

        SILVA, M. S. JQuery Mobile: desenvolva aplicações web para dispositivos móveis com HTMLS, CCSS3, AJAX, jQuery e jQuery UI. São Paulo: Novatec, 2012.

        TERUEL, E. C. HTML 5. São Paulo: Erica, 2012.
    }{
        BORGES JÚNIOR, M. P. Aplicativos móveis: aplicativos para dispositivos móveis usando C\#.Net com a ferramenta visual Studio.NET e MySQL e SQL Server. Rio de Janeiro: Ciência Moderna, 2005. 130p.

        DEITEL, H. M.; DEITEL, P. J. Java: como programar. 8. ed. São Paulo: Bookman, 2010.

        FLATSCHART, F. HTML 5: embarque imediato. Rio de Janeiro: Brasport, 2011.

        LECHETA, R. R. Google Android: aprenda a criar aplicações para dispositivos móveis com o Android SDK. 3. ed. São Paulo: Novatec, 2013.
    } 
     % Fredy 06/03/2023
    \competencias{
        cg-aprender/{ce-ap-1, ce-ap-2, ce-ap-4},
        cg-produzir/{ce-pro-2, ce-pro-4, ce-pro-5},
        cg-atuar/{ce-atuar-1, ce-atuar-2, ce-atuar-3, ce-atuar-4}
        }
}