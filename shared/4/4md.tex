\disciplina{md}{
    \titulo      {}{Matemática Discreta}
    \objetivo    {Familiarizar os estudantes com a estrutura das demonstrações matemáticas, através da apresentação de diversos exemplos e exercícios; capacitar os estudantes a deduzir e utilizar fatos e noções elementares sobre números, conjuntos, relações, funções e grafos; estimular os estudantes a utilizar raciocínio indutivo em suas análises.}
    \requisitos  {N/A} % xxxxxxx
    \recomendadas{Lógica Matemática}
    \ementa      {Introdução à matemática discreta. Apresentação de estratégias de demonstração de teoremas com detalhamento de indução matemática. Introdução à teoria dos números, somatórios e produtórios, e teoria dos conjuntos, com apresentação de propriedades matemáticas e demonstrações das mesmas. Apresentação de relações, relações de equivalência e relações de ordem. Noções de funções, funções injetoras, funções sobrejetoras e funções bijetoras. Introdução a grafos com apresentação de conceitos, como: conectividade e subgrafos, orientação e caminhos, graus e cortes, laços e arestas paralelas, coloração; além da introdução de categorias de grafos, como: árvores e circuitos, grafos bipartidos, eulerianos e hamiltonianos, grafos planares e duais. Problematização com exemplos práticos da computação.}
    \creditos    {4 total (4 teóricos)}
    %    \extra       {4 horas}
    \codigo      {DC}{1001500}
    \bibliografia {
        GERSTING, J. L. Fundamentos matemáticos para a ciência da computação: um tratamento moderno de matemática discreta. 5. ed. Rio de Janeiro: LTC, c2004.

        SCHEINERMAN, E. R. Matemática discreta: uma introdução. São Paulo: Thomson Learning Edições, 2006.

        GOODAIRE, E. G. Goodaire; PARMENTER, Michael M. Discrete mathematics with graph theory. 3. ed. Upper Saddle River, NJ: Pearson Prentice Hall, c2006.
    }{
        ROSEN, K. H. Discrete mathematics and its applications. 7th. ed. New York: McGraw Hill, 2013.

        GOMIDE, A.; STOLFI, J. Elementos de Matemática Discreta para Computação. 238 p. 2011.

        VELLEMAN D. How to Prove It, A Structured Approach, 2a. Edição, Cambridge, 2006.

        LEHMAN, E.; LEIGHTON, F. T.; MEYER, A. R. Mathematics for Computer Science. 2017.

        STEIN, C.; DRYSDALE, R. L.; BOGART, K. Matemática discreta para ciência da computação. São Paulo: Pearson Education do Brasil, 2013.

        FEOFILOFF, P.; KOHAYAKAWA, Y.; WAKABAYASHI, Y. Uma Introdução Sucinta à Teoria dos Grafos. 2011.
    }
}