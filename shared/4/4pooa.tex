\disciplina{pooa}{
    \titulo      {7-9}{Programação Orientada a Objetos Avançada}
    \objetivo    {Estimular o estudante a programar utilizando estruturas que facilitem a implementação, manutenção e evolução de software. Familiarizar o estudante com os princípios SOLID (responsabilidade única, aberto-fechado, substituição de Liskov, segregação de interface e inversão de dependência) da orientação a objetos. Capacitar o estudante a criar software orientado a objetos que utiliza os conceitos básicos da programação orientada a objetos (abstração, classes, objetos, atributos e métodos, encapsulamento/visibilidade, herança, composição/agregação, sobrecarga, polimorfismo de inclusão, classes abstratas, polimorfismo paramétrico, modularização, alocação dinâmica de objetos, tratamento de exceções) de forma a corretamente seguir os princípios SOLID.}
    \requisitos  {Algoritmos e Estruturas de Dados 1 e Programação Orientada a Objetos} % % xxx?)
    \recomendadas{N/A}
    \ementa      {Histórico da orientação a objetos. Princípios SOLID (responsabilidade única, aberto-fechado, substituição de Liskov, segregação de interface e inversão de dependência). Atribuição de responsabilidade em programas orientados a objetos. Padrões de projeto em nível de implementação. Prática de desenvolvimento de software orientado a objetos seguindo os princípios SOLID.}
    \creditos    {4 total (2 teóricos, 2 práticos)}
    %    \extra       {x horas}
    \codigo      {DC}{1001521}
    \bibliografia { %deixar linhas em branco para separar os livros
        Craig Larman. Utilizando UML e Padrões. Uma Introdução à Análise e ao Projeto Orientados a Objetos e ao Desenvolvimento Iterativo - 3a edição. Bookman, 2006. 696p. ISBN 8560031529.

        Dave West. Use a Cabeça! Análise \& Projeto Orientado ao Objeto. Alta Books, 2007. 472p. ISBN 978-85-7608-145-6.

        Erich Gamma, Richard Helm, John Vlissides, Ralph Johnson. Padrões de Projeto - Soluções Reutilizaveis de Software Orientado a Objetos. Bookman, 2003. 368p. ISBN 8573076100
    }{
        DEITEL, H.M. \& DEITEL, P. J. - C++ Como Programar, 5ed, Pearson Prentice Hall, 2006

        PIZZOLATO, E. B. - Introdução à programação orientada a objetos com C++ e Java, EdUFSCar, 2010

        ECKEL, B. Thinking in C++. 2ed. Upper Saddle River: Prentice Hall, 2000.

        SILVA FILHO, A. M. Introdução à programação orientada a objetos com C++, Elsevier, 2010

        DEITEL, Paul J.; DEITEL, Harvey M. C++ for programmers. Upper Saddle River, NJ: Prentice Hall, 2009. 1000 p. (Deitel Developer Series). ISBN 10-0-13-700130-9.

        SCHILDT, Herbert. C++: the complete reference. 4. ed. New York: McGraw Hill, c2003. 1023 p. ISBN 0-07-222680-3.
    }
    %Fredy 13/03/2023
    \competencias{
        cg-aprender/{ce-ap-1, ce-ap-2, ce-ap-3},
        cg-produzir/{ce-pro-1, ce-pro-2, ce-pro-4},
        cg-atuar/{ce-atuar-1, ce-atuar-3, ce-atuar-4},
    }
}