\disciplina{plp}{
    \titulo      {7-9}{Paradigmas de Linguagens de Programação}
    \objetivo    {Familiarizar o estudante com diferentes paradigmas de programação, com foco na programação lógica e funcional. Habilitar o estudante a reconhecer as características, vantagens, desvantagens e aplicabilidade de cada paradigma em diferentes situações. Capacitar o estudante a desenvolver programas utilizando os paradigmas de programação lógica e funcional.}
    \requisitos  {Projeto e Análise de Algoritmos} % % xxx?)
    \recomendadas{N/A}
    \ementa      {Motivações para o estudo dos diferentes paradigmas de programação, a evolução histórica e alguns dos principais fatores que definem características das linguagens. Influências da arquitetura de máquina, das metodologias de desenvolvimento de software, e como se verificam as características de modularidade, extensibilidade, efeito colateral e o método de implementação da linguagem em cada paradigma. Conceitos de linguagens imperativas, como amarração de variáveis a escopo, tipo, memória e valor, métodos de passagem de parâmetros, aspectos de implementação de subprogramas, funcionamento da pilha de execução. Conceitos das linguagens de programação lógica, com foco nos aspectos de linguagem declarativa, interpretada, simbólica, e o uso das estruturas de listas e de recursão. Conceitos das linguagens de programação funcional, com foco nos aspectos de linguagem declarativa, interpretada, simbólica, e o uso das estruturas de listas e de recursão. Desenvolvimento de programas com versões imperativas, lógicas e funcionais.}
    \creditos    {4 total (2 teóricos, 2 práticos)}
    %    \extra       {x horas}
    \codigo      {DC}{1001511}
    \bibliografia { %deixar linhas em branco para separar os livros
        SEBESTA, Robert W.. Conceitos de linguagens de programação. [Concepts of programming languages]. José Carlos Barbosa dos Santos (Trad.). 5 ed. Porto Alegre: Bookman, 2003. 638 p. ISBN 85-363- 0171-6. (Disponível na BCo)

        GHEZZI, Carlo; JAZAYERI, Mehdi. Conceitos de linguagens de programação. [Programming language concepts]. Paulo A.S. Veloso (Trad.). Rio de Janeiro: Campus, 1985. 306 p. ISBN 85-7001- 204-7.  (Disponível na BCo)

        SETHI, Ravi. Programming languages: concepts and constructs. 2 ed. Reading: Addison-Wesley, c1996. 640 p. ISBN 0-201- 59065-4.  (Disponível na BCo)

        Nicoletti, Maria do Carmo. A Cartilha da Lógica - 3ª Ed. LTC. 2017. 235p. ISBN 85-2162-693-2
    }{
        FURTADO, Antonio Luz. Paradigmas de linguagens de programação. Campinas: UNICAMP, 1986. 146 p.  (Disponível na BCo)

        SILVA, José Carlos G. da; ASSIS, Fidelis Sigmaringa G. de. Linguagens de programação: conceitos e avaliação; Fortran, C, Pascal, Modula-2, Ada, Chill. São Paulo: McGraw-Hill do Brasil, 1988. 213 p. (Disponível na BCo)

        BRATKO, Ivan. Prolog: programming for artificial intelligence. 2 ed. Harlow: Addison-Wesley, 1990. 597 p. -- (International Computer Science Series) ISBN 0-201- 41606-9.  (Disponível na BCo)

        LISP / W783L.2 WINSTON, Patrick Henry; HORN, Berthold Klaus Paul. Lisp. 2 ed. Reading: Addison-Wesley, 1984. 434 p. ISBN 0-201- 08372-8.  (Disponível na BCo)
    }

    % Inserido por Murillo R. P. Homem, em 01/04/2023
    \dataatualizacao{16/10/23} % Jander, Alexandre, Edilson, Fredy, Márcio, Alan
    \competencias{
        % cg-aprender/{ce-ap-1, ce-ap-2, ce-ap-3, ce-ap-4},
        % cg-atuar/{ce-atuar-1, ce-atuar-2, ce-atuar-3, ce-atuar-4},
        % cg-produzir/{ce-pro-1, ce-pro-2, ce-pro-3},
        % Para
        cg-aprender/{ce-ap-1, ce-ap-2, ce-ap-4},
        cg-atuar/{ce-atuar-1},
        cg-produzir/{ce-pro-2, ce-pro-3},
    }    
}