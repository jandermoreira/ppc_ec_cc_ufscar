\disciplina{aed1}{
    \titulo      {2}{Algoritmos e Estruturas de Dados 1}
    \objetivo    {Tornar os estudantes aptos a utilizar técnicas básicas de programação em seus projetos; capacitar os estudantes a reconhecer, implementar e modificar algoritmos e estruturas de dados básicos; familiarizar os estudantes com noções de projeto e análise de algoritmos, através do estudo de uma linguagem algorítmica, exemplos e exercícios práticos; estimular os estudantes a avaliar quais técnicas de programação, algoritmos e estruturas de dados se adequam melhor a cada situação, problema ou aplicação.}
    \requisitos  {Construção de Algoritmos e Programação} % xxxxxxx
    \recomendadas{N/A}
    \ementa      {Introdução à recursão, com algoritmos e aplicações. Visão intuitiva sobre análise de correção (invariantes) e eficiência (complexidade) de algoritmos. Apresentação de busca linear e binária. Apresentação de algoritmos de ordenação elementares (insertion sort, selection sort e bubble sort). Apresentação de programação por retrocesso (backtracking) e enumeração. Noções de tipos abstratos de dados. Detalhamento de estruturas de dados como: listas (alocação estática e dinâmica, circulares, duplamente ligadas e com nó cabeça), matrizes e listas ortogonais, pilhas e filas (alocação sequencial e ligada) com aplicações. Detalhamento de árvores (definição, representação e propriedades), árvores binárias (manipulação e percursos) e árvores de busca (operações de busca, inserção e remoção). Apresentação de filas de prioridade com detalhamento das implementações triviais e com heap (alocação ligada e sequencial). Apresentação de exemplos e exercícios práticos, os quais podem envolver estruturas de dados compostas (como vetores de listas ligadas) e diferentes abordagens algorítmicas (gulosa, divisão e conquista, programação dinâmica, backtracking, busca em largura, etc).}
    \creditos    {4 total (4 teóricos)}
    % \extra       {4 horas}
    \codigo      {DC}{1001502}
    \bibliografia {
        FEOFILOFF. P. Algoritmos em Linguagem C, Elsevier, 2009.

        AARON M.; TENENBAUM, Y. L.; AUGENSTEIN,  M. J. Estruturas de dados usando C. São Paulo: Pearson Makron Books, 2009.

        FERRARI, R., RIBEIRO, M. X., DIAS, R. L., FALVO, M. Estruturas de Dados com Jogos. Rio de Janeiro – Elsevier, 2014.
    }{
        SEDGEWICK,  R. Algorithms in C++, Parts 1-4: fundamentals, data structures, sorting, searching. 3rd. ed., Boston: Addison - Wesley, 1998.

        SEDGEWICK, R. Algorithms in C++, Part 5: graph algorithms. 3rd. ed., Boston: Addison-Wesley, 2001.

        ZIVIANI, N. Projetos de algoritmos: com implementações em Pascal e C. 3. ed. rev. e ampl. São Paulo: Cengage Learning, 2012.

        ROBERTS,  E.S. Programming Abstractions in C: a Second Course in Computer Science, Addison-Wesley, 1997.

        CIFERRI,  R.R. Programação de Computadores, Edufscar, 2009.
    }
    % Jander, em 15/2/23
    \dataatualizacao{9/10/23} % Jander, Alexandre, Edilson, Fredy
    \competencias{
        % cg-produzir/{ce-pro-1, ce-pro-2, ce-pro-4, ce-pro-5},
        % cg-atuar/{ce-atuar-1, ce-atuar-2, ce-atuar-4},
        % cg-empreender/{ce-emp-1, ce-emp-2, ce-emp-4, ce-emp-5},
        % Para:
        cg-produzir/{ce-pro-1, ce-pro-2, ce-pro-4, ce-pro-5},
        cg-atuar/{ce-atuar-1, ce-atuar-4, ce-atuar-5},
        cg-empreender/{ce-emp-1, ce-emp-2},
    }
}