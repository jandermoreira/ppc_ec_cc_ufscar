\disciplina{cap}{
    \titulo      {1}{Construção de Algoritmos e Programação}
    \objetivo    {Tornar os estudantes aptos a utilizar pensamento computacional e algorítmico para proposição de soluções de problemas. Capacitar os estudantes a mapear tais soluções em programas usando linguagem de programação.}
    \requisitos  {N/A} % xxxxxxx
    \recomendadas{N/A}
    \ementa      {Noções gerais da computação: organização de computadores, programas, linguagens e aplicações. Detalhamento de algoritmos estruturados e programação: tipos básicos de dados. Representação e manipulação de dados. Estruturas de controle de fluxo (condicionais e repetições). Modularização (sub-rotinas, passagem de parâmetros e escopo). Documentação. Estruturação básica de dados: variáveis compostas heterogêneas (registros) e homogêneas (vetores e matrizes). Operações em arquivos e sua manipulação. Alocação dinâmica de memória e ponteiros.}
    \creditos    {8 total (4 teóricos, 4 práticos)}
    %    \extra       {4 horas}
    \codigo      {DC}{1001350}
    \bibliografia {
        CIFERRI. R.R. Programação de Computadores, Edufscar, 2009.

        MEDINA, M. ; FERTIG. C. Algoritmos e Programação: Teoria e Prática, Novatec, 2005.

        SENNE,  E. Primeiro Curso de Programação em C, Visual Books, 2003.

        TREMBLAY, J.P.; BUNT. R.B. Ciência dos Computadores, McGraw-Hill, 1981.

        KERNIGHAN, B.W. ; RITCHIE, D.M. The C Programming Language (2nd Edition), 1988.
    }{
        HARBISON, S.P.; STEELE., G.L. C: a reference manual, 2002.

        KOCHAN; S.G. Programming in C: A complete introduction to the C programming language, 2004.

        KING,  K.N. C Programming: A Modern Approach, Norton \& Company, 1996.
    }
    % Jander, em 15/2/23
    \dataatualizacao{9/10/23} % Jander, Alexandre, Edilson, Fredy
    \competencias{%
        % Revisto sem modificações
        cg-aprender/{ce-ap-4},
        cg-produzir/{ce-pro-2, ce-pro-4},
        cg-atuar/{ce-atuar-1, ce-atuar-2},
        cg-empreender/{ce-emp-2},
    }
}
