\disciplina{ori}{
    \titulo      {5}{Organização e Recuperação da Informação}
    \objetivo    {Tornar os estudantes aptos a solucionar problemas que envolvem a organização e recuperação de informações armazenadas em arquivos. Capacitar os estudantes a implementar estruturas de dados adequadas à organização e busca de informação em meios externos. Familiarizar os estudantes com o projeto e a análise de algoritmos para lidar com informações em disco, através de exemplos e exercícios práticos. Estimular os estudantes a avaliar quais técnicas de programação, algoritmos e estruturas de dados se adequam melhor a cada situação, problema ou aplicação.}
    \requisitos  {Algoritmos e Estruturas de Dados 1} % xxxxxxx
    \recomendadas{N/A}
    \ementa      {Apresentação dos conceitos de representação, organização, armazenamento e recuperação de dados em memória secundária. Noções sobre a estrutura física de dispositivos de armazenamento secundário (discos magnéticos, fitas magnéticas, discos de estado sólido e novas tecnologias). Apresentação do conceito de organização de arquivos: arquivos dos tipos binários e texto, campos, registros e reaproveitamento de espaço na remoção lógica de registros. Apresentação de conceitos e implementação de índices: índice linear, índice multinível, índices primário e secundário, estruturas de árvores de múltiplos caminhos (árvores B, B+, B* e B virtual com buffer-pool). Noções sobre índices para dados não convencionais (árvores métricas, quadtrees e índices bitmap). Apresentação de algoritmos para o processamento cossequencial de listas em memória secundária e ordenação externa. Apresentação do conceito e implementação para hashing externo: funções e espalhamento, baldes, hash dinâmico, uso de hash como mecanismo de indexação. Apresentação de conceitos de compressão de dados sem perda de informação (Huffman, LZW ou similares). Apresentação da organização de memória interna: métodos sequenciais e não sequenciais; coleta de lixo.}
    \creditos       {4 total (4 teóricos)}
    %    \extra       {4 horas}
    \codigo      {DC}{1001487}
    \bibliografia {
        M. J. Folk, B. Zoellick. File Structures, Second Edition. Addison-Wesley, Hardcover, Published June 1992.

        Nivio Ziviani. Projetos de algoritmos: com implementações em Pascal e C. 3. ed. rev. e ampl. São Paulo: Cengage Learning, 2012.

        Adam Drozdek. Estruturas de dados e algoritmos em C++. São Paulo: Cengage Learning, 2010.

        T.H. Cormen, C.E. Leiserson, R.L. Rivest, C. Stein, Introduction to Algorithms, 3rd ed., McGraw-Hill, 2009.
    }{
        Yedidyah Langsam, Moshe J. Augenstein, Aaron M. Tenenbaum. Data structures using C and C++. 2. ed. Upper Sadle River: Prentice Hall, 1996.

        Aaron M. Tenenbaum, Yedidyah Langsam, Moshe J. Augenstein. Estruturas de dados usando C. São Paulo: Pearson Makron Books, 2009.

        Jayme Luiz Szwarcfiter, Lilian Markenzon. Estruturas de Dados e seus Algoritmos. 3a edição, Rio de Janeiro: LTC, 2010.

        Nivio Ziviani. Projeto de algoritmos: com implementações em Java e C++. 2. ed. São Paulo: Cengage Learning, 2011.
    }
    \dataatualizacao{9/10/23} % Jander, Alexandre, Edilson, Fredy
    \competencias{
        % cg-aprender/{ce-ap-1, ce-ap-2},
        % cg-produzir/{ce-pro-2, ce-pro-4},
        % cg-atuar/{ce-atuar-4, ce-atuar-5}
        % Para:
        cg-aprender/{ce-ap-2},
        cg-produzir/{ce-pro-2, ce-pro-4},
        cg-atuar/{ce-atuar-1, ce-atuar-4},
        cg-buscar/{ce-busc-4},
    }
}