\disciplina{circeletronicos1}{
    \titulo      {5}{Circuitos Eletrônicos 1}
    \objetivo    {Promover o entendimento das características físicas dos materiais adotados para emergência dos comportamentos não lineares dos dispositivos eletrônicos básicos. Caracterizar os dispositivos eletrônicos básicos e suas propriedades não lineares. Desenvolver habilidades de modelagem, análise e síntese de circuitos eletrônicos. Gerar a capacitação em modelagem de circuitos eletrônicos por regiões de comportamento linear e resolução com verificação de hipóteses. Apresentar e desenvolver projetos dos principais circuitos funcionais e aplicações.}
    \requisitos  {Circuitos Elétricos} % xxxxxxx OU Fisica Experimental B % Sugestão Sabrina CoC 
    \recomendadas{N/A}
    \ementa      {Características e comportamentos de sistemas não lineares. Estratégias de análise de sistemas não lineares. Vantagens e desvantagens de sistemas não lineares. Materiais semicondutores básicos e suas propriedades. Concepção de dispositivos eletrônicos básicos. Caracterização do diodo: comportamentos e modelos: ideal, aproximado e teórico. Circuitos com diodos: modelagem e estratégia de análise de sistemas não lineares por regiões de comportamento. Síntese de circuitos com diodos. Caracterização do transistor de junção bipolar (TJB): comportamentos, modelos e configurações. Ponto de operação e circuito de polarização. Circuitos com transistores: modelagem e estratégia de análise. Síntese de circuitos com transistores TJB. Aplicações. Amplificadores operacionais: conceituação e propriedades. Análise e projeto de circuitos com base em amplificadores operacionais. Aplicações usuais e relevantes de amplificadores operacionais.}
    \creditos    {6 total (4 teóricos, 2 práticos)}
    %    \extra       {3 horas}
    \codigo      {DC}{1001533}
    \bibliografia {AMARAL, A. M. Raposo. Análise de circuitos e dispositivos eletrónicos. Porto: Publindústria, Edições Técnicas, 2013.

    TOOLEY, M.. Circuitos eletrônicos: fundamentos e aplicações. Rio de Janeiro: Elsevier, 2007.

    BOYLESTAD, R. L.; NASHELSKY, Louis. Dispositivos eletrônicos e teoria de circuitos. 11. ed. São Paulo: Pearson, 2013.
    }
    {
        MALVINO, A. P.; BATES, D. J. Eletrônica. 7. ed. Porto Alegre, RS: AMGH Editora, 2007.

    COMER, D. J.; COMER, Donald. Fundamentos de projeto de circuitos eletrônicos. Rio de Janeiro: LTC, 2005.

    SILVA, M. M. Introducao aos circuitos electricos e electronicos. 2. ed. Lisboa: Fundacao Calouste Gulbenkian, 2001

    HAYT, W.H.; Neudeck, G. W. ; Electronic Circuit Analysis and Design; Edition 2; Wiley;0 1984.

    BATARSEH, I.; Ahmad H.; Power Electronics: Circuit Analysis and Design; Springer; 2018.
    }

    \dataatualizacao{12/12/23} % Luciano
    \competencias{
        cg-aprender/{ce-ap-1, ce-ap-3},
        cg-produzir/{ce-pro-1, ce-pro-2, ce-pro-4},
        cg-atuar/{ce-atuar-1, ce-atuar-4}
    }
}
