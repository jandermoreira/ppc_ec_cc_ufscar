\disciplina{fisica1}{
    \titulo      {2}{Física 1}
    \objetivo    {Introduzir os princípios básicos da Física Clássica (Mecânica), tratados
    de forma elementar, desenvolvendo no estudante a intuição necessária para analisar fenômenos físicos sob os pontos de vista qualitativo e quantitativo. Despertar o interesse e ressaltar a necessidade do estudo desta matéria, mesmo para não especialistas.}
    \requisitos  {N/A} % xxxxxxx
    \recomendadas{N/A}
    \ementa      {Movimento de uma partícula em 1D, 2D e 3D; Leis de Newton; Aplicações das Leis de Newton – Equilíbrio de Líquidos (Arquimedes) - Forças Gravitacionais; Trabalho e Energia; Forças Conservativas – Energia Potencial; Conservação da Energia (Equação de Bernoulli); Sistemas de Várias Partículas – Centro de Massa; Colisões; Conservação do Momento Linear.
    }
    \creditos    {4 total (4 teóricos)}
    %    \extra       {x horas}
    \codigo      {DF}{09.901-5}
    \bibliografia {
        HALLIDAY, D.; RESNICK, R.; WALKER. J. Fundamentos de Física, volume 1, Mecânica, 9ª edição, GEN/LTC 2012.

        YOUNG, H. D.; FRIEDMAN. R. A.; SEARS ; ZEMANSKY. Física I: Mecânica, 12.ed.. Pearson, São Paulo.

        CHAVES,  A.; SAMPAIO,  J.F. Física Básica: Mecânica.
    }{
        NUSSENZVEIG, H.M. Curso de Física Básica, v. 1

        TIPLER,  P. A.; . MOSCA,  G. Física para cientista e engenheiros, v. 1, Mecânica, 6. ed., GEN/LTC, 2008.

        SERWAY R. A.; JEWETT, Jr. J. W. Princípios de Física, v. 1, Mecânica. 3 ed.. Editorial Thomson. 2005.

        FEYNMAN R.P.; Lectures on Physics, v. 1.

        KELLER, F.J.; GETTYS,  W.E.; SKOVE, M.J. Física, v. 1
    }

}
