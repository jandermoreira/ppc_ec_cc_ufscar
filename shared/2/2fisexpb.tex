\disciplina{fisexpb}{
    \titulo      {3}{Física Experimental B}
    \objetivo    {Ao final da disciplina, o aluno deverá ter pleno conhecimento dos conceitos básicos, teórico-experimentais, de eletricidade, magnetismo e óptica geométrica. - Conhecerá os princípios de funcionamento e dominará a utilização de instrumentos de medidas elétricas, como: osciloscópio, voltímetro, amperímetro e ohmímetro. Saberá a função de vários componentes passivos, e poderá analisar e projetar circuitos elétricos simples, estando preparado para os cursos mais avançados, como os de Eletrônica. - Em óptica geométrica, verificará experimentalmente, as leis da reflexão e refração.
    }
    \requisitos  {N/A} % xxxxxxx
    \recomendadas{N/A}
    \ementa      {Medidas elétricas;Circuitos de corrente contínua; Indução eletromagnética; Resistência, capacitância e indutância; Circuitos de corrente alternada; Óptica geométrica: Dispositivos e instrumentos; Propriedades elétricas e magnéticas da matéria}
    \creditos    {4 total (4 práticos)}
    %    \extra       {x horas}
    \codigo      {DF}{09.111-1}
    \bibliografia {HALLIDAY, D.; RESNICK, R.; WALKER, J. Fundamentos de física. [Fundamentals of physics]. Gerson Bazo Costamilan (Trad.). 4 ed. Rio de Janeiro: LTC, c1993.

    TIPLER, P. A., Física para cientistas e engenheiros. [Physics for scientists and engineers]. Horacio Macedo (Trad.). 4 ed. Rio de Janeiro: LTC, c2000.

    NUSSENZVEIG, H. M. Curso de física básica. São Paulo: Edgard Blucher, 1997.
    }
    {
        BROPHY, J. J. Eletronica basica. Julio Cesar Goncalves Reis (Trad.). 3 ed. Rio de Janeiro: Guanabara Dois, 1978.

    CUTLER, P. Analise de circuitos CC, com problemas ilustrativos. Adalton Pereira de Toledo (Trad.). Sao Paulo: McGraw-Hill do Brasil, 1976.

    CUTLER, P. Analise de circuitos CA: com problemas ilustrativos. Adalton Pereira de Toledo (Trad.). Sao Paulo: McGraw-Hill do Brasil, 1976.

    NUSSENZVEIG, H. M.. Curso de Fisica Basica. 3 ed. Sao Paulo: Edgard Blucher, 1996.

    SERWAY, R. A. Física para cientistas e engenheiros com fisica moderna. [Physics for scientists and engineers with modern physics]. Horacio Macedo (Trad.). 3 ed. Rio de Janeiro: LTC, c1996.

    HALLIDAY, D; RESNICK, R; KRANE, K. S. Fisica III e IV. [Physics]. Denise Helena Sotero da Silva (Trad.). 4 ed. Rio de Janeiro: LTC, c1996.
    }

    % Inserido por Edilson Kato, em 11/03/2024
     \competencias{
        cg-aprender/{ce-ap-1, ce-ap-2, ce-ap-3, ce-ap-4},
        cg-atuar/{ce-atuar-1, ce-atuar-2, ce-atuar-3, ce-atuar-4},
        }
}
