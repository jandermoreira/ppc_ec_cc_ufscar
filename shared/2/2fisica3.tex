\disciplina{fisica3}{
    \titulo      {3}{Física 3}
    \objetivo    {Nesta disciplina serão ministrados aos estudantes os fundamentos de eletricidade e magnetismo e suas aplicações. Os estudantes terão a oportunidade de aprender as equações de Maxwell. Serão criadas condições para que os mesmos possam adquirir uma base sólida nos assuntos a serem discutidos, resolver e discutir questões e problemas ao nível do que será ministrado e de acordo com as bibliografias recomendadas.}
    \requisitos  {Física 1} % 99015 OU 520136 OU 98108 OU 90018 OU 24023 OU 98019 OU 98051 OU 342092 OU 520098
    \recomendadas{N/A}
    \ementa      {Carga elétrica, força de Coulomb e conceito de campo elétrico; Cálculo do campo elétrico por integração direta e através da Lei de Gauss. Aplicações; Potencial elétrico. Materiais dielétricos e Capacitores; Corrente elétrica, circuitos simples e circuito RC; Campo magnético; Cálculo do campo magnético: Lei de Ampère e Biot-Savart; Indução eletromagnética e Lei de Faraday; Indutância e circuito RL; Propriedades magnéticas da matéria: diamagnetismo, paramagnetismo e ferromagnetismo.}
    \creditos    {4 total (4 teóricos)}
    %    \extra       {x horas}
    \codigo      {DF}{09.903-1}
    \bibliografia {
        Halliday, D.; Resnick , R.; Walker. J. Fundamentos da Física, 6ª ed., Rio de Janeiro: LTC, 2003.

        Young, H. D.; Friedman. R. A. Física III: Eletromagnetismo, 12ª ed., São Paulo: Addison Wesley, 2008.

        Tipler,  P. A.; Mosca. G. Física para cientistas e engenheiros, 5ª ed., Rio de Janeiro: Editora LTC, 2006.
    }{
        Serway R. A. ; Jewett Jr. J. W. Física: para cientistas e engenheiros, [Rio de Janeiro: LTC, 1996] ou [São Paulo: Cengage Learning, 2008].

        Nussenzveig,  H. M. Curso de Física Básica, 1. ed., São Paulo: Edgard Blucher, 1997.

        Keller , F. J.; Gettys, W. E.; Skove. M. J. Física, São Paulo: Makron Books, c1999.

        Feynman, R. P; Leighton, R. B.; Sands. M. The Feynman lectures on physics. Reading: Addison-Wesley, c1963.

        Chaves, A. S. Física: curso básico para estudantes de ciências físicas e engenharias. Rio de Janeiro: Reichmann \& Affonso, 2001.
    }
    % Inserido por Edilson Kato, em 11/03/2024
     \competencias{
        cg-aprender/{ce-ap-1, ce-ap-2, ce-ap-3, ce-ap-4},
        cg-atuar/{ce-atuar-1, ce-atuar-2, ce-atuar-3, ce-atuar-4},
        }
}
