\disciplina{fisexpa}{
    \titulo      {2}{Física Experimental A}
    \objetivo    {Treinar o aluno para desenvolver atividades em laboratório. Familiarizá-lo
    com instrumentos de medidas de comprimento, tempo e temperatura. Ensinar o aluno a organizar dados experimentais, a determinar e processar erros, a construir e analisar gráficos, para que possa fazer uma avaliação crítica de seus resultados. Verificar experimentalmente as leis da Física.}
    \requisitos  {N/A} % xxxxxxx
    \recomendadas{N/A}
    \ementa      {Medidas e erros experimentais; Cinemática e dinâmica de partículas; Cinemática e dinâmica de corpos rígidos; Mecânica de meios contínuos;Termometria e calorimetria.}
    \creditos    {4 total (4 práticos)}
    %    \extra       {x horas}
    \codigo      {DF}{09.110-3}
    \bibliografia {
        INMETRO. Avaliação de dados de medição: guia para a expressão de incerteza de medição – GUM 2008. Traduzido de: Evaluation of measurement data: guide to the expression of uncertainty in measurement – GUM 2008. 1ª Ed. Duque de Caxias, RJ: INMETRO/CICMA/SEPIN, 2012. Disponível em:< http://www.inmetro.gov.br/infotec/publicacoes/gum\_final.pdf>. Acesso em: 13 Mar. 2013.

        HALLIDAY, D.; RESNICK, R.; WALKER, J. Fundamentos de Física: mecânica. [Fundamentals of physics]. Gerson Bazo Costamilan (Trad.). 4. ed. Rio de Janeiro: LTC, 1993

        VUOLO, J. H. Fundamentos da Teoria de Erros. 2. ed. São Paulo, SP: Editora Edgard Blücher LTDA
    }{
        INMETRO. Vocabulário internacional de termos fundamentais e gerais de Metrologia: portaria INMETRO nº 029 de 1995. 5. ed. Rio de Janeiro: Editiora SENAI, 2007.

        NUSSENZVEIG, H. M. Curso de Física Básica, 3. ed. São Paulo: Editora Edgard Blücher LTDA, 1996.

        CAMPOS, A. A; ALVES, E.S; SPEZIALI, N.L. Física Experimental Básica na Universidade, 2. ed. Belo Horizonte: Editora UFMG, 2008.

        DUPAS, M. A. Pesquisando e normalizando: noções básicas e recomendações úteis para a elaboração de trabalhos científicos. 6. ed. São Carlos: Editora EdUFSCar, 2009.

        WORSNOP, B. L.; FLINT, H. T. Curso Superior de Física Práctica - Tomo I. Buenos Aires: EUDEBA, 1964.
    }
    % Inserido por Edilson Kato, em 11/03/2024
     \competencias{
        cg-aprender/{ce-ap-1, ce-ap-2, ce-ap-3, ce-ap-4},
        cg-atuar/{ce-atuar-1, ce-atuar-2, ce-atuar-3, ce-atuar-4},
        }
}
