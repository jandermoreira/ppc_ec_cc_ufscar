\disciplina{cg}{
    \titulo      {7-9}{Computação Gráfica}
    \objetivo    {Familiarizar o estudante com os conceitos fundamentais da área; capacitar o estudante a compreender a organização e as funcionalidades de sistemas gráficos; capacitar o estudante a implementar abordagens básicas na solução de problemas em computação gráfica.}
    \requisitos  {Geometria Analítica e Algoritmos e Estruturas de Dados 1} % % xxx?)
    \recomendadas{N/A}
    \ementa      {Introdução à computação gráfica; apresentar os tipos de equipamentos e tecnologias atuais disponíveis em computação gráfica; algoritmos básicos: aspectos geométricos e transformações (problemática associada e algoritmos). Noções da teoria de cores. Aprofundamento em modelagem de objetos bidimensionais e tridimensionais. Apresentação de projeções planares. Aprofundamento em transformações de visualização, determinação de superfícies visíveis e técnicas de iluminação e sombreamento. Visão geral de programação com pacotes gráficos padrões. Noções de gerenciamento de eventos. Noções de animação.}
    \creditos    {4 total (2 teóricos, 2 práticos)}
    %    \extra       {x horas}
    \codigo      {DC}{1001536}
    \bibliografia { %deixar linhas em branco para separar os livros
        ANGEL, E. and Shreiner D. Interactive Computer Graphics: A Top-Down Approach With WebGL, 7th ed., Pearson 2014.

        FOLEY, J. et al. Computer graphics: principles and practice, 3rd ed., Addison-Wesley Professional, 2013, 1264 p.

        SHREINER, Dave et al. OpenGL Programming Guide: The Official Guide to Learning OpenGL, Version 4.3, 8th ed.,  Addison-Wesley, 2013, 935 p.
    }{
        COHEN, Marcelo; MANSSOUR, Isabel. OpenGL - Uma Abordagem Prática e Objetiva. São Paulo: Novatec, 2006. 486 p.

        AZEVEDO, Eduardo; CONCI, Aurea. Computação Gráfica. Geração de Imagem - Volume 1- Teoria e Prática. Elsevier, 2003. 384 p

        VELHO, L. e GOMES J. M. Fundamentos da Computação Gráfica. Rio de Janeiro: IMPA, 2008.
    }
    % Inserido por Murillo R. P. Homem, em 08/03/2023
    %\dataatualizacao{12/12/23} % Luciano      
    %\competencias{   
    %     cg-aprender/{ce-ap-1, ce-ap-2, ce-ap-3, ce-ap-4},
    %     cg-produzir/{ce-pro-1, ce-pro-2, ce-pro-3, ce-pro-4, ce-pro-5},
    %     cg-atuar/{ce-atuar-1, ce-atuar-2, ce-atuar-3}
    %     }

    % Atualizado em 21/02/2024
    % Inserido por Murillo Rodrigo Petrucelli Homem 
    \competencias{   
        cg-produzir/{ce-pro-1,ce-pro-2},
        cg-atuar/{ce-atuar-1,ce-atuar-4},
        cg-pautar/{ce-paut-4}
    }    
}
