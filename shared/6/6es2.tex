\disciplina{es2}{
    \titulo      {7-9}{Engenharia de Software 2}
    \objetivo    {Habilitar o estudante a gerenciar o processo de desenvolvimento de um sistema de software; habilitar o estudante a aplicar testes funcionais e estruturais em sistemas de software; familiarizar o estudante com conceitos de qualidade de software e fazer com que ele consiga refletir esses conceitos na prática.}
    \requisitos  {Engenharia de Software 1} % % xxx?)
    \recomendadas{N/A}
    \ementa      {Aprofundamento sobre Ciclo de Vida de Desenvolvimento e Manutenção de Sistemas. Modelos de Processo e Metodologias Ágeis: características, diretrizes de escolha e simulação. Técnicas de gerenciamento de projetos de software (local e distribuído geograficamente): Gerenciamento de configuração e de versões. Técnicas de Verificação e Validação de Software: Testes Funcionais e Estruturais; Conceituação e Exemplificação de Tipos de Manutenção de Software; Caracterização de qualidade de software e seu emprego/manutenção ao longo das fases do desenvolvimento; métricas, smells e refatorações. Visão geral sobre modelos de melhoria de processo.}
    \creditos    {4 total (4 teóricos)}
    %    \extra       {x horas}
    \codigo      {DC}{1001516}
    \bibliografia { %deixar linhas em branco para separar os livros
        PRESSMAN, Roger S.; MAXIM, Bruce R. Engenharia de software: uma abordagem profissional. 8. ed. Porto Alegre: AMGH, 2016. 940 p. ISBN 9788580555332. Disponível na BCo.

        SOMMERVILLE, Ian. Engenharia de software. 9. ed. São Paulo: Pearson Prentice Hall, 2011. 529 p. ISBN 97885793611081. Disponível na BCo.

        DELAMARO, Márcio E.; MALDONADO, José C.; JINO, Mario. Introdução ao teste de software. Rio de Janeiro: Elsevier : Campus, 2007, 394 p. ISBN 9788535226348. Disponível na BCo.
    }{
        GORTON, Ian. Essential Software Architecture. Springer-Verlag, Germany, 2016. ISBN 9783642191763

        COHN, Mike. Agile estimating and planning. Upper Saddle River, NJ: Prentice Hall Professional Technical Reference, 2010. 330 p. (Robert C. Martin Series). ISBN 9780131479418.

        THE CAPABILITY maturity model: guidelines for improving the software process. Boston: Addison-Wesley, 2001. 441 p. (The SEI Series in Software Engineering). ISBN 0-201-54664-7.

        PFLEEGER, Shari Lawrence. Engenharia de software: teoria e prática. 2. ed. São Paulo: Prentice Hall, 2004. 537 p. ISBN 85-87918-31-1.
    }
        %Fredy 13/03/2023
    \dataatualizacao{23/10/23} % Edilson, Márcio, Luciano, Menotti, Helio, Jander, 
     \competencias{
        cg-aprender/{ce-ap-1, ce-ap-2, ce-ap-3},
        cg-produzir/{ce-pro-1, ce-pro-2, ce-pro-4},
        cg-atuar/{ce-atuar-1, ce-atuar-3, ce-atuar-4},
    }
}