\disciplina{pvd}{
    \titulo      {7-9}{Processamento e Visualização de Dados}
    \objetivo    {Capacitar o estudante a entender e trabalhar com os procedimentos necessários para transformar dados possibilitando a análise e visualização destes por ferramentas computacionais, garantindo qualidade e minimizando distorções. Familiarizar o estudante com os princípios e técnicas de visualização da informação e como trabalhar com eficiência considerando os recursos gráficos atuais, por software e/ou hardware especializados.}
    \requisitos  {Computação Gráfica} % % xxx?)
    \recomendadas{N/A}
    \ementa      {Introdução ao conceito de conjuntos de dados e aprofundamento na análise estatística apropriada para técnicas de mineração de dados. Apresentação dos modelos básicos de preparação de dados. Apresentação de técnicas para lidar com valores ausentes e com dados ruidosos. Técnicas para redução de dados; seleção de atributos e instâncias. Amostragem. Discretização. Introdução a dados tabulares. Modelos de projeções multidimensionais,  hierárquicas e gráficos tridimensionais.}
    \creditos    {4 total (2 teóricos, 2 práticos)}
    %    \extra       {x horas}
    \codigo      {DC}{1001514}
    \bibliografia { %deixar linhas em branco para separar os livros
        Colin Ware, Information Visualization (Third Edition), Elsevier, 2012

        Kamber, Micheline; Han, Jiawei ;Pei, Jian. Data Mining: Concepts And Techniques, 3o edition, Morgan Kaufmann, 2011.

        Salvador García, Julián Luengo, Francisco Herrera. Data Preprocessing in Data Mining (Intelligent Systems Reference Library), Springer 2015.
    }{
        Information Visualization: Design for Interaction (2nd Edition). Robert Spence, Prentice Hall, 2007.

        Jake VanderPlas, Title Python Data Science Handbook: Essential Tools for Working with Data. O'Reilly Media; 2016; eBook (GitHub)

        TELEA, A. Data Visualization: Principles and Practice. A.K. Peters, 2007.

        WARD, M. O.; GRINSTEIN, G.; KELM, D. Interactive Data Visualization. A.K. Peters Ltd., 2010.

        GERALD, F.; HANSFORD, D. Mathematical Principles for Scientific Computing and Visualization. A.K.Peters Ltd. , 2008

        An Introduction to data cleaning with R, 2013. (\url{https://cran.r-project.org/doc/contrib/de_Jonge+van_der_Loo-Introduction_to_data_cleaning_with_R.pdf})

        7 Steps to Mastering Data Preparation with Python
        https://www.kdnuggets.com/2017/06/7-steps-mastering-data-preparation-python.html

        Levine, D. C. et al. Estatística: Teoria e Aplicações. 5ª ed. Rio de Janeiro: LTC, 2008

    }
    %Fredy 13/03/2023
    \dataatualizacao{06/11/23} % Kelen, Luciano, Fedy, Alexandre, Kato, Helio, Jander, Menotti, Orides        
   \competencias
    {
        %cg-aprender/{ce-ap-1, ce-ap-2, ce-ap-3},
        %cg-produzir/{ce-pro-1, ce-pro-2, ce-pro-4},
        %cg-atuar/{ce-atuar-1, ce-atuar-3, ce-atuar-4},
        cg-aprender/{ce-ap-1, ce-ap-3},
        cg-gerenciar/{ce-ger-3},
        cg-atuar/{ce-atuar-3, ce-atuar-4},
        cg-buscar/{ce-busc-3, ce-busc-4}        
    }
}