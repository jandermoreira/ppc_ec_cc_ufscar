\disciplina{ia}{
    \titulo      {6}{Inteligência Artificial}
    \objetivo    {Capacitar o estudante para utilizar representação de conhecimento na construção de algoritmos a partir dos conceitos da IA. Propiciar ao estudante a aquisição dos conceitos relacionados à busca, representação de conhecimento, raciocínio automático e aprendizado de máquina. Desenvolver no estudante a competência para saber identificar problemas que podem ser resolvidos com técnicas da IA e quais técnicas podem ser adequadas a cada problema.}
    \requisitos  {Algoritmos e Estruturas de Dados 1} %  ou Programação e Algoritmos 2.

    \recomendadas{Probabilidade e Estátistica} %

    \ementa      {Caracterização da área de IA. Apresentação de métodos de busca desinformada e informada para a resolução de problemas: busca em largura, busca de custo uniforme, busca em profundidade, subida da encosta, têmpera simulada, algoritmos evolutivos. Introdução à representação de conhecimento baseada em lógica. Visão geral de métodos de raciocínio e inferência: algoritmos de encadeamento para frente e para trás, resolução e programação lógica. Introdução à representação de conhecimento incerto: quantificação de incerteza e raciocínio probabilístico. Noções de aprendizado de máquina supervisionado e não-supervisionado: classificação, regressão e agrupamento.}
    \creditos    {4 total (2 teóricos, 2 práticos)}
    %    \extra       {x horas}
    \codigo      {DC}{1001336}
    \bibliografia {
        RUSSELL, Stuart J; NORVIG, Peter. Artificial intelligence: a modern approach. 3. ed. Upper Saddle River: Prentice-Hall, c2010. 1131 p. ISBN 978-0-13-604259-4.
        % 
        LUGER, George F. Artificial intelligence: Structures and strategies for complex problem solving. 5. ed. Harlow: Addison Wesley Longman, c2005. 824 p. ISBN 0-321-26318-9.
        % 
        BRATKO, Ivan. Prolog: programming for artificial intelligence. 2. ed. Harlow: Addison-Wesley, 1990. 597 p. (International Computer Science Series). ISBN 0-201-41606-9.
    }
    {
        MITCHELL, Tom M. Machine learning. Boston: MCB/McGraw-Hill, 1997. 414 p. (McGraw-Hill Series in Computer Science). ISBN 0-07-042807-7;
    %
    BITTENCOURT, Guilherme. Inteligência artificial: ferramentas e teorias. 3. ed. Florianópolis, SC: Editora da UFSC, 2006. 371 p. : il., tabs. (Série Didática). ISBN 8532801382;
    %
    FACELI, Katti; LORENA, Ana Carolina; GAMA, João; CARVALHO, André Carlos Ponce de Leon Ferreira de. Inteligência artificial: uma abordagem de aprendizado de máquina. Rio de Janeiro: LTC, 2011. 378 p. ISBN 9788521618805;
    %
    COPPIN, Ben. Inteligência artificial. Grupo Gen-LTC, 2015.
    }
    \dataatualizacao{23/10/23} % Edilson, Márcio, Luciano, Menotti, Helio, Jander, 
    \competencias{
        cg-aprender/{ce-ap-1, ce-ap-3, ce-ap-4},
        cg-produzir/{ce-pro-1, ce-pro-2, ce-pro-4, ce-pro-5},
        cg-empreender/{ce-emp-3, ce-emp-4},
        cg-atuar/{ce-atuar-1, ce-atuar-3, ce-atuar-4}
    }
}