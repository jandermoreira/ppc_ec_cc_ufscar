\disciplina{es1}{
    \titulo      {4}{Engenharia de Software 1}
    \objetivo    {Capacitar os estudantes a realizar levantamento de requisitos; capacitar os estudantes a elaborar modelos (diagramas) que traduzem os requisitos em uma solução de software de qualidade; tornar os estudantes aptos a especificar diagramas que cobrem vários níveis de abstração de um sistema de software; habilitar os estudantes a refletir sobre a modelagem de sistemas não triviais, como de tempo real, embarcados, ferramentas, etc.}
    \requisitos  {Programação Orientada a Objetos} % xxxxx
    \recomendadas{N/A}
    \ementa      {Histórico da Engenharia de Software. Visão sobre Ciclo de Vida de Desenvolvimento de Sistemas de Software. Detalhamento do processo de gerenciamento de requisitos com ênfase na elicitação e especificação: documento de requisitos e casos de uso. Detalhamento do Processo de Conversão de Requisitos em Modelos Conceituais (Diagramas de Classes e Diagramas de Sequência do Sistema - DSS). Introdução à Modelagem comportamental: Diagramas de Estado em nível de análise. Introdução ao Projeto de Software. Detalhamento da conversão dos modelos de análise em Modelos de Projeto: Diagrama de Classes e de Pacotes (Subsistemas). Apresentação do conceito de modularização (agrupamento de classes que atendem a determinado critério). Conversão dos Modelos de Análise em Modelos Projeto: Diagramas de Sequência/Colaboração. Diagrama de Estados em nível de projeto. Utilização de Diagramas de Componentes para modularização do sistema. Utilização de diagramas de implantação.}
    \creditos    {4 total (4 teóricos)}
    %    \extra       {x horas}
    \codigo      {DC}{1001530}
    \bibliografia {
        SOMMERVILLE, Ian. Engenharia de software. 9. ed. São Paulo: Pearson Prentice Hall, 2011. 529 p. ISBN 97885793611081;

        PRESSMAN, Roger S.; MAXIM, Bruce R. Engenharia de software: uma abordagem profissional. 8. ed. Porto Alegre: AMGH, 2016. 940 p. ISBN 9788580555332.

        FURLAN, José Davi. Modelagem de objetos através da UML - Unified Modeling Languagem. São Paulo: Makron Books, 1998. 329 p. ISBN 85-346-0924-1.

        BLAHA, Michael; RUMBAUGH, Michael. Modelagem e projetos baseados em objetos com UML 2. 2. ed. Rio de Janeiro: Elsevier, 2006. 496 p. ISBN 8535217533.
    }
    {
        HUMPHREY, Watts S. A discipline for software engineering. Reading: Addison-Wesley, 1995. 789 p. (SEI Series in Software Engineering). ISBN 0-201-54610-8.

    PFLEEGER, Shari Lawrence. Engenharia de software: teoria e prática. 2. ed. São Paulo: Prentice Hall, 2004. 537 p. ISBN 85-87918-31-1.

    ENGINEERING and managing software requirements. Berlin: Springer, 2006. AURUM, Aybüke; WOHLIN, C.(Eds.), 478 p. (Institute for nonlinear science). ISBN 3-540-25043-3.
    }

    % Engenharia de Software 1
    % Inserido por Murillo R. P. Homem, em 09/02/2023
    % Compilado a partir dos formulários preenchidos por Valter Camargo e Auri Vincenzi
    % \competencias{
    %     cg-aprender/{ce-ap-1, ce-ap-2},
    %     cg-produzir/{ce-pro-1, ce-pro-2},
    %     cg-atuar/{ce-atuar-1, ce-atuar-2, ce-atuar-3, ce-atuar-5},
    %     cg-gerenciar/{ce-ger-1, ce-ger-3},
    %     cg-empreender/{ce-emp-1, ce-emp-2}
    % }
    \dataatualizacao{23/10/23} % Edilson, Márcio, Luciano, Menotti, Helio, Jander    
    \competencias{
        cg-aprender/{ce-ap-3, ce-ap-4},
        cg-gerenciar/{ce-ger-1, ce-ger-3},
        cg-produzir/{ce-pro-1, ce-pro-5},
    }

    % Murillo: retirei os itens ce-pro-3, ce-pro-5

}