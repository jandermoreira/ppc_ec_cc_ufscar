\disciplina{pdi}{
    \titulo      {7-9}{Processamento Digital de Imagens}
    \objetivo    {Habilitar o estudante a aplicar técnicas de melhoramento e segmentação de imagens digitais; habilitar o estudante a realizar o pré-processamento de imagens para a subsequente aplicação de técnicas de aprendizado de máquina; capacitar o estudante a identificar as técnicas mais adequadas a serem aplicadas dependendo do tipo de imagem a ser processada.}
    \requisitos  {Construção de Algoritmos e Programação, Cálculo Diferencial e Integral I e Geometria Analítica} % % xxx?)
    \recomendadas{N/A}
    \ementa      {Visão biológica e artificial. Visão geral das etapas de um sistema de processamento de imagens. Apresentação de técnicas de modificação de histogramas. Detalhamento sobre filtragem espacial de imagens (filtros lineares e não-lineares). Aprofundamento sobre filtragem de imagens no domínio da frequência. Apresentação de processamento multiresolução. Apresentação detalhada sobre processamento morfológico. Visão geral sobre técnicas de representação e descrição de imagens. Apresentação de técnicas de segmentação de imagens.}
    \creditos    {4 total (2 teóricos, 2 práticos)}
    %    \extra       {x horas}
    \codigo      {DC}{1001527}
    \bibliografia { %deixar linhas em branco para separar os livros
        R. C. Gonzalez e R. E. Woods, “Digital Image Processing” (3rd Edition), Prentice-Hall, 2008.  (disponível na BCo - UFSCar)

        A. K. Jain, “Fundamentals of digital image processing”, Prentice-Hall, 1989. (disponível na BCo - UFSCar)

        H. Pedrini e W. Robson, “Análise de imagens digitais: princípios, algoritmos e aplicações”, Thomson Learning, 2008. (disponível na BCo - UFSCar)
    }{
        D. A. Forsyth and J. Ponce, "Computer Vision: A Modern Approach", Prentice Hall, 2003. (disponível na BCo).

        R. Szeliski, “Computer Vision: Algorithms and Applications”, Springer, 2010 (http://szeliski.org/Book/). (disponível on-line)

        M. Nixon, A. S. Aguado, “Feature Extraction \& Image Processing for Computer Vision”, (2nd Edition), Academic Press, 2008.

        Openheim, A. V. and Schafer, R. W., Discrete-Time Signal Processing, Prentice-Hall, 1989 (disponível na BCO - UFSCar)

        Proakis, J. G. and Manolakis, D. G., Digital Signal Processing: Principles, Algorithms and Applications, MacMIllan, 1992 (disponível na BCO – UFSCar)

    }
    %Fredy 13/03/2023
    \dataatualizacao{4/9/23} % Jander
    \competencias{
        cg-aprender/{ce-ap-1, ce-ap-2, ce-ap-4},
        cg-produzir/{ce-pro-1, ce-pro-2, ce-pro-3, ce-pro-4, ce-pro-5},
        cg-atuar/{ce-atuar-1, ce-atuar-2, ce-atuar-3},
    }
}