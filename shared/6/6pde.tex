\disciplina{pde}{
    \titulo      {7-9}{Processamento de Dados em Escala}
    \objetivo    {Familiarizar o estudante com os desafios, técnicas e ferramentas de processar dados em larga escala; proporcionar ao estudante uma visão geral dos problemas complexos enfrentados ao se processar dados com severos requisitos em termos de volume, velocidade e variedade e sobre os paradigmas e ferramentas disponíveis.}
    \requisitos  {Sistemas Operacionais}
    \recomendadas{N/A}
    \ementa      {Apresentação de problemas reais causados pelo aumento do volume, variedade ou velocidade com que dados são disponibilizados e devem ser processados. Apresentação de modelos de programação em larga escala para programação em lotes, como Mapreduce e ferramentas associadas. Apresentar modelos de programação e ferramentas para processamento de fluxos contínuos de dados. Apresentar as principais arquiteturas para processamento de dados em escala.}
    \creditos    {4 total (4 teóricos)}
    %    \extra       {x horas}
    \codigo      {DC}{1001515}
    \bibliografia { %deixar linhas em branco para separar os livros
        Lin, Jimmy, and Chris Dyer. "Data-intensive text processing with MapReduce." Synthesis Lectures on Human Language Technologies 3.1 (2010): 1-177.

        White, Tom. Hadoop: The definitive guide. " O'Reilly Media, Inc.", 2012.
    }{
        Marz, Nathan, and James Warren. "A new paradigm for Big Data." Big Data: Principles and best practices of scalable real-time data systems. Shelter Island, NY: Manning Publications (2014).
    }
        %Fredy 13/03/2023
     \competencias{
        cg-aprender/{ce-ap-1, ce-ap-2, ce-ap-3},
        cg-atuar/{ce-atuar-1, ce-atuar-3, ce-atuar-4},
    }
}