\disciplina{compiladores}{
    \titulo      {7-9}{Construção de Compiladores}
    \objetivo    {Habilitar o estudante a não ser apenas um utilizador de linguagens existentes, mas sim um projetista; capacitar o estudante com habilidades para criar suas próprias linguagens para situações de diferentes domínios. Desenvolver no estudante a competência para construir um compilador completo utilizando ferramentas de auxílio à construção automática.}
    \requisitos  {Teoria da Computação e Construção de Algoritmos e Programação}
    \recomendadas{N/A}
    \ementa      {Apresentação e contextualização sobre Compiladores. Visão geral sobre a estrutura de um compilador (etapas de front-end/análise e etapas de back-end/síntese). Detalhamento da etapa de Análise Léxica. Detalhamento da etapa de Análise Sintática: Análise Sintática Descendente. Análise Sintática Ascendente. Detalhamento da etapa de Análise semântica. Detalhamento da etapa de Geração e otimização de código. Noções de Manipulação de erros. Apresentação de algumas ferramentas de auxílio à construção de um compilador. Aprofundamento no projeto e na implementação de um compilador completo, traduzindo uma linguagem de programação simplificada para código executável em arquitetura física ou virtual. Aprofundamento no projeto e na implementação de um compilador (análise léxica, análise sintática, análise semântica e geração de código ou interpretação) para um domínio de escolha do estudante.}
    \creditos    {4 total (4 teóricos)}
    %    \extra       {x horas}
    \codigo      {DC}{1001497}
    \bibliografia { %deixar linhas em branco para separar os livros
        ALFRED V. AHO. et al. Compiladores: princípios, técnicas e ferramentas. 2. ed. São Paulo: Pearson Addison-Wesley, 2007. x, 634 ISBN 9788588639249 - disponível na BCo - UFSCar.

        LOUDEN, Kenneth C. Compiladores: princípios e práticas. São Paulo: Pioneira Thomson Learning, 2004. 569 p. ISBN 85-221-0422-0 - disponível na BCo - UFSCar.

        COOPER, Keith D. ; Torczon, Linda. Engineering a compiler. 2nd. ed. Amsterdam: Elsevier, 2012. xxiii, 800 p. : il., tabs. ISBN 9780120884780 - disponível na BCo - UFSCar.
    }{
        NETO, João José. Introdução à Compilação. 2a. ed. Rio de Janeiro: Elsevier, 2016. 307 p. ISBN 9788535278101.

        PARR, Terence. The Definitive ANTLR 4 Reference. IN: The Pragmatic Bookshelf, 2013. 328 p. ISBN 9781934356999.

        DELAMARO, Márcio E. Como Construir um Compilador Utilizando Ferramentas Java. IN: Novatec, 2004. 308p. ISBN 8575220551.

        MAK, Ronald. Writing compilers and interpreters: a modern software engineering approach using Java. 3rd. ed. Indianapolis, IN: Wiley Publishing, 2009. xxiii, 840 p. : il., tabs. ISBN 9780470177075.
    }
        %Fredy 13/03/2023
     \competencias{
        cg-aprender/{ce-ap-1, ce-ap-2, ce-ap-3},
        cg-produzir/{ce-pro-1, ce-pro-2, ce-pro-4},
        }
}