\disciplina{lm}{
    \titulo      {7-9}{Lógica Matemática}
    \objetivo    {Desenvolver nos estudantes a capacidade de raciocínio lógico e abstrato no intuito de capacitar o estudante a propor algoritmos rápidos e eficientes; ensinar aos estudantes os fundamentos sobre sistemas dedutivos e formalismos da lógica clássica; tornar os estudantes aptos a conhecer os conceitos da Lógica Proposicional e da Lógica de Primeira Ordem e suas aplicações na computação.}
    \requisitos  {N/A} % % xxx?)
    \recomendadas{N/A}
    \ementa      {Apresentação da Lógica proposicional: proposições atômicas, conectivos, proposições compostas, fórmulas bem formadas, linguagem proposicional, semântica (interpretações e modelos), consequência lógica, equivalência lógica, dedução, formas normais, notação clausal, cláusulas de Horn, regras de inferência, argumentos, o princípio da resolução. Apresentação da Lógica de primeira ordem (lógica de predicados): alfabetos, termos, fórmulas bem formadas, linguagem de primeira ordem, escopo de quantificadores, variáveis livres e ligadas, semânticas (modelos), consequência lógica, equivalência lógica, dedução, skolemização, formas normais, quantificação universal, notação clausal, cláusulas de Horn, substituição e unificação, unificadores mais gerais, o princípio de resolução.}
    \creditos    {4 total (4 teóricos)}
    %    \extra       {x horas}
    \codigo      {DC}{1001529}
    \bibliografia { %deixar linhas em branco para separar os livros
        NICOLETTI, M.C. A Cartilha da Lógica. Série de Apontamentos, 2 ed.. São Carlos: EdUFSCar, 2009. 233 p. - Disponível na BCo

        SILVA, F. S. C.; FINGER, M., MELO, A.C.V. Lógica para computação. Thomson, 2010. - Disponível na BCo

        SOUZA, J. N. Lógica para ciência da computação: uma introdução concisa. Elsevier, 2008. - Disponível na BCo
    }{
        LEVADA, A. L. M. Fundamentos de lógica matemática. Coleção UAB-UFSCar, Sistemas de Informação. 2010, 170 pgs.

        GERSTING, J. L. Fundamentos Matemáticos para a Ciência da Computação: um tratamento moderno de matemática discreta. 5 ed. Rio de Janeiro: LTC, 2004. - Disponível na BCo
    }

    % Inserido por Murillo R. P. Homem, em 01/04/2023
    \dataatualizacao{12/12/23} % Luciano
    \competencias{
        % Inserido por Murillo R. P. Homem, em 01/04/2023
        %cg-aprender/{ce-ap-1, ce-ap-2, ce-ap-3, ce-ap-4},
        %cg-atuar/{ce-atuar-1, ce-atuar-2, ce-atuar-3, ce-atuar-4},
        cg-aprender/{ce-ap-1, ce-ap-4},
        cg-produzir/{ce-pro-2, ce-pro-4},        
        cg-atuar/{ce-atuar-1, ce-atuar-2, ce-atuar-3, ce-atuar-4},        
        }  
}