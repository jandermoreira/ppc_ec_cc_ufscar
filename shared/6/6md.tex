\disciplina{md}{
    \titulo      {7-9}{Matemática Discreta}
    \objetivo    {Familiarizar os estudantes com a estrutura das demonstrações matemáticas, através da apresentação de diversos exemplos e exercícios;
    Capacitar os estudantes a deduzir e utilizar fatos e noções elementares sobre números, conjuntos, relações, funções e grafos;
    Tornar os estudantes aptos a analisar cenários e situações envolvendo probabilidade;
    Estimular os estudantes a utilizar raciocínio indutivo em suas análises.}
    \requisitos  {N/A} % % xxx?)
    \recomendadas{N/A}
    \ementa      {Introdução à matemática discreta; Apresentação de estratégias de demonstração de teoremas com detalhamento de indução matemática; Introdução a teoria dos números, somatórios e produtórios, e teoria dos conjuntos, com apresentação de propriedades matemáticas e demonstrações das mesmas; Apresentação de relações, relações de equivalência e relações de ordem; Noções de funções, funções injetoras, funções sobrejetoras e funções bijetoras; Introdução a grafos com apresentação de conceitos, como: conectividade e subgrafos, orientação e caminhos, graus e cortes, laços e arestas paralelas, emparelhamento e coloração; além da introdução de categorias de grafos, como: árvores, circuitos e grafos bipartidos, eulerianos, hamiltonianos, planares e duais; Problematização com exemplos práticos da computação.}
    \creditos    {4 total (4 teóricos)}
    %    \extra       {x horas}
    \codigo      {DC}{1001500}
    \bibliografia { %deixar linhas em branco para separar os livros
        K. H. Rosen, Discrete Mathematics and its Applications. 7ª Edição, McGraw-Hill. 2013.

        LEHMAN, E.; LEIGHTON, F. T.; MEYER, A. R. Mathematics for Computer Science. 2017. Disponível em: \url{https://courses.csail.mit.edu/6.042/spring17/mcs.pdf}.

        GOMIDE, A.; STOLFI, J. Elementos de Matemática Discreta para Computação. 238 p. 2011. Disponível em: \url{http://www.ic.unicamp.br/~stolfi/cursos/MC358-2012-1-A/docs/apostila.pdf}

        SCHEINERMAN, Edward R. Matemática discreta: uma introdução. São Paulo: Thomson Learning Edições, 2006. 532 p. ISBN 85-221-0291-0. Disponível na BCO.
    }{
        D. Velleman, How to Prove It, A Structured Approach, 2a. Edição, Cambridge, 2006.

        GERSTING, Judith L. Fundamentos matemáticos para a ciência da computação: um tratamento moderno de matemática discreta. 5. ed. Rio de Janeiro: LTC, c2004. 597 p. ISBN 978-85-216-1422-7. Disponível na BCO.

        STEIN, C.; DRYSDALE, R. L.; BOGART, K. Matemática discreta para ciência da computação. São Paulo: Pearson Education do Brasil, 2013. 394 p.
    }

    % Inserido por Murillo R. P. Homem, em 01/04/2023
    \competencias{
        cg-aprender/{ce-ap-1, ce-ap-2, ce-ap-3, ce-ap-4},
        cg-atuar/{ce-atuar-1, ce-atuar-2, ce-atuar-3, ce-atuar-4},
        } 
}