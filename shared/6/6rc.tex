\disciplina{rc}{
    \titulo      {7}{Redes de Computadores}
    \objetivo    {Estudar as redes de computadores, abordando suas operações, funcionalidades e serviços. Apresentar tecnologias de conexão existentes, abordando aspectos de hardware e de protocolos e o projeto físico e lógico de redes.}
    \requisitos  {Sistemas Operacionais} % xxxxx
    \recomendadas{N/A}
    \ementa      {Transmissão de dados: camadas física e de enlace, sinalização, modulação e codificação, framing, endereçamento, camadas física e de enlace. Endereçamento lógico e físico, encaminhamento, roteamento e mobilidade na Internet. Endereçamento físico e lógico, roteamento fixo e dinâmico, mobilidade de nós, encaminhamento de pacotes. Controle de fluxo e de congestionamento: latência, bufferbloat, bandwidth, throughput, controle de fluxo fim a fim, controle de congestionamento na rede. Gerenciamento de rede: configuração, desempenho, contabilização, falha e segurança. Redes definidas por software (SDN), Redes de sensores, redes móveis, redes ad-hoc e redes veiculares. Qualidade de Serviço (QoS).}
    \creditos    {4 total (4 teóricos)}
    %    \extra       {x horas}
    \codigo      {DC}{1001504}
    \bibliografia {TANENBAUAN, A. "Computer Networks". Prentice-Hall, 3. ed., 1996. %G 004.6 T164c.2 (BCo) 
    % 
    KUROSE, J. F. ; ROSS, K. W. Redes de Computadores e a Internet: Uma abordagem top-down. Pearson Addison Wesley– 6ª Edição, 2014
    % 
    PETERSON, L. L.; DAVIE, B. S. Computer Networks: A Systems Approach, 5. ed,,  Editora Elsevier}
    {COMER, D. E. Redes de Computadores e Internet, 6. ed, Editora Bookman, 2016}

    % Fredy Valente 06/03/2023
    \dataatualizacao{06/11/23} % Kelen, Luciano, Fedy, Alexandre, Kato, Helio, Jander, Menotti, Orides    
    \competencias
    {
        cg-aprender/{ce-ap-1, ce-ap-2, ce-ap-4},
        cg-produzir/{ce-pro-2, ce-pro-4, ce-pro-5},
        cg-atuar/{ce-atuar-1, ce-atuar-2, ce-atuar-3, ce-atuar-4}
    }
}