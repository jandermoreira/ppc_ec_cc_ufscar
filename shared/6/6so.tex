\disciplina{so}{
    \titulo      {5}{Sistemas Operacionais}
    \objetivo    {Familiarizar os estudantes com Sistemas Operacionais, apresentando seus objetivos, suas funcionalidades e aspectos de suas organizações internas. Familiarizar os estudantes com as políticas para o gerenciamento de processos e recursos. Familiarizar os estudantes com as funcionalidades providas pelos Sistemas Operacionais como gerenciadores de recursos. Tornar o estudante ciente dos algoritmos e das abstrações utilizadas em projetos de sistemas operacionais para o gerenciamento de atividades a executar (processos e threads) e para o armazenamento de dados (arquivos). Habilitar o estudante a identificar os requisitos existentes para diferentes tipos de sistemas computacionais e suas implicações no projeto do sistema operacional (sistemas de tempo-real, servidores, dispositivos com capacidades de software e hardware limitadas). Tornar os estudantes aptos a criar programas que usem eficientemente os recursos e serviços providos por sistemas operacionais. Tornar os estudantes aptos a entender e atuar no projeto e no desenvolvimento de sistemas operacionais.}
    \requisitos  {Arquitetura e Organização de Computadores 1}
    \recomendadas{N/A}
    \ementa      {Introdução. Interface do SO. Processos, threads e gerenciamento do processador. Gerenciamento de memória. Comunicação e sincronização de processos e threads. Gerenciamento de armazenamento. Estudo de caso com sistemas operacionais.}
    \creditos    {6 total (4 teóricos, 2 práticos)}
    %    \extra       {x horas}
    \codigo      {DC}{1001535}
    \bibliografia {TANENBAUM, A.S. "Sistemas Operacionais Modernos", 2. ed., Pearson Prentice Hall, 2008.
    % 
    SILBERSCHATZ, A.; GALVIN, P. B.; Gagne, G. Fundamentos de sistemas operacionais. Trad. 6. ed. LTC, 2009.
    % 
    TANENBAUM, A. S.; WOODHULL, A.S. Operating systems: design and implementation. 3 ed. Pearson Prentice Hall, 2009.}
    {STALLINGS, W. "Operating System: Internals and Design Principles",  6. ed., Prentice Hall, 2008. ISBN-10: 0136006329, ISBN-13: 978-0136006329.
    %
    MACHADO, F.B., MAIA, L.P. "Arquitetura de Sistemas Operacionais", 4. ed., LTC, 2007.ISBN: 8521615485, ISBN-13: 9788521615484.
    %
    DEITEL, H.M.; DEITEL, P.J. ; CHOFFNES. "Sistemas Operacionais", PRENTICE HALL BRASIL, 2007. ISBN: 8576050110, ISBN-13: 9788576050117.
    %
    GUIMARAES, C. C. Principios de sistemas operacionais. Rio de Janeiro: Campus, 1980. 222 p.
    %
    KIRNER, C.; MENDES, S. B. T. Sistemas operacionais distribuidos: aspectos gerais e analise de sua estrutura. Rio de Janeiro: Campus, 1988. 184 p. ISBN 85-7001-475- }

    % Comentado por Jander, para remover o erro de duplicação de competências
    % \competencias{
    % % Sistemas Operacionais
    % % Inserido por Murillo R. P. Homem, em 09/02/2023
    % % Compilado a partir do formulário preenchido por Hélio Guardia
    %     cg-aprender/{ce-ap-1, ce-ap-2, ce-ap-4},
    %     cg-produzir/{ce-pro-1, ce-pro-2, ce-pro-4},
    %     cg-atuar/{ce-atuar-1, ce-atuar-4}
    % }
    
    % Fredy Valente 06/03/2023
    \dataatualizacao{06/11/23} % Kelen, Luciano, Fedy, Alexandre, Kato, Helio, Jander, Menotti, Orides
    \competencias{
        %cg-aprender/{ce-ap-1, ce-ap-2, ce-ap-4},
        %cg-produzir/{ce-pro-1, ce-pro-2, ce-pro-4},
        %cg-atuar/{ce-atuar-1,  ce-atuar-4}
        cg-aprender/{ce-ap-1, ce-ap-2, ce-ap-4},
        cg-produzir/{ce-pro-2, ce-pro-4},
        cg-atuar/{ce-atuar-4, ce-atuar-5}
    }
}