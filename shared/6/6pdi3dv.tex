\disciplina{pdi3dv}{
    \titulo      {7-9}{Processamento Digital de Imagens 3D e Vídeos}
    \objetivo    {Habilitar o estudante apto a visualizar e processar imagens tridimensionais, bem como sequências temporais de imagens (vídeos); habilitar o estudante a aplicar técnicas eficientes de processamento de imagens, essenciais para a análise de imagens 3D e vídeos.}
    \requisitos  {Processamento Digital de Imagens} % % xxx?)
    \recomendadas{N/A}
    \ementa      {Revisão sobre técnicas básicas de processamento de imagens. Apresentação de ferramentas e técnicas de visualização de imagens 3D e vídeo. Visão geral sobre formatos de vídeos. Aprofundamento sobre os desafios encontrados em imagens 3D não-isotrópicas. Análise de movimento (estimação e estabilização de movimento, fluxo ótico, rastreamento de objetos). Processamento espaço-temporal. Apresentação sobre técnicas de interpolação. Detalhamento sobre cálculo de esqueleto em 3D. Visão geral de técnicas de detecção de estruturas tubulares.}
    \creditos    {4 total (2 teóricos, 2 práticos)}
    %    \extra       {x horas}
    \codigo      {DC}{1001480}
    \bibliografia { %deixar linhas em branco para separar os livros
        D. A. Forsyth and J. Ponce, "Computer Vision: A Modern Approach", Prentice Hall, 2003. (disponível na BCo).

        R. C. Gonzalez and R. E. Woods, “Digital Image Processing” (3rd Edition), Prentice-Hall, 2008. (disponível na BCo).

        A. Bovik, “Handbook of image and video processing” (2. ed), Elsevier Academic Press, 2005.
    }{
        A. Kaebler and G. Bradski, “Learning OpenCV - Computer Vision in C++ with the OpenCV library” (1st. Edition), O’Reilly, 2017.

        J. W. Woods, “Multidimensional signal, image, and video processing and coding”, Elsevier, 2006.

        A. M. Tekalp, “Digital video processing”, Prentice Hall Press, 2015.

        Openheim, A. V. and Schafer, R. W., Discrete-Time Signal Processing, Prentice-Hall, 1989 (disponível na BCO - UFSCar)

        Proakis, J. G. and Manolakis, D. G., Digital Signal Processing: Principles, Algorithms and Applications, MacMIllan, 1992 (disponível na BCO – UFSCar)
    }
        %Fredy 13/03/2023
    \dataatualizacao{06/11/23} % Kelen, Luciano, Fedy, Alexandre, Kato,    Helio, Jander, Menotti, Orides        
    \competencias
    {
        %cg-aprender/{ce-ap-1, ce-ap-2, ce-ap-3},
        %cg-produzir/{ce-pro-1, ce-pro-2, ce-pro-4},
        %cg-atuar/{ce-atuar-1, ce-atuar-3, ce-atuar-4},
        cg-aprender/{ce-ap-1, ce-ap-4},
        cg-produzir/{ce-pro-2, ce-pro-4, ce-pro-5},
        cg-atuar/{ce-atuar-1, ce-atuar-2, ce-atuar-3}
    }
}