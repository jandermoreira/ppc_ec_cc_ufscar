\disciplina{am1}{
    \titulo      {7-9}{Aprendizado de Máquina 1}
    \objetivo    {Familiarizar o estudante com conceitos básicos e algoritmos de aprendizado de máquina supervisionado e não-supervisionado. Capacitar o estudante a identificar quais algoritmos de aprendizado de máquina e quais ferramentas podem ser adequados a cada problema. Capacitar o estudante a realizar a análise de resultados desses algoritmos.}
    \requisitos  {Inteligência Artificial e Probabilidade e Estátistica}
    \recomendadas{N/A}
    \ementa      {Apresentação de conceitos básicos e exemplos de aplicação de Aprendizado de Máquina. Noções de ferramentas e linguagens apropriadas para AM. Visão geral sobre aprendizado supervisionado: classificação, regressão e seleção de modelos e generalização. Detalhamento sobre técnicas de avaliação e comparação de modelos de classificação. Visão geral sobre aprendizado não-supervisionado: agrupamento, aprendizado competitivo e regras de associação. Introdução a técnicas de pré-processamento e redução de dimensionalidade: seleção e transformação de atributos e pré-processamento de dados não estruturados.}
    \creditos    {4 total (4 teóricos)}
    %    \extra       {x horas}
    \codigo      {DC}{1001524}
    \bibliografia { %deixar linhas em branco para separar os livros
        MITCHELL, Tom M. Machine learning. Boston: MCB/McGraw-Hill, 1997. 414 p. (McGraw-Hill Series in Computer Science). ISBN 0-07-042807-7

        WITTEN, Ian H.; FRANK, Eibe. Data mining: practical machine learning tools and techniques. 2. ed. San Francisco: Elsevier, c2005. 524 p. (The Morgan Kaufmann Series in Data Management Systems). ISBN 0-12-088407-0.

        ALPAYDIN, Ethem. Introduction to machine learning. Cambridge: MIT Press, c2004. 415 p. (Adaptive Computation and Machine Learning). ISBN 0-262-01211-1.
    }{
        FACELI, Katti; LORENA, Ana Carolina; GAMA, João; CARVALHO, André Carlos Ponce de Leon Ferreira de. Inteligência artificial: uma abordagem de aprendizado de máquina. Rio de Janeiro: LTC, 2011. 378 p. ISBN 9788521618805.

        PANG-NING, Tan; STEINBACH, Michael; KUMAR, Vipin. Introduction data mining. Boston: Pearson Education, c2006. 769 p. ISBN 0-321-32136-7.

        BISHOP, Christopher M. Pattern recognition and machine learning. New York: Springer, c2006. 738 p. (Information Science and Statistics). ISBN 978-0-387-31073-2.

        HAYKIN, Simon S. Neural networks and learning machines. 3. ed. Upper Saddle River: Pearson Education, 2008. 906 p. ISBN 978-0-13-147139-9.

        SILVA, Ivan Nunes da; SPATTI, Danilo Hernane; FLAUZINO, Rogério Andrade. Redes neurais artificiais: para engenharia e ciências aplicadas. São Paulo: Artliber, 2010. 399 p. ISBN 978-85-88098-53-4.
    }
        %Fredy 13/03/2023
     \competencias{
        cg-aprender/{ce-ap-1, ce-ap-2, ce-ap-3},
        cg-atuar/{ce-atuar-1, ce-atuar-3, ce-atuar-4},
    }
}