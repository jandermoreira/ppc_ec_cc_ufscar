\documentclass[aspectratio=169]{beamer}
\usetheme{Berkeley}
\usecolortheme{beaver}
\logo{\includegraphics[width=1.5cm]{logo}}

\usepackage[utf8]{inputenc}
\usepackage[portuguese]{babel}


\title{Projetos Pedagógicos dos Cursos}
\author{Departamento de Computação}
\date{\today}

\begin{document}

\frame{\titlepage}
 
\begin{frame}{Como compilar o projeto de cada curso?}
    \begin{itemize}
        \item Clique no arquivo .tex principal do projeto de cada curso e com ele selecionado clique em ``Recompilar''.
        \item As opções são:
        \begin{itemize}
            \item \texttt{enc/ppc\_enc.tex}
            \item \texttt{bcc/pp\_bcc.tex}
        \end{itemize}
    \end{itemize}
\end{frame}

\begin{frame}{As disciplinas estão ordenadas por eixos}
    \begin{enumerate}
        \item Fundamentos de Matemática e Estatística			\item Física
        \item Eletrônica
        \item Algoritmos e Programação
        \item Arquitetura de Computadores
        \item Metodologias e Técnicas de Computação	
        \item Engenharia e Sistemas	
        \item Humanas, Estágios e TCCs (Há um template nesta pasta!)
        \item Especialização e Trilhas	
        \item Tópicos (Ementas abertas)
    \end{enumerate}
    Na importação, basta digitar \texttt{\textbackslash input\{} seguido do número e uma lista das opções já é apresentada. 
\end{frame}

\begin{frame}{Documentos com as fichas}
    \begin{itemize}
        \item Obrigatórias: \url{https://docs.google.com/document/d/1tsAxg0g9l__Sk1p1Pfcro1SfYT2nffcR6NFcrUDf9UA/edit}
        \item Optativas: 
        \url{https://docs.google.com/document/d/1SvJaQCb_BmQxaPE7-K8BUhPJ7SeW4QhnHr6Ak3p42JE/edit}
    \end{itemize}
\end{frame}

\end{document}
